\documentclass[a4paper,12pt]{article}  % Clase de documento

\usepackage[
  a4paper,
  left=3cm,
  right=3cm,
  top=2.5cm,
  bottom=2.5cm
]{geometry}

% Paquetes esenciales
\usepackage[utf8]{inputenc}         % Codificación UTF-8
\usepackage[T1]{fontenc}            % Soporte para caracteres en PDF
\usepackage[main=spanish]{babel}    % Traducir a español títulos y otros elementos
\usepackage{amsmath, amssymb}       % Matemáticas avanzadas
\usepackage{graphicx}               % Para incluir imágenes
\usepackage{wrapfig}                % Figuras que se envuelven en el texto
\usepackage{subcaption}             % Leyendas en figuras anidadas
\usepackage[most]{tcolorbox}        % Paquete para crear cajas de texto
\usepackage[table]{xcolor}          % Paquete para crear colores
\usepackage{hyperref}               % Enlaces internos y externos
\usepackage{array, colortbl}        % Tablas
\usepackage{soul}                   % Resaltado de texto
\usepackage{enumitem}               % Listas personalizadas (itemize, enumerate)
\usepackage{cancel}                 % Cancelar términos en ecuaciones
\usepackage{textcomp}               % Símbolos como el centavo (¢)

\usepackage{tikz}
\usepackage{pgfplots}
\pgfplotsset{compat=1.18}
\usepgfplotslibrary{fillbetween}

% Referencias
\usepackage[backend=biber,style=alphabetic]{biblatex}
\addbibresource{references.bib}

\usepackage{xcolor}
\usepackage{titlesec}

% Definir colores personalizados
\definecolor{dark_blue}{RGB}{0, 102, 204}
\definecolor{dark_red}{RGB}{204, 0, 0}
\definecolor{soft_green}{RGB}{0, 153, 0}
\definecolor{intense_blue}{RGB}{30, 144, 255}
\definecolor{highlight}{RGB}{255, 230, 153}
\definecolor{lightyellow}{RGB}{255, 249, 171}
\definecolor{lightdanger}{RGB}{255, 205, 210}
% Colores adquiridos de https://latexcolor.com/
\definecolor{brandeisblue}{rgb}{0.0, 0.44, 1.0}
\definecolor{asparagus}{rgb}{0.53, 0.66, 0.42}
\definecolor{airforceblue}{rgb}{0.36, 0.54, 0.66}
\definecolor{chamoisee}{rgb}{0.63, 0.47, 0.35}
\definecolor{champagne}{rgb}{0.97, 0.91, 0.81}
\definecolor{applegreen}{rgb}{0.55, 0.71, 0.0}

\sethlcolor{highlight}
% Cambiar el color de los títulos
\titleformat{\section}
  {\color{dark_blue}\normalfont\Large\bfseries} % Estilo del título de sección
  {\thesection}{1em}{} % Numeración y espacio

\titleformat{\subsection}
  {\color{dark_red}\normalfont\large\bfseries} % Estilo del título de subsection
  {\thesubsection}{1em}{}

\titleformat{\subsubsection}
  {\color{soft_green}\normalfont\normalsize\bfseries} % Estilo del título de subsubsection
  {\thesubsubsection}{1em}{}

\renewcommand{\paragraph}[1]{%
  \noindent\textcolor{purple}{\textbf{#1}}%
  \hspace{1.2em}%
}

\renewcommand{\subparagraph}[1]{%
  \noindent{\textbf{#1}}%
  \hspace{1.2em}%
}

\newcounter{ejemplo}[section]
\renewcommand{\theejemplo}{\thesection\Alph{ejemplo}}

\newcommand{\ejemplo}{%
    \refstepcounter{ejemplo}%
    \textbf{Ejemplo \thesection.\Alph{ejemplo}}%
}

% Configuración de las listas
\setlist[itemize]{label=\textbullet} % Cambiar el símbolo de las listas itemize
\setlist{topsep=3pt,parsep=0pt} % Eliminar espacio entre la lista y el texto

% Definir comandos personalizados
\newcommand{\vect}[1]{\mathbf{#1}} % Vectores en negrita
\newcommand{\diff}[2]{\frac{d#1}{d#2}} % Derivada
\newcommand{\pdiff}[2]{\frac{\partial #1}{\partial #2}} % Derivada parcial

% Tile para resúmenes
\tcbset{colback=blue!5!white, colframe=airforceblue!75!black, fonttitle=\bfseries}

% Tile para cuadros de recordatorio
\tcbset{
  remember/.style={
    colback=chamoisee!5!white,
    colframe=champagne!75!black,
    fonttitle=\bfseries,
    arc=6pt
  }
}

% Datos curiosos
\tcbset{
  interesting_data/.style={
    colback=chamoisee!5!white,
    colframe=applegreen!75!black,
    fonttitle=\bfseries,
    arc=6pt
  }
}

% Sección resaltada para escribir conclusiones bonitas
\tcbset{
  myconclusion/.style={
    enhanced,
    colback=lightyellow,
    colframe=white, % blanco hace invisible el borde
    boxrule=0pt,    % grosor del borde: 0pt = sin borde
    arc=6pt,        % redondeado de bordes
    boxsep=5pt,     % separación entre texto y borde
    left=5pt,
    right=5pt,
    top=5pt,
    bottom=5pt
  }
}

% Sección resaltada para escribir conclusiones bonitas
\tcbset{
  mydanger/.style={
    enhanced,
    colback=lightdanger,
    colframe=white, % blanco hace invisible el borde
    boxrule=0pt,    % grosor del borde: 0pt = sin borde
    arc=6pt,        % redondeado de bordes
    boxsep=5pt,     % separación entre texto y borde
    left=5pt,
    right=5pt,
    top=5pt,
    bottom=5pt
  }
}

\hypersetup{
    colorlinks=true,
    linkcolor=black,
    urlcolor=intense_blue,
    citecolor=black
}
\graphicspath{ {./images/} }

\begin{document}

  \begin{titlepage}
  \centering
  \vspace*{2cm} % Espacio superior

  {\scshape\LARGE Universidad de Mendoza\par}
  \vspace{2cm}
  {\large Basado en el programa de la materia\par}
  \vspace{1.5cm}

  {\Huge\bfseries Algebra Lineal\par}
  \vspace{0.5cm}
  {\Large\itshape Resumen de la Materia\par} % Subtítulo opcional

  \vspace{1.5cm}
  \includegraphics[width=0.45\textwidth]{images/cover.png} % Imagen centrada
  \vspace{2.5cm}

  {\Large Elio Valentino Anci\par}
  {\large Ingeniería en Computación\par}

  \vfill

  {\large \today\par}
\end{titlepage}
  \newpage

  \tableofcontents
  \newpage

  \section{Programación Lineal}

\subsection{Introducción}

La programación lineal es una técnica matemática de optimización que se utiliza para encontrar la mejor solución posible a un problema cuando se tienen recursos limitados y múltiples opciones para usar esos recursos.

¿Qué es exactamente? Es un método que permite maximizar o minimizar una \hl{función objetivo} (como ganancias, costos, tiempo, etc.) sujeta a un conjunto de \hl{restricciones lineales}. Tanto la función objetivo como las restricciones se expresan mediante ecuaciones o inecuaciones lineales, es decir, donde las variables aparecen elevadas solo a la primera potencia.

\noindent\textbf{Componentes principales:}
\begin{itemize}
  \item \textbf{Función objetivo}: Lo que queremos optimizar (maximizar ganancias o minimizar costos, por ejemplo)
  \item \textbf{Variables de decisión}: Las cantidades que podemos controlar y necesitamos determinar
  \item \textbf{Restricciones}: Las limitaciones del problema (presupuesto disponible, tiempo, materiales, capacidad de producción, etc.)
\end{itemize}

\subsection{Problemas de Optimización}

En un \textit{problema de optimización}, se busca maximizar o minimizar una cantidad específica llamada \hl{objetivo}, la cual depende de un número finito de variables de entrada. Estas variables pueden ser independientes entre sí o estar relacionadas a través de una o más restricciones.

\ejemplo\label{ej:ppl_lineal}: El siguiente problema:
\begin{align*}
  \text{max:} \quad         &z = 3x_1 + 2x_2 \\[3pt]
  \text{sujeto a:} \quad    &x_1 + x_2 \leq 10 \\
                            &x_1,x_2 \geq 0
\end{align*}
es un problema de optimización para el objetivo \(z\). Las variables de entrada son \(x_1\) y \(x_2\) y se denominan \textit{variables de decisión}, que deben cumplir dos restricciones: \(x_1 + x_2 \leq 10\) y \(x_1,x_2 \geq 0\).

El problema del ejemplo \ref{ej:ppl_lineal} pide maximizar \(z = 3x_1 + 2x_2\), es decir, encontrar los valores de \(x_1\) y \(x_2\) que maximicen \(z\) bajo las restricciones dadas.

En general, se llama PPL a un problema de optimización que se puede resolver mediante técnicas de programación lineal. Un problema de programación matemático es lineal si tanto la función objetivo \(z\) como las restricciones son lineales. Esto es:
\begin{align*}
  \text{optimizar:} \quad         &z = f(x_1, x_2, \ldots, x_n) \\[3pt]
  \text{sujeto a:} \quad    &g_i(x_1, x_2, \ldots, x_n) \thicksim  b_i \qquad \forall i \in \{1, 2, \ldots, m\} \\
\end{align*}
donde las restricciones son ecuaciones o desigualdades lineales, es decir emplean alguno de los símbolos \(\leq\), \(\geq\) o \(=\).

\subsection{Planteamiento de problemas}

Los problemas de optimización se plantean muy a menudo verbalmente, es decir, en palabras. Ya se verá un ejemplo a continuación (ejemplo \ref{ej:ppl_verbal}) de un problema de optimización planteado verbalmente. El procedimiento para la solución consiste en realizar un modelo del problema para poder resolverlo mediante técnicas de programación lineal.

\begin{tcolorbox}[interesting_data, title=¿Existe solo una forma de plantear problemas?]
  Existen múltiples formas de plantear un problema de optimización, por lo que los dos métodos que se proponen en este documento puedes tomarlos como sugerencias.
\end{tcolorbox}

\begin{quote}
  \ejemplo\label{ej:ppl_verbal}: Una compañía de petróleos produce en sus refinerías gasóleo (\(G\)), gasolina sin plomo (\(P\)) y gasolina súper (\(S\)) a partir de dos tipos de crudos, \(C_1\) y \(C_2\). Las refinerías están dotadas de dos tipos de tecnologías. La tecnología nueva (\(T_n\)) utiliza en cada sesión de destilación \(7\) unidades de \(C_1\) y \(12\) de \(C_2\), para producir \(8\) unidades de \(G\), \(6\) de \(P\) y \(5\) de \(S\). Con la tecnología antigua (\(T_a\)), se obtienen en cada destilación \(10\) unidades de \(G\), \(7\) de \(P\) y \(4\) de \(S\), con un gasto de \(10\) unidades de \(C_1\) y \(8\) de \(C_2\).

  Estudios de la demanda permiten estimar que para el próximo mes se deben producir al menos \(900\) unidades de \(G\), \(300\) de \(P\) y entre \(800\) y \(1700\) de \(S\). La disponibilidad de \(C_1\) es de \(1400\) unidades y de \(C_2\) de \(2000\) unidades. Los beneficios por unidad producida son:
  \begin{table}[ht]
    \centering
    \begin{tabular}{|c|c|c|c|}
    \hline
    Gasolina & \textit{G} & \textit{P} & \textit{S} \\ \hline
    Beneficio/u & 4 & 6 & 7 \\ \hline
    \end{tabular}
  \end{table}

  La compañía desea conocer cómo utilizar ambos procesos de destilación, que se pueden utilizar total o parcialmente, y los crudos disponibles para que el beneficio sea el máximo.
\end{quote}
\vspace{5mm}
\hrule
\vspace{5mm}

Este problema es un ejemplo típico de un PPL verbal. Hay una cantidad que se desea optimizar y un conjunto de restricciones que se deben cumplir. En este caso, se desea maximizar el beneficio y las restricciones son las cantidades de crudos disponibles y las cantidades de gasolina que se deben producir. Más adelante vamos a plantear el PPL asociado a este problema y vamos a resolver otros problemas. De momento con darnos a la idea de cómo es un problema verbal es suficiente.

\subsubsection{Métodos de planteamiento de PPLs}

Como dijimos anteriormente, existen múltiples formas de plantear un problema de optimización, por lo que los dos métodos que se proponen son el método recomendado por el libro Bronson (sacado de la cátedra) y el método que se propone en el libro de Alfaomega (\cite{PPL_Alfaomega}). 

Luego de describir los métodos de planteamiento de PPLs, el ejemplo \ref{ej:modelo_de_ppl_verbal} muestra el planteo del ejemplo \ref{ej:ppl_verbal} con el método de Alfaomega.

\begin{tcolorbox}[title=Método del libro Bronson]
  Luego de tener un enunciado verbal de un problema de optimización, se deben seguir los siguientes pasos:
  \begin{enumerate}
    \item Determínese la cantidad que se optimizará y exprésese como una función matemática. Hacer esto sirve para definir las variables de decisión.
    \item Identifíquese todos los requerimientos, restricciones y limitaciones estipulados, y exprésense matemáticamente. Estos requerimientos constituyen las restricciones.
    \item Exprésense todas aquellas condiciones ocultas. Tales condiciones no están estipuladas explícitamente, pero se hacen evidentes a partir de la situación física para la que se está planteando el modelo. Generalmente, involucran requerimientos de no negatividad o de ser enteras, para las variables de decisión.
  \end{enumerate}  
\end{tcolorbox}

\noindent Por otro lado, el método que se proponen en el libro PPL de Alfaomega es el siguiente:
\begin{tcolorbox}[title=Método del libro Alfaomega]
  \begin{enumerate}
    \item \textit{Reconocimiento de las variables de decisión}: las variables de decisión son las variables sobre las que el decisor tiene control y que se suponen continuas. Representan productos o bienes a producir, almacenar o vender, disponibilidad o adquisición de materias primas, etc.
    \item \textit{Identificación de las restricciones}: las restricciones representan las limitaciones o requisitos y definen la \textit{región factible} del problema. Representan el deseo de no exceder un valor específico (\(\leq\)), no descender por debajo de un valor particular (\(\geq\)) o ser igual a un valor particular (\(=\)).
    \item \textit{Obtención de la función objetivo}: la función objetivo es la que se quiere maximizar o minimizar. Representa el beneficio, renta, ganancias, costos, etc.
    \item \textit{Formulación del PPL}: cuando se tienen los tres pasos anteriores se puede formular el PPL. La solución de este PPL se denomina \textit{solución óptima} y se estudiará en la proxima sección.
  \end{enumerate}
\end{tcolorbox}

\begin{tcolorbox}[mydanger]
  \textbf{Cuidado:} El paso 2 de la metodología de Alfaomega incluye \textit{todas} las restricciones, incluso aquellas que no están en el enunciado verbal (como no negatividad por ejemplo).
\end{tcolorbox}

\vspace{5mm}
\hrule
\vspace{5mm}

\begin{quote}
  \ejemplo\label{ej:modelo_de_ppl_verbal}: Tomemos el PPL verbal del ejemplo \ref{ej:ppl_verbal}. Propongo al lector que intente plantear el PPL asociado (no resolver) y luego comparar con la solución. En este caso vamos a modelar el PPL siguiendo el método recomendado por el libro de Alfaomega (segundo método).

  \noindent\textbf{Paso 1: Reconocimiento de las variables de decisión}

  Si bien la consigna puede resultar un poco enredada al principio, si prestamos atención, vemos que al final nos pregunta sobre \hl{cómo usar los procesos de destilación}. Esto nos da un indicio de que lo que se quiere es saber cuánto usar el proceso \(T_n\) y el proceso \(T_a\) para maximizar el beneficio.

  Entonces, las variables sobre las que el vendedor tiene control son:
  \begin{itemize}
    \item \(x_1 ~\rightarrow\) cantidad de sesiones de destilación usando el proceso \(T_n\)
    \item \(x_2 ~\rightarrow\) cantidad de sesiones de destilación usando el proceso \(T_a\)
  \end{itemize}

  \noindent\textbf{Paso 2: Identificación de las restricciones}

  Tenemos restricciones debidas a las limitaciones en la disponibilidad de ambos tipos de crudos:
  \begin{itemize}
    \item Para \(C_1\): \(7x_1 + 10x_2 \leq 1400\)
    \item Para \(C_2\): \(12x_1 + 8x_2 \leq 2000\)
  \end{itemize}

  \noindent También se tienen restricciones según las necesidades de refinado que se requieren:
  \begin{itemize}
    \item Para Gasóleo (\(G\)): \(8x_1 + 10x_2 \geq 900\)
    \item Para Sin Plomo (\(P\)): \(6x_1 + 7x_2 \geq 300\)
    \item Para Super (\(S\)): \((5x_1 + 4x_2 \geq 800) \wedge (5x_1 + 4x_2 \leq 1700)\)
  \end{itemize}

  \noindent Además no debemos olvidar de tener en cuenta las restricciones implícitas, ya que es obvio que no se pueden realizar destilaciones negativas, se tiene:
  \begin{itemize}
    \item \(x_1 \geq 0\)
    \item \(x_2 \geq 0\)
  \end{itemize}

  \noindent\textbf{Paso 3: Identificación de la función objetivo}

  El beneficio será la suma de las cantidades de cada refinado que se vende, por lo que:
  \begin{itemize}
    \item Cantidad de Gasóleo: \(8x_1 + 10x_2 ~ \rightarrow G \times \text{Precio: } 4(8x_1 + 10x_2) = 32x_1 + 40x_2\)
    \item Cantidad de Sin Plomo: \(6x_1 + 7x_2 ~ \rightarrow P \times \text{Precio: } 6(6x_1 + 7x_2) = 36x_1 + 42x_2\)
    \item Cantidad de Super: \(5x_1 + 4x_2 ~ \rightarrow S \times \text{Precio: } 7(5x_1 + 4x_2) = 35x_1 + 28x_2\) 
  \end{itemize}

  \noindent Por lo tanto, la función objetivo es:
  \begin{align*}
    z &= 32x_1 + 40x_2 + 36x_1 + 42x_2 + 35x_1 + 28x_2 \\
      &= 103x_1 + 110x_2
  \end{align*}

  \noindent\textbf{Paso 4: Formulación del PPL}
  \begin{align*}
    \text{maximizar:} \quad   &z = 103x_1 + 110x_2 \\[3pt]
    \text{sujeto a:} \quad    &7x_1 + 10x_2 \leq 1400 \\
                              &12x_1 + 8x_2 \leq 2000 \\
                              &8x_1 + 10x_2 \geq 900 \\
                              &6x_1 + 7x_2 \geq 300 \\
                              &5x_1 + 4x_2 \geq 800 \\
                              & 5x_1 + 4x_2 \leq 1700 \\
                              &x_1 \geq 0 \\
                              &x_2 \geq 0
  \end{align*}
\end{quote}

Es muy importante entender la metodología del planteo, ya que si el modelo realizado no es adecuado, el resultado obtenido cuando se resuelva el problema será, muy probablemente, incorrecto.

Al intentar plantear el problema del ejemplo \ref{ej:ppl_verbal} tal vez te hayan surgido algunas dudas, así que vamos a realizar un pequeño repaso sobre algunos detalles en el procedimiento del ejemplo \ref{ej:modelo_de_ppl_verbal}.

\noindent\textbf{¿Por qué se tomaron \(T_n\) y \(T_a\) como variables de decisión y no \(C_1\) y \(C_2\)?}

Tal vez podrías pensar que \(C_1\) y \(C_2\) pueden ser buenos candidatos a variables de decisión, ya que el vendedor puede controlar cuánta cantidad de cada tipo de curdo usa en cada tecnología. Entonces intentemos generar un modelo con \(C_1\) y \(C_2\) como variables de decisión.

Como ya sabemos que las variables de decisión son \(C_1\) y \(C_2\) entonces el paso 1 resulta:
\begin{itemize}
  \item \(x_1 ~\rightarrow\) cantidad de crudo \(C_1\) utilizado
  \item \(x_2 ~\rightarrow\) cantidad de crudo \(C_2\) utilizado
\end{itemize}

\noindent En el paso 2 tenemos las restricciones:
\begin{enumerate}
  \item Según la tecnología usada
  \item Según la cantidad de crudo disponible
  \item Según la cantidad de refinado que se debe producir dependiendo de la tecnología usada
  \item Restricción de no negatividad (no se pueden refinar cantidades negativas)
\end{enumerate}
Si bien de alguna forma rebuscada puede llegarse a un PPL equivalente, vemos que es mucho más complicado, ya que la tercer restricción está relacionando los crudos con el destilado y la tecnología, por lo que serán restricciones largas y complejas de modelar correctamente.

Es de mucha importancia tomarse el tiempo de analizar bien cuáles son las variables de decisión, ya que de ahí partirá todo el modelo. En caso de que usted elija utilizar el procedimiento de formulación de modelos del libro de Bronson, el paso 1 tiene este riesgo implícito, ya que si se genera una función objetivo cuyas variables de decisión no sean las adecuadas, el modelo puede terminar por ser incorrecto.

En este documento se ha usado el enfoque de Alfaomega ya que la idea de tener un paso específico para analizar el enunciado y tomar las variables de decisión es muy importante y no debe pasarse por alto.

\subsection{Resolución de Programas Lineales}

En esta materia se ven dos métodos para resolver PPLs:
\begin{itemize}
  \item \textbf{Método Gráfico}: se basa en la representación gráfica de las restricciones y de la solución. Tiene la ventaja de ser muy intuitivo y sencillo de entender, pero tiene la desventaja de ser poco eficiente y muy limitado en cuanto al número de variables.
  \item \textbf{Método Analítico}: Es más general que el método gráfico ya que no tiene la desventaja de límite de variables. Este método se conoce como el método Simplex y es muy eficiente y algorítmico. La desventaja es que requiere la transformación previa al \textit{formato estándar} añadiendo variables de holgura y/o variables artificiales.
\end{itemize}
Más adelante veremos en profundidad los dos métodos, así que no se preocupe por el vocabulario desconocido.

\subsubsection{Típos de solución}

Todo PPL puede tener alguna de las siguientes soluciones:
\begin{itemize}
  \item Solución optima única
  \item Solución optima múltiple
  \item Problema no acotado
  \item Problema infactible
  \item Rayo óptimo
\end{itemize}

A continuación vamos a profundizar un poco sobre qué significa cada tipo de solución, para que al momento de resolver PPLs sepa identificar qué tipo de solución tiene.

\paragraph{Solución óptima única}

\noindent Es el caso más común y deseable. Existe exactamente un punto en la región factible donde la función objetivo alcanza su valor óptimo (máximo o mínimo) y presenta las siguientes características:
\begin{itemize}
  \item La función objetivo tiene una pendiente única que no es paralela a ninguna cara de la región factible
  \item Gráficamente, la línea de la función objetivo ``toca'' la región factible en un solo vértice
  \item Matemáticamente, existe un único vector \(x^*\) que optimiza la función
\end{itemize}

\paragraph{Solución óptima múltiple (infinitas soluciones óptimas)}

Existen infinitos puntos que proporcionan el mismo valor óptimo de la función objetivo. Esto ocurre cuando:
\begin{itemize}
  \item La función objetivo es paralela a una de las caras (aristas) de la región factible
  \item Todos los puntos en esa arista proporcionan el mismo valor óptimo
\end{itemize}
Este tipo de soluciones tienen las siguientes características:
\begin{itemize}
  \item Cualquier combinación convexa de dos vértices óptimos adyacentes también es óptima
  \item La solución es un segmento de línea completo en la frontera de la región factible
\end{itemize}

Si el problema tiene una solución óptima múltiple, entonces no existe una preferencia por una solución sobre otra, ya que todas son óptimas. La respuesta que usted elija darle al problema será cualquiera de las soluciones óptimas, y será considerada correcta.

\paragraph{Problema no acotado}

La función objetivo puede crecer (o decrecer) indefinidamente sin violar ninguna restricción. Estos problemas presentan las siguientes características:
\begin{itemize}
  \item La región factible se extiende infinitamente en la dirección que mejora la función objetivo
  \item No existe un valor máximo (o mínimo) finito para la función objetivo
  \item Indica generalmente un error en la formulación del problema
\end{itemize}

\ejemplo : Maximizar \(z = x_1 + x_2\) sujeto solo a \(x_1 \geq 0, x_2 \geq 0\) (sin restricciones superiores).

\paragraph{Problema infactible}

No existe ningún punto que satisfaga simultáneamente todas las restricciones del problema. Este tipo de problemas pueden ocurrir en dos casos:
\begin{itemize}
  \item Las restricciones son contradictorias entre sí
  \item El conjunto de restricciones no tiene intersección común
\end{itemize}
Y presentan las siguientes características:
\begin{itemize}
  \item La región factible está vacía
  \item No hay solución posible al problema
  \item Matemáticamente: el conjunto \(\{x : Ax \leq b, x \geq 0\} = \emptyset\)
\end{itemize}

\ejemplo : Sea el siguiente PPL:
\begin{align*}
  \text{maximizar:} \quad   &z = x_1 + x_2 \\[3pt]
  \text{sujeto a:} \quad    &x_1 + x_2 \leq 5 \\
                            &x_1 + x_2 \geq 10 \\
                            &x_1, x_2 \geq 0
\end{align*}
Estas restricciones son imposibles de satisfacer simultáneamente.

\paragraph{Rayo óptimo}

Este es un caso especial del problema no acotado donde existe una dirección específica (rayo) a lo largo de la cual la función objetivo mejora indefinidamente. Este tipo de problemas presentan las siguientes características:
\begin{itemize}
  \item Existe un punto factible \(x_0\) y una dirección \(d\) tal que \(x_0 + \lambda d\) es factible para todo \(\lambda \geq 0\)
  \item La función objetivo mejora a lo largo de esta dirección: \(c^T d > 0\) (para maximización)
  \item El rayo representa la dirección de crecimiento ilimitado
\end{itemize}
La diferencia con la solución de tipo ``no acotado'' es que el rayo óptimo especifica exactamente la dirección del crecimiento infinito, mientras que ``no acotado'' es la conclusión general.

\paragraph{Identificación en el método simplex}

En el método simplex, la identificación de los tipos de soluciones se realiza de la siguiente manera:
\begin{itemize}
  \item \textbf{Única/Múltiple:} Se identifica en la tabla final del simplex
  \item \textbf{No acotado:} Aparece cuando una variable no básica tiene coeficientes no positivos en su columna
  \item \textbf{Infactible:} Se detecta cuando aparecen variables artificiales con valor positivo en la solución final
  \item \textbf{Rayo óptimo:} Se determina por la dirección correspondiente a la variable que causa el comportamiento no acotado
\end{itemize}

Cada tipo de solución tiene implicaciones importantes para la interpretación y aplicación práctica del modelo de optimización. De momento no se preocupe por entender el método simplex, ya que se verá en detalle en la siguiente sección.

\subsubsection{Método Gráfico}

Como se mencionó anteriormente, el método gráfico es muy intuitivo, y por ende, ideal para aprender los conceptos de PPLs. Para entender el método gráfico lo más sencillo es analizarlo con un ejemplo.

\ejemplo\label{ej:mtd_grfico}: Un expendio de carnes de la cuidad acostumbra a preparar carne para albondigón con una combinación de carne molida de res y carne molida de cerdo. La carne de res contiene \(80\%\) de carne y \(20\%\) de grasa, y le cuesta a la tienda \textcent \textit{80} por libra; la carne de cerdo contiene \(68\%\) de carne y \(32\%\) de grasa y cuesta \textcent \textit{60} por libra 

¿Qué cantidad de cada tipo de carne debe emplear la tienda en cada libra de albondigón, si se desea minimizar el costo y mantener el contenido de grasa no mayor al \(25\%\)?

\noindent\textbf{Resolución del ejemplo \ref{ej:mtd_grfico}:}
\begin{quote}
  \textbf{1. Identificación de variables de decisión:}
  Vemos que el resultado es un albondigón, y este se prepara con \(x_1\) libras de carne de res y \(x_2\) libras de carne de cerdo. Por lo que las variables de decisión son:
  \begin{itemize}
    \item \(x_1 ~\rightarrow\) cantidad de carne de res por libra de albondigón
    \item \(x_2 ~\rightarrow\) cantidad de carne de cerdo por libra de albondigón
  \end{itemize}

  \textbf{2. Identificación de restricciones:}
  Vemos que la cantidad total de grasa del albondigón debe ser menor o igual al 25\%, y que la cantidad de carne de res más la cantidad de carne de cerdo debe ser igual a 1 libra.

  Sumado a esto, también se tiene que las cantidades de carne de res y carne de cerdo no pueden ser negativas. Entonces, resulta:
  \begin{align*}
    0.2x_1 + 0.32x_2 &\leq 0.25 \\
    x_1 + x_2 &= 1 \\
    x_1, x_2 &\geq 0
  \end{align*}
  \textbf{3. Identificación de la función objetivo:}

  Vemos que la función objetivo es el costo total, que es la suma del costo de la carne de res y la carne de cerdo. Por lo que la función objetivo es:
  \[
    z = 80x_1 + 60x_2
  \]

  \textbf{4. Formulación del PPL:}

  Juntando la función objetivo con las restricciones, resulta en el siguiente PPL:
  \begin{align*}
    \text{minimizar:} \quad   &z = 80x_1 + 60x_2 \\[3pt]
    \text{sujeto a:} \quad    &0.2x_1 + 0.32x_2 \leq 0.25 \\
                              &x_1 + x_2 = 1 \\
                              &x_1, x_2 \geq 0
  \end{align*}
  \textbf{5. Resolución del PPL:}

  Para resolver el PPL usando el método gráfico, debemos graficar las restricciones, la región factible y la función objetivo. 

  \noindent ¿Cómo se realiza la gráfica de la función objetivo? 

  Para realizar la gráfica de la función objetivo debemos establecer un valor para el costo, y verificar si se encuentra en la región factible. Por ejemplo, nosotros vamos a elegir un valor inicial de \(z = 50\), y resolver la ecuación \(50 = 80x_1 + 60x_2\) para obtener la ecuación de la recta que representa la función objetivo.
  \begin{align*}
    50 &= 80x_1 + 60x_2 \\
    x_2 &= -\frac{4}{3}x_1 + \frac{5}{6}
  \end{align*}
  Y listo, ahora solo debemos graficar las restricciones e identificar la región factible, que la marcaremos de color \textcolor{cyan}{cyan}. 
    
  \begin{figure}[ht]
  \centering
  \begin{tikzpicture}
  \begin{axis}[
      xlabel={$x_1$},
      ylabel={$x_2$},
      xmin=0, xmax=1.5,
      ymin=0, ymax=1.2,
      grid=major,
      axis lines=center,
      legend pos=north east,
      width=10cm,
      height=8cm
  ]

  % Restricción 1: 0.2x1 + 0.32x2 <= 0.25
  \addplot[orange, thick, domain=0:1.25, name path=A] {-(4/7)*x + (5/7)};
  \addlegendentry{\(0.2x_1 + 0.32x_2 \leq 0.25\)}

  % Restricción 2: x1 + x2 <= 1
  \addplot[blue, thick, domain=0:1, name path=B] {1-x};
  \addlegendentry{\(x_1 + x_2 = 1\)}

  % Función objetivo (para un valor de z=50): 50 = 80x1 + 60x2
  \addplot[teal, thick, domain=0:1, name path=C] {-(4/3)*x + (5/6)};
  \addlegendentry{\(50 = 80x_1 + 60x_2\)}

  % Restricciones de no negatividad
  \addplot[black, thick, domain=0:10, name path=D] {0};
  \addplot[black, thick] coordinates {(0,0) (0,10)};

  % % Región factible para la restricción 1 sombreada
  % \addplot[orange!30, opacity=0.3] fill between[
  %     of=A and D,
  %     soft clip={domain=0:1.25}
  % ];

  % Región factible
  \addplot[cyan, ultra thick , domain=2/3:1, name path=E] {1-x};

  % Puntos vértices de la región factible
  \addplot[only marks, mark=*, mark size=3pt, color=black] 
  coordinates {(1,0) (2/3,1/3)};

  % Etiquetas de los vértices
  \node at (axis cs:1,0) [above right] {B (1,0)};
  \node at (axis cs:2/3,1/3) [above right] {A \(\left(\frac{2}{3},\frac{1}{3}\right)\)};

  \end{axis}
  \end{tikzpicture}
  \caption{Gráfico del PPL}
  \label{fig:ppl}
  \end{figure}

  \noindent En el gráfico se puede observar que la región factible es el segmento color cyan. En este caso, el valor de costo \(z=50\) no se encuentra dentro de la región factible. Buscar este valor a prueba y error no es lo más conveniente, entonces vamos a usar un teorema que nos asegura que la solución óptima se encuentra en un vértice del \textbf{polígono} de la región factible. Más adelante se verá la demostración de este teorema.

  Entonces, basado en el teorema, la solución óptima se encuentra en el vértice \textit{A} o en el vértice \textit{B}. A simple vista en el gráfico podemos ver que la función objetivo está más cerca del origen de coordenadas en el punto \textit{A} y por lo tanto el costo será menor, sin embargo veremos el costo en ambos puntos para mostrar que el costo aumenta a medida que la función objetivo se aleja del origen de coordenadas.

  \begin{align*}
    z_A &= 80\left(\frac{2}{3}\right) + 60\left(\frac{1}{3}\right) \\
    z_A &= \frac{220}{3} \approx \boxed{73.33} \\[5pt]
    z_B &= 80(1) + 60(0) = 80
  \end{align*}

  \textbf{6. Respuesta:} 

  En base a la resolución del PPL, se debe emplear \({2}/{3}\) de carne de res y \({1}/{3}\) de carne de cerdo por libra de albondigón para mantener el contenido de grasa no mayor al 25\%, y que el costo sea el menor posible, que es \$\(73.33\).

\end{quote}

\hrule
\vspace{5mm}

\paragraph{Resolución Bónus}

\ejemplo\label{ej:resolucion_ppl_petroleos_grafico}: Resolución del PPL del problema de la compañía de petróleos. 
\begin{quote}
  \textbf{1. Formulación del PPL:}

  A continuación se mostrará la resolución del ejercicio visto en el ejemplo \ref{ej:modelo_de_ppl_verbal}. Recordando el PPL ya modelado, se tiene:
  \begin{align*}
    \text{maximizar:} \quad   &z = 103x_1 + 110x_2 \\[3pt]
    \text{sujeto a:} \quad    &7x_1 + 10x_2 \leq 1400 \\
                              &12x_1 + 8x_2 \leq 2000 \\
                              &8x_1 + 10x_2 \geq 900 \\
                              &6x_1 + 7x_2 \geq 300 \\
                              &5x_1 + 4x_2 \geq 800 \\
                              & 5x_1 + 4x_2 \leq 1700 \\
                              &x_1 \geq 0 \\
                              &x_2 \geq 0
  \end{align*}

  \textbf{2. Resolución gráfica:}

  En este caso se tienen muchas más restricciones que en el ejemplo \ref{ej:mtd_grfico}, sin embargo siempre se terminará formando un polígono, donde los vertices indicarán las posibles soluciones óptimas. Para facilitar la resolución gráfica, escribiremos las restricciones en la forma de ecuaciones y reordenaremos los términos de cada ecuación para que queden en la forma de \(y = mx + b\).
  \begin{align*}
    7x_1 + 10x_2 = 1400 \quad &\rightarrow \quad x_2 = -\frac{7}{10}x_1 + 140 \\
    12x_1 + 8x_2 = 2000 \quad &\rightarrow \quad x_2 = -\frac{3}{2}x_1 + 250 \\
    8x_1 + 10x_2 = 900  \quad &\rightarrow \quad x_2 = -\frac{4}{5}x_1 + 90 \\
    6x_1 + 7x_2 = 300  \quad &\rightarrow \quad x_2 = -\frac{6}{7}x_1 + \frac{300}{7} \\
    5x_1 + 4x_2 = 800  \quad &\rightarrow \quad x_2 = -\frac{5}{4}x_1 + 200 \\
    5x_1 + 4x_2 = 1700 \quad &\rightarrow \quad x_2 = -\frac{5}{4}x_1 + 425
  \end{align*}
  Y para las restricciones de no negatividad, sabemos que la región factible se tiene que encontrar el el primer cuadrante.

  Por último, antes de graficar, vamos a suponer algún valor de ganancia de la función objetivo, por ejemplo \(z = 17000\), y resolver la ecuación \(17000 = 103x_1 + 110x_2\) para obtener la ecuación de la recta que representa la función objetivo.
  \[
    17000 = 103x_1 + 110x_2 \quad \rightarrow \quad x_2 = -\frac{103}{110}x_1 + \frac{1700}{11}
  \]

  Entonces, el gráfico (figura \ref{fig:ppl-maximizacion}) de la región factible y la función objetivo con la ganancia \(z = 17000\) es el siguiente:
  \begin{figure}[ht]
    \centering
    \begin{tikzpicture}
    \begin{axis}[
        xlabel={$x_1$},
        ylabel={$x_2$},
        xmin=0, xmax=300,
        ymin=0, ymax=199,
        grid=major,
        legend pos=north east,
        width=13cm,
        height=9.5cm,
        legend style={font=\footnotesize}
    ]
    
    % Restricción 1: 7x₁ + 10x₂ ≤ 1400 → x₂ ≤ 140 - 0.7x₁
    \addplot[blue, thick, dashed, domain=0:200] {140-0.7*x};
    \addlegendentry{$7x_1 + 10x_2 \leq 1400$}
    
    % Restricción 2: 12x₁ + 8x₂ ≤ 2000 → x₂ ≤ 250 - 1.5x₁
    \addplot[red, thick, dashed, domain=0:166.67] {250-1.5*x};
    \addlegendentry{$12x_1 + 8x_2 \leq 2000$}
    
    % Restricción 3: 8x₁ + 10x₂ ≥ 900 → x₂ ≥ 90 - 0.8x₁
    \addplot[green!70!black, thick, dashed, domain=0:112.5] {90-0.8*x};
    \addlegendentry{$8x_1 + 10x_2 \geq 900$}
    
    % Restricción 4: 6x₁ + 7x₂ ≥ 300 → x₂ ≥ 42.86 - 0.857x₁
    \addplot[orange, thick, dashed, domain=0:50] {(300/7)-(6/7)*x};
    \addlegendentry{$6x_1 + 7x_2 \geq 300$}
    
    % Restricción 5: 5x₁ + 4x₂ ≥ 800 → x₂ ≥ 200 - 1.25x₁
    \addplot[purple, thick, dashed, domain=0:160] {200-1.25*x};
    \addlegendentry{$5x_1 + 4x_2 \geq 800$}
    
    % Restricción 6: 5x₁ + 4x₂ ≤ 1700 → x₂ ≤ 425 - 1.25x₁
    \addplot[brown, thick, dashed, domain=0:300] {425-1.25*x};
    \addlegendentry{$5x_1 + 4x_2 \leq 1700$}
    
    % Región factible aproximada (necesitarías calcular los vértices exactos)
    % Esta es una aproximación visual de la región factible
    \fill[cyan!25, opacity=0.5] 
        (1200/11,700/11) -- (275/2,43.75) -- (500/3,0) -- (160,0) -- cycle;

    % Función objetivo
    \addplot[magenta, very thick, domain=0:165] {-(103/110)*x + (1700/11)};
    \addlegendentry{$z = 17000$}
    
    % Algunos puntos de intersección importantes (aproximados)
    \addplot[only marks, mark=*, mark size=3pt, color=black] 
    coordinates {(1200/11,700/11) (275/2,43.75) (500/3,0) (160,0)};
    
    % Etiquetas para algunos puntos clave
    \node at (axis cs:1200/11,700/11) [below left] {\footnotesize A \(\left(\frac{1200}{11},\frac{700}{11}\right)\)};
    \node at (axis cs:275/2,43.75) [above right] {\footnotesize B \(\left(\frac{275}{2},\frac{175}{4}\right)\)};
    \node at (axis cs:160,0) [above right] {\footnotesize C \(\left(160,0\right)\)};
    \node at (axis cs:500/3,0) [above left] {\footnotesize D \(\left(\frac{500}{3},0\right)\)};
    \end{axis}
    \end{tikzpicture}
    \caption{Región factible del PPL de ejemplo \ref{ej:modelo_de_ppl_verbal} en color \textcolor{cyan}{cyan}}
    \label{fig:ppl-maximizacion}
  \end{figure}

  Vemos que la la función objetivo \(z\) se encuentra dentro de la región factible, sin embargo, no es la solución óptima. Se puede ver que el punto más alejado del origen de coordenadas que puede tomar la función objetivo es el vértice \textit{B}. Por lo tanto, la solución óptima es el vértice \textit{B}, que es el punto \(\left(275/2,175/4\right)\). Entonces:
  \[
    z_B = 103\left(\frac{275}{2}\right) + 110\left(\frac{175}{4}\right) = \boxed{18975}
  \]

  \textbf{3. Respuesta:}

  En base a la resolución del PPL, se deben realizar \(137.5\) sesiones de destilación usando la tecnología nueva (\(T_n\)) y \(43.75\) sesiones de destilación usando la tecnología antigua (\(T_a\)), para obtener la mayor ganancia posible, que es \$\(18975\).

  \begin{tcolorbox}[remember]
    Aunque la respuesta pueda resultar confusa, recordemos que el enunciado decía que los procesos de destilación \textit{se pueden utilizar total o parcialmente}, por lo que la respuesta toma valores fraccionarios. Si se quisiera que la respuesta fuera un número entero, entonces se agregan las respectivas restricciones y se resuelve el PPL de la misma manera, tomando únicamente los puntos enteros de \(x_1\) y \(x_2\).
  \end{tcolorbox}

\end{quote}

\subsubsection{Introducción al método Simplex}

Como anteriormente mencionamos, el método simplex es un algoritmo que nos permite resolver problemas de optimización lineal. Aunque disponemos del método gráfico, cuando el número de \textit{variables de decisión} es grande (\(> 2\)), el método gráfico es impracticable, por lo que utilizar el método analítico se vuelve una necesidad.

Además, en general, la mayoría de los problemas de optimización suelen tener más de dos variables de decisión, por lo que la necesidad de generalizar la búsqueda de la solución optima es un tema de interés.

\paragraph{¿Qué hace el método Simplex?}

El método Simplex es un algoritmo iterativo que parte de una solución básica factible (es decir, un vértice del conjunto factible del problema) y se \hl{desplaza de un vértice a otro adyacente}, mejorando en cada paso el valor de la función objetivo, hasta que no se puede mejorar más, lo que significa que se ha llegado a la \textbf{solución óptima}.

Visualmente, imagina un poliedro (la región factible), donde cada vértice es una posible solución básica. El Simplex camina por los bordes del poliedro en dirección ascendente (en problemas de maximización) hasta llegar al punto más alto.

\paragraph{Estructura general del método}

El método se basa en representar el problema en \textbf{forma tabular}. Cada paso del algoritmo consiste en:
\begin{enumerate}
  \item \textbf{Iniciar en una solución básica factible inicial}: esta primera solución se obtiene en la \textit{Fase I} del Simplex. 
  \item \textbf{Seleccionar una variable entrante}: mediante la \textit{regla del criterio} se selecciona aquella que más mejora la función objetivo
  \item \textbf{Seleccionar una variable saliente}: para mantener la factibilidad (cumplir restricciones)
  \item \textbf{Actualizar la tabla (pivoteo)}: para generar una nueva solución básica
  \item \textbf{Repetir el proceso}: hasta que no haya más mejoras posibles
\end{enumerate} 
No se preocupe por entender el concepto de \textit{Fase I} y la \textit{regla del criterio}, ya que se explicarán en detalle más adelante.

\begin{tcolorbox}[interesting_data, title=Forma tabular]
  Que el método Simplex se base en representar el problema en ``forma tabular'' significa que todas las ecuaciones (la función objetivo y las restricciones) del problema de Programación Lineal se organizan en una matriz estructurada, una tabla, llamada ``tableau Simplex''
\end{tcolorbox}

Ahora, vamos a ver algunos conceptos necesarios para comprender la formulación del método, pero el método en sí se verá en el la sección \ref{sec:simplex}, ya que es bastante largo.

\vspace{3mm}

\paragraph{Forma matricial de un PPL}

Un PPL puede ser representado en forma matricial de la siguiente manera:
\begin{align*}
  \text{optimizar:} \quad   &z = C^TX \\[3pt]
  \text{sujeto a:} \quad    &AX \thicksim  B \\
\end{align*}
donde:
\begin{itemize}
  \item \(C\) es la matriz (o el vector) de coeficientes de la función objetivo, generalmente se le llama vector de costos, ya que cada coeficiente representa el costo de una variable de decisión. En la formula se denota como \(C^T\) para indicar que es una matriz transpuesta,
  \item \(X\) es la matriz (o el vector) de variables de decisión. Este vector debe incluir todas las variables del PPL, tanto las variables de decisión originales como las agregadas (como variables de holgura, superfluas, etc que se verán más adelante),
  \item \(A\) es la matriz de coeficientes de las restricciones,
  \item \(B\) es la matriz (o el vector) de términos independientes de las restricciones.
\end{itemize}
\vspace{5pt}
\noindent De forma explícita se puede ver como:
\begin{align*}
  \text{optimizar:} \quad   &z = c_{1}x_{1} + c_{2}x_{2} + \cdots + c_{n}x_{n} \\[3pt]
  \text{sujeto a:} \quad    &a_{11}x_{1} + a_{12}x_{2} + \cdots + a_{1n}x_{n} \thicksim b_{1} \\[3pt]
                            &a_{21}x_{1} + a_{22}x_{2} + \cdots + a_{2n}x_{n} \thicksim b_{2} \\[3pt]
                            &\quad \vdots \\[3pt]
                            &a_{m1}x_{1} + a_{m2}x_{2} + \cdots + a_{mn}x_{n} \thicksim b_{m}
\end{align*}
donde:
\[
  C = \begin{pmatrix} c_1 \\ c_2 \\ \vdots \\ c_n \end{pmatrix},\ X = \begin{pmatrix} x_1 \\ x_2 \\ \vdots \\ x_n \end{pmatrix},\ A = \begin{pmatrix} a_{11} & a_{12} & \cdots & a_{1n} \\ a_{21} & a_{22} & \cdots & a_{2n} \\ \vdots & \vdots & \ddots & \vdots \\ a_{m1} & a_{m2} & \cdots & a_{mn} \end{pmatrix},\ B = \begin{pmatrix} b_1 \\ b_2 \\ \vdots \\ b_m \end{pmatrix}
\]

Esta forma es conveniente ya que permite representar el PPL de una manera más compacta y permite referirnos a una parte específica del problema, como la matriz de los coeficientes (\(A\)) para referirnos a las restricciones, el vector de costos (\(C\)) para referirnos a los coeficientes de la función objetivo y el vector de términos independientes (\(B\)) para referirnos a los términos independientes de las restricciones.

Con estos conceptos estamos listos para ver la forma de resolución de PPLs mediante el método Simplex.

\subsection{El método Simplex}
\label{sec:simplex}

El método simplex se mueve de un vértice del poliedro factible a un vértice adyacente que mejora el valor de la función objetivo. Repasando lo que vimos anteriormente cada iteración corresponde a:
\begin{itemize}
  \item \textbf{Evaluación del vértice actual}: Verificar si es óptimo mediante los costos reducidos
  \item \textbf{Selección de arista}: Elegir la arista (dirección) que más mejora la función objetivo
  \item \textbf{Movimiento}: Desplazarse a lo largo de la arista hasta el siguiente vértice
  \item \textbf{Actualización de la tabla}: El nuevo vértice se convierte en la nueva solución básica
\end{itemize}
Más adelante veremos un ejemplo simple para comprender mejor el método Simplex, pero antes necesitamos ver qué necesitamos para usar el método Simplex.

\paragraph{Condiciones para usar el método Simplex}

Entender con claridad \textit{cuáles son las condiciones necesarias para aplicar el método Simplex} es fundamental, ya que no todo problema lineal puede resolverse directamente con él sin alguna transformación previa. De forma resumida, para aplicar Simplex se necesita:
\begin{enumerate}
  \item Convertir el problema a forma estándar,
  \item Tener una solución básica factible inicial,
  \item Asegurar que el problema tiene solución factible y acotada.
\end{enumerate}
En general, para poder aplicar el método simplex, se expresa el PPL en su forma matricial, ya que permite realizar las transformaciones y operaciones de forma ordenada.

A continuación veremos cada una de las condiciones en detalle.

\subsubsection{Forma estándar del problema}
\label{sec:forma_estandar}

Un PPL está en su forma estándar si cumple las siguientes condiciones:
\begin{itemize}
  \item La función objetivo debe ser de maximización,
  \item Todas las variables deben ser mayores o iguales a cero (restricción de no negatividad).
  \item Todas las restricciones deben estar escritas como igualdades (no desigualdades).
\end{itemize}
Ya puede intuir, por los ejemplos vistos, que no todos los PPLs cumplen con estas condiciones, por lo que se requiere transformar el PPL a su forma estándar.

\paragraph{Transformación de un PPL en su forma estándar}

Como vimos anteriormente, un PPL está en su forma estándar si cumple las condiciones vistas. Para trabajar de forma ordenada vamos a llamar a cada condición de la siguiente manera:
\begin{enumerate}
  \item \textbf{Condición de maximización}: la función objetivo debe ser de maximización,
  \item \textbf{Condición de no negatividad}: todas las variables de decisión deben ser positivas: \(\forall x_i,\ x_i \geq 0;\ i = 1,2,\ldots,n\)
  \item \textbf{Condición no desigualdad}: todas las restricciones deben ser de la forma \(A_iX = b_i\). Para lograr esto se introducen \hl{variables de holgura} y \hl{variables superfluas}.
\end{enumerate}
En un momento se explicará el concepto de variables de holgura y variables superfluas.

Para tener un contexto previo, ver el siguiente video: \href{https://www.youtube.com/watch?v=6f5K3O7yUzU}{\texttt{forma estándar en programación lineal - YouTube}}, donde se muestran algunos ejemplos de cómo transformar un PPL a forma estándar. De igual manera, en este documento se mostrará el proceso de transformación a formato estándar de un PPL. 

\ejemplo\label{ej:transformacion_ppl_forma_estandar}: Transformación de un PPL a forma estándar.
\begin{quote}
  Consideremos el siguiente PPL (ejemplo del video recomendado):
  \begin{align*}
    \text{maximizar:} \quad   &z = -8x_1 + 16x_2 - 4x_3 \\[3pt]
    \text{sujeto a:} \quad    &5x_1 + 8x_3 \geq 100 \\
                              &-x_1 + 6x_2 \geq 100 \\
                              &-6x_2 + 4x_3 \leq 100 \\
                              &x_1 \geq 0 \\
                              &x_2 \text{ Libre} \\
                              &x_3 \leq0
  \end{align*}

  \noindent\textbf{1. Condición de maximización:}

  El PPL ya cumple con la condición de maximización, por lo que no se requiere ninguna transformación. Para aquellos PPL que buscan \textit{minimizar}, que no cumplen esta condición, al multiplicar la función objetivo por \(-1\) se puede transformar en una maximización. 

  \noindent\textbf{2. Condición de no negatividad:}

  En este caso, tenemos las tres posibilidades en las variables de decisión:
  \begin{itemize}
    \item Libre: \(x_2\)
    \item Positiva: \(x_1\)
    \item Negativa: \(x_3\)
  \end{itemize}

  La única variable que cumple con la condición de no negatividad es \(x_1\), por lo tanto esta variable se mantiene como está, es decir, \(x_1 \geq 0\). 

  Las variables que no cumplen con esta condición son \(x_2\) y \(x_3\), por lo que debemos operar sobre ellas para hacer cumplir la condición. Veamos el caso de \(x_3\) primero: 

  La variable \(x_3\) es negativa, para transformar la restricción a positiva debemos realizar un cambio de variable, tal que:
  \[
    x_3 = -y_3 \quad \rightarrow \quad y_3 \geq 0
  \]

  \noindent donde \(y_3\) es una variable que reemplaza a \(x_3\) y que es positiva. Esta variable debe cambiarse en todo el PPL, es decir, en la función objetivo y en las restricciones. 

  Para el caso de \(x_2\), la variable puede tomar cualquier valor, por lo que para que estrictamente cumpla la condición de no negatividad podemos realizar lo siguiente:
  \[
    x_2 = y_2 - y_2' \quad \rightarrow \quad y_2,\ y_2' \geq 0
  \]

  \noindent donde \(y_2\) y \(y_2'\) son variables que reemplazan a \(x_2\) y que cumplen con la condición de no negatividad. Estas variables deben reemplazar a \(x_2\) en todo el PPL, al igual que \(x_3\) se reemplaza por \(y_3\). Reemplazando las variables en el PPL, obtenemos:
  \begin{align*}
    \text{maximizar:} \quad   &z = -8x_1 + 16y_2 - 16y_2' + 4y_3 \\[3pt]
    \text{sujeto a:} \quad    &5x_1 - 8y_3 \geq 100 \\
                              &-x_1 + 6y_2 - 6y_2' \geq 100 \\
                              &-6y_2 + 6y_2' - 4y_3 \leq 100 \\
                              &x_1,\ y_2,\ y_2',\ y_3 \geq 0
  \end{align*}
  Listo, la condición de no negatividad se cumple, ahora veamos la tercera condición.

  \noindent\textbf{3. Condición de no desigualdad:}

  La tercera condición es que todas las restricciones deben ser de la forma \(Ax = b\). En este caso, ninguna de las restricciones cumple con esta condición, por lo que debemos operar sobre todas ellas. 

  Cuando analizamos la tercera condición, podemos encontrarnos con tres casos posibles:
  \begin{enumerate}
    \item Restricción de la forma \(A_i x \leq b_i\): agregar una \hl{\textit{variable de holgura}}
    \item Restricción de la forma \(A_i x \geq b_i\): agregar una \hl{\textit{variable superflua}}
    \item Restricción de la forma \(A_i x = b_i\): cumple con la condición de forma estándar, por lo que no se necesita ninguna variable adicional
  \end{enumerate}
  \noindent donde \(A_i\) es la i-ésima restricción y \(b_i\) es su término independiente.
  \begin{tcolorbox}[interesting_data, title=¿Qué significa variable de holgura?]
    La \textbf{variable de holgura} es aquella que se agrega a una restricción de la forma \(A_i x \leq b_i\) para transformarla en una igualdad, y representa la cantidad de recursos de los que se disponían pero no se usaron. 

    \textit{Ejemplo}: si se disponen de 10 horas totales para fabricar un producto, pero solo se usan 8 horas, entonces la variable de holgura representa las \(2\) horas que se disponían pero no se usaron. 
  \end{tcolorbox}

  \begin{tcolorbox}[interesting_data, title=¿Qué significa variable superflua?]
    Por otro lado, la \textbf{variable superflua} es aquella que se agrega a una restricción de la forma \(A_i x \geq b_i\) para transformarla en una igualdad, y representa la cantidad en que el lado izquierdo de la restricción \textbf{excede} el requisito mínimo establecido por el lado derecho.

    \textit{Ejemplo}: si se requieren como mínimo 54 toneladas de mineral, pero se extraen 60 toneladas, entonces la variable superflua vale \(60 - 54 = 6\) toneladas, y representa el exceso de toneladas de mineral que se extrae. 
  \end{tcolorbox}

  \noindent Volviendo al problema original, las restricciones que no cumplen la tercer condición son:
  \begin{align*}
    5x_1 - 8y_3 &\geq 100 \\
    -x_1 + 6y_2 - 6y_2' &\geq 100 \\
    -6y_2 + 6y_2' - 4y_3 &\leq 100
  \end{align*}
  Para la tercera restricción, tenemos el caso 1. Para este caso debemos agregar una \textit{variable de holgura}:
  \[
    -6y_2 + 6y_2' - 4y_3 \leq 100 \quad \rightarrow \quad -6y_2 + 6y_2' - 4y_3 + h_1 = 100
  \]
  Luego, para la primera y la segunda restricción tenemos el caso 2. Para este caso debemos agregar una \textit{variable superflua}:
  \begin{align*}
    5x_1 - 8y_3 \geq 100 \quad &\rightarrow \quad 5x_1 - 8y_3 - s_1 = 100 \\
    -x_1 + 6y_2 - 6y_2' \geq 100 \quad &\rightarrow \quad -x_1 + 6y_2 - 6y_2' - s_2 = 100
  \end{align*}

  Estas variables que se han agregado a las restricciones deben aparecer en la función objetivo, sin embargo, como no son variables de decisión, su coeficiente en la función objetivo debe ser cero. Por lo que el PPL transformado a forma estándar es:
  \begin{align*}
    \text{maximizar:} \quad   &z = -8x_1 + 16y_2 - 16y_2' + 4y_3 - 0s_1 - 0s_2 + 0h_1 \\[3pt]
    \text{sujeto a:} \quad    &5x_1 + 0y_2 - 0y_2' - 8y_3 - s_1 - 0s_2 + 0h_1 = 100 \\
                              &-x_1 + 6y_2 - 6y_2' + 0y_3 - 0s_1 - s_2 + 0h_1 = 100 \\
                              &0x_1 -6y_2 + 6y_2' - 4y_3 - 0s_1 - 0s_2 + h_1 = 100 \\
                              &x_1,\ y_2,\ y_2',\ y_3,\ h_1,\ s_1,\ s_2 \geq 0
  \end{align*}
  Si quitamos todos los términos con coeficiente nulo de las restricciones queda:
  \begin{align*}
    \text{maximizar:} \quad   &z = -8x_1 + 16y_2 - 16y_2' + 4y_3 - 0s_1 - 0s_2 + 0h_1 \\[3pt]
    \text{sujeto a:} \quad    &5x_1 - 8y_3 - s_1 = 100 \\
                              &-x_1 + 6y_2 - 6y_2' - s_2 = 100 \\
                              &-6y_2 + 6y_2' - 4y_3 + h_1 = 100 \\
                              &x_1,\ y_2,\ y_2',\ y_3,\ h_1,\ s_1,\ s_2 \geq 0
  \end{align*}
\end{quote}

\subsubsection{Existencia de una solución básica factible inicial (SBF)}

Este es el ``\textit{punto más delicado}''. Para comenzar el método Simplex, se necesita un punto inicial que cumpla todas las restricciones (factible) y sea una solución básica (es decir, con tantas variables básicas como ecuaciones, y el resto en cero).

Como ya vimos, si las restricciones son de tipo \(\leq\) y se agregan variables de \textit{holgura positiva}, entonces la \textbf{SBF} inicial está dada simplemente por poner en cero las variables originales y tomar las variables de holgura como solución. Por otro lado, si las restricciones son de tipo \(\geq\) o \(=\), o si el sistema \textbf{no} tiene una SBF evidente, entonces se debe recurrir a un método auxiliar como el método de la fase I del Simplex o el método de las dos fases.

\paragraph{¿Qué es una solución básica?}

En términos del álgebra lineal, una \textit{solución básica} es una solución del sistema lineal:
\[
AX = B
\]
Este sistema, representado de forma matricial, no es más que las restricciones del PPL en forma estándar. Si desea repasar el tema de sistemas de ecuaciones lineales puede consultar el capítulo \ref{sec:sel}.

\paragraph{¿Cómo se define una \textit{solución básica}?}

Para construir una \textit{solución básica}, se hace lo siguiente:
\begin{enumerate}
  \item Se eligen arbitrariamente \(m\) variables del vector \(X\), a las que se llama \textbf{variables básicas}.
  \item El resto de las \(n - m\) variables se fijan en cero; estas se llaman \textbf{variables no básicas}.
  \item Luego se resuelve el sistema lineal con esas \(m\) incógnitas (las básicas), usando las \(m\) ecuaciones.
\end{enumerate}
Esto es posible si las columnas de \(A\) correspondientes a las variables básicas son \textbf{linealmente independientes}, lo que permite resolver el sistema.

\paragraph{¿Y qué es una \textit{solución básica factible} (SBF)?}

Una solución básica factible es una solución básica que además satisface:
\[
x \geq 0
\]
Es decir, todas las variables (básicas y no básicas) deben ser \textbf{mayores o iguales a cero}. Esta condición es necesaria porque el método Simplex trabaja solamente en la región factible del espacio, que está limitada por las restricciones del problema.

\paragraph{Intuición geométrica}

En geometría, la región factible de un problema lineal es un \textit{poliedro} (en 2D, un polígono; en 3D, un poliedro tridimensional). Cada \textit{vértice} (o esquina) de esa región corresponde a una \textbf{solución básica factible}.

El método Simplex camina de un vértice al siguiente, buscando mejorar el valor de la función objetivo, hasta que ya no puede mejorar más.

Por eso se necesita comenzar desde un vértice: es decir, desde una \textbf{SBF}.

\ejemplo\label{ej:busqueda_sbf}{: Búsqueda de SBF}

Supón que tienes este sistema (ya convertido a forma estándar):
\begin{align*}
  x_1 + x_2 + x_3 &= 4 \\
  2x_1 + 3x_2 + x_4 &= 7 \\
  x_1, x_2, x_3, x_4 &\geq 0
 \end{align*}

Aquí hay 4 variables y 2 ecuaciones. Una solución básica se obtiene eligiendo, por ejemplo, las variables \(x_1\) y \(x_3\) como básicas, y fijando \(x_2 = x_4 = 0\).

Al resolver el sistema para \(x_1\) y \(x_3\), obtenemos \(x_1 = 3.5;\, x_3 = 0.5\). Como todas las variables son no negativas, entonces \(x_1 = 3.5;\, x_3 = 0.5\) es una \hl{solución básica factible}.

Ahora veamos qué pasa si elegimos las variables \(x_1\) y \(x_2\) como básicas, y fijamos \(x_3 = x_4 = 0\). Al resolver el sistema para \(x_1\) y \(x_2\) el resultado da \(x_1 = 5;\, x_2 = -1\). Como \(x_2\) es negativo, entonces \textbf{no es} una solución básica factible, simplemente es una solución básica al sistema.
\vspace{5mm}
\hrule

\subsubsection{Condición de factibilidad y acotación}

Esta es la última condición que debe cumplir el problema para que el Simplex sea aplicable.

El método Simplex solo es aplicable si el problema tiene una solución factible. Si no la tiene, el Simplex lo detectará (normalmente en la fase I).

El problema también debe ser acotado, es decir, si la función objetivo puede crecer indefinidamente sin violar las restricciones, el Simplex lo reportará como no acotado.

En términos simples, con el método simplex buscamos una solución factible \textbf{única}.

\subsection{Inicialización del método Simplex}

\subsubsection{Generación de solución factible inicial}

Como vimos, Simplex también requiere de una solución básica factible (SBF) inicial ya que el algoritmo se mueve de una solución factible básica a otra, mejorando el valor de la función objetivo en cada iteración. Anteriormente, en el ejemplo \ref{ej:busqueda_sbf} vimos, tras elegir arbitrariamente las variables básicas, que dimos con una SBF y luego, tras elegir nuevamente otras variables básicas, dimos con una solución básica pero no factible. Ese problema tenía pocas variables de decisión, por lo que si queremos encontrar una SBF podemos ir probando hasta dar con alguna. Sin embargo, si el problema tiene muchas variables de decisión, es difícil encontrar una SBF inicial, y probar todas las posibilidades no es lo más eficiente. Por lo que existen dos métodos para generar una SBF inicial:
\begin{itemize}
  \item \hl{Inspección directa}: si es fácil identificar un punto que satisface todas las restricciones, se puede usar directamente.
  \item \hl{Método de la Fase I del Simplex}: para aquellos casos donde no es fácil aplicar la inspección directa, se utiliza este método. Este método consiste en cuatro pasos:
  \begin{enumerate}
    \item Se construye un problema auxiliar donde se agregan \textit{variables artificiales} para convertir el problema original en uno con una solución factible evidente
    \item Se minimiza la suma de esas variables artificiales
    \item Si en el óptimo esta suma es cero, se ha encontrado una solución factible al problema original
    \item Si no es cero, el problema original no tiene solución factible (es incompatible)
  \end{enumerate}
\end{itemize}

\noindent El método de la inspección directa se vió en el ejemplo \ref{ej:busqueda_sbf}. Básicamente consiste en analizar las restricciones y verificar si se puede generar una solución básica que sea factible. Ahora veamos el método de la Fase I.

\subsubsection{Método de la Fase I: construcción del problema auxiliar}

Esta fase tiene como objetivo encontrar una solución básica factible inicial para un problema de programación lineal (PPL) cuando no está disponible de forma directa, aunque también puede usarse para aquellos problemas que tienen una SBF evidente.

Recordemos que el método Simplex parte de una SBF y se mueve de una a otra mejorando el valor de la función objetivo. Por eso, si no contamos con una SBF al comienzo, necesitamos construir un problema auxiliar que nos la proporcione. Esto es lo que hace la Fase I.

El objetivo de la Fase I es formular y resolver un problema auxiliar que:
\begin{itemize}
  \item Sea fácil de resolver
  \item Tenga una SBF evidente
  \item Su solución, si es factible, nos permita obtener una SBF \textbf{del problema original}
\end{itemize}

Este método consiste en cuatro pasos generales:
\begin{enumerate}

  \item \textbf{Convertir el PPL original a forma estándar}: Esto incluye que todas las restricciones sean igualdades (introduciendo variables de holgura, exceso, etc.) y todas las variables estén acotadas inferiormente por cero.

  \item \textbf{Identificar las restricciones problemáticas}: Por ejemplo, si una ecuación tiene una constante en el lado derecho negativa, o si las variables artificiales son necesarias para armar una base inicial.

  \item \textbf{Agregar variables artificiales}: Se utilizan cuando las restricciones del problema no permiten obtener una solución básica factible inicial (SBF inicial) de forma directa, es decir, cuando no se puede formar una matriz identidad a partir de las variables de holgura. Esto se hace cuando:
    \begin{itemize}
      \item En el caso \(\geq\), se agregan \textit{variables superfluas}, que restan una variable: \(-s_i\), lo cual da lugar a columnas con \(-1\) en vez de \(1\), y por tanto no son vectores \(e_i\).
      \item En el caso \(=\), al no haber ni suma ni resta de variables de holgura, no se genera ninguna columna tipo \(e_i\) automáticamente.
    \end{itemize}
    donde las columnas \(e_i\) son columnas que tienen un elemento 1 y el resto cero y permiten formar una matriz identidad.
  \begin{tcolorbox}[myconclusion]
    Las variables artificiales son variables auxiliares o ficticias que se introducen en el PPL para poder iniciar el algoritmo Simplex. No tienen un significado en el problema original; son puramente un ``truco matemático''.
  \end{tcolorbox}

  \item \textbf{Formular un problema auxiliar (problema de la Fase I)}: Se define una nueva función objetivo auxiliar: Minimizar la suma de todas las variables artificiales. Esta función objetivo representa el “costo” de alejarse del espacio factible del problema original.

  \item \textbf{Resolver el problema auxiliar con el método Simplex}: 
  \begin{itemize}
    \item Si la solución óptima del problema auxiliar tiene valor cero, significa que se ha encontrado una SBF para el problema original.
    \item Si la solución óptima tiene valor distinto de cero, el problema original no es factible (no existe ninguna solución que satisfaga todas las restricciones).
  \end{itemize}

  \item \textbf{Eliminar las variables artificiales}: Si la Fase I fue exitosa, se eliminan las variables artificiales (si aún están presentes en la base se deben reemplazar mediante pivoteos), y se continúa con la \textit{Fase II}, ahora con una SBF válida y la función objetivo original.
\end{enumerate}

El problema auxiliar busca un punto factible \hl{minimizando} el ``uso'' de las variables artificiales. Si puede lograr que todas ellas sean cero, significa que existe una combinación de las variables reales (originales y de holgura) que satisfacen todas las restricciones. Esa es justamente una solución básica factible. Veamos un ejemplo de la Fase I.

\begin{quote}
  \ejemplo\label{ej:fase_1} Sea el siguiente problema de programación lineal:
  \begin{align*}
    \text{maximizar} \quad  &z = 3x_1 + 2x_2\\[3pt]
    \text{sujeto a:} \quad  &x_1 + x_2 = 4\\
                            &x_1 - x_2 \geq 2\\
                            &x_1, x_2 \geq 0
  \end{align*}
  Observamos que:
  \begin{itemize}
    \item La primera restricción ya está en forma de igualdad, pero no tiene una variable de holgura asociada que permita incluir una columna identidad. Por lo tanto debemos agregar una variable artificial.
    \item La segunda es una desigualdad \(\geq\), por lo tanto debemos restar una variable superflua, por lo tanto, también debemos agregar una variable artificial.
  \end{itemize}
  
  \subparagraph{Paso 1: Convertimos a forma estándar y agregamos variables artificiales}
  \begin{enumerate}
    \item A la primera ecuación (que ya es igualdad), agregamos una variable artificial \(a_1\), porque no hay variable de holgura ni exceso que permita formar la base inicial.
    \item A la segunda desigualdad, restamos una variable de exceso \(s_2\), y agregamos una variable artificial \(a_2\).
  \end{enumerate}
  El sistema queda:
  \begin{align*}
    x_1 + x_2 + a_1 &= 4\\
    x_1 - x_2 - s_2 + a_2 &= 2
  \end{align*}  
  Con condiciones:
  \[
    x_1, x_2, s_2, a_1, a_2 \geq 0
  \]

  \subparagraph{Paso 2: Definimos el problema auxiliar}
  
  La función objetivo auxiliar es:
  \[
    \omega = a_1 + a_2
  \]
  Nuestro nuevo problema (auxiliar) es:
  \begin{align*}
    \text{minimizar} \quad  &\omega = a_1 + a_2\\[3pt]
    \text{sujeto a:} \quad  &x_1 + x_2 + a_1 = 4\\
                            &x_1 - x_2 - s_2 + a_2 = 2\\
                            &x_1, x_2, s_2, a_1, a_2 \geq 0
  \end{align*}

  Para las restricciones, la base inicial está formada por las variables artificiales \(a_1\) y \(a_2\), porque aparecen con coeficiente 1 y sólo en una ecuación (forman una matriz identidad).
  \begin{align*}
    \begin{pmatrix}
      1 & 1 & 0 & 1 & 0\\
      1 & -1 & -1 & 0 & 1
    \end{pmatrix}
    \begin{pmatrix}
      x_1\\
      x_2\\
      s_2\\
      a_1\\
      a_2
    \end{pmatrix}
    =
    \begin{pmatrix}
      4\\
      2
    \end{pmatrix}
  \end{align*}
  Esto significa que, si establecemos como variables básicas a \(a_1\) y \(a_2\), y las demás en cero, obtenemos una solución básica factible para el problema auxiliar, ya que \(a_1 = 4\) y \(a_2 = 2\).

  De este modo, ya tenemos una \hl{SBF inicial} para el problema auxiliar.
  
  \subparagraph{Paso 3: Aplicamos el método Simplex}
  
  A partir de esta situación, podríamos construir la tabla Simplex con \(a_1\) y \(a_2\) en la base, y comenzar a iterar para minimizar \(\omega\). Si al finalizar el valor óptimo de \(\omega\) es cero, habremos encontrado una combinación de \(x_1\), \(x_2\) y \(s_2\) que satisface las restricciones sin necesidad de variables artificiales, es decir, una solución básica factible del problema original.
  
  Si en cambio \(\omega > 0\), entonces el problema original no tiene solución factible.  
\end{quote}

\begin{tcolorbox}[interesting_data, title=Siempre recordar lo que se busca en la Fase I]
  \begin{itemize}
    \item \textbf{Objetivo}: Minimizar \(\omega = \sum(\text{artificiales})\)
    \item \textbf{Costos}: Coeficientes de artificiales = 1, resto = 0
    \item \textbf{Costos reducidos}: Miden impacto en suma de artificiales
    \item \textbf{Criterio}: Costo reducido más negativo entra
    \item \textbf{Meta}: Llegar a \(\omega = 0\)
  \end{itemize}
\end{tcolorbox}
Ahora profundizaremos más en lo que dice este cuadro. 

\subsubsection{Construcción de la primera tabla del método Simplex (problema auxiliar)}

Una vez que tienes el problema auxiliar planteado, el siguiente paso es construir la primera tabla del método Simplex para el problema auxiliar, resolverlo con iteraciones hasta obtener una SBFI, y a partir de ahí: si \(\omega = 0\), comenzar el Simplex para el problema original con esa base.

Entonces, siguiendo con el ejemplo \ref{ej:fase_1}, vamos a armar la primer tabla Simplex. Para armar la primer tabla Simplex debemos tener en cuenta algunos detalles, como la construcción del vector de costos y la selección de la base. Aunque en el ejemplo \ref{ej:fase_1} más o menos vimos estos conceptos, ahora los veremos más en detalle.

\paragraph{Contrucción del vector de costos \(c\) para la Fase I}

Recordando que la función objetivo del problema auxiliar es \(\omega = a_1 + a_2\), por lo tanto el \textbf{vector de costos} asociado a todas las variables \textbf{del PPL auxiliar} (en el orden \(x_1, x_2, s_2, a_1, a_2\)) es: 
\[
  \hat{c} = (0, 0, 0, 1, 1)
\]
Ya que \(x_1\), \(x_2\) y \(s_2\) no participan en la función objetivo del problema auxiliar, su coeficiente es cero.

Un detalle muy importante es que el problema auxiliar es de \textit{minimización}, por lo que no está en formato estándar. Por ello debemos multiplicar a la función objetivo por (\(-1\)), resultando:
\[
  \text{max} \quad -\omega = - a_1 - a_2 \quad \rightarrow \quad \hat{c}=(0,0,0,-1,-1)
\]

\paragraph{Elección de la base inicial}

Para poder aplicar el método Simplex, se necesita una \textit{base inicial} formada por un conjunto de variables básicas tales que el sistema:
\[
A_{\beta}X_{\beta} = B
\]
tenga una solución factible (es decir, \(X_\beta \geq 0\)).
\begin{tcolorbox}[remember, title=Aclaración]
  Cuando se usa \(A_\beta\) o \(X_\beta\) se refiere al vector de \textbf{variables básicas} y sus respectivos coeficientes. Eso no tiene nada que ver con la matriz \(B\).
\end{tcolorbox}

En el problema auxiliar hemos introducido \(a_1\) y \(a_2\) de modo que:
\begin{itemize}
  \item Cada una aparece \textbf{una sola vez} en una ecuación,
  \item Con coeficiente 1,
  \item Y no aparece en las demás ecuaciones.
\end{itemize}
Este tipo de estructura es ideal para que las variables artificiales sirvan como base inicial. Por lo tanto, tomamos como base inicial:
\[
  \beta = \{a_1, a_2\}
\]
Esto garantiza que la matriz base \(A_\beta\) es la \textit{matriz identidad} \(I_2\), y por lo tanto, \(X_\beta = B\) tiene solución inmediata. En forma matricial es:
\begin{align*}
  A_\beta =
  \begin{pmatrix}
    1 & 0\\
    0 & 1
  \end{pmatrix}
  \quad
  X_\beta =
  \begin{pmatrix}
    a_1\\
    a_2
  \end{pmatrix}
  \quad
  B =
  \begin{pmatrix}
    4\\
    2
  \end{pmatrix}
\end{align*}
Entonces:
\begin{align*}
  A_\beta X_\beta = B \quad \rightarrow \quad {a_1} = 4, ~~ {a_2} = 2
\end{align*}
Entonces ya tenemos una \textbf{SBF} inicial (como vimos anteriormente).

\paragraph{Introducción de costos reducidos}

El costo reducido \(\bar{c}_j\) se define como:
\[
  \bar{c}_j = c_j - c_\beta^T A_\beta^{-1} A_j
\]
donde:
\begin{itemize}
  \item \(\bar{c}_j\) es el vector de costos reducidos correspondiente a la variable \(x_j\) que está fuera de la base (candidata a entrar),
  \item \(c_j\) es el valor \textit{j-ésimo} del vector de costos de la función objetivo,
  \item \(c_\beta^T\) es el vector de costos de la base (que es la base actual),
  \item \(A_\beta^{-1}\) es la matriz inversa de la matriz de los coeficientes de las restricciones que forman la base actual,
  \item \(A_j\) es la columna del sistema que corresponde a la variable \(x_j\) que está fuera de la base (candidata a entrar).
\end{itemize}

\begin{tcolorbox}[mydanger]
  Atención en esta parte, suele ser un poco complicada de entender al principio. Recomiendo ir tomando nota de los distintos costos.
\end{tcolorbox}

Antes de ver en qué consiste el cálculo de costos reducidos y para qué sirve, vamos a ver en detalle qué vectores se utilizan en la formula y cómo vamos a llamarlos.

\vspace{5mm}

\subparagraph{1. Vector de costos \textit{c}}

Anteriormente vimos que el vector \textit{c} es el vector de coeficientes de la función objetivo del problema original. Si el problema original es:
\[
  \max z = c_1 x_1 + c_2 x_2 + \cdots + c_n x_n
\]
entonces el vector \(c\) es:
\[
  c = \begin{bmatrix} c_1 & c_2 & \cdots & c_n \end{bmatrix}
\]
Este vector se utiliza una vez que el problema auxiliar ha sido resuelto, y se quiere retomar el método Simplex para optimizar el problema original.

\vspace{5mm}

\subparagraph{2. Vector de costos \(c_\omega\) (función auxiliar)}

Este es el vector de coeficientes de la \hl{función auxiliar} \(\omega = a_1 + a_2 + \cdots + a_k\) que se introduce para hallar una solución básica factible inicial.

En este caso, el vector \(c_\omega\) es algo así como:
\[
  c_\omega = \begin{bmatrix} 0 & 0 & \cdots & 1 & 1 & \cdots \end{bmatrix}
\]
donde los ceros corresponden a las variables originales y de holgura, y los unos a las variables artificiales.

Este vector solo se usa mientras resolvemos el problema auxiliar (es decir, en la Fase I). Los costos reducidos que se calculan durante la Fase I (con \(c_\omega\)) determinan el pivoteo para minimizar \(\omega\).

\vspace{5mm}

\subparagraph{3. Vector de costos de la base \(c_\beta\)}

Este vector contiene los \textit{costos asociados a las variables que actualmente están en la base}.

Para ilustrar mejor este vector, supongamos el mismo ejemplo que vimos anteriormente (ejemplo \ref{ej:fase_1}).

\begin{table}[htbp]
  \centering
  \renewcommand\cellalign{tl}
  \renewcommand\cellgape{\Gape[4pt]}
  \begin{tabular}{c|c}
  \textit{PPL Original} & \textit{PPL Auxiliar} \\
  \hline
  \makecell[l]{
    \(\text{maximizar} \quad z = 3x_1 + 2x_2\)\\[3pt]
    \(\text{sujeto a:} \quad x_1 + x_2 = 4\)\\
    \hspace{4.5em}\(x_1 - x_2 - s_2 = 2\)\\
    \hspace{4.5em}\(x_1, x_2, s_2 \geq 0\)
  }
  &
  \makecell[l]{
    \(\text{maximizar} \quad -\omega = -a_1 - a_2\)\\[3pt]
    \(\text{sujeto a:} \quad x_1 + x_2 + a_1 = 4\)\\
    \hspace{4.5em}\(x_1 - x_2 - s_2 + a_2 = 2\)\\
    \hspace{4.5em}\(x_1, x_2, s_2, a_1, a_2 \geq 0\)
  }
  \\
  \end{tabular}
  \caption{Recordatorio de PPL Original y PPL Auxiliar de ejemplo \ref{ej:fase_1}}
  \label{tab:comparacion_ppl}
\end{table}

Ahora, recordando que la base \(\beta\) es el conjunto de las variables básicas que dan solución factible al problema, podríamos construir una base para cada uno de los PPLs del cuadro \ref{tab:comparacion_ppl}. Por ejemplo, para el PPL auxiliar, vamos a elegir la base \(\beta_\omega = \{a_1, a_2\}\)\footnote{la base del PPL auxiliar se ha elegido a propósito, para que se note como se forma una matriz identidad}, y para el PPL original, la base podría ser \(\beta = \{x_1, x_2\}\).

Entonces el vector \(c_\beta\) para cada uno de los PPLs sería:
\begin{itemize}
  \item Para el PPL original (\(\beta\)): \(c_\beta = \begin{bmatrix} 3 & 2 \end{bmatrix}\) 
  \item Para el PPL auxiliar (\(\beta_\omega\)): \(c_{\omega\beta} = \begin{bmatrix} -1 & -1 \end{bmatrix}\)
\end{itemize}
\begin{tcolorbox}[remember, title=Aclaración]
  Vea que el vector de costos \(c_\beta\) no es más que los costos asociados a la base \(\beta\) que se esté usando en ese momento.
\end{tcolorbox}

\subparagraph{4. La matriz \(A_\beta^{-1}\)}

Es la matriz inversa de la submatriz base (\(A_\beta\)), que está formada por las columnas del sistema \(A\) que corresponden a las variables en la base actual.


Esta fórmula nos dice: si agrego la variable \(x_j\) (actualmente fuera de la base), ¿cuánto cambia la función objetivo?
\begin{itemize}
  \item Si \(\bar{c}_j < 0\) en maximización (o \(\bar{c}_j > 0\) en minimización), es favorable introducir \(x_j\).
  \item Si \(\bar{c}_j = 0\), la variable es \textit{degenerada} (puede introducirse sin cambiar el valor de la función objetivo).
  \item Si \(\bar{c}_j > 0\) (en maximización), introducir esa variable empeora la función objetivo.
\end{itemize}

Continuemos con el ejemplo del cuadro \ref{tab:comparacion_ppl}. Entonces, las bases y las submatrices asociadas a cada base para cada PPL son:
\begin{align*}
  \beta = \{x_1,x_2\} \quad \rightarrow& \quad A_\beta = \begin{bmatrix} 1 & 1\\ 1 & -1 \end{bmatrix}\\[4pt]
  \beta_\omega = \{a_1, a_2\} \quad \rightarrow& \quad A_{\omega\beta} = \begin{bmatrix} 1 & 0\\ 0 & 1 \end{bmatrix}
\end{align*}
Entonces, debemos calcular la inversa para cada una de las matrices base:
\begin{align*}
  A_\beta^{-1} &= \begin{bmatrix} 1 & 1\\ 1 & -1 \end{bmatrix}^{-1} = \begin{bmatrix} 0.5 & 0.5\\ 0.5 & -0.5 \end{bmatrix}\\[4pt]
  A_{\omega\beta}^{-1} &= \begin{bmatrix} 1 & 0\\ 0 & 1 \end{bmatrix}^{-1} = \begin{bmatrix} 1 & 0\\ 0 & 1 \end{bmatrix}
\end{align*}
Con esto, podemos entonces, reemplazar en la fórmula de costos reducidos:
\[
  \bar{c}_j = c_j - c_\beta^T A_\beta^{-1} A_j
\]
Entonces, para el PPL \hl{auxiliar}, el costo reducido para la variable \(x_1\) sería:
\begin{enumerate}
  \item Costo de la variable en la función objetivo: \(c_j = 0\)
  \item Vector de costos de la base: \(c_\beta^T = \begin{bmatrix} -1 & -1 \end{bmatrix}\)
  \item Matriz inversa de la matriz de los coeficientes de las restricciones que forman la base actual: \(A_\beta^{-1} = \begin{bmatrix} 1 & 0\\ 0 & 1 \end{bmatrix}\)
  \item Columna del sistema que corresponde a la variable \(x_1\) que está fuera de la base (candidata a entrar): \(A_{x1} = \begin{bmatrix} 1\\ 1 \end{bmatrix}\)
\end{enumerate}
Entonces resulta:
\begin{align*}
  \bar{c}_{x1} = 0 - \begin{bmatrix} -1 & -1 \end{bmatrix} \begin{bmatrix} 1 & 0\\ 0 & 1 \end{bmatrix} \begin{bmatrix} 1\\ 1 \end{bmatrix} = 2
\end{align*}

Y para el PPL \hl{original}, el costo reducido para la variable \(s_1\):
\begin{enumerate}
  \item Costo de la variable en la función objetivo: \(c_j = 0\)
  \item Vector de costos de la base: \(c_\beta^T = \begin{bmatrix} 3 & 2 \end{bmatrix}\)
  \item Matriz inversa de la matriz de los coeficientes de las restricciones que forman la base actual: \(A_\beta^{-1} = \begin{bmatrix} 0.5 & 0.5\\ 0.5 & -0.5 \end{bmatrix}\)
  \item Columna del sistema que corresponde a la variable \(s_1\) que está fuera de la base (candidata a entrar): \(A_{s1} = \begin{bmatrix} 0\\ -1 \end{bmatrix}\)
\end{enumerate}
\begin{align*}
  \bar{c}_{s1} = 0 - \begin{bmatrix} 3 & 2 \end{bmatrix} \begin{bmatrix} 0.5 & 0.5\\ 0.5 & -0.5 \end{bmatrix} \begin{bmatrix} 0\\ -1 \end{bmatrix} = 0.5
\end{align*}

\begin{tcolorbox}[remember, title=¿Para qué usamos costos reducidos?]
  Los costos reducidos indican para cada variable cuánto se modificaría la función objetivo si esa variable entra en la base.
\end{tcolorbox}

\paragraph{Armado de la tabla Simplex}

% En la primera tabla Simplex que armamos:

% \[
% \begin{array}{c|ccccc|c}
% \text{Base} \, \beta & x_1 & x_2 & s_2 & a_1 & a_2 & \text{B} \\
% \hline
% a_1 & 1 & 1 & 0 & 1 & 0 & 4 \\
% a_2 & 1 & -1 & -1 & 0 & 1 & 2 \\
% \hline
% z & -2 & 0 & 1 & 0 & 0 & -6 \\
% \end{array}
% \]

% Fijate en la **última fila** de la tabla (la fila de $z$). Es allí donde se reflejan los **costos reducidos** de las variables **no básicas**. Cada entrada de esa fila es precisamente:

% \[
% \bar{c}_j = c_j - z_j
% \]

% * En este caso, $x_1$ tiene un costo reducido de $-2$
% * $x_2$ tiene costo reducido 0
% * $s_2$ tiene costo reducido $1$
% * $a_1$ y $a_2$ tienen costos reducidos $0$ porque ya están en la base, y sus coeficientes están anulados por la forma canónica

% ¿Qué representa el valor $-2$ en $x_1$?

% Significa que si 
% reemplazás una de las variables básicas actuales por $x_1$, la función $z = -\omega$ aumentaría 2 unidades por cada unidad que entre $x_1$ en la base. Como estamos maximizando $-\omega$, esto equivale a **minimizar $\omega$**.

% Es por eso que $x_1$ es la **mejor candidata para entrar a la base** en esta iteración.



% Los costos reducidos en la fila de $z$ indican para cada variable **cuánto se modificaría la función objetivo si esa variable entra en la base**. En el método Simplex:

% * Se elige como **entrante** a la variable con el **costo reducido más negativo** (en maximización)
% * Se detiene el algoritmo cuando **todos los costos reducidos son mayores o iguales a cero**

% ¿Te gustaría que calcule explícitamente los costos reducidos en este ejemplo usando la fórmula $\bar{c}_j = c_j - c_B^T B^{-1} A_j$? Podemos hacerlo paso a paso.


% Ahora si, el siguiente paso consiste en aplicar el método Simplex para hallar una \textbf{solución básica factible inicial (SBFI)} que sirva de punto de partida para el problema original. Para ello, se construye la \textbf{primera tabla del método Simplex correspondiente al problema auxiliar}.

% \paragraph{Componentes de la tabla}

% La tabla inicial debe contener:

% \begin{itemize}
%   \item Todas las \textbf{variables del sistema}: variables originales \(x_i\), variables de holgura \(h_i\) (si las hay), y variables artificiales \(a_i\).
%   \item Una \textbf{columna de términos independientes} o la matriz \(B\) (lado derecho de las restricciones).
%   \item Una \textbf{fila correspondiente a la función objetivo auxiliar} \(\omega = a_1 + a_2 + \cdots + a_k\), en forma canónica.
%   \item Una \textbf{identificación de la base inicial}, que estará compuesta por las variables artificiales que se agregaron para formar una base identidad.
% \end{itemize}

% \subsubsection*{Forma general de la tabla}

% La forma general de la tabla Simplex es la siguiente:

% \[
% \begin{array}{c|cccccc|c}
% \text{Base} & x_1 & x_2 & \cdots & h_i & a_j & \cdots & \text{Término independiente} \\
% \hline
% a_1 & a_{11} & a_{12} & \cdots & a_{1i} & a_{1n} & \cdots & b_1 \\
% a_2 & a_{21} & a_{22} & \cdots & a_{2i} & a_{2n} & \cdots & b_2 \\
% \vdots & \vdots & \vdots & & \vdots & \vdots & & \vdots \\
% a_k & a_{k1} & a_{k2} & \cdots & a_{ki} & a_{kn} & \cdots & b_k \\
% \hline
% \omega & c_1 & c_2 & \cdots & c_i & c_n & \cdots & -\sum b_i \\
% \end{array}
% \]

% \textbf{Notas:}
% \begin{itemize}
%   \item No confundir la base \(\beta=\{a_1,a_2\}\) con los coeficientes de las restricciones \(A\) (\(a_{11},a_{12},\dots\)) 
%   \item La fila de \(\omega\) está escrita en forma canónica. Es decir, los coeficientes se obtienen sustituyendo cada variable básica en la función objetivo auxiliar \(\omega\), lo cual produce la expresión: 
%   \[
%   \omega = -x_1 - x_2 - \cdots \quad \text{(solo si están presentes en la base)}
%   \]
%   y un término independiente igual a \( -\sum b_i \), si todos los coeficientes de las variables artificiales en \(\omega\) son 1.
%   \item El signo negativo en la función objetivo responde al hecho de que el método Simplex estándar resuelve problemas de \textbf{maximización}, por lo que al resolver el problema auxiliar se transforma el objetivo \( \min \omega \) en \( \max (-\omega) \).
% \end{itemize}

% \begin{tcolorbox}[myconclusion]
%   Ahora continuaremos con el ejemplo \ref{ej:fase_1} para visualizar el contenido de la tabla de forma numérica.
% \end{tcolorbox}

% \[
% \begin{array}{c|rrrrr|r}
% \text{Base} & x_1 & x_2 & s_2 & a_1 & a_2 & \text{Término independiente} \\
% \hline
% a_1 & 1 & 1 & 0 & 1 & 0 & 4 \\
% a_2 & 1 & -1 & -1 & 0 & 1 & 2 \\
% \hline
% z = -\omega & -2 & 0 & 1 & 0 & 0 & -6 \\
% \end{array}
% \]


% \paragraph{Procedimiento}

% A partir de esta tabla inicial, se aplica el método Simplex como en cualquier problema estándar:

% \begin{itemize}
%   \item Se identifica la variable entrante mediante el criterio de optimalidad (el coeficiente más negativo en la fila de \(\omega\)).
%   \item Se determina la variable saliente mediante el criterio del cociente mínimo.
%   \item Se realizan operaciones fila para obtener la nueva tabla (pivoteo).
%   \item Se repite el proceso hasta que todos los coeficientes en la fila de \(\omega\) sean mayores o iguales a cero (óptimo alcanzado).
% \end{itemize}

% Una vez obtenida la solución óptima del problema auxiliar, se evalúa el valor de \(\omega\). Si es cero, se ha obtenido una SBFI para el problema original. En ese caso, se eliminan las variables artificiales de la tabla (pivoteando si aún permanecen en la base) y se continúa con el método Simplex aplicado al problema original.

\subsection{Iteraciones del método Simplex}
\label{sec:iteraciones_simplex}

Una vez se ha construido la tabla Simplex inicial, se procede a iterar el método Simplex hasta alcanzar la solución óptima. El proceso de iteración no es complicado, de hecho, lo más largo y tedioso suele ser el proceso previo de construcción de la tabla Simplex inicial. De igual manera, depende mucho del problema que se esté resolviendo. Simplex es un método general que sirve para cualquier PPL que pueda ser transformado para cumplir con las condiciones del método Simplex.

A continuación veremos el proceso de iteración del método Simplex paso a paso.

\subsubsection{Algoritmo Simplex Paso a Paso (después de construir la tabla inicial)}
Para mostrar el algoritmo paso a paso, vamos a continuar con el ejemplo con el que trabajamos en el capítulo \ref{sec:tabla_simplex}. Para ello vamos a tener en cuenta las siguientes consideraciones:
\begin{itemize}
  \item Asumimos un problema de \textbf{maximización} y está en su forma estándar.
  \item La tabla Simplex muestra \(-\bar{c}_j\) en la fila \(z\).
  \item Variables artificiales están presentes (Fase I), pero el algoritmo es idéntico para Fase II.
\end{itemize}

Pero antes de continuar con el ejemplo, veamos el algoritmo paso a paso.

\paragraph{Paso 1: Verificar optimalidad}

El objetivo de este paso es evaluar si la solución básica actual (correspondiente a la tabla del método Simplex que ya has construido) es óptima. Esto se hace observando la fila de los coeficientes reducidos (comúnmente llamada fila \(z\)) en la tabla.

En términos prácticos, debes observar los coeficientes de la fila \(z\), excluyendo la columna del término independiente \(B\):
\begin{itemize}
  \item Si todos los coeficientes son mayores o iguales a cero, es decir, \(\geq 0\), entonces ya no es posible mejorar más el valor de la función objetivo. Esto indica que la solución actual es \hl{óptima}.
  \item Si algún coeficiente es negativo, significa que al aumentar el valor de la variable correspondiente, se podría mejorar (reducir, en caso de minimización) el valor de la función objetivo. Por tanto, aún no se ha alcanzado la solución óptima, y se debe continuar con la iteración (Paso 2).
\end{itemize}

El fundamento teórico tras este paso es el siguiente: En el método Simplex, el valor de cada coeficiente en la fila \(z\) representa el costo reducido de introducir la variable correspondiente en la base. Estos coeficientes suelen expresarse como \(-\bar{c}_j\), donde:
\begin{itemize}
  \item \(-\bar{c}_j = c_j - c_B^T B^{-1} A_j\), el costo reducido asociado a la variable \(x_j\).
  \item \(c_j\): coeficiente de \(x_j\) en la función objetivo.
  \item \(c_\beta\): vector de costos de las variables básicas actuales.
  \item \(A_\beta\): matriz de columnas básicas.
  \item \(A_j\): columna de la matriz de restricciones correspondiente a la variable \(x_j\).
\end{itemize}
Entonces:
\begin{itemize}
  \item Si \(-\bar{c}_j \geq 0\) para todos los \(j\), eso significa que \(\bar{c}_j \leq 0\). En este caso, introducir cualquier variable no básica en la base no mejora el valor de la función objetivo, por lo tanto, la solución es óptima.
  \item En cambio, si \(-\bar{c}_j < 0\) para algún \(j\), es decir, \(\bar{c}_j > 0\), entonces aumentar \(x_j\) mejoraría la función objetivo (en un problema de minimización), y se debe continuar iterando.
\end{itemize}

Este paso responde a la pregunta: ¿ya hemos optimizado la función objetivo con las variables básicas actuales?

Si todos los coeficientes en la fila \(z\) son positivos o cero, entonces sí: no hay variables no básicas que puedan mejorar el valor de \(z\).

Si hay coeficientes negativos, entonces no: existe al menos una variable que, al entrar a la base, podría mejorar el resultado. En ese caso, se procede al siguiente paso del algoritmo (selección de variable entrante y saliente).

\begin{tcolorbox}[title=Resumen del paso 1]
  \noindent \textbf{Objetivo}: Determinar si la solución actual es óptima.

  \noindent \textbf{Regla}:
  \begin{itemize}
    \item Si todos los coeficientes en la fila \(z\) (excluyendo la columna \textit{B}) son \(\geq 0\) (positivos) \(\rightarrow\) Solución óptima alcanzada.
    \item Si hay algún coeficiente \(< 0\) (negativo) \(\rightarrow\) Ir al Paso 2.
  \end{itemize}

  \noindent \textbf{Fundamento}:
  \begin{itemize}
    \item Coeficiente en fila \(z = -\bar{c}_j\).
    \item \(-\bar{c}_j \geq 0\) implica \(\bar{c}_j \leq 0\) (no hay variables que mejoren \(z\)).
  \end{itemize}
\end{tcolorbox}

\paragraph{Paso 2: Seleccionar variable entrante}

El paso 2 del método Simplex, denominado ``Seleccionar variable entrante'', es el segundo paso de cada iteración una vez que se ha verificado que la solución actual no es óptima. Aquí se toma una decisión clave: qué variable no básica se incorporará a la base, desplazando a alguna de las actuales.

El objetivo es identificar cuál variable \textbf{no básica} debe ingresar a la base. Esta elección busca mejorar el valor de la función objetivo \(z\) de la manera más eficiente posible, dado el estado actual del sistema.

Para hacer esta selección, se utiliza la fila \(z\) de la tabla simplex, donde están los costos reducidos \(-\bar{c}_j\) (es decir, la ganancia marginal de introducir cada variable no básica):
\begin{itemize}
  \item Se identifica el coeficiente más negativo en la fila \(z\) (excluyendo el término independiente). Esto se debe a que un coeficiente más negativo implica que aumentar esa variable no básica reduce el valor de \(z\) (recordando que en un problema de minimización, buscamos reducir \(z\) al máximo).
  \item Si hay empate entre varios coeficientes igualmente negativos, se puede:
  \begin{itemize}
    \item Elegir cualquiera (esto puede llevar a diferentes caminos, pero en teoría todos válidos), o
    \item Aplicar una regla de desempate, como la Regla de Bland, que elige la variable de menor índice para evitar ciclos.
  \end{itemize}
\end{itemize}

Desde el punto de vista teórico:
\begin{itemize}
  \item Los coeficientes de la fila \(z\) están dados por \(-\bar{c}_j = c_j - c_\beta^T A_\beta^{-1} A_j\). Como explicamos antes, este valor representa cuánto cambiará la función objetivo si se introduce una unidad de la variable \(x_j\) a la solución (es decir, si se la incorpora a la base).
  \item Cuanto más negativo sea \(-\bar{c}_j\), más positivo será \(\bar{c}_j\), y por tanto, mayor será la reducción de \(z\) si se introduce esa variable.
  \item En otras palabras, estamos buscando la variable con el mayor potencial para reducir el valor de la función objetivo, que es precisamente aquella con el \(-\bar{c}_j\) más negativo.
\end{itemize}

Este paso responde a la pregunta: ¿cuál de las variables no básicas, si la incorporamos a la base, más mejora la función objetivo?

La respuesta es: aquella cuya columna en la fila \(z\) tiene el valor más negativo. Aumentar esa variable generará la mayor mejora inmediata en \(z\), por lo tanto es la candidata ideal para entrar en la base.

\begin{tcolorbox}[title=Resumen del paso 2]
  \noindent \textbf{Objetivo}: Elegir la variable que ingresa a la base para mejorar \(z\).
  
  \noindent \textbf{Regla}:
  \begin{itemize}
    \item Seleccionar la columna con el coeficiente más negativo en la fila \(z\).
    \item Si hay empate, elegir cualquiera (o usar regla de Bland).
  \end{itemize}
  
  \noindent \textbf{Fundamento}: El coeficiente más negativo en fila \(z\) (\(-\bar{c}_j\)) corresponde al \(\bar{c}_j\) más positivo, que genera la mayor mejora en \(z\).
\end{tcolorbox}

\paragraph{Paso 3: Seleccionar variable saliente (Ratio Test)}

El \textit{paso 3 del método Simplex}, denominado ``\textit{Seleccionar variable saliente (Ratio Test)}'', completa la decisión de qué variable debe entrar y cuál debe salir de la base en la iteración actual. Este paso asegura que el movimiento hacia una nueva solución básica \textbf{sea factible}, es decir, no viole las restricciones del problema.

El objetivo es determinar \textit{cuál de las variables básicas actuales debe salir de la base} al incorporar la nueva variable entrante. Esto se hace asegurando que el nuevo punto al que se avanza permanezca dentro de la región factible (es decir, que todas las variables sigan siendo no negativas).

Una vez que se ha elegido la variable entrante (la columna con el coeficiente más negativo en la fila \(z\)), se aplica el llamado \textbf{``Test del cociente''} o \emph{Ratio Test} para decidir la variable saliente:
\begin{enumerate}
  \item Trabajar \textbf{solo con las filas correspondientes a las restricciones},
    es decir, las que representan variables básicas actuales.
  \item Observar la \textbf{columna de la variable entrante}.
  \item Considerar \textbf{únicamente los coeficientes positivos} de esa columna
    (coeficientes \(>0\)). Esto es fundamental porque un coeficiente negativo
    implicaría que al aumentar la variable entrante, la variable básica
    correspondiente aumentaría también, lo cual puede violar la factibilidad.
  \item Para cada fila con coeficiente positivo en la columna de la variable
    entrante, calcular el \textbf{cociente}:
    \[
      \mathrm{Ratio}_i
      = \frac{B_i}{A_{i,\mathrm{entrante}}}
    \]
    donde
    \begin{itemize}
      \item \(B_i\): el término independiente de la fila \(i\) (columna ``\textit{B}'').
      \item \(A_{i,\mathrm{entrante}}\): el valor de la columna
        de la variable entrante en la fila \(i\).
    \end{itemize}
  \item Elegir la fila con \textbf{el menor cociente positivo}: es la que
    \emph{llega primero a cero} al aumentar la variable entrante, es decir,
    la \emph{primera en volverse no factible} si seguimos avanzando en esa
    dirección.
  \item La \textbf{variable básica} correspondiente a esa fila es la que
    \textbf{debe salir} de la base.
\end{enumerate}

Desde el punto de vista teórico, este paso garantiza que el movimiento en la
dirección de la variable entrante:

\begin{itemize}
  \item Sea \textbf{factible}, es decir, que todas las variables básicas
    permanezcan no negativas.
  \item No viole ninguna restricción de desigualdad original.
\end{itemize}

Al elegir la \textbf{cantidad máxima que puede incrementarse la variable
entrante} sin hacer negativa alguna variable básica, se asegura que el nuevo
vértice factible es \emph{vecino del actual} y se mantiene dentro del
\emph{poliedro de soluciones factibles}.

Este paso refleja una transición a lo largo de una \emph{arista del poliedro
factible} en la dirección que más mejora \(z\), deteniéndose justo en el
siguiente vértice (nueva solución básica).

\subsection*{Caso especial: problema no acotado}

Si \textbf{todos los coeficientes en la columna de la variable entrante son
\(\leq 0\)}, entonces no se puede aplicar el Ratio Test, ya que:

\begin{itemize}
  \item Aumentar la variable entrante haría que todas las variables básicas
    crezcan (o no se afecten), \emph{nunca decrezcan}, lo que implica que no hay
    ningún límite superior que detenga el crecimiento de la variable entrante.
\end{itemize}

Este escenario indica que el problema \textbf{no está acotado}, es decir, que la
función objetivo \emph{puede seguir mejorando indefinidamente}, y por lo tanto
\textbf{no existe una solución óptima finita}.

Este paso responde a la pregunta:
\begin{quote}
  \textbf{¿Qué variable debe salir de la base para mantener la factibilidad cuando
  una nueva variable entra?}
\end{quote}

La respuesta se obtiene aplicando el Ratio Test: se calcula el cociente entre
el término independiente y el coeficiente de la variable entrante en cada fila,
y se elige el menor cociente positivo. La variable básica correspondiente a
esa fila es la que sale.

\begin{tcolorbox}[title=Resumen del paso 3]
  \noindent \textbf{Objetivo}: Determinar qué variable abandona la base.
  
  \noindent \textbf{Regla}:
  \begin{enumerate}
    \item En la columna de la variable entrante, considerar solo coeficientes \(> 0\).
    \item Calcular ratio para cada fila \(i\):
     \[
     \text{Ratio}_i = \frac{B_i}{A_{i,\text{entrante}}}
     \]
    \item Seleccionar la fila con el menor ratio positivo.
    \item La variable básica de esa fila es la variable saliente.
  \end{enumerate}
  
  \noindent \textbf{Caso especial}: Si todos los coeficientes \(\leq 0\) \(\rightarrow\) Problema no acotado (no existe solución óptima finita).
\end{tcolorbox}

\paragraph{Paso 4: Realizar pivoteo}

El \textit{Paso 4 del método Simplex}, frecuentemente llamado
\textit{``Paso de pivoteo''}, es el proceso algebraico que actualiza toda la
tabla del método Simplex luego de decidir qué variable entra y cuál sale de la
base. Su objetivo es producir una \textbf{nueva solución básica factible}, en
la que se espera que el valor de la función objetivo haya mejorado.

El objetivo de este paso es \textbf{actualizar la tabla} del método Simplex de
acuerdo con el cambio de base (variable entrante y saliente) que se determinó
en el Paso 3. Esta actualización garantiza que la nueva tabla represente un
sistema equivalente, pero con una \textbf{nueva base} que contiene la variable
entrante.

\begin{enumerate}
  \item \textbf{Identificar el pivote}:\\
    El \textbf{pivote} es el elemento ubicado en la \textbf{intersección} de
    la columna de la \textbf{variable entrante} y la fila de la
    \textbf{variable saliente}. Este valor será utilizado para transformar la
    fila y la columna asociadas.

  \item \textbf{Normalizar la fila pivote}:\\
    Se transforma la fila pivote dividiendo cada uno de sus elementos
    (incluyendo el término independiente y el coeficiente en la fila \(z\), si
    corresponde) por el valor del pivote. El propósito es convertir el pivote
    en 1, lo que permitirá más adelante convertir los demás elementos de su
    columna en cero.\\
    \emph{Ejemplo}: si el pivote es 3 y la fila pivote es \((3,\;6,\;9)\), después
    de la normalización se convierte en \((1,\;2,\;3)\).

  \item \textbf{Hacer ceros en la columna pivote (excepto en fila pivote)}:\\
    Se aplica \emph{eliminación gaussiana} para anular todos los demás
    elementos de la \textbf{columna de la variable entrante}, excepto el 1 que
    ahora ocupa la posición del pivote. Para cada fila \(i\) (incluyendo la fila
    \(z\)), se realiza:
    \[
      \text{Nueva Fila}_i
      = \text{Fila}_i
      - \bigl(\text{Coeficiente}_{i,\mathrm{entrante}} \times \text{Fila Pivote}\bigr).
    \]
    Esto asegura que en todas las demás filas, la columna correspondiente a la
    variable entrante tenga valor 0, excepto en la fila pivote donde tiene 1,
    convirtiendo la columna en parte de una matriz identidad.

  \item \textbf{Actualizar la base}:\\
    En la columna de variables básicas (frecuentemente llamada ``Base''), se
    reemplaza la variable que salió con la variable que entró. Esto formaliza
    el cambio de base, reflejando la nueva estructura de la solución básica
    factible.
\end{enumerate}
Este paso es fundamental porque garantiza que:
\begin{itemize}
  \item La nueva tabla es \textbf{equivalente al sistema original}, pero
    reescrita en función de una nueva base.
  \item El sistema se mantiene en forma \textbf{canónica} (o forma estándar), es decir, la base
    sigue representada por una submatriz identidad.
  \item La nueva solución sigue siendo factible (todas las variables básicas
    siguen siendo \(\ge 0\)) y, si el procedimiento se realizó correctamente,
    \textbf{mejora o mantiene} el valor de la función objetivo.
\end{itemize}

Geométricamente, este paso corresponde a \textit{avanzar de un vértice del
poliedro (o politopo) factible al siguiente vértice adyacente}, a lo largo de una arista
que mejora el valor de la función objetivo.

Este paso responde a la pregunta:
\begin{quote}
  \textbf{¿Cómo actualizamos la tabla para reflejar el nuevo vértice factible
  al que nos hemos movido?}
\end{quote}

La respuesta es aplicar el \textbf{proceso de pivoteo}, que consiste en:
\begin{enumerate}
  \item Normalizar la fila pivote.
  \item Anular el resto de la columna pivote.
  \item Actualizar la base.
\end{enumerate}

Al final, se obtiene una nueva tabla lista para iniciar nuevamente el Paso 1.

\begin{tcolorbox}[title=Resumen del paso 4]
  \noindent \textbf{Objetivo}: Actualizar la tabla para la nueva base.
  
  \noindent \textbf{Pasos}:
  \begin{enumerate}
    \item \textbf{Identificar el pivote}: Intersección de columna entrante y fila saliente.
    \item \textbf{Normalizar la fila pivote}: Dividir toda la fila por el valor del pivote para convertirlo en 1. Por ejemplo si el pivote es 3, dividir toda la fila por 3.
    \item \textbf{Hacer ceros en la columna pivote (excepto en fila pivote)}: Para cada fila \(i\) (incluyendo fila \(z\)):
       \[
       \text{Nueva Fila}_i = \text{Fila}_i - (\text{Coeficiente}_{i,\text{entrante}} \times \text{Fila Pivote})
       \]
    \item \textbf{Actualizar la base}: Reemplazar variable saliente por variable entrante en la columna ``Base''.
  \end{enumerate}
\end{tcolorbox}

\paragraph{Paso 5: Actualizar la fila \(z\) y \textit{B} y repetir el proceso}

Realmente este no es un paso explícito como el resto de pasos, pero me parece útil mencionarlo para que puedas verificar que la tabla sigue siendo consistente. Además me parece muy importante destacar que si el problema ha llegado a su solución óptima, se detecta en el paso 1, no en este paso.

\begin{tcolorbox}[title=Resumen del paso 5]
  \noindent \textbf{Objetivo}: Actualizar la fila \(z\) y \textit{B} y repetir el proceso.
  
  \noindent \textbf{Nota}: La fila \(z\) se actualiza en el Paso 4 (operaciones fila). Si todo es consistente, repetir.
  
  \noindent \textbf{Verificar consistencia}:
  \begin{itemize}
    \item El nuevo \textit{B} en fila \(z\) debe ser el valor actual de \(z\).
    \item Coeficientes de variables básicas en fila \(z\) deben ser 0.
  \end{itemize}
\end{tcolorbox}

\subsection{Ejecución del método con un ejemplo}

Continuando con el ejemplo de la sección anterior:

Tabla inicial:

\[
  A=\left[
  \begin{NiceMatrix}[code-before = 
    \rectanglecolor{red!20}{2-1}{4-1}
    \rectanglecolor{green!20}{2-4}{4-4}
    \rectanglecolor{blue!20}{2-5}{4-5}
  ]
  x_1 & y_2 & y_2' & y_3 & h_1 & s_1 & s_2 \\
  \hline
  5 & 0 & 0 & -8 & 0 & -1 & 0 \\
  -1 & 6 & -6 & 0 & 0 & 0 & -1 \\
  0 & -6 & 6 & -4 & 1 & 0 & 0
  \end{NiceMatrix}
  \right]
\]

\[
\begin{array}{c|ccccc|c}
\beta & x_1 & x_2 & s_1 & a_1 & a_2 & B \\
\hline
a_1 & 1 & 1 & 0 & 1 & 0 & 4 \\
a_2 & 1 & -1 & -1 & 0 & 1 & 2 \\
\hline
z & -2 & 0 & 1 & 0 & 0 & -6 \\
\end{array}
\]

| Base | \(x_1\) | \(x_2\) | \(s_2\) | \(a_1\) | \(a_2\) | RHS  |
|------|---------|---------|---------|---------|---------|------|
| \(a_1\) | 1       | 1       | 0       | 1       | 0       | 4    |
| \(a_2\) | 1       | -1      | -1      | 0       | 1       | 2    |
| \(z\)  | -2      | 0       | 1       | 0       | 0       | -6   |

---

#### Iteración 1:
1. Paso 1: Coeficientes en \(z\): \([-2, 0, 1, 0, 0]\). Hay negativo (\(-2\)) → No óptimo.
2. Paso 2: Variable entrante = \(x_1\) (coeficiente más negativo: \(-2\)).
3. Paso 3 (Ratio Test):
   - Columna \(x_1\): \([1, 1]^T\) (ambos \(> 0\)).
   - Ratios: \(\frac{4}{1} = 4\), \(\frac{2}{1} = 2\).
   - Menor ratio: \(2\) → Fila de \(a_2\) sale.
4. Paso 4 (Pivoteo):
   - Pivote: \(1\) (intersección \(a_2\) y \(x_1\)).
   - Normalizar fila \(a_2\): Ya es 1.
   - Operaciones fila:
     - Fila \(a_1\): \([1, 1, 0, 1, 0, 4] - (1 \times [1, -1, -1, 0, 1, 2]) = [0, 2, 1, 1, -1, 2]\)
     - Fila \(z\): \([-2, 0, 1, 0, 0, -6] - (-2 \times [1, -1, -1, 0, 1, 2]) = [0, -2, -1, 0, 2, -2]\)
   - Nueva base: \(\{a_1, x_1\}\).

Tabla actualizada:
| Base | \(x_1\) | \(x_2\) | \(s_2\) | \(a_1\) | \(a_2\) | RHS  |
|------|---------|---------|---------|---------|---------|------|
| \(a_1\) | 0       | 2       | 1       | 1       | -1      | 2    |
| \(x_1\) | 1       | -1      | -1      | 0       | 1       | 2    |
| \(z\)  | 0       | -2      | -1      | 0       | 2       | -2   |

---

#### Iteración 2:
1. Paso 1: Coeficientes en \(z\): \([0, -2, -1, 0, 2]\). Hay negativo (\(-2\)) → No óptimo.
2. Paso 2: Variable entrante = \(x_2\) (coeficiente más negativo: \(-2\)).
3. Paso 3 (Ratio Test):
   - Columna \(x_2\): \([2, -1]^T\). Solo \(a_1\) tiene coeficiente \(> 0\) (2).
   - Ratio: \(\frac{2}{2} = 1\) → Fila de \(a_1\) sale.
4. Paso 4 (Pivoteo):
   - Pivote: \(2\) (intersección \(a_1\) y \(x_2\)).
   - Normalizar fila \(a_1\): Dividir por 2 → \([0, 1, 0.5, 0.5, -0.5, 1]\).
   - Operaciones fila:
     - Fila \(x_1\): \([1, -1, -1, 0, 1, 2] - (-1 \times [0, 1, 0.5, 0.5, -0.5, 1]) = [1, 0, -0.5, 0.5, 0.5, 3]\)
     - Fila \(z\): \([0, -2, -1, 0, 2, -2] - (-2 \times [0, 1, 0.5, 0.5, -0.5, 1]) = [0, 0, 0, 1, 1, 0]\)

Tabla actualizada:
| Base | \(x_1\) | \(x_2\) | \(s_2\)  | \(a_1\)  | \(a_2\)  | RHS  |
|------|---------|---------|----------|----------|----------|------|
| \(x_2\) | 0       | 1       | 0.5      | 0.5      | -0.5     | 1    |
| \(x_1\) | 1       | 0       | -0.5     | 0.5      | 0.5      | 3    |
| \(z\)  | 0       | 0       | 0        | 1        | 1        | 0    |

---

#### Final de Fase I:
- Paso 1: Coeficientes en \(z\): \([0, 0, 0, 1, 1] \geq 0\) → Solución óptima de Fase I.
- Valor de \(z = 0\) → Factible. Pasamos a Fase II.

---

### Transición a Fase II
1. Eliminar columnas de variables artificiales (\(a_1, a_2\)).
2. Reemplazar fila \(z\) por la función objetivo original.
3. Recalcular costos reducidos para la base actual.

Supongamos objetivo original: Maximizar \(x_1 + x_2\).  
- Nueva fila \(z\): Coeficientes \([-1, -1, 0]\) (porque muestra \(-\bar{c}_j\)).
- Cálculo:
  - Base actual: \(\{x_1, x_2\}\), \(c_B = [1, 1]\).
  - \(\bar{c}_{s_2} = 0 - [1, 1] \cdot [-0.5, 0.5]^T = 0 - (0) = 0\).
  - Valor de \(z = [1, 1] \cdot [1, 3]^T = 4\).

Tabla Fase II:
| Base | \(x_1\) | \(x_2\) | \(s_2\)  | RHS  |
|------|---------|---------|----------|------|
| \(x_2\) | 0       | 1       | 0.5      | 1    |
| \(x_1\) | 1       | 0       | -0.5     | 3    |
| \(z\)  | 0       | 0       | 0        | 4    |

- Solución óptima: \(x_1 = 3\), \(x_2 = 1\), \(z = 4\).

---

### Diagrama de Flujo del Algoritmo Simplex
```mermaid
graph TD
    A[Inicio con tabla inicial] --> B[Verificar optimalidad]
    B -->|Todos coeficientes ≥ 0| C[¡Solución óptima!]
    B -->|Hay coeficiente < 0| D[Seleccionar variable entrante]
    D --> E[Ratio Test]
    E -->|Todos coeficientes ≤ 0| F[¡Problema no acotado!]
    E -->|Encontrar pivote| G[Pivoteo]
    G --> H[Actualizar tabla]
    H --> B
```

### Consejos clave
1. Fila \(z\) siempre muestra \(-\bar{c}_j\).
2. Ratio Test ignora coeficientes \(\leq 0\).
3. Variables artificiales:
   - Fase I: Minimizar \(\omega = \sum a_i\) (transformada a \(z = -\omega\)).
   - Si \(z_{\text{Fase I}} < 0\) → Problema infactible.
4. Fase II: Usar solución de Fase I y función objetivo original.

\subsection{Resumen de contenidos}

\subsubsection{Método simplex}

El método simplex puede resumirse con un diagrama de flujo que muestra los pasos del algoritmo. Es importante que usted haya leído todo el capitulo para entender los detalles implícitos en cada parte.

\begin{figure}[ht]
  \centering
  \includegraphics[width=0.8\textwidth]{simplex.png}
\end{figure}

% Diagrama de flujo usando tikz (no me convence el resultado, por eso dejo la imagen)
% \begin{tikzpicture}[node distance=1.5cm]
%   \node (start) [startstop] {Inicio: problema en forma estándar};
%   \node (base) [process, below of=start] {Seleccionar base inicial factible};
%   \node (optimal) [decision, below of=base, yshift=-0.5cm] {¿Solución óptima?};
%   \node (choose) [process, right of=optimal, xshift=5.5cm] {Elegir variable entrante};
%   \node (pivot) [process, below of=choose] {Determinar variable saliente};
%   \node (update) [process, below of=pivot] {Actualizar tabla (pivote)};
%   \node (end) [startstop, below of=optimal, yshift=-2.3cm] {Fin: solución óptima alcanzada};

%   % Coordenadas auxiliares
%   \coordinate (optwest) at (optimal.west);
%   \coordinate (endwest) at (end.west);
%   \coordinate (mid1) at ($(optwest)+(-1.5,0)$);
%   \coordinate (mid2) at ($(mid1 |- endwest)$);

%   % Flechas
%   \draw [arrow] (start) -- (base);
%   \draw [arrow] (base) -- (optimal.north);
%   \draw [arrow] (optimal.east) -- node[above] {No} (choose.west);
%   \draw [arrow] (choose) -- (pivot);
%   \draw [arrow] (pivot) -- (update);
%   \draw [arrow] (update.west) -- ++(-3.5,0) -| (optimal.south);
%   \draw [arrow] (optwest) -- (mid1) -- node[midway,left] {Sí} (mid2) -- (endwest);
% \end{tikzpicture}



\paragraph{Costos de Penalización}

La introducción de variables de holgura y superfluas no altera ni la naturaleza de las restricciones ni al objetivo. Por consiguiente, estas variables se incorporan a la función objetivo con coeficientes cero. Las variables artificiales, sin embargo, cambian la naturaleza de als restricciones. Ya que se agregan a un solo lado de una desigualdad, el nuevo sistema es equivalente al sistema anterior de restricciones sólo si las variables artificiales son cero. Para garantizar estas condiciones en la solución óptima las variables artificiales se introducen a la función objetivo con coeficientes muy grandes (si hay que maximizar la ganancia, una ganancia muy pequeña o negativa implica una gran penalización, por el contrario si hay que minimizar costos, un costo grande implica una gran penalización para esa variable). Estos coeficientes se denotan generalmente como \(+M\) (para minimización) o \(-M\) (para maximización).

Los coeficientes \(\pm M\) donde \(M\) se considera un número positivo muy grande, representan el severo costo de penalización que se impone a las variables artificiales cuando se activan en la solución óptima. 
  \newpage
  
  \section{Espacios vectoriales}

\subsection{Cuerpo matemático}

Un cuerpo (o campo) \(K\) es, a grandes rasgos, un conjunto no vacío provisto de dos operaciones binarias, la suma \((+)\) y el producto \((\cdot)\), que satisfacen los siguientes requisitos:

\noindent \textbf{1. Estructura aditiva}
\begin{itemize}
  \item La suma es asociativa y conmutativa.
  \item Existe un elemento neutro aditivo, denotado \(0\), tal que \(a+0=0+a=a\) para todo \(a\in K\).
  \item Cada elemento \(a\in K\) tiene un inverso aditivo \(-a\) tal que \(a+(-a)=0\).
\end{itemize}

\noindent \textbf{2. Estructura multiplicativa}
\begin{itemize}
  \item El producto es asociativo y conmutativo.
  \item Existe un elemento neutro multiplicativo, denotado \(1\neq 0\), tal que \(a\cdot 1 = 1\cdot a = a\) para todo \(a\in K\).
  \item Cada elemento distinto de cero, \(a\neq 0\), tiene un inverso multiplicativo \(a^{-1}\) tal que \(a\cdot a^{-1}=1\).
\end{itemize}

\noindent \textbf{3. Distributividad}

El producto distribuye sobre la suma:
\[
  a\cdot(b+c) = a\cdot b + a\cdot c,\quad
  (a+b)\cdot c = a\cdot c + b\cdot c
  \quad\forall\,a,b,c\in K.
\]

En conjunto, estas propiedades garantizan que en \(K\) podemos manejar sumas, restas, productos y divisiones (salvo por cero) con la misma libertad aritmética que conocemos en \(\mathbb{R}\).

\ejemplo{ Cuerpos matemáticos habituales}
\begin{itemize}
  \item El cuerpo de los números reales, \(\mathbb{R}\).
  \item El cuerpo de los números complejos, \(\mathbb{C}\).
  \item El cuerpo de los números racionales, \(\mathbb{Q}\).
\end{itemize}

Cuando definimos un espacio vectorial, escogemos uno de esos cuerpos como conjunto de escalares; de este modo, todas las multiplicaciones por escalares vienen ``heredadas'' de la estructura de \(K\). Por ejemplo: En un espacio vectorial sobre \(\mathbb{R}\), cualquier \(k\in\mathbb{R}\) puede multiplicar a los vectores. En un espacio vectorial sobre \(\mathbb{C}\), permitimos escalares complejos, lo cual es fundamental, por ejemplo, en espacios de funciones de variable compleja.

Así, el cuerpo \(K\) proporciona el marco algebraico (con sus operaciones y axiomas) que garantiza que la multiplicación escalar en el espacio vectorial se comporte correctamente.

\subsection{Espacio vectorial}

Sea \textit{V} un conjunto no vacío, y \(\left(K, +, \cdot\right)\) un cuerpo (se trabajará con el cuerpo de los números reales y el cuerpo de los números complejos), se definen una operación binaria interna y una operación binaria externa en \textit{V} como sigue:
\begin{align*}
  +:& V \times V \rightarrow V \quad \text{tal que}~~ (u,v) \rightarrow u+v \\
  \cdot ~ :& K \times V \rightarrow V \quad \text{tal que}~~ (k,u) \rightarrow k\cdot u
\end{align*}

La notación:
\[
+\;:\;V\times V\longrightarrow V
\]
se interpreta de la manera siguiente:
\begin{itemize}
  \item \textbf{Dominio}: \(
  V \times V
  \) es el \textbf{producto cartesiano} de \(V\) consigo mismo. Un elemento de \(V \times V\) es un \textbf{par ordenado} donde \(u \in V\) y \(v \in V\). Decimos ``relación de \(V\) consigo mismo'' porque la suma toma dos vectores de \(V\).
  \item \textbf{Imagen o codominio}: \(V\) indica que el resultado de aplicar la operación ``\(+\)'' a cualquier par \((u,v)\) debe ser otro elemento de \(V\). Es la condición de \textit{clausura}: sumar dos vectores nos vuelve a dar un vector \hl{del mismo espacio}.
\end{itemize}

En términos coloquiales: ``\(+\) es una regla que, a cada par de vectores \((u,v)\), le asigna un único vector \(u+v\), y este resultado siempre pertenece a \(V\).''

De modo análogo, la multiplicación por escalar se escribe
\[
  \cdot\;:\;K\times V\;\longrightarrow\;V,
\]
donde:
\begin{itemize}
  \item El dominio \(K\times V\) agrupa un escalar \(k\in K\) y un vector \(u\in V\).
  \item El codominio \(V\) exige que el producto \(k\cdot u\) tenga como resultado siempre un vector en \(V\).
\end{itemize}
Así quedan definidas de manera precisa las dos operaciones básicas de cualquier espacio vectorial:
\begin{itemize}
  \item La suma toma dos vectores y los combina en uno solo.
  \item El producto escalar toma un escalar y un vector, y produce un vector.
\end{itemize}

Definir explícitamente estos mapeos te permite luego verificar los axiomas (asociatividad, conmutatividad, distributividad, existencia de neutros e inversos, etc.), pues todos ellos se expresan como igualdades entre resultados de estas dos funciones.

Se dice que \textit{V} con dichas operaciones es un \textit{K} espacio vectorial si se verifican los siguientes axiomas:
\begin{itemize}
  \item \textbf{A1} - Asociativa: Cualesquiera sean \(u,v\) y \(w\) de \(V\): \[
    (u+v)+w = u + (v+w)
  \]
  \item \textbf{A2} - Existencia de elemento neutro: Existe \(\vec{0}\) en \(V\) para todo \(u\) de \(V\):\[
    u+\vec{0}=\vec{0}+u = u
  \]
  \item \textbf{A3} - Existencia de elementos opuestos: Para todo \(u\) de \(V\) existe \(-u\) también en \(V\):\[
    u + (-u) = (-u) + u = \vec{0}
  \]
  \item \textbf{A4} - Conmutativa: Cualesquiera sean \(u\) y \(v\) de \(V\): \[
    u + v = v + u
  \]
  \item \textbf{A5} - Para todo \(k\) de \(K\), cualesquiera sean \(u\) y \(v\) de \(V\): \[
    k\cdot (u+v)=k\cdot u+k\cdot v
  \]
  \item \textbf{A6} - Cualesquiera sean \(k\) y \(k'\) de \(K\), para todo \(u\) de \(V\):\[
    (k + k')\cdot u = k \cdot u + k' \cdot u
  \]
  \item \textbf{A7} - Cualesquiera sean \(k\) y \(k'\) de \(K\), para todo \(u\) de \(V\):\[
    (k\cdot k')\cdot u = k \cdot (k' \cdot u)
  \]
  \item \textbf{A8} - Para todo \(u\) de \(V\): \[
    1 \cdot u = u \quad \text{(siendo 1 la unidad del cuerpo)}
  \]
\end{itemize}
A los elementos del espacio vectorial se los llama \hl{vectores}.

\begin{quote}
  \ejemplo{ Defina un espacio vectorial de los polinomios de grado menor o igual a dos}

  \textbf{Solución}: Para definir el espacio vectorial de polinomios de grado a lo sumo dos, procedemos del siguiente modo:

  Sea \(K\) el cuerpo de los números reales (o complejos) y consideremos el conjunto
  \[
  V = \bigl\{\,p(x) = a_0 + a_1x + a_2x^2 \;\big|\; a_0,a_1,a_2 \in K\bigr\}.
  \]
  En otras palabras, \(V\) es el conjunto de todas las funciones polinómicas cuyo grado es menor o igual a dos.

  A continuación, definimos en \(V\) las operaciones de suma y multiplicación por escalar exactamente como las de los polinomios:
  \begin{itemize}
    \item Para cualesquiera \(p(x),q(x)\in V\), la suma se define punto a punto:
    \[
      (p+q)(x) \;=\; p(x)+q(x) \;=\; \bigl(a_0+b_0\bigr) + \bigl(a_1+b_1\bigr)x + \bigl(a_2+b_2\bigr)x^2,
    \]
    donde \(p(x)=a_0+a_1x+a_2x^2\) y \(q(x)=b_0+b_1x+b_2x^2\).

  \item Para cualquier escalar \(k\in K\) y \(p(x)\in V\):
    \[
      (k\cdot p)(x) \;=\; k\,p(x) \;=\; (k\,a_0) + (k\,a_1)x + (k\,a_2)x^2.
    \]
  \end{itemize}

  Ahora bien, para ver que efectivamente \((V,+,\cdot)\) es un espacio vectorial sobre \(K\), basta observar que:
  \begin{enumerate}
    \item La suma de dos polinomios de grado \(\le2\) sigue siendo de grado \(\le2\) y lo mismo vale para la multiplicación por escalar.

    \item Asociatividad y conmutatividad de la suma, existencia del elemento neutro (el polinomio \(0(x)=0\)) y de inversos aditivos (para \(p(x)\), su inverso es \(-p(x)\)) se heredan directamente de las propiedades algebraicas de los coeficientes en \(K\).

    \item Distributividad de la multiplicación escalar respecto de la suma de polinomios y de escalares, compatibilidad de la multiplicación de escalares ( \((k\,k')p = k(k'p)\) ) y existencia del escalar unidad como operador neutro (\(1\cdot p = p\)) se verifican ecuación a ecuación, ya que se reducen a propiedades en el cuerpo \(K\).
  \end{enumerate}

  Por tanto, todos los axiomas A1 a A8 se cumplen en este caso concreto.
  \begin{tcolorbox}[myconclusion]
    Si no se está seguro de que todos los axiomas se cumplen (como en este caso), se puede demostrar cada uno de ellos para confirmarlo. Si los ocho axiomas se cumplen, entonces el conjunto es un espacio vectorial
  \end{tcolorbox}
\end{quote}

\begin{tcolorbox}[interesting_data, title=¿Puedo crear un conjunto cualquiera y verificar si es un espacio vectorial?]
  Exacto. El procedimiento general es precisamente ese: uno elige un conjunto \(V\) y un cuerpo de escalares \(K\) (por ejemplo \(\mathbb{R}\) o \(\mathbb{C}\)), define en \(V\) dos operaciones: la suma de dos elementos de \(V\) y la multiplicación de un escalar de \(K\) por un elemento de \(V\) y, si logra demostrar que estas operaciones satisfacen los ocho axiomas (A1-A8), entonces \((V,+,\cdot)\) es un espacio vectorial sobre \(K\). A partir de ese momento, cualquier objeto que pertenezca a \(V\) se denomina vector.

  No hay más requisitos ocultos: lo importante es que la suma y el producto por escalar estén bien definidos (es decir, que siempre produzcan un elemento de \(V\)) y que cumplan las propiedades de asociatividad, conmutatividad, existencia de neutros e inversos aditivos, distributividad y compatibilidad con la estructura del cuerpo.
\end{tcolorbox}

\paragraph{Consecuencias de la definición de espacio vectorial}

La definición de espacio vectorial implica que se cumplan las siguientes propiedades:
\begin{itemize}
  \item \textbf{P1}: \(0\cdot u = \vec{0} \qquad \forall u \in V\)
  \item \textbf{P2}: \(k \cdot \vec{0} = \vec{0} \qquad \forall k \in K\)
  \item \textbf{P3}: \((-1) \cdot u = -u \qquad \forall u \in V\)
\end{itemize}

\begin{tcolorbox}[mydanger]
  Cuidado: Es muy importante tener en cuenta que \[
  \text{Axiomas A1-A8} \quad \implies \quad \text{Propiedades P1-P3}
  \]
  Si solo se te dan P1, P2 y P3 sin garantizar que las operaciones están bien definidas, que la suma sea asociativa, que haya un neutro, o que se cumplan las reglas de distribución y compatibilidad, entonces no puedes concluir que estás en un espacio vectorial.
\end{tcolorbox}

\subsection{Conjunto de vectores: familia libre y familia ligada}

Un conjunto ordenado de vectores: \(F = \left\{u_1,u_2,\cdots,u_m\right\}\) de un espacio vectorial \(V(K)\) se denomina familia de vectores.

\subsubsection{Combinación lineal}

Una combinación lineal de los vectores de una familia con escalares \(a_1, a_2, \cdots, a_m\) es \textit{siempre} un vector del espacio vectorial \(V(K)\).
\[
  u = a_1 u_1 + a_2 u_2 + \cdots + a_m u_m \qquad \forall u \in V
\]
Si el vector \(u\) de \(V\) es nulo, entonces la combinación lineal es:
\[
  \vec{0} = a_1 u_1 + a_2 u_2 + \cdots + a_m u_m
\]
Esta combinación lineal es un sistema de ecuaciones lineales homogéneo, si al resolverlo resulta que los escalares son todos nulos \((a_1,a_2,\cdots,a_m = 0)\) entonces la familia de vectores \(F=\left\{u_1, u_2, \cdots, u_m \right\}\) recibe el nombre de \textbf{familia libre}.

En el caso contrario, si al menos uno de los escalares es no nulo, la familia \(F\) se denomina \textbf{ligada}.

Los vectores de una familia libre son linealmente independientes, esto significa que ninguno de ellos se puede expresar como combinación lineal de los demás de la familia. Los vectores de una familia ligada son linealmente dependientes.

\subsubsection{Propiedades de una familia libre}

Los vectores de una familia libre (linealmente independientes) presentan las siguientes propiedades:
\begin{enumerate}
  \item Si \(F\) es una familia libre, el vector nulo no pertenece a ella. \\ Demostración: Sea \(F=\left\{\vec{0},u_1, u_2, \cdots, u_m\right\}\) una familia de vectores \(V(K)\), en la combinación lineal: \[
    \vec{0} = a_0 \vec{0} + a_1 u_1 + a_2 u_2 + \cdots + a_m u_m
  \]
  Si suponemos que \(a_1, a_2, \cdots, a_m = 0\) resulta entonces \(\vec{0}=a_0 \vec{0}\). Pero por la propiedad \textbf{P2} estamos admitiendo que \(a_0\) no es necesariamente nulo, lo que implica que la familia no es libre.
  \item Si una familia de vectores de \(V\) consta de un solo vector no nulo, es una familia libre. \\ Demostración: Sea \(u_1 \neq \vec{0}\) y \(F=\left\{u_1\right\}\) una familia de vectores de \(V(K)\), la combinación lineal \(\vec{0} = a_1 u_1\) resulta que es verdadera si \(a_1 = 0\) o bien si \(u_1 = \vec{0}\). Sin embargo partimos de que \(u_1 \neq \vec{0}\), por lo que \(a_1 = 0\), lo que implica que \(F\) es una familia libre.
\end{enumerate}

\subsubsection{Propiedades de una familia ligada}

Los vectores de una familia ligada (linealmente dependientes) presentan las siguientes propiedades:
\begin{enumerate}
  \item Si el vector nulo pertenece a una familia \(F\) de vectores de \(V\), \(F\) es una familia ligada. (La demostración es análoga a la propiedad 1 del encabezado anterior).
  \item Si \(F\) es una familia ligada de vectores de \(V\), al menos uno de sus vectores se puede expresar como combinación lineal de los demás. \\ Demostración: Sea \(F=\left\{u_1, u_2, \cdots, u_m\right\}\) una familia ligada en \(V(K)\), y sea la combinación lineal del vector nulo:\[
    \vec{0} = a_1 u_1 + a_2 u_2 + \cdots + a_m u_m
  \]
  Por ser \(F\) ligada, al menos uno de los escalares \(a_i \neq 0\). Supongamos que \(a_1 \neq 0\), entonces: \[
    \vec{0} = a_1 u_1 + a_2 u_2 + \cdots + a_m u_m, ~~ a_1 \neq 0 \implies u_1 = -\frac{a_2}{a_1}u_2 - \cdots - \frac{a_m}{a_1}u_m
  \]
  Lo que nos está indicando que el vector \(u_1\) es combinación lineal de los demás vectores de la familia \(F\).
\end{enumerate}

\subsubsection{Familia generatriz}

Una familia de vectores de \(V(K): F=\left\{u_1,u_2,\cdots,u_m\right\}\) recibe el nombre de familia generatriz o generadora si \textit{todo} vector del espacio vectorial \(V\) se puede expresar como combinación lineal de los vectores de \(F\). Los vectores de \(F\) reciben el nombre de \textit{vectores generadores}.

Es decir, cualquiera sea \(u\) de \(V\), ess posible expresarlo de la siguiente manera:\[
  u = a_1 u_1 + a_2 u_2 + \cdots + a_m u_m
\]

\teorema{Dos vectores de un espacio vectorial \(V\) son linealmente dependientes (L.D.) si y sólo si uno de ellos es un múltiplo escalar del otro.}

\teorema{Sea \(K\) un cuerpo, y \(K^n\) el espacio vectorial de dimensión \(n\) sobre \(K\). Entonces, todo conjunto de \(n\) vectores linealmente independientes en \(K^n\) forma una base y, en consecuencia, genera todo \(K^n\).}
\label{teo:generador_li}

Por ejemplo, si \(K=\mathbb{R}\) (cuerpo de los números reales), el teorema se aplica como sigue: Todo conjunto de \(n\) vectores linealmente independientes (L.I.) en \(\mathbb{R}^n\) genera \(\mathbb{R}^n\)

\subsection{Base y dimensión}

Una familia de vectores \(F = \left\{u_1, u_2, \cdots, u_m\right\}\) de \(V(K)\) se denomina \hl{base} de \(V\) si es a la vez familia \textbf{libre y generatriz}.

Esto equivale a decir que todo \(u \in V\) se puede expresar como combinación lineal de los vectores de \(F\), o bien:
\[
  u = a_1 u_1 + a_2 u _2 + \cdots + a_m u_m
\] 
y además si \(u\) es el vector nulo entonces:
\[
  \vec{0} = a_1 u_1 + a_2 u_2 + \cdots + a_m u_m \qquad a_i = 0; ~~ i=1,2,\cdots,m
\]
es decir, los vectores de \(F\) son L.I.

Hemos visto en el teorema \ref{teo:generador_li} que todo conjunto de \(n\) vectores linealmente independientes (L.I.) en \(\mathbb{R}^n\) genera a \(\mathbb{R}^n\). Entonces, de acuerdo a la definición de base, se tiene que:
\begin{tcolorbox}
  \centering
  Todo conjunto de \(n\) vectores L.I. en \(K^n\), constituye una base de \(K^n\)
\end{tcolorbox}

Si el espacio vectorial \(V\) tiene una base finita, es decir, con finitos elementos, entonces la \textbf{dimensión} de \(V\) que se denota \(\text{dim}V\), es el número de vectores que tiene una base cualquiera de \(V\). Este último recibe el nombre de \textit{espacio vectorial de dimensión finita}. En cualquier otro caso, se dice que \(V\) es un espacio vectorial de dimensión infinita.

\begin{itemize}
  \item Todo vector de \(V\) se expresa de manera única en cada base.
  \item El número de vectores de una base de \(V\) se denomina \hl{dimensión} del espacio vectorial.
  \item Todas las bases de un espacio vectorial de dimensión finita \textit{n} tienen exactamente \textit{n} vectores.
  \item Existen espacios vectoriales de dimensión infinita.
  \item El espacio vectorial \(\left\{\vec{0}\right\}\) tiene dimensión cero por definición.
  \item Si \textit{V} es un espacio vectorial de dimensión finita \textit{n}:
  \begin{enumerate}
    \item \(n+1\) vectores de \(V\) son linealmente dependientes, lo que equivale a decir que una familia libre tiene a lo más \(n\) elementos (ver teorema \ref{teo:base_ld}).
    \item una familia generatriz tiene como mínimo \textit{n} elementos (ver teorema \ref{teo:elementos_de_una_base}).
    \item toda familia libre de \(V\) con \(n\) elementos es una base de \(V\).
    \item toda familia generatriz de \(V\) con \(n\) elementos es una base de \(V\).
  \end{enumerate}
\end{itemize}

\teorema{Si \(\left\{v_1, v_2, \cdots, v_n\right\}\) es una base de un espacio vectorial \(V\), entonces todo conjunto con más de \(n\) vectores es linealmente dependiente (L.D.).}
\label{teo:base_ld}

\teorema{Si \(\left\{u_1, u_2, \cdots, u_m\right\}\) y \(\left\{v_1, v_2, \cdots, v_n\right\}\) son bases del espacio vectorial \(V\), entonces \(m=n\)}
\label{teo:elementos_de_una_base}

\subsubsection{Componentes de un vector relativas a una base}

Sea \(V(K)\) un espacio vectorial de dimensión \(m\) y \(F = \left\{u_1,u_2,\cdots,u_m\right\}\) una base de \(V\), un vector \(u\) cualquiera de \(V\) se expresa de manera única:
\[
  u = a_1 u_1 + a_2 u_2 + \cdots + a_m u_m
\]
Los escalares que permiten esta combinación lineal reciben el nombre de componentes del vector \(u\) y se puede anotar asi:
\[
  u = a_1 u_1 + a_2 u_2 + \cdots + a_m u_m = \begin{pmatrix}
    a_1 \\ a_2 \\ \vdots \\ a_m
  \end{pmatrix}
\]
El vector de coordenadas de \(u\) relativo a la base \(B\) se denota \((u)_B\):
\[
  (u)_B = (a_1, a_2, \cdots, a_m)
\]
Si \(v\) es otro vector de \(V\) se expresa en \(B\) de la siguiente manera:
\[
  v = b_1 u_1 + b_2 u_2 + \cdots + b_m u_m = \begin{pmatrix}
    b_1 \\ b_2 \\ \vdots \\ b_m
  \end{pmatrix}
\]
Las componentes del vector \(u+v\) son las suma de las componentes de \(u\) más las componentes de \(v\), y las componentes del vector \(k\cdot u\) son las que se obtienen de multiplicar \(k\) por cada una de las componentes de \(u\). De forma explícita:
\begin{align*}
  u + v &= (a_1 u_1 + a_2 u_2 + \cdots + a_m u_m) + (b_1 u_1 + b_2 u_2 + \cdots + b_m u_m) \\[3pt]
  u + v &= \begin{pmatrix}
    a_1 \\ a_2 \\ \vdots \\ a_m
  \end{pmatrix} + \begin{pmatrix}
    b_1 \\ b_2 \\ \vdots \\ b_m
  \end{pmatrix} = \begin{pmatrix}
    a_1 + b_1 \\ a_2 + b_2 \\ \vdots \\ a_m + b_m
  \end{pmatrix} \\[10pt]
  k\cdot u &= k \cdot \begin{pmatrix}
    a_1 \\ a_2 \\ \vdots \\ a_m
  \end{pmatrix} = \begin{pmatrix}
    k \cdot a_1 \\ k \cdot a_2 \\ \vdots \\ k \cdot a_m
  \end{pmatrix}
\end{align*}

\subsubsection{Cambio de base}

Lo desarrollaremos para un espacio vectorial de dimensión 2 y luego lo generalizaremos a un espacio de dimensión finita \textit{n}.

\paragraph{Cambio de base en \(\mathbb{R}^2\)}

Sea \(V(K)\) un espacio vectorial y \(F_1 = \left\{u_1, u_2\right\}\) y \(F_2 = \left\{v_1, v_2\right\}\) dos bases de \(V\), un vector \(w\) de \(V\) se expresa de manera única en \(F_1\) y de manera única en \(F_2\):
\begin{gather*}
  (w)_{F1} = \alpha_1 u_1 + \alpha_2 u_2 = \begin{pmatrix}
    \alpha_1 \\ \alpha_2
  \end{pmatrix} \qquad (w)_{F2} = \beta_1 v_1 + \beta_2 v_2 = \begin{pmatrix}
    \beta_1 \\ \beta_2
  \end{pmatrix} \\[3pt]
  w = \alpha_1 u_1 + \alpha_2 u_2 = \beta_1 v_1 + \beta_2 v_2
\end{gather*}
Si se sabe que las componentes de los vectores de \(F_2\) en la base de \(F_1\) son: 
\[
  v_1 = \begin{pmatrix}
    a_{11} \\ a_{21}
  \end{pmatrix} = a_{11} u_1 + a_{21} u_2 \qquad \text{y} \qquad v_2 = \begin{pmatrix}
    a_{12} \\ a_{22}
  \end{pmatrix} = a_{12} u_1 + a_{22} u_2
\]
Es posible relacionar ambas expresiones de \(w\):
\begin{align*}
  w &= \alpha_1 u_1 + \alpha_2 u_2 = \beta_1 v_1 + \beta_2 v_2 = \beta_1 \begin{pmatrix}
    a_{11} \\ a_{21}
  \end{pmatrix} + \beta_2 \begin{pmatrix}
    a_{12} \\ a_{22}
  \end{pmatrix} \\
  &= \beta_1 (a_{11} u_1 + a_{21}u_2) + \beta_2 (a_{12}u_1 + a_{22}u_2)\\[5pt]
  w &= \alpha_1 u_1 + \alpha_2 u_2 = \beta_1 a_{11} u_1 + \beta_1 a_{21} u_2 + \beta_2 a_{12}u_1 + \beta_2 a_{22}u_2\\
  &= (a_{11}\beta_1 + \beta_2 a_{12}) u_1 + (\beta_1 a_{21} + \beta_2 a_{22}) u_2
\end{align*}
Por igualación podemos decir que:
\begin{align*}
  \alpha_1 &= \beta_1 a_{11} + \beta_2 a_{12} \\ 
  \alpha_2 &= \beta_1 a_{21} + \beta_2 a_{22}
\end{align*}
Que es un sistema de dos ecuaciones con dos incógnitas, con solución única, ya que dijimos que cada vector se expresa de manera única en cada base. Resolver el sistema equivale a hallar las componentes de \(w\) en \(F_2\) a partir de \(F_1\) o viceversa.

En forma matricial lo podemos expresar así:
\begin{align*}
  X = P \cdot X' ~ : ~ \begin{pmatrix}
    \alpha_1 \\ \alpha_2
  \end{pmatrix} = \begin{pmatrix}
    a_{11} & a_{12} \\ 
    a_{21} & a_{22}
  \end{pmatrix} \cdot \begin{pmatrix}
    \beta_1 \\ \beta_2
  \end{pmatrix}
\end{align*}
Donde \(X\) contiene a las componentes de \(w\) en \(F_1\), \(P\) contiene en cada columna a las componentes de los vectores de \(F_2\), y se le denomina matriz de pasaje, y \(X'\) contiene a las componentes de \(w\) en \(F_2\).

Lo visto se generaliza a un espacio vectorial de dimensión finita \(n\).

\subsection{Subespacio vectorial}

Sea \(V(K)\) un espacio vectorial y \(S\) un subconjunto no vacío de \(V\), si \(S\) es espacio vectorial sobre \(K\) respecto a las mismas operaciones definidas en \(V\), es un subespacio de \(V\). En otras palabras: \(S(K)\) subespacio de \(V(K)\)

\paragraph{Parte estable (propiedad)}

Un subconjunto \(S\) no vacío de \(V(K)\) es estable para las combinaciones lineales si se verifica que para cualesquier par de elementos \(u,v\) de \(S\) y para cualquier escalar \(k\) de \(K\), los vectores \(u+v\) y \(k\cdot u\) pertenecen también a \(S\). \(S\) recibe el nombre de parte estable.
\[
  S ~ \text{es parte estable} ~ \Longleftrightarrow ~ S \subset V \land S \neq \emptyset \land \begin{cases}
    u + v \in S \quad \forall u, v \in S\\
    k \cdot u \in S \quad \forall k \in K \land \forall u \in S
  \end{cases}
\]
Se puede demostrar que si ``\(S\) es parte estable de \(V(K)\) es un subespacio vectorial de \(V\)''.

Por tanto, para comprobar que \(S\) es parte estable basta con verificar:
\begin{enumerate}
  \item \(S\neq\emptyset\) (o, mejor, que \(\mathbf{0}\in S\)).
  \item Si \(u,v\in S\) entonces \(u+v\in S\).
  \item Si \(u\in S\) y \(k\in K\) entonces \(k\cdot u\in S\).
\end{enumerate}

Con eso ya queda demostrado que \(S\) hereda todos los axiomas de espacio vectorial y es, en consecuencia, un subespacio de \(V\).


\subsubsection{Subespacios triviales}

Sea \(V(K)\) un espacio vectorial, los siguientes reciben el nombre de \textit{subespacios triviales} o impropios: \(\left\{\vec{0}\right\} (K) ~~ \text{y} ~~ V(K)\), es decir el conjunto formado exclusivamente por el vector nulo y el mismo \(V\), cualquier otro subespacio de \(V\) recibe el nombre de subespacio propio.

Todo espacio vectorial admite al menos los subespacios triviales.

\subsubsection{Dimensión de los subespacios}

Sea el espacio vectorial \(V(K)\), de dimensión finita \(n\), todo subespacio de \(V\) tiene dimensión finita \(m\) tal que \(m \leq n\).

\begin{itemize}
  \item Por convención el subespacio trivial \(U=\left\{\vec{0}\right\}\) como carece de base, entonces se dice que su dimensión es cero, \(\text{dim}\left\{\vec{0}\right\}=0\), en este caso el único subespacio es él mismo, por lo tanto mantiene la dimensión nula.
  \item Si \(\text{dim }V=1\), admite subespacios de dimensión 0 y dimensión 1. Esto significa que sólo admite a los subespacios triviales, ya que no existe otro subespacio de la misma dimensión incluido él.
  \item Si \(\text{dim }V = 2\), admite los subespacios triviales, de dimensión 0 y de dimensión 2, pero también admite subespacios de dimensión 1.
  \item En general si \(\text{dim }V=n\) admite a los subespacios triviales de dimensión 0 y \textit{n} y además todos los de dimensión \(m\) tal que \(0 \leq m \leq n\)
\end{itemize}
  \newpage
  
  \section{Matrices}

Si desea repasar la unidad de matrices, de forma resumida, puede ver la sección \ref{sec:repaso_matrices}.
  \newpage
  
  \section{Espacios Métricos}

\subsection{Función distancia}

Sea \(E\) un conjunto no vacío cuyos elementos se denominan ``puntos''. Se llama \textit{función distancia} a toda aplicación que a cada par de puntos de \(E\) les asigna un número real, verificando las siguientes propiedades:

\[
  d: E \times E \rightarrow \mathbb{R}, \quad \text{tal que } d(a,b) = h
\]

\begin{itemize}
  \item Simetría: \(d(a,b) = d(b,a)\)
  \item Identidad: \(d(a,b) = 0 \iff a=b\)
  \item Desigualdad triangular: \(d(a,c) \leq d(a,b) + d(b,c)\)
\end{itemize}

Un par \((E, d)\), donde \(E\) es un conjunto y \(d\) una función distancia definida sobre él, se denomina \hl{espacio métrico}.

\begin{tcolorbox}[remember, title=Observación]
  En el caso particular en que \(E\) sea un subconjunto de un espacio vectorial \(V(K)\), como por ejemplo \(\mathbb{R}^n\), es posible asociar a cada par ordenado de vectores \((a,b)\) el vector de desplazamiento \(\vec{ab} = b - a = d(a,b)\). En tales contextos, la distancia entre puntos puede definirse a partir de la norma de dicho vector, por ejemplo, utilizando la norma euclídea (o pre-hilbertiana):
  \[
    d(a,b) = \| b - a \| = \sqrt{(b_1 - a_1)^2 + \cdots + (b_n - a_n)^2}
  \]
  En este caso, el espacio métrico sería \(\left(\mathbb{R}^n,\left\lVert a-b \right\rVert \right)\), aunque profundizaremos más sobre esto en la próxima sección.

  Sin embargo, es importante destacar que la definición general de espacio métrico no requiere que \(E\) tenga estructura vectorial; basta con que exista una función distancia que cumpla las propiedades mencionadas.
\end{tcolorbox}

\subsection{Función norma}

\subsubsection{En un espacio vectorial real \(V(\mathbb{R})\)}

Sea \(V(\mathbb{R})\) un espacio vectorial sobre \(\mathbb{R}\). Se denomina \textit{norma} a una función que asigna a cada vector \(u \in V\) un número real no negativo, cumpliendo las siguientes condiciones:
\begin{align*}
  \left\lVert \cdot \right\rVert : V &\longrightarrow \mathbb{R} \\
  u &\longmapsto \left\lVert u \right\rVert
\end{align*}
Para todo \(u, v \in V\) y \(t \in \mathbb{R}\), la función \(\left\lVert \cdot \right\rVert\) debe verificar:

\begin{itemize}
  \item \textbf{Positividad:} \(\left\lVert u \right\rVert \geq 0\), y \(\left\lVert u \right\rVert = 0 \iff u = \vec{0}\)
  \item \textbf{Homogeneidad absoluta:} \(\left\lVert t \cdot u \right\rVert = \left|t\right| \cdot \left\lVert u \right\rVert\)
  \item \textbf{Desigualdad triangular:} \(\left\lVert u + v \right\rVert \leq \left\lVert u \right\rVert + \left\lVert v \right\rVert\)
\end{itemize}

Cuando se ha definido una función norma sobre el espacio vectorial \(V\), al par \((V, \left\lVert \cdot \right\rVert)\) se lo denomina \textbf{espacio normado}.

\vspace{5mm}

\paragraph{Consecuencias de la definición}

A partir de las propiedades que cumple la norma, se deducen inmediatamente los siguientes resultados:

\[
  \left\lVert \vec{0} \right\rVert = 0, \qquad
  \left\lVert -u \right\rVert = \left\lVert u \right\rVert, \qquad
  \left\lVert u - v \right\rVert = \left\lVert v - u \right\rVert
\]

\paragraph{Vector unitario (o normalizado):}

Un vector \(u \in V\) tal que \(\left\lVert u \right\rVert = 1\) se denomina \textbf{vector unitario}. En general, si \(u \neq \vec{0}\), siempre es posible normalizarlo mediante:
\[
u' = \frac{u}{\left\lVert u \right\rVert} \quad \Rightarrow \quad \left\lVert u' \right\rVert = 1
\]

\noindent \ejemplo{ Algunas funciones norma son las siguientes}
\begin{enumerate}

  \item \textbf{En \(\mathbb{R}\): valor absoluto}

  Sea \(E = \mathbb{R}\). La norma natural sobre este espacio es el valor absoluto:
  \[
    \left\lVert x \right\rVert = |x|
  \]

  \item \textbf{En \(\mathbb{R}^n\): normas \(p\)}

  Sea \(x = (x_1, x_2, \dots, x_n) \in \mathbb{R}^n\). Se definen distintas normas en función de su forma:

  \begin{itemize}
    \item \textbf{Norma uno (o norma \(L^1\)):}
    \[
      \left\lVert x \right\rVert_1 = \sum_{i=1}^{n} |x_i|
    \]

    \item \textbf{Norma dos (o euclídea, o \(L^2\)):}
    \[
      \left\lVert x \right\rVert_2 = \sqrt{\sum_{i=1}^{n} x_i^2}
    \]

    \item \textbf{Norma \(p\) (para \(p \geq 1\)):}
    \[
      \left\lVert x \right\rVert_p = \left( \sum_{i=1}^{n} |x_i|^p \right)^{\frac{1}{p}}
    \]

    \item \textbf{Norma infinito (o \(L^\infty\)):}
    \[
      \left\lVert x \right\rVert_\infty = \max \{ |x_i| : 1 \leq i \leq n \}
    \]
  \end{itemize}

  \item \textbf{En \(C([a,b])\): funciones continuas en un intervalo}

  Sea \(f \in C([a,b])\), es decir, una función continua definida sobre \([a,b]\). Se definen:

  \begin{itemize}
    \item \textbf{Norma \(L^2\):}
    \[
      \left\lVert f \right\rVert_2 = \sqrt{ \int_a^b f(x)^2 \, dx }
    \]

    \item \textbf{Norma supremo (o norma \(L^\infty\)):}
    \[
      \left\lVert f \right\rVert_\infty = \max_{x \in [a,b]} |f(x)|
    \]
  \end{itemize}

  \item \textbf{Normas matriciales}

  Sea \(A = (a_{ij}) \in \mathbb{R}^{n \times n}\). Se definen:

  \begin{itemize}
    \item \textbf{Norma matricial uno:}
    \[
      \left\lVert A \right\rVert_1 = \max_{1 \leq j \leq n} \sum_{i=1}^{n} |a_{ij}|
    \]
    Es el máximo valor absoluto de las sumas por columnas.

    \item \textbf{Norma matricial infinito:}
    \[
      \left\lVert A \right\rVert_\infty = \max_{1 \leq i \leq n} \sum_{j=1}^{n} |a_{ij}|
    \]
    Es el máximo valor absoluto de las sumas por filas.

    \item \textbf{Norma matricial dos:}
    \[
      \left\lVert A \right\rVert_2 = \sqrt{ \lambda_{\max} }
    \]
    donde \(\lambda_{\max}\) es el mayor valor propio de \(A^T A\). Esta norma es inducida por la norma euclídea.

    \item \textbf{Norma de Frobenius:}
    \[
      \left\lVert A \right\rVert_F = \sqrt{ \sum_{i=1}^{n} \sum_{j=1}^{n} a_{ij}^2 }
    \]
    Coincide con la norma euclídea sobre el espacio \(\mathbb{R}^{n^2}\).
  \end{itemize}

\end{enumerate}

\subsubsection{Norma y distancia}

Siempre es posible definir una distancia inducida por una función norma tal que:
\[
  d(u,v) = \left\lVert u-v\right\rVert \qquad \text{o también} \qquad d(a,b) = \left\lVert a-b \right\rVert
\]
Por lo tanto, todo \textbf{espacio normado}, es también \textbf{espacio métrico}.

\subsection{Producto interior (producto escalar)}

\subsubsection{En un espacio vectorial real \(V(\mathbb{R})\)}

Sea \(V(\mathbb{R})\) un espacio vectorial sobre el cuerpo \(\mathbb{R}\). Se denomina \textbf{producto escalar} (o \textbf{producto interior}) a una aplicación que asigna a cada par de vectores un número real, cumpliendo ciertas propiedades:
\begin{align*}
  \left\langle \cdot, \cdot \right\rangle : V \times V &\longrightarrow \mathbb{R} \\
  (u,v) &\longmapsto \left\langle u, v \right\rangle
\end{align*}
Para todos \(u, v, w \in V\) y para todo escalar \(t \in \mathbb{R}\), el producto escalar debe verificar:

\begin{itemize}
  \item \textbf{Positividad:} \(\left\langle u, u \right\rangle \geq 0\), y \(\left\langle u, u \right\rangle = 0 \iff u = \vec{0}\)
  \item \textbf{Simetría:} \(\left\langle u, v \right\rangle = \left\langle v, u \right\rangle\)
  \item \textbf{Linealidad en la segunda variable:}
  \[
    \left\langle u, v + w \right\rangle = \left\langle u, v \right\rangle + \left\langle u, w \right\rangle
    \quad \text{y} \quad
    \left\langle u, t v \right\rangle = t \left\langle u, v \right\rangle
  \]
\end{itemize}

Cuando un producto escalar está definido sobre el espacio vectorial \(V\), al par \(\left(V, \left\langle \cdot, \cdot \right\rangle\right)\) se lo denomina un \textbf{espacio prehilbertiano}.

\paragraph{Consecuencias de la definición}
\[
\left\langle u, v \right\rangle = 0 \quad \Rightarrow \quad \text{los vectores \(u\) y \(v\) son ortogonales}
\]

\ejemplo{ Algunos productos interiores}

\begin{enumerate}
  \item \textbf{Producto escalar usual en \(\mathbb{R}^n\):}

  Sea \(u = (u_1, \dots, u_n)\) y \(v = (v_1, \dots, v_n)\), se define:
  \[
    \left\langle u, v \right\rangle = \sum_{i=1}^{n} u_i v_i
  \]

  \item \textbf{Producto escalar no usual en \(\mathbb{R}^2\):}

  Para \(u = (x, y)\), \(v = (x', y')\), definimos:
  \[
    \left\langle u, v \right\rangle = x x' - y x' - x y' + 4 y y'
  \]
  Este producto escalar es válido si cumple las propiedades definitorias.

  \item \textbf{Producto escalar en el espacio \(P_n(\mathbb{R})\):}

  Para polinomios reales de grado menor o igual a \(n\):
  \[
    \left\langle u, v \right\rangle = \int_0^1 u(x) v(x) \, dx
  \]

  \item \textbf{Producto escalar en el espacio \(M_{n \times n}(\mathbb{R})\):}

  Para matrices reales cuadradas de orden \(n\):
  \[
    \left\langle A, B \right\rangle = \mathrm{tr}(B^T A)
  \]
  donde \(\mathrm{tr}(\cdot)\) denota la traza de una matriz.
\end{enumerate}

\subsubsection{Norma y distancia inducidas por el producto interior}

Todo producto interior induce una norma sobre el espacio vectorial \(V\), definida por:
\[
\left\lVert u \right\rVert = \sqrt{ \left\langle u, u \right\rangle }
\]
Por lo tanto, todo \textbf{espacio prehilbertiano} es también un \textbf{espacio normado}.

Además, esta norma induce una distancia (o métrica) sobre \(V\), dada por:
\[
d(u, v) = \left\lVert u - v \right\rVert = \sqrt{ \left\langle u - v, u - v \right\rangle }
\]
Por lo tanto, todo \textbf{espacio prehilbertiano} es también un \textbf{espacio métrico}.

\subsection{Espacios complejos con producto interno}

Sea \(V\) un espacio vectorial sobre el cuerpo \(\mathbb{C}\) de los números complejos. Se denomina \textbf{producto interior} (o \textbf{producto hermítico}) a una función que asigna a cada par de vectores un número complejo, cumpliendo ciertas propiedades fundamentales:

\begin{align*}
  \left\langle \cdot , \cdot \right\rangle : V \times V &\longrightarrow \mathbb{C} \\
  (u, v) &\longmapsto \left\langle u, v \right\rangle
\end{align*}

Para todo \(u, v, w \in V\) y \(t \in \mathbb{C}\), se deben cumplir las siguientes propiedades:

\begin{itemize}
  \item \textbf{Linealidad en la segunda variable:}
  \[
    \left\langle u, v + w \right\rangle = \left\langle u, v \right\rangle + \left\langle u, w \right\rangle
  \]
  \item \textbf{Conjugación en la primera variable:}
  \[
    \left\langle t \cdot u, v \right\rangle = \overline{t} \cdot \left\langle u, v \right\rangle
  \]
  \item \textbf{Hermiticidad:}
  \[
    \left\langle u, v \right\rangle = \overline{ \left\langle v, u \right\rangle }
  \]
  \item \textbf{Positividad definida:}
  \[
    \left\langle u, u \right\rangle \in \mathbb{R}, \quad \left\langle u, u \right\rangle \geq 0, \quad \text{y} \quad \left\langle u, u \right\rangle = 0 \iff u = \vec{0}
  \]
\end{itemize}

Definido un producto interior que cumple estas propiedades, al par \(\left(V, \left\langle \cdot , \cdot \right\rangle\right)\) se lo denomina un \textbf{espacio unitario}.

\ejemplo{Ejemplos de producto interior sobre \(\mathbb{C}\):}

\begin{enumerate}
  \item \textbf{Producto interior canónico en \(\mathbb{C}^n\):}

  Para \(u = (x_1, \dots, x_n)\) y \(v = (x'_1, \dots, x'_n)\), se define:
  \[
    \left\langle u, v \right\rangle = x_1 \cdot \overline{x'_1} + \cdots + x_n \cdot \overline{x'_n}
    = \sum_{k=1}^{n} x_k \cdot \overline{x'_k}
  \]

  \item \textbf{Producto interior en el espacio \(C^*_{[a,b]}\):}

  Sea \(C^*_{[a,b]}\) el conjunto de funciones continuas con valores complejos definidas en \([a,b]\), es decir:
  \[
    f(t) = f_1(t) + i f_2(t), \quad \text{con } f_1, f_2 \in C_{[a,b]}(\mathbb{R})
  \]
  El producto interior se define como:
  \[
    \left\langle f, g \right\rangle = \int_{a}^{b} f(t) \cdot \overline{g(t)} \, dt
  \]

  \item \textbf{Producto interior en \(M_n(\mathbb{C})\):}

  Sea \(M_n(\mathbb{C})\) el espacio de matrices complejas cuadradas de orden \(n\). Se define:
  \[
    \left\langle A, B \right\rangle = \sum_{i,j=1}^{n} a_{ij} \cdot \overline{b_{ij}} 
  \]
  Este producto también puede expresarse como:
  \[
    \left\langle A, B \right\rangle = \mathrm{tr}(B^* A)
  \]
  donde \(B^*\) denota la matriz adjunta de \(B\) (conjugada transpuesta).
\end{enumerate}

Una gran parte de las propiedades de los espacios prehilbertianos se extiende naturalmente a los espacios unitarios.

\paragraph{Norma y distancia inducidas por el producto hermítico}

Como en el caso real, todo producto hermítico induce una norma sobre el espacio \(V\):
\[
  \left\lVert u \right\rVert = \sqrt{ \left\langle u, u \right\rangle } \quad \in \mathbb{R}_{\geq 0}
\]

A partir de esta norma, se puede definir una función distancia (o métrica) mediante:
\[
  d(u, v) = \left\lVert u - v \right\rVert = \sqrt{ \left\langle u - v, u - v \right\rangle }
\]

Esta distancia es siempre un número real no negativo, por lo que todo \textbf{espacio unitario} es también un \textbf{espacio métrico}.

\subsection{Bases ortogonales y ortonormadas}

\subsubsection{Vectores y bases ortogonales}

Sea \(V(K)\) un espacio vectorial en el que se ha definido un producto interior. Dados \(u, v \in V\), si
\[
\left\langle u, v \right\rangle = 0,
\]
entonces los vectores \(u\) y \(v\) se dicen \textbf{ortogonales}.

\textbf{Nota:} El vector nulo \(\vec{0}\) es ortogonal a todo vector de \(V\).

Una familia de vectores \(F = \{u_1, u_2, \dots, u_n\}\) se denomina un \textbf{conjunto ortogonal} si:
\[
\left\langle u_i, u_j \right\rangle = 0 \quad \text{para todo } i \neq j
\]

Si además el conjunto \(F\) es una base del espacio \(V\), entonces se dice que \(F\) es una \textbf{base ortogonal}.

\subsubsection{Vectores y bases normalizados}

Sea \(V(K)\) un espacio vectorial provisto de una norma. Un vector \(u \in V\) se dice \textbf{normalizado} o \textbf{unitario} si su norma es igual a uno:
\[
\left\lVert u \right\rVert = 1
\]

Todo vector no nulo puede ser normalizado mediante:
\[
u' = \frac{u}{\left\lVert u \right\rVert}, \quad \text{de modo que } \left\lVert u' \right\rVert = 1
\]

Una familia de vectores \(F = \{u_1, u_2, \dots, u_n\}\) se dice un \textbf{conjunto normalizado} si:
\[
\left\lVert u_i \right\rVert = 1 \quad \text{para todo } i
\]

Si además el conjunto \(F\) es una base del espacio \(V\), se dice que \(F\) es una \textbf{base normalizada}.

\subsubsection{Bases ortonormadas}

Una familia de vectores de \(V\) que sea simultáneamente ortogonal y normalizada se denomina \textbf{ortonormada}. Es decir:
\[
\left\langle u_i, u_j \right\rangle = 
\begin{cases}
1 & \text{si } i = j \\
0 & \text{si } i \neq j
\end{cases}
\]

Si esta familia constituye además una base de \(V\), entonces se dice que es una \textbf{base ortonormada}.

\begin{table}[H]
  \centering
  \begin{tabular}{c|c|c}
    \textbf{Tipo de base} & \textbf{Vectores ortogonales} & \textbf{Vectores normalizados} \\
    \hline
    Arbitraria & No necesariamente & No necesariamente \\
    Ortogonal & Sí: \(\left\langle u_i, u_j \right\rangle = 0\) si \(i \neq j\) & No necesariamente \\
    Ortonormal & Sí: \(\left\langle u_i, u_j \right\rangle = 0\) si \(i \neq j\) & Sí: \(\left\lVert u_i \right\rVert = 1\) \\
  \end{tabular}
  \caption{Comparación entre bases en espacios con producto interior}
\end{table}

Una base ortonormal es el caso más estructurado: los vectores son mutuamente ortogonales y además tienen norma unitaria. Toda base ortonormal es ortogonal, pero no toda base ortogonal está normalizada.
  \newpage
  
  \section{Funciones Lineales}

Una función lineal (también llamada transformación lineal o aplicación lineal) es una función entre dos espacios vectoriales que preserva la estructura algebraica de los vectores, es decir, respeta la suma de vectores y la multiplicación por escalares.

\textbf{Definición}: Sea \(f: V \to U\) una función entre espacios vectoriales sobre el mismo cuerpo \(K\). 
\begin{align*}
  f: V &\rightarrow U\\
  v &\rightarrow u
\end{align*}
que a cada vector \(v\) de \(V\) le hace corresponder un único vector \(f(v)\) de \(U\).

Diremos que \(f\) es una aplicación lineal si para todo \(v, w \in V\) y todo escalar \(t \in K\), se cumple:
\begin{itemize}
  \item \(f(v + w) = f(v) + f(w)\)
  \item \(f(t \cdot v) = t \cdot f(v)\)
\end{itemize}

La primera condición se la conoce como \textit{aditividad}, indica que la función respeta las operaciones internas definidas en los espacios vectoriales involucrados, la segunda se denomina \textit{homogeneidad} y está indicando que la función también respeta las operaciones externas. 

\begin{tcolorbox}[remember, title=Aclaración]
  \(V(K)\) y \(U(K)\) son dos espacios vectoriales definidos sobre un mismo cuerpo \(K\) (por ejemplo, \(\mathbb{R}\) o \(\mathbb{C}\)).
  
  La función \(f: V \rightarrow U\) toma un vector \(v \in V\) y lo lleva a un vector \(f(v) \in U\).
  
  Es importante entender que tanto el dominio como el codominio son espacios vectoriales, por lo tanto, en ambos hay operaciones de suma y producto por escalar.
\end{tcolorbox}

En otras palabras, una función \(f: V \rightarrow U\) es lineal si cumple dos condiciones fundamentales para todo \(v, w \in V\) y \(t \in K\):
\begin{enumerate}
  \item Aditividad (respeta la suma):
    \[
     f(v + w) = f(v) + f(w)
    \]
  \item Homogeneidad (respeta el producto por escalares):
    \[
      f(t \cdot v) = t \cdot f(v)
    \]
\end{enumerate}

Estas dos propiedades garantizan que la función ``conserva'' la estructura lineal. En otras palabras, las combinaciones lineales en el espacio de partida se transforman en combinaciones lineales en el espacio de llegada, de la misma forma.

\paragraph{Propiedades de una función lineal}

Si \(f\) es una función lineal de \(V(K)\) en \(U(K)\) entonces:
\begin{enumerate}
  \item \(f\left(\vec{0}\right) = \vec{0}, \qquad \left(\vec{0} \in V,\vec{0} \in U\right)\)

  Demostración:\[
    f\left(\vec{0}\right) = f\left(0 \cdot v\right) = 0 \cdot f(v) = \vec{0}
  \]
  \item \(f(-v)=-f(v), \quad \forall v \in V\)
  
  Demostración: \[
    f(-v) = f(-1 \cdot v) = -1 \cdot f(v) = -f(v)
  \]
  \item \(f(v-w)=f(v) - f(w) \quad \forall v,w \in V\)
  
  Demostración: \[
    f(v-w) = f(v) + f(-w) = f(v) + (-f(w)) = f(v) - f(w)
  \]
\end{enumerate}

\begin{tcolorbox}[interesting_data, title=Nota conceptual]
  Las propiedades desarrolladas son sumamente importantes. Por ejemplo la primer propiedad distingue claramente las funciones lineales de otras funciones. Por ejemplo, si una función \(f\) no cumple que \(f\left(\vec{0}\right) = \vec{0}\), entonces automáticamente no es lineal.
\end{tcolorbox}

Ahora para todo par de escalares \(a,b \in K\) y para todo par de vectores \(v,w \in V\), obtenemos imponiendo ambas condiciones de linealidad:
\[
  F(av + bw) = a\cdot f(v) + b\cdot f(w) \qquad \text{se utiliza para definirlas}
\]
Con mayor generalidad, para escalares \(a_i \in K\) y vectores \(v_i \in V\), llegamos:
\[
  F(a_1 v_1 + a_2 v_2 + \cdots + a_n v_n) = a_1 f(v_1) + a_2 f(v_2) + \cdots + a_n f(v_n)
\]
\begin{tcolorbox}[myconclusion]
  Expresión utilizada para demostrar los teoremas.
\end{tcolorbox}

\begin{quote}
  \ejemplo{ Veamos algunos ejemplos}
  \begin{enumerate}[label=\alph*.]
    \item Sea \(A\) cualquier matriz \(m \times n\) sobre un cuerpo \(K\). Como se señaló previamente, \(A\) determina una aplicación \(F:K^n \rightarrow K^m\) mediante la asignación \(v \rightarrow Av\) (aquí los vectores de \(K^n\) y \(K^m\) se escriben como columnas). Afirmamos que \(F\) es lineal. Esto es debido a que, por propiedades de las matrices:
    \begin{align*}
      F(v + w) &= A(v+w) = Av + Aw = F(v) + F(w) \\
      F(kv) &= A(kv) = kAv = kF(v)
    \end{align*}
    donde \(v, w \in K^n\) y \(k \in K\).
    \item Sea \(F:\mathbb{R}^3 \rightarrow \mathbb{R}^3\) la aplicación <<proyección>> en el plano \(xy: F(x,y,z)= (x,y,0)\). Probemos que \(F\) es lineal. Sean \(v=(a,b,c)\) y \(w=(a',b',c')\). Entonces:
    \begin{align*}
      F(v+w) &= F(a+a', b+b', c+c') = (a+a', b+b',0) = \\
            &= (a,b,0) + (a',b',0) = F(v) + F(w)
    \end{align*}
    y para todo \(k \in \mathbb{R}\):
    \[
      F(kv) = F(ka,kb,kc) = (ka,kb,0) = k(a,b,0) = kF(v)
    \]
    O sea, \(F\) es lineal.
    \item Sea \(F: \mathbb{R}^2 \rightarrow \mathbb{R}^2\) la aplicación de <<traslación>> definida según \(F(x,y) = (x+1,y+2)\). Obsérvese que \(F(0)=(0,0)=(1,2)\neq 0\). Es decir, el vector cero no se aplica sobre el vector cero. Por consiguiente \(F\) no es lineal.
    \item Sea \(F:V\rightarrow U\) la aplicación que asigna \(0 \in U\) a todo \(v \in V\). Para todo par ded vectores \(v,w \in V\) y todo \(k \in K\) tenemos:
    \[
      F(v+w) = 0 = 0+0 = F(v) + F(w) \qquad \text{y} \qquad F(kv) = 0 = k0 = kF(v)
    \]
    Así \(F\) es lineal. Llamamos a \(F\) la \textit{aplicación cero} y la denotaremos normalmente por \(0\).
    \item Consideremos la aplicación identidad \(I:V\rightarrow V\) que aplica cada \(v\in V\) en si mismo. Para todo par de vectores \(v, w \in V\) y todo par de escalares \(a,b \in K\):
    \[
      I(av+bw)= av + bw = aI(v) + bI(w)
    \]
    Así \(I\) es lineal.
    \item Sea \(V\) el espacio vectorial de los polinomios en la variable \(t\) sobre el cuerpo real \(\mathbb{R}\). La aplicación derivada \(\mathbf{D}:V\rightarrow V\) y la aplicación integral \(\mathbf{J}:V\rightarrow \mathbb{R}\), son lineales. La razón es que, según se demuestra en el cálculo, para todo par de vectores \(u,v \in V\) y todo \(k\in \mathbb{R}\):
    \[
      \frac{d(u+v)}{dt} = \frac{du}{dt} + \frac{dv}{dt} \qquad \text{y} \qquad \frac{ku}{dt} = k\frac{du}{dt}
    \]
  \end{enumerate}
 \end{quote}

\subsubsection{Clasificación de las funciones lineales}

Se tiene una función lineal \(f\) de \(V(K)\) en \(W(K)\):
\begin{itemize}
  \item Si \(f\) es inyectiva recibe el nombre de \textit{monomorfismo}
  \item Si \(f\) es suryectiva recibe el nombre de \textit{epimorfismo}
  \item Si \(f\) es biyectiva recibe el nombre de \textit{isomorfismo}
  \item Si en \(f \quad V=W\) recibe el nombre de \textit{endomorfismo}
  \item Si \(f\) es un endomorfismo y es biyectiva recibe el nombre de \textit{automorfismo} 
\end{itemize}

\subsubsection{Núcleo e imagen de una función lineal}

Sea \(f\) una función lineal de \(V(K)\) en \(U(K)\) \(\left(f:V\rightarrow U\right)\), asociados a esta función existen dos subconjuntos, uno del conjunto de partida y otro del conjunto de llegada, que reciben el nombre de \textit{núcleo} (o \textit{Ker}) de la función e imagen de la función, los definimos:
\[
\text{Ker } f = N(f) = \left\{ v \in V \mid f(v) = \vec{0}_U \right\}
\]
\begin{center}
  son todos los vectores de \(V\) que tienen como imagen al vector nulo de \(U\)
\end{center}
\[
\text{Im } f = \left\{u\in U \mid \exists v \in V ~\text{ tal que }~ f(v) = u \right\}
\]
o también en una forma más abreviada:
\[
\text{Im } f = \left\{f(v) \mid v \in V\right\}
\]
\begin{center}
  son todos los vectores de \(U\) que son imagen de algún vector \(v \in V\)
\end{center}

Los subconjuntos \(N(f)\) e \(\text{Im } f\) son subespacios de \(V\) y de \(U\) respectivamente.

\begin{quote}
  \ejemplo{ Consideremos la siguiente función lineal:}
  \[
    f(x) = 3x
  \]
  Esta es una función lineal \(f:\mathbb{R} \rightarrow \mathbb{R}\).
  \begin{itemize}
    \item \textbf{Núcleo:}
    \[
      \text{Ker } f = \left\{x \in \mathbb{R} \mid f(x) = 0\right\} = \{0\}
    \]
    \item \textbf{Imagen:}
    \[
      \text{Im } f = \left\{f(x) \mid x \in \mathbb{R}\right\} = \left\{3x \mid x \in \mathbb{R}\right\} = \mathbb{R}
    \]
  \end{itemize}
  Aquí el \textit{único} elemento que se anula es el \(0\), pero la imagen es \textit{todo} \(\mathbb{R}\).

  \vspace{5mm}

  \ejemplo{ Ahora consideremos la función trivial de \(\mathbb{R}\) a \(\mathbb{R}\):}
  \[
    f(x) = 0
  \]
  Esta función también es lineal, y \(f:\mathbb{R} \rightarrow \mathbb{R}\)
  \begin{itemize}
    \item \textbf{Núcleo:}
    \[
      \text{Ker } f = \left\{x \in \mathbb{R} \mid f(x) = 0\right\} = \mathbb{R}
    \]
    \item \textbf{Imagen:}
    \[
      \text{Im } f = \{0\}
    \]
  \end{itemize}
  En este caso, \textit{todo} el dominio se anula y la imagen es el conjunto reducido que contiene solo al cero.
\end{quote}

\begin{tcolorbox}[title=Resumen para fijar la idea]
  \begin{itemize}
    \item El \textbf{núcleo} es un subconjunto del \textbf{dominio} \(V\): son los vectores que van a parar al cero del codominio.
    \item La \textbf{imagen} es un subconjunto del codominio \(U\): son todos los valores posibles que puede tomar \(f(v)\)
  \end{itemize}

  \vspace{5mm}

  El núcleo responde a la pregunta: ``¿Qué vectores se anulan bajo \(f\)?''

  La imagen responde a: ``¿Qué vectores pueden obtenerse como salida de \(f\)''
\end{tcolorbox}

\subsubsection{Los subespacios Núcleo y Imagen de \(f\)}

\paragraph{El conjunto núcleo de \(f\) es un subespacio del dominio}

\textbf{Proposición}: Sea \(f: V \to U\) una aplicación lineal entre espacios vectoriales. Entonces, el conjunto núcleo de \(f\),
\[
\text{Ker } f = \{ v \in V \mid f(v) = 0 \}
\]
es un subespacio vectorial de \(V\).

\textit{Demostración}
\begin{quote}
  \begin{enumerate}
    \item \textbf{No vacío}:
    
    El vector nulo del dominio \(\vec{0}_V \in V\) cumple que:
      \[
      f(\vec{0}_V) = \vec{0}_U
      \]
    por lo tanto, \(\vec{0}_V \in \text{Ker } f\), y así \(\text{Ker } f \ne \emptyset\).
  
    \item \textbf{Cerrado bajo la suma}:
    
      Sean \(v_1, v_2 \in \text{Ker } f\). Entonces:
     \[
     f(v_1) = 0 \quad \text{y} \quad f(v_2) = 0
     \]
     Como \(f\) es lineal:
     \[
     f(v_1 + v_2) = f(v_1) + f(v_2) = 0 + 0 = 0
     \]
     Entonces \(v_1 + v_2 \in \text{Ker } f\).
  
    \item \textbf{Cerrado bajo el producto por escalares}:
     
      Sea \(t \in K\) y \(v \in \text{Ker } f\), es decir, \(f(v) = 0\). Entonces:
     \[
     f(t \cdot v) = t \cdot f(v) = t \cdot 0 = 0
     \]
     Por lo tanto, \(t \cdot v \in \text{Ker } f\).
  \end{enumerate}
  
  Conclusión: Se cumplen las tres condiciones necesarias para que \(\text{Ker } f\) sea un subespacio de \(V\).
  \(\blacksquare\)
\end{quote}

\paragraph{El conjunto imagen de \(f\) es un subespacio del codominio}


\textbf{Proposición}: Sea \(f: V \to U\) una aplicación lineal entre espacios vectoriales sobre un mismo cuerpo \(K\). Entonces el conjunto
\[
\text{Im } f = \{ u \in U \mid \exists v \in V \text{ tal que } f(v) = u \}
\]
es un subespacio vectorial de \(U\).

\textit{Demostración:}
\begin{quote}
  \begin{enumerate}
    \item \textbf{No vacío}:
      Como \(f\) es lineal, se cumple que:
      \[
      f(\vec{0}_V) = \vec{0}_U
      \]
      Entonces, \(\vec{0}_U \in \text{Im } f\), por lo tanto \(\text{Im } f \ne \emptyset\).

    \item \textbf{Cerrado bajo suma y producto por escalares} (en una sola propiedad):

      Sean \(u_1, u_2 \in \text{Im } f\).
      Por definición, existen \(v_1, v_2 \in V\) tales que:
      \[
      f(v_1) = u_1 \quad \text{y} \quad f(v_2) = u_2
      \]
      Sean \(a, b \in K\) escalares arbitrarios. Como \(f\) es lineal:
      \[
      f(a v_1 + b v_2) = a f(v_1) + b f(v_2) = a u_1 + b u_2
      \]
      Esto significa que \(a u_1 + b u_2 \in \text{Im } f\).
  \end{enumerate}

  Conclusión: La imagen de \(f\) cumple las condiciones necesarias para ser subespacio de \(U\): no es vacía, y es cerrada bajo combinaciones lineales.
  \(\blacksquare\)
\end{quote}

\begin{tcolorbox}[title=Observaciones]
  Note que no necesitamos comprobar ``cerrado bajo suma'' y ``cerrado bajo producto escalar'' por separado, ya que probar cerrado bajo combinaciones lineales es más general y suficiente.
\end{tcolorbox}


%%%% A partir de aquí he copiado todo tal y como sale en el apunte de artal. Luego lo modificaré.

Ejemplos
\label{ej:ejemplos_a_revisar}
\begin{enumerate}[label=\alph*.]
  \item Sea \(F:\mathbb{R}^3 \rightarrow \mathbb{R}^3\) la aplicación proyección sobre el plano \(xy\). Esto es,
  \[
    F(x,y,z)=(x,y,0)
  \]
  (Véase la figura {insertar figura aquí}) Claramente la imagen de \(F\) es todo el plano \(xy\). Es decir, 
  \[
    \text{Im } F = \left\{(a,b,c):c\in \mathbb{R}\right\}
  \]
  Nótese que el núcleo de \(F\) es el eje \(z\). O sea,
  \[
    \text{Ker } F = \left\{(0,0,c):c\in \mathbb{R}\right\}
  \]
  ya que estos puntos y solamente éstos se aplican en el vector cero \(0=(0,0,0)\).
  \item Sean \(V\) el espacio vectorial de los polinomios sobre \(\mathbb{R}\) y \(\mathbf{T}:V\rightarrow V\) el operador tercera derivada, esto es,
  \[
    \mathbf{T}\left[f(t)\right] = \frac{d^3 f}{dt^3}
  \]
  A veces se emplea la notación \(\mathbf{T}=\mathbf{D}^3\), donde \(\mathbf{D}\) es la aplicación derivada. Tendremos:
  \[
    \text{Ker } T = \left\{\text{polinomios de grado} \leq 2\right\}
  \]
  porque \(\mathbf{T}(at^2+bt+c)=0\) pero \(\mathbf{T}(t^n)\neq 0\) para \(n > 3\). Por otra parte
  \[
    \text{Im } T = V
  \]
  puesto que todo polinomio en \(V\) es la tercera derivada de algún polinomio.
\end{enumerate}

\paragraph{Teorema de la imagen de un sistema de generadores}

Supongamos ahora que los vectores \(v_1, \cdots, v_n\) generan \(V\) y que \(F:V\rightarrow U\) es lineal. Probemos que los vectores \(F(v_1),\cdots, F(v_n)\in U\) generan \(\text{Im } F\); en ese caso, \(F(v)=u\) para algún vector \(v\in V\). Como los \(v_i\) generan \(V\) y \(v \in V\), existen escalares \(a_1, \cdots, a_n\) tales que:
\[
  v = a_1 v_1 + a_2 v_2 + \cdots + a_n v_n
\]  
En consecuencia:
\[
  u = F(v) = F(a_1 v_1 + a_2 v_2 + \cdots + a_n v_n) = a_1 F(v_1) + a_2 F(v_2) + \cdots + a_n F(v_n)
\]
y por lo tanto los vectores \(F(v_1), \cdots , F(v_n)\) generan \(\text{Im } F\).

Establezcamos formalmente este útil resultado.

Supongamos que \(v_1, v_2, \cdots, v_n\) generan un espacio vectorial \(V\) y que \(F:V\rightarrow U\) es lineal.

Entonces \(F(v_1), F(v_2), \cdots, F(v_n)\) generan \(\text{Im } F\).

\subsection{Rango y nulidad de una aplicación lineal}

Hasta aquí no hemos relacionado la noción de dimensión con la de aplicación lineal \(F : V \rightarrow U\). En pos casos enq ue \(V\) es de dimensión finita, tenemos la relación fundamental que sigue:

\teorema
Sea \(V\) de dimensión finita y sea \(F:V\rightarrow U\) una aplicación lineal. Entonces:
\begin{equation}
  \dim V = \dim( \text{Ker } F) + \dim(\text{Im } F)
  \label{eq:teorema_dimensión_nucleo_imagen}  
\end{equation}

El teorema proporciona, la siguiente fórmula para \(F\) cuando \(V\) tiene dimensión finita:

Es decir, la suma de las dimensiones de la imagen y el núcleo de una aplicación lineal es igual a la dimensión de su dominio. 

Se ve fácilmente que la ecuación (?9.1) se rige para la aplicación proyección de \(F\) del ejemplo (?9.5a - creo que es el primer ejemplo de \ref{ej:ejemplos_a_revisar}). Allí la imagen (plano \(xy\)) y el núcleo (eje \(z\)) de \(F\) tienen dimensiones 2 y 1, respectivamente, mientras que el dominio \(\mathbb{R}^3\) de \(F\) tiene dimensión 3.

\textbf{Nota}: Sea \(F:V\rightarrow U\) una aplicación lineal. Se define el \textit{rango} de \(F\) como la dimensión de su imagen y la \textit{nulidad} de \(F\) como la dimensión de su núcleo; esto es:
\[
  \text{rango } F = \dim(\text{Im } F) \qquad \text{y} \qquad \text{nulidad } F = \dim(\text{Ker } F)
\]
\[
\text{rango } F + \text{nulidad } F = \dim V
\]
Recordemos que el rango de una matriz se definió en origen como la dimensión de su espacio columna y de su espacio fila. Obsérvese que si ahora \(A\) como una aplicación lineal, las dos definiciones se corresponden, porque la imagen de \(A\) es precisamente su espacio columna.

Ejemplo

Sea \(F:\mathbb{R}^4 \rightarrow \mathbb{R}^3\) la aplicación lineal definida por:
\[
  F(x,y,s,t) = (x-y+s+t,x+2s-t,x+y+3s-3t)
\]
\begin{enumerate}[label=\alph*.]
  \item Encontramos una base y la dimensión de la imagen de \(F\).
  
  Hallamos la imagen de los vectores de la base usual de \(\mathbb{R}^4\):
  \begin{align*}
    F(1,0,0,0) = (1,1,1)\phantom{-} &\qquad F(0,0,1,0) = (1,2,3)\\
    F(0,1,0,0) = (-1,0,1) &\qquad F(0,0,0,1) = (1,-1,-3)
  \end{align*}
  Según la proposición (??) (revisar), los vectores imagen generan \(\text{Im } F\); por eso construimos la matriz cuyas filas son estos vectores imagen y la reducimos por filas a forma escalonada:
  \[
  \begin{pmatrix}
    1 & 1 & 1 \\
    -1 & 0 & 1 \\
    1 & 2 & 3 \\
    1 & -1 & -3
  \end{pmatrix} \sim \begin{bmatrix}
    1 & 1 & 1 \\
    0 & 1 & 2 \\
    0 & 1 & 2 \\
    0 & -2 & -4
  \end{bmatrix} \sim \begin{bmatrix}
    1 & 1 & 1 \\
    0 & 1 & 2 \\
    0 & 0 & 0 \\
    0 & 0 & 0
  \end{bmatrix}
  \]
  De este modo, \((1,1,1)\) y \((0,1,2)\) constituyen una base de \(\text{Im } F\), luego \(\dim(\text{Im } F) = 2\) o, en otras palabras, rango \(F=2\).
  \item Encontraremos una base y la dimensión del núcleo de la aplicación \(F\).

  Hacemos \(F(v)=0\), donde \(v=(x,y,z,t)\):
  \[
    F(x,y,s,t) = (x-y+s+t, x+2s-t, x+y+3s-3t) = (0,0,0)
  \]
  Igualamos entre si las componentes correspondientes para formar el siguiente sistema homogéneo cuyo espacio solución es \(\text{Ker } F\):
  \begin{align*}
    \begin{cases}
      x - y + \phantom{1}s + \phantom{1}t = 0 \\
      x \phantom{+ 1y } + 2s - \phantom{1}t = 0 \\
      x + y + 3s - 3t = 0
    \end{cases} \quad \text{o} \quad \begin{cases}
      x - y + \phantom{1}s + \phantom{1}t = 0 \\
      \phantom{x +} \phantom{1}y + \phantom{1}s - 2t = 0 \\
      \phantom{x +} 2y + 2s - 4t = 0
    \end{cases} \quad \text{o} \quad \begin{cases}
      x-y+s+\phantom{1}t=0\\
      \phantom{x+} y+s-2t = 0
    \end{cases}
  \end{align*}
  Las variables libres son \(s\) t \(t\), luego \(\dim(\text{Ker } F)=2\) o \(\text{nulidad } F =2\). Tomamos:
  \begin{itemize}
    \item \(s=-1,t=0\), para obtener la solución \((2,1,-1,0)\).
    \item \(s=0,t=1\), para obtener la solución \((1,2,0,1)\).
  \end{itemize}
  Así \((2,1,-1,0)\) y \(1,2,0,1\) constituyen una base de \(\text{Ker } F\). (Obsérvese que \(\text{rango } F + \text{nulidad } F = 2 + 2 = 4\)), que es la dimensión del dominio \(\mathbb{R}^4\) de \(F\).
\end{enumerate}

\subsection{Aplicación a los sistemas de ecuaciones lineales}

Consideremos un sistema de \(m\) ecuaciones lineales con \(n\) incógnitas sobre un cuerpo \(K\):
\begin{align*}
  a_{11} x_1 + a_{12} x_2 + \cdots + a_{1n} x_n &= b_1 \\
  a_{21} x_1 + a_{22} x_2 + \cdots + a_{2n} x_n &= b_2\\
  \vdots  \qquad \qquad ~ & \\
  a_{m1} x_1 + a_{m2} x_2 + \cdots + a_{mn} x_n &= b_m
\end{align*}
que es equivalente a la ecuación matricial
\[
  Ax=b
\]
donde \(A=(a_{ij})\) es la matriz de los coeficientes y \(x=(x_i)\) y \(b=(b_i)\) los vectores columna de las incógnitas y de las constantes, respectivamente. Ahora la matriz \(A\) puede verse también como la aplicación lineal 
\[
  A:K^n \rightarrow K^m
\]
Siendo así, la solución de la ecuación \(A:K^n \rightarrow K^m\). Más aún, la solución de la ecuación homogénea asociada \(Ax=0\) puede verse como el núcleo de la aplicación lineal \(A:K^n \rightarrow K^m\).

El teorema \ref{eq:teorema_dimensión_nucleo_imagen} relativo a aplicaciones lineales nos conduce a la relación:
\[
  \dim(\text{Ker } A = \dim(K^n) - \dim(\text{Im } A) = n - \text{rango } A)
\]
Pero \(n\) es exactamente el número de incógnitas en el sistema homogéneo \(Ax=0\).

\subsection{Aplicaciones lineales singulares y no singulares (isomorfismo)}

Se dice que una aplicación lineal \(F: V \rightarrow U\) es \textit{singular} si la imagen de algún vector no nulo bajo \(F\) es \(0\), es decir, si existe algún \(v \in V\) tal que \(v \neq 0\) pero \(F(v) = 0\). De esta manera, \(F:V \rightarrow U\) es \textit{no singular} si unicamente \(0 \in V\) se aplica en \(0 \in U\) o, equivalentemente, si su núcleo consiste solamente en el vector cero: \(\text{Ker } F = \left\{0\right\}\).

\teorema Supongamos que una aplicación lineal \(F: V \rightarrow U\) es no singular. En tal caso, la imagen de cualquier conjunto linealmente independiente es linealmente independiente.

\textbf{Demostración}: Supongamos que \(v_1, v_2, \cdots, v_n\) son vectores linealmente independientes en \(V\). Afirmamos que \(F(v_1),F(v_2),\cdots, F(v_n)\) son también linealmente independientes. Supongamos que:
\[
  a_1 F(v_1) + a_2 F(v_2) + \cdots + a_n F(v_n) = 0 \quad \text{donde } a_i ~ \text{pertenece a } K
\]
dado que \(F\) es lineal \(F(a_1 v_1 + a_2 v_2 + \cdots + a_n v_n) = 0\) por lo tanto,
\[
  \left(a_1 v_1 + a_2 v_2 + \cdots + a_n v_n\right) \in \text{Ker } F
\]
Pero \(F\) es no singular, esto es el \(\text{Ker }F = \{0\}\), luego \(a_1 v_1 + a_2 v_2 + \cdots + a_n v_n = 0\). Siendo linealmente independientes los \(v_i\), todos los \(a_i\) son cero. De acuerdo con esto los \(F(v_i)\) son linealmente independientes.

\paragraph{Isomorfismos}

Supongamos que una aplicación lineal \(F:V \rightarrow U\) es inyectiva. Entonces solo \(0 \in V\) puede aplicarse en \(0 \in U\) y por tanto \(F\) es no singular. El recíproco es cierto también, porque si \(F\) es no singular y \(F(v)=F(w)\), necesariamente \(F(v-w) = F(v) - F(w) = 0\), luego \(v-w = 0\) o \(v=w\). Así \(F(v) = F(w)\) implica \(v=w\), o sea, \(F\) es inyectiva. Hemos demostrado pues,

\textbf{Proposición}: Una aplicación lineal \(F:V\rightarrow U\) es inyectiva si y solo si es no singular.

Recordemos que una aplicación \(F:V\rightarrow U\) recibe el nombre de isomorfismo si es lineal y biyectiva, esto es, si es lineal, inyectiva y suprayectiva. Recordemos, asimismo, que se dice que un espacio vectorial \(V\) es isomorfo a otro \(U\), escrito \(V \simeq U\), si existe algún isomorfismo \(F:V\rightarrow U\).

En este caso es aplicable el siguiente teorema:
\begin{quote}
  Supongamos que \(V\) tiene dimensión finita y que \(F:V \rightarrow U \) es lineal. Entonces \(F\) es un isomorfismo si y solo si no es singular.

  Demostración:
  Si \(F\) es un isomorfismo, solo \(0\) se aplica en \(0\), de modo que \(F\) es no singular. Supongamos que \(F\) es no singular. Entonces \(\dim(\text{Ker } F) = 0\). Según el teorema (libro:9.5??), \(\dim(V) = \dim(\text{Ker } F) + \dim(\text{Im } F)\).
  Así pues, \(\dim(U) = \dim(V) = \dim(\text{Im } F)\). Como \(U\) tiene dimensión finita, \(\text{Im } F = U\) y por ende \(F\) es suprayectiva. Por lo tanto, \(F\) es simultáneamente inyectiva y suprayectiva, es decir \(F\) es un isomorfismo.
\end{quote}

\subsection{Operaciones con aplicaciones lineales}

Podemos combinar aplicaciones lineales de varias maneras consiguiendo nuevas aplicaciones lineales. Estas operaciones son de gran importancia y se utilizan a lo largo de todo el texto.

Supongamos que \(F:V \rightarrow U\) y \(G:V \rightarrow U\) son aplicaciones lineales entre espacios vectoriales sobre un cuerpo \(K\). Definimos la suma \(F+G\) como la aplicación de \(V\) en \(U\) que asigna \(F(v) + G(v)\) a \(v\in V\):
\[
  (F+G)(v) = F(v) + G (v)
\]
Asimismo, para cualquier escalar \(k \in K\), definimos el producto \(kF\) como la aplicación en \(U\) que asigna \(kF(v)\) a \(v \in V\):
\[
  (kF)(v) = kF (v)
\]
Probamos ahora que si \(F\) y \(G\) son lineales, también lo son \(F + G\) y \(kF\). Tenemos, para todo par de vectores \(v,w \in V\) y todo par de escalares \(a,b \in K\).
\begin{align*}
  (F+G)(av + bw) &= F(av + bw) + G(av + bw) = \\
                 &=aF(v) + bF(w) + aG(v) + bG(w) = \\
                 &=a(F(v) + G(v)) + b(F(w) + G(w)) = \\
                 &=a(F+G)(v) + b(F+G)(w) =\\[3pt]
  (kF)(av + bw) &= kF(av + bw) = k(aF(v)+bF(w))= \\
                &= akF(v) + bkF(w) = a(kF)(v) + b(kF)(w)
\end{align*}
Así pues \(F+G\) y \(kF\) son lineales.

Disponemos del siguiente teorema:
\teorema Sean \(V\) y \(U\) espacios vectoriales sobre un cuerpo \(K\). Ka colección de todas las aplicaciones lineales de \(V\) en \(U\), con las operaciones de suma y producto por un escalar anteriores, es un espacio vectorial sobre \(K\).

El espacio vectorial del teorema suele denotarse por:
\[
\text{Hom}(V,U)
\]
Aquí \(\text{Hom}\) viene de la palabra homomorfismo. En caso de ser \(V\) y \(U\) de dimensión finita, se aplica el teorema que enseguida enunciamos,

Supongamos \(\dim V = m\) y \(\dim U=n\). En tal caso, \(\dim(\text{Hom}(V,U)) = m\cdot n\).

\subsection{Composición de aplicaciones lineales}

Sean \(U, V\) y \(W\) espacios vectoriales sobre un mismo cuerpo \(K\) y \(F : V \rightarrow U\) y \(G: U \rightarrow W\) aplicaciones lineales:
\begin{center}
\begin{tikzpicture}[
  node distance=3cm,
  every node/.style={font=\large},
  set/.style={circle, draw, minimum size=1cm, align=center},
  arrow/.style={-Stealth, thick}
  ]

  \node[set] (V) {\(V\)};
  \node[set, right=of V] (U) {\(U\)};
  \node[set, right=of U] (W) {\(W\)};

  \draw[arrow] (V) -- node[above] {\(F\)} (U);
  \draw[arrow] (U) -- node[above] {\(G\)} (W);

\end{tikzpicture}
\end{center}

  \draw[arrow] (V) -- node[above] {\(f\)} (U);
Recordemos que la función compuesta \(G \circ F\) es la aplicación de \(V\) en \(W\) definida por \((G \circ F) (v) = G(F(v))\). Probemos que \(G\circ F\) es lineal siempre que lo sean \(F\) y \(G\). Tenemos, para todo par de vectores \(v,w \in V\) y todo par de escalares \(a,b \in K\),
\begin{align*}
  (G \circ F)(av + bw) &= G(F(av+bw)) = G(aF(v)+bF(w)) =\\
                       &=aG(F(v)) + bG(F(w)) = a(G \circ F)(v) + b(G \circ F)(w)
  
\end{align*}
Es decir \(G \circ F\) es lineal.

La composición de aplicaciones lineales está relacionada con la suma y el producto por un escalar como sigue:

Sean \(V, U\) y \(W\) espacios vectoriales sobre \(K\), \(F\) y \(F'\) aplicaciones lineales de \(V\) en \(U\), \(G\) y \(G'\) aplicaciones lineales de \(U\) en \(W\) y sea \(k \in K\). Entonces:
\begin{align*}
  G \circ (F+F') = G\circ F + G \circ F'\\
  (G+G') \circ F = G \circ F + G' \circ F \\
  k(G\circ F) = (kG) \circ F = G \circ (kF)
\end{align*}
  \newpage
  
  \section{Diagonalización}

Sea \(f\) un endomorfismo en el espacio vectorial \(V(K)\), y \(A\) su matriz asociada en cierta base \(B\)
\begin{align*}
  f:V&\rightarrow V \\
  u &\rightarrow f(u) \quad \text{tal que} \quad A\cdot [u] = [f(u)]
\end{align*}
Un \textbf{vector} \(u\) de \(V\), no nulo, es un \textbf{vector propio} o característico del endomorfismo \(f\), si y sólo si existe un escalar \(\lambda\) tal que:
\[
  f(u) = \lambda \cdot u \quad \text{lo que equivale a que} \quad A \cdot [u] = \lambda[u]
\]
\(u\) es un vector propio relativo al \textbf{escalar \(\lambda\)} que recibe el nombre de \textbf{valor propio}.

\textbf{Proposición}: Si \(u\) es un vector propio del endomorfismo \(f\) relativo al valor propio \(\lambda\), resulta que todo vector linealmente dependiente de \(u\) también es un vector propio correspondiente al mismo valor propio \(\lambda\).

\textit{Demostración}: \(f\) es un endomorfismo y \(u\) es un vector propio de él relativo al valor propio \(\lambda\), por lo tanto:
\[
  f(u) = \lambda \cdot u
\]
Busquemos ahora la imagen de \(v\) por \(f\):
\[
  f(v) = f(t \cdot u) = t \cdot f(u) = t \cdot (\lambda \cdot u) = \lambda \cdot (t\cdot u) = \lambda \cdot v
\]
Lo que indica que \(v\) también es un vector propio relativo al valor propio \(\lambda\): \(f(v) = \lambda \cdot v\)

Como \(u\) es no nulo, el conjunto de los vectores linealmente dependientes a él, \(v = t \cdot u\), determinan un subespacio de \(V\):
\[
  U = \left\{v \in V \mid \exists t \in K ~~ \text{tal que} ~~ v = t\cdot u\right\}
\]
Este \textbf{conjunto que contiene a los vectores propios} generados por el vector propio \(u\), es un subespacio vectorial de \(V\) que recibe el nombre de \textbf{espacio propio o característico} del endomorfismo \(f\), respecto al valor propio \(\lambda\).

Veamos ahora de qué forma se pueden determinar los valores, vectores y espacios propios:

\(f\) es un endomorfismo en \(V\) del cual conocemos su matriz asociada \(A\) y queremos hallar los vectores que hacen posible \(A \cdot [u] = \lambda [u]\) con \(A\in M_{n\times n}\)

Esta expresión la podemos escribir:
\[
  A \cdot [u] = \lambda \cdot I \cdot [u] \qquad I : \text{matriz identidad en } M_{n\times n}
\]
Por lo tanto:
\[
  A \cdot [u] - \lambda \cdot I \cdot [u] = 0 \qquad 0:\text{matriz nula}
\]
O bien:
\[
  (A-\lambda \cdot I) \cdot [u] = 0
\]
Para que el vector \(u\), \textbf{no nulo}, verifique esta última expresión para algún \(\lambda\), que es un sistema de ecuaciones homogéneo, debe verificarse que:
\[
\det(A-\lambda \cdot I) = 0
\]
Lo que equivale a decir que el sistema homogéneo admite solución no nula.

La expresión: \(\det(A-\lambda \cdot I) = 0\) se conoce como ecuación característica, los escalares \(\lambda\) que la satisfacen son los valores propios.

Sustituyendo los valores propios \(\lambda\) en el sistema de ecuaciones lineales homogéneo:
\[
  (A-\lambda \cdot I) \cdot [u] = 0
\]
se obtiene los vectores propios y espacios característicos respectivos.

\ejemplo{ Sea \(f\) un endomorfismo en \(\mathbb{R}^2\) cuya matriz asociada en la base canónica es:}
\[
  A = \begin{pmatrix}
    3 & 0 \\
    8 & -1
  \end{pmatrix}
\]
\begin{enumerate}
  \item Buscar valores propios a partir de la ecuación característica: \(\det(A-\lambda \cdot I) = 0\)
  \[
    A = \begin{pmatrix}
      3 & 0 \\ 8 & -1
    \end{pmatrix}, \quad \lambda \cdot I = \lambda \cdot \begin{pmatrix}
      1 & 0 \\ 0 & 1
    \end{pmatrix} = \begin{pmatrix}
      \lambda & 0 \\
      0 & \lambda
    \end{pmatrix}
  \]
  \[ 
    \quad A - \lambda \cdot I = \begin{pmatrix}
      3 - \lambda & 0 \\ 8 & -1-\lambda
    \end{pmatrix}
  \]
  \begin{align*}
    \det(A-\lambda \cdot I) &= \begin{vmatrix}
      3-\lambda & 0 \\ 8 & -1-\lambda
    \end{vmatrix} = (3-\lambda) \cdot (-1-\lambda) - 0 \\
    &= \lambda ^2 -2 \lambda  - 3 = 0 \implies \lambda_1 = 3 ~,~~ \lambda_2 = -1
  \end{align*}
  Resolviendo la ecuación característica obtuvimos los valores propios del endomorfismo dado:
  \[
    \lambda_1 = 3 \quad \text{y} \quad \lambda_2 = -1
  \]
  \item Buscar los vectores propios relativos a los valores propios hallados anteriormente, para ello planteamos el sistema homogéneo: \((A-\lambda \cdot I)\cdot [u] = 0\)
  \[
    (A-\lambda \cdot I)\cdot [u] = \begin{pmatrix}
      3-\lambda & 0 \\ 8 & -1-\lambda
    \end{pmatrix} \cdot \begin{pmatrix}
      x \\ y
    \end{pmatrix} = \begin{pmatrix}
      0 \\ 0
    \end{pmatrix}, ~~ \text{multiplicando matricialmente:}
  \]
  \[
    \begin{cases}
      (3-\lambda)x + 0y = 0 \\
      8x + (-1-\lambda)y = 0
    \end{cases} \quad \text{sustituimos } \lambda \text{ por los valores encontrados}
  \]
  Si \(\lambda = \lambda_1 = 3\):
  \[
    \begin{cases}
      (3-3)x + 0y = 0 \\
      8x + (-1-3)y = 0
    \end{cases} \qquad \implies y = 2x
  \]
  Por lo tanto los vectores \(u = (x ~~~ 2x)^T\) son solución del sistema para \(\lambda = \lambda_1 = 3\), de aquí que:
  \[
    U_1 = \left\{u \in \mathbb{R}^2 \mid u = \begin{pmatrix}
      x \\ 2x
    \end{pmatrix} ~~ \text{para algún} ~ x \in \mathbb{R}\right\}
  \]
  es el espacio propio relativo al valor propio \(\lambda_1 = 3\), un vector propio a este mismo escalar sería por ejemplo: \(u_1 = (1 ~~~ 2)^T\)

  Si \(\lambda = \lambda_2 = -1\):
  \[
    \begin{cases}
      (3-(-1))x + 0y = 0 \\
      8x + (-1-(-1))y = 0
    \end{cases} \quad \implies\quad \begin{array}{c}
      4x = 0 \\
      8x = 0
    \end{array} \quad \implies x=0
  \]
  Por lo tanto los vectores \(u=(0 ~~~ y)^T\) son solución del sistema para \(\lambda = \lambda_2 = -1\), de aquí que:
  \[
    U_2 = \left\{u \in \mathbb{R}^2 \mid u = \begin{pmatrix}
      0 \\ y
    \end{pmatrix} ~~ \text{para algún} ~ y \in \mathbb{R}\right\}
  \]
  es el espacio propio relativo al valor propio \(\lambda_2 = -1\), un vector propio relativo a este mismo escalar sería por ejemplo: \(u_2 = (0 ~~~ 1)^T\)
\end{enumerate}

\subsection{Diagonalización por vectores propios}

\subsubsection{Diagonalización}

Dada una matriz cuadrada \(A\) asociada a un endomorfismo \(f\) en \(V\):

\textbf{La matriz \textit{A} es diagonalizable} si existe una matriz \(P\) \textbf{invertible}, tal que se verifique que:
\[
  P^{-1} \cdot A \cdot P = D~ , ~~~ \text{siendo } D \text{ una matriz diagonal}
\]
\paragraph{Procedimiento para diagonalizar una matriz a partir de los vectores propios}

\begin{enumerate}[label=\(\arabic{*}^\circ\)]
  \item se buscan los vectores propios del endomorfismo \(f\)
  \item se forma con ellos una matriz \(P\)
  \item se analiza si \(P\) es invertible, en caso afirmativo se determina \(P^{-1}\)
  \item se realiza la multiplicación matricial \(P^{-1}\cdot A \cdot P = D\). Esta matriz \(D\) tiene en su diagonal principal los valores propios del endomorfismo \(f\). 
\end{enumerate}
Si \(P\) no es invertible, entonces es imposible diagonalizar a la matriz \(A\).

\ejemplo{ Retomemos el ejemplo anterior, teníamos un endomorfismo en \(\mathbb{R}^2\) cuya matriz asociada es}
\[
  A = \begin{pmatrix}
    3 & 0 \\ 8 & -1
  \end{pmatrix}
\]
dijimos que los vectores \(u_1=(1 ~~~ 2)^T\) y \(u_2 =(0 ~~~ 1)\) eran vectores propios relativos a los vectores propios \(\lambda_1 = 3\) y \(\lambda_2 = -1\)

Formamos con ellos una matriz \(P\):
\[
  P = \begin{pmatrix}
    1 & 0 \\
    2 & 1 
  \end{pmatrix}
\]
Analizamos si \(P\) es invertible: \(\det P = 1\), por lo tanto existe \(P^{-1}\):
\[
P^{-1} = \begin{pmatrix}
  1 & 0 \\
  -2 & 1
\end{pmatrix}
\]
Esto implica que \(A\) es diagonalizable, relizamos la multiplicación matricial:
\[
P^{-1} \cdot A \cdot P = D = \begin{pmatrix}
  3 & 0 \\
  0 & -1
\end{pmatrix}
\]
Observemos que en la diagonal están los valores propios del endomorfismo.

\teorema{Si la matriz \(A\), asociada al endomorfismo \(f\) de \(V_n\), tiene \(n\) valores propios diferentes, entonces \(A\) es diagonalizable.}

\teorema{Si el endomorfismo \(f\) de \(v_n\) tiene \(n\) vectores propios linealmente independientes, \(A\) es diagonalizable.}

\subsection{Matrices simétricas congruentes}

Se dice que una matriz \(M\) es congruente a otra \(A\) si existe una matriz no singular \(P\) tal que:
\[
  M = P^T \cdot A \cdot P
\]
La congruencia es una relación de equivalencia.

Si la matriz \(A\) es simétrica significa que \(A = A^T\), entonces podemos hacer:
\[
  M^T = (P^T \cdot A \cdot P)^T = P^T \cdot A \cdot (P^T)^T = P^T \cdot A \cdot P = M
\]
Por lo tanto, si \(A\) es simétrica, entonces \(M\) también lo es.

Las matrices diagonales son simétricas, se puede demostrar que únicamente matrices simétricas son congruentes a matrices diagonales.

\subsubsection{Diagonalización ortogonal}

Una matriz cuadrada y simétrica \(A\), asociada a un endomorfismo \(f\) en \(V\), es \textbf{ortogonalmente diagonalizable} si existe una matriz \(P\) tal que:
\[
  P^T \cdot A \cdot P = D \qquad \text{sea diagonal}
\]
\(A\) es diagonalizable ortogonalmente solamente si es \textbf{simétrica}, es decir, que \(A^T=A\)

\paragraph{Procedimiento para diagonalizar ortogonalmente una matriz simétrica a partir de los vectores propios}

\begin{enumerate}[label=\(\arabic{*}^\circ\)]
  \item Se hallan los valores propios relativos al endomorfismo \(f\)
  \item Se determinan vectores propios correspondientes a los valores propios
  \item Se ortonormalizan dichos vectores propios
  \item Se forma con los vectores ya ortonormalizados una matriz \(P\) y se realiza la multiplicación matricial \(P^T \cdot A \cdot P\)
\end{enumerate}
La matriz que se obtiene es diagonal.

\ejemplo

Sea \(A\) una matriz asociada a un endomorfismo en \(\mathbb{R}^2\), tal que \(A^T = A\), definida a continuación
\[
  A = \begin{pmatrix}
    3 & 1 \\
    1 & 3
  \end{pmatrix}
\]
Buscamos los valores propios a partir de la ecuación característica: \(\det (A - \lambda \cdot I) = 0\)
\begin{align*}
A - \lambda \cdot I = \begin{pmatrix}
  3-\lambda & 1 \\
  1 & 3-\lambda
\end{pmatrix} \quad \implies \det (A-\lambda \cdot I) &= (3-\lambda)(3-\lambda)-1\\
&= \lambda^2 -6\lambda + 8 = 0
\end{align*}
Los valores propios son: \(\lambda_1 = 4\) y \(\lambda_2=2\)

Buscamos vectores propios a relativos a esos valores propios:
\begin{align*}
  (A-\lambda \cdot I) \cdot [u] &= [0] \\
  \begin{pmatrix}
    3-\lambda & 1 \\
    1 & 3-\lambda
  \end{pmatrix} \cdot \begin{pmatrix}
    x\\ y
  \end{pmatrix} &= \begin{pmatrix}
    0 \\ 0
  \end{pmatrix} \quad \implies \quad \begin{cases}
    (3-\lambda)x + y =0\\
    x +(3-\lambda) \cdot y = 0
  \end{cases} 
\end{align*}
Resolviendo el sistema para \(\lambda_1 = 4\) resulta que un vector propio puede ser \(u_1 = (1 ~~~ 1)^T\) y para el valor \(\lambda_2 = 2\) un vector propio sería por ejemplo \(u_2 = (-1 ~~~ 1)^T\)

Tenemos que normalizar a estos vectores, que ya son ortogonales:
\begin{align*}
  \left\langle u_1, u_2\right\rangle = 0 \qquad \text{por lo cual son ortogonales} \\  
  \left\lVert u_1\right\rVert = \sqrt{2} ~~~ \text{y} ~~~ \left\lVert u_2\right\rVert = \sqrt{2} \quad \text{por lo tanto no son normados}  
\end{align*}
Entonces normalizamos \(u_1\) y \(u_2\)
\[
u'_1 = \frac{u_1}{\left\lVert u_1\right\rVert } = \begin{pmatrix}
  \frac{1}{\sqrt{2}} \\ \frac{1}{\sqrt{2}}
\end{pmatrix} \qquad \text{y} \qquad u'_2 = \frac{u_2}{\left\lVert u_2\right\rVert } = \begin{pmatrix}
  -\frac{1}{\sqrt{2}} \\ \frac{1}{\sqrt{2}}
\end{pmatrix}
\]
Armamos una nueva matriz con ellos:
\[
  P = \begin{pmatrix}
    \frac{1}{\sqrt{2}} & -\frac{1}{\sqrt{2}} \\
    \frac{1}{\sqrt{2}} & \frac{1}{\sqrt{2}}
  \end{pmatrix}
\]
Esta matriz es ortogonal, las matrices ortogonales verifican que su inversa es igual a su transpuesta.

Si hacemos:
\[
  P^T \cdot A \cdot P = \begin{pmatrix}
    4 & 0 \\
    0 & 2
  \end{pmatrix} \qquad \text{que es una matriz diagonal}
\]
En conclusión \(A\) es ortogonalmente diagonalizable.

\(P\) es una matriz ortogonal ya que \(P^T = P^{-1}\) (la inversa coincide con la transpuesta)

\subsection{Diagonalización y formas cuadráticas}

Las \textbf{Formas Cuadráticas} surgen de una diversidad de problemas relativos a distintos contextos y áreas del conocimiento.

Por ejemplo, una ecuación de la forma \(Ax^2+2Bxy+Cy^2+Dx+Ey+F = 0\) en la cual los coeficientes \(A,B,C\) no son nulos a la vez, se denomina \textbf{ecuación cuadrática}.

En ese ejemplo a la expresión \(F(x,y) = Ax^2 + 2Bxy + Cy^2\) se denomina \hl{forma cuadrática} asociada.

\ejemplo{ \(2x^2 + y^2 - 12x - 4y+18 =0\) es una \textbf{ecuación cuadrática} y tiene asociada la \textbf{forma cuadrática} \(F(x,y)=2x^2 + y^2\)}

Es decir, en la forma cuadrática se tienen \textbf{solamente} los términos de grado dos.

En este otro caso: \(Ax^2 + By^2+Cz^2+2Dxy+2Exz+2Fyz+Gx+Hy+Iz+J=0\), siendo al menos uno de los coeficientes \(A,B,C,D,E\) o \(F\) no nulo, la ecuación cuadrática tiene asociada una forma cuadrática:
\[
F(x,y,z) = Ax^2 + By^2 + C^2 + 2Dxy + Exz + 2Fyz
\]
\begin{tcolorbox}[myconclusion]
  \textbf{Observemos algo}: todos los términos son de grado dos, pero hay varios términos cruzados, es decir, hay términos donde hay un producto entre dos variables diferentes \((xy,xz,yz)\)
\end{tcolorbox}

\paragraph{Generalizando a \(n\) variables}

En general una forma cuadrática en las variables \(x_1,x_2,\cdots,x_n\) con las constantes \(C_{ij}\), no todas nulas, es un polinomio, donde cada término es de grado dos y lo podemos expresar:
\[
  F(x_1,x_2,\cdots,x_n) = \sum C_{ij} ~ x_i ~ x_j
\]
La forma cuadrática está diagonalizada si se puede expresar como sigue:
\[
  F(x_1,x_2,\cdots,x_n) = C_{11} x_1^2 + C_{22} x_2^2 + \cdots + C_{nn} x_n^2
\]
Es decir, que cada término de la forma cuadrática diagonalizada es de grado dos, pero \textbf{no hay ningún término cruzado}.

\paragraph{Representación matricial}

La forma cuadrática puede expresarse en forma matricial, para cierto \(X=(x_1,x_2,\cdots,x_n)^T\)
\[
  F(X) = X^T \cdot A \cdot X
\]
\begin{tcolorbox}[remember, title=Aclaración tipográfica]
  En este caso se usa \(X^T\) para representar la matriz transpuesta. Algunos textos optan por \(X^t\), que representa lo mismo.
\end{tcolorbox}
En un ejemplo dado anteriormente \(F(X) = F(x,y) = 2x^2 + y^2\) es:
\[
  \begin{pmatrix}
    x & y
  \end{pmatrix} \cdot \begin{pmatrix}
    2 & 0 \\
    0 & 1
  \end{pmatrix} \cdot \begin{pmatrix}
    x \\ y
  \end{pmatrix}
\]
Si consideramos la forma cuadrática:
\[
  F(x,y,z) = Ax^2 + By^2 + Cz^2 + 2Dxy + 2Exz + 2Fyz
\]
Se puede expresar en forma matricial \(F(X) = X^T \cdot A \cdot X \), en donde:
\[
X = \begin{pmatrix}
  x \\ y \\z
\end{pmatrix} \qquad A = \begin{pmatrix}
  A & D & E\\
  D & B & F\\
  E & F & C
\end{pmatrix}
\]
La matriz \(A\) es simétrica tal como la hemos definido y contiene los coeficientes de cada término de la forma cuadrática en cierto orden.

En la diagonal principal se pueden observar los coeficientes de aquellos términos que no son cruzados. Los demás elementos de la matriz resultan la mitad de los que corresponden a los términos cruzados.

Es importante establecer que la matriz \(A\) que representa a la forma cuadrática, tal como la acabamos de definir es siempre simétrica, sin embargo, pueden existir otras matrices no simétricas que también representan a la misma forma cuadrática, pero que no son de interés en este apartado.

\ejemplo{La forma cuadrática}
\[
F(x,y,z) = x^2 - 6 x y + 8 y^2 - 4 x z + 10 y z + 7 z^2
\]
puede representarse a través de la matriz 
\[
A = \begin{pmatrix}
  1 & -3 & -2 \\
  -3 & 8 & 5 \\
  -2 & 5 & 7
\end{pmatrix}
\]
o bien la matriz:
\[
M = \begin{pmatrix}
  1 & -6 & -4 \\
  0 & 8 & 10 \\
  0 & 0 & 7
\end{pmatrix}
\]
en ambos casos el  producto \(X^T \cdot A \cdot X\) resultará en la forma cuadrática dada.

\begin{tcolorbox}[myconclusion]
  Nos interesan las matrices simétricas que representan a las formas cuadráticas, ya que son las que podremos diagonalizar.
\end{tcolorbox}

\subsubsection{Diagonalización de una forma cuadrática}

Se tiene una forma cuadrática en las variables \(x_1,x_2,\cdots, x_n\), representada matricialmente
\[
  F(X) = X^T \cdot A \cdot X
\]
Es posible realizar un cambio de variables de tal forma que se considere \(X=P\cdot Y\) para \(y_1,y_2,\cdots,y_n\), y de este modo obtener
\[
  F(Y) = (P\cdot Y)^T \cdot A \cdot (P\cdot Y)
\]
Trabajando algebraicamente resulta
\[
  F(Y) = Y^T\cdot (P^T \cdot A \cdot P) \cdot Y
\]
Como \(A\) es una matriz simétrica tal que \(B = P^T \cdot A \cdot P\) donde \(B\) es una matriz diagonal congruente con \(A\) bajo congruencia, resulta ser la representación matricial de la forma cuadrática diagonalizada en las nuevas variables \(y_1,y_2,\cdot, y_n\)
\[
  F(Y) = Y^T \cdot B \cdot Y
\]
En conclusión, se dice que  la sustitución lineal \(X=P\cdot Y\) diagonaliza la forma cuadrática \(F(X)\) si la matriz representativa de \(F(Y)\) es una matriz diagonal.


La matriz \(B=P^T \cdot A \cdot P\) es congruente a la matriz \(A\) quien es la matriz simétrica, por lo cual hay un teorema que se puede demostrar que dice:
\begin{quote}
  \teorema{``Sea \(F(X) = X^T\cdot A \cdot X\) una forma cuadrática real, con \(A\) simétrica, entonces siempre existe una sustitución lineal \(X=P\cdot Y\) que diagonaliza a \(F\)''}
\end{quote}

\ejemplo{ \(F(x,y,z) = 6x^2-4xy-2xz+6y^2-2yz+5z^2\)}

La matriz \(A\) que la representa es
\[
  A = \begin{pmatrix}
    6 & -2 & -1 \\
    -2 & 6 & -1 \\
    1 & 1 & 5
  \end{pmatrix}
\]
La diagonalizamos, para ello buscamos los valores y vectores propios a partir de 
\[
\det(A-\lambda I) = 0 \quad \rightarrow \quad \begin{vmatrix}
  6-\lambda & -2 & -1 \\
  -2 & 6-\lambda & -1 \\
  -1 & -1 & 5-\lambda
\end{vmatrix}= 0 \quad \rightarrow \quad \begin{array}{c}
  \lambda_1 = 8 \\
  \lambda_2 = 6 \\
  \lambda_3 = 3  
\end{array}
\]
Resolviendo los respectivos sistemas de ecuaciones homogéneos se obtienen los vectores propios:
\[
  v_1 = \begin{pmatrix}
    -1 \\ 1 \\ 0
  \end{pmatrix} \qquad v_2 = \begin{pmatrix}
    -1 \\ -1 \\ 2
  \end{pmatrix} \qquad v_3 = \begin{pmatrix}
    1 \\ 1 \\ 1
  \end{pmatrix}
\]
Podemos calcular los productos escalares y comprobar que son ortogonales entre sí, pero no son normados, entonces los normalizamos:
\[
  v_1 = \begin{pmatrix}
    -\frac{1}{\sqrt{2}} \\ \frac{1}{\sqrt{2}} \\ 0
  \end{pmatrix} \qquad v_2 = \begin{pmatrix}
    -\frac{1}{\sqrt{6}} \\ -\frac{1}{\sqrt{6}} \\ \frac{2}{\sqrt{6}}
  \end{pmatrix} \qquad v_3 = \begin{pmatrix}
    \frac{1}{\sqrt{3}} \\ \frac{1}{\sqrt{3}} \\ \frac{1}{\sqrt{3}}
  \end{pmatrix}
\]
Ahora armamos la matriz \(P\):
\[
  P = \begin{pmatrix}
    -\frac{1}{\sqrt{2}} & -\frac{1}{\sqrt{6}} & \frac{1}{\sqrt{3}}\\ 
    \frac{1}{\sqrt{2}}  & -\frac{1}{\sqrt{6}} & \frac{1}{\sqrt{3}} \\ 
    0 & \frac{2}{\sqrt{6}} & \frac{1}{\sqrt{3}}
  \end{pmatrix}
\]
Realizando el producto matricial se obtiene la matriz diagonal congruente con \(A\):
\[
  P^T \cdot A \cdot P = D = \begin{pmatrix}
    8 & 0 & 0 \\
    0 & 6 & 0 \\
    0 & 0 & 3
  \end{pmatrix}
\]
que diagonaliza a la cuadrática dada tal que haciendo una sustitución lineal para variables \(a,b,c\) resulta:
\[
  F(a,b,c) = 8a^2 + 6b^2 + 3c^2
\]
\begin{tcolorbox}[mydanger]
  \textbf{Importante}: No siempre al diagonalizar una matriz simétrica se obtienen vectores ortogonales por lo que es suficiente con normalizarlos para obtener la cuadrática diagonalizada.
\end{tcolorbox}
Entonces si los vectores no fueran ortogonales hay que ortonormalizarlos. Existe un método que se denomina de \textit{Gran Schmidt} que permite ortonormalizar vectores, pero que no es objeto de estudio en esta materia.
  \newpage
  
  \printbibliography
\end{document}