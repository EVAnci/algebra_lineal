\documentclass[a4paper,12pt]{article}  % Clase de documento

\usepackage[
  a4paper,
  left=3cm,
  right=3cm,
  top=2.5cm,
  bottom=2.5cm
]{geometry}

% Paquetes esenciales
\usepackage[utf8]{inputenc}         % Codificación UTF-8
\usepackage[T1]{fontenc}            % Soporte para caracteres en PDF
\usepackage[main=spanish]{babel}    % Traducir a español títulos y otros elementos
\usepackage{amsmath, amssymb}       % Matemáticas avanzadas
\usepackage{amsthm}                 % Para el entorno proof
\usepackage{graphicx}               % Para incluir imágenes
\usepackage{wrapfig}                % Figuras que se envuelven en el texto
\usepackage{subcaption}             % Leyendas en figuras anidadas
\usepackage[most]{tcolorbox}        % Paquete para crear cajas de texto
\usepackage[table]{xcolor}          % Paquete para crear colores
\usepackage{hyperref}               % Enlaces internos y externos
\usepackage{array, colortbl}        % Tablas
\usepackage{soul}                   % Resaltado de texto
\usepackage{enumitem}               % Listas personalizadas (itemize, enumerate)
\usepackage{cancel}                 % Cancelar términos en ecuaciones
\usepackage{textcomp}               % Símbolos como el centavo (¢)
\usepackage{nicematrix}             % Matrices mejoradas
\usepackage{tabularx, float}        % Tablas personalizadas
\usepackage{makecell}

\usepackage{tikz}
\usetikzlibrary{shapes.geometric, arrows.meta, positioning, calc}
\usepackage{tikz-3dplot}
\usetikzlibrary{babel}

\usepackage{pgfplots}
\pgfplotsset{compat=1.18}
\usepgfplotslibrary{fillbetween}


\usepackage{fancyhdr}
\usepackage{lastpage} % opcional: para "página X de Y"
\pagestyle{fancy}
\fancyhf{} % limpia encabezados/pies
\renewcommand{\headrulewidth}{0pt} % sin línea horizontal (pon valor si quieres)
\fancyfoot[R]{\scriptsize\color{gray} Anci, Elio Valentino } 
\fancyfoot[C]{\thepage}
\setlength{\headheight}{8pt} % ajusta según tamaño de fuente; evita warnings


% Referencias
\usepackage[backend=biber,style=alphabetic]{biblatex}
\addbibresource{references.bib}

\input{styles/general_styles.tex}
\graphicspath{ {./images/} }

\begin{document}

  \begin{titlepage}
  \thispagestyle{empty}
  \centering
  \vspace*{2cm} % Espacio superior

  {\scshape\LARGE Universidad de Mendoza\par}
  \vspace{2cm}
  {\large Basado en el programa de la materia\par}
  \vspace{1.5cm}

  {\Huge\bfseries Algebra Lineal\par}
  \vspace{0.5cm}
  {\Large\itshape Cuadernillo de la Materia\par} % Subtítulo opcional

  \vspace{1.5cm}
  \includegraphics[width=0.45\textwidth]{images/cover.png} % Imagen centrada
  \vspace{2.5cm}

  {\Large Anci Elio Valentino\par}
  {\large Ingeniería en Computación\par}

  \vfill

  {\large \today\par}
\end{titlepage}

  \newpage

  \section*{Aclaraciones}

  Antes de empezar el cuadernillo, revisa las últimas páginas (recursos adicionales y bibliografía - sección \ref{sec:recursos_adicionales}). Allí encontrarás enlaces para colaborar, reportar errores y descargar la versión más reciente con las correcciones.

  \tableofcontents
  \newpage

  \section{Programación Lineal}

\subsection{Introducción}

La programación lineal es una técnica matemática de optimización que se utiliza para encontrar la mejor solución posible a un problema cuando se tienen recursos limitados y múltiples opciones para usar esos recursos.

¿Qué es exactamente? Es un método que permite maximizar o minimizar una \hl{función objetivo} (como ganancias, costos, tiempo, etc.) sujeta a un conjunto de \hl{restricciones lineales}. Tanto la función objetivo como las restricciones se expresan mediante ecuaciones o inecuaciones lineales, es decir, donde las variables aparecen elevadas solo a la primera potencia.

\noindent\textbf{Componentes principales:}
\begin{itemize}
  \item \textbf{Función objetivo}: Lo que queremos optimizar (maximizar ganancias o minimizar costos, por ejemplo)
  \item \textbf{Variables de decisión}: Las cantidades que podemos controlar y necesitamos determinar
  \item \textbf{Restricciones}: Las limitaciones del problema (presupuesto disponible, tiempo, materiales, capacidad de producción, etc.)
\end{itemize}

\subsection{Problemas de Optimización}

En un \textit{problema de optimización}, se busca maximizar o minimizar una cantidad específica llamada \hl{objetivo}, la cual depende de un número finito de variables de entrada. Estas variables pueden ser independientes entre sí o estar relacionadas a través de una o más restricciones.

\ejemplo\label{ej:ppl_lineal}: El siguiente problema:
\begin{align*}
  \text{max:} \quad         &z = 3x_1 + 2x_2 \\[3pt]
  \text{sujeto a:} \quad    &x_1 + x_2 \leq 10 \\
                            &x_1,x_2 \geq 0
\end{align*}
es un problema de optimización para el objetivo \(z\). Las variables de entrada son \(x_1\) y \(x_2\) y se denominan \textit{variables de decisión}, que deben cumplir dos restricciones: \(x_1 + x_2 \leq 10\) y \(x_1,x_2 \geq 0\).

El problema del ejemplo \ref{ej:ppl_lineal} pide maximizar \(z = 3x_1 + 2x_2\), es decir, encontrar los valores de \(x_1\) y \(x_2\) que maximicen \(z\) bajo las restricciones dadas.

En general, se llama PPL a un problema de optimización que se puede resolver mediante técnicas de programación lineal. Un problema de programación matemático es lineal si tanto la función objetivo \(z\) como las restricciones son lineales. Esto es:
\begin{align*}
  \text{optimizar:} \quad         &z = f(x_1, x_2, \ldots, x_n) \\[3pt]
  \text{sujeto a:} \quad    &g_i(x_1, x_2, \ldots, x_n) \thicksim  b_i \qquad \forall i \in \{1, 2, \ldots, m\} \\
\end{align*}
donde las restricciones son ecuaciones o desigualdades lineales, es decir emplean alguno de los símbolos \(\leq\), \(\geq\) o \(=\).

\subsection{Planteamiento de problemas}

Los problemas de optimización se plantean muy a menudo verbalmente, es decir, en palabras. Ya se verá un ejemplo a continuación (ejemplo \ref{ej:ppl_verbal}) de un problema de optimización planteado verbalmente. El procedimiento para la solución consiste en realizar un modelo del problema para poder resolverlo mediante técnicas de programación lineal.

\begin{tcolorbox}[interesting_data, title=¿Existe solo una forma de plantear problemas?]
  Existen múltiples formas de plantear un problema de optimización, por lo que los dos métodos que se proponen en este documento puedes tomarlos como sugerencias.
\end{tcolorbox}

\ejemplo\label{ej:ppl_verbal}: Una compañía de petróleos produce en sus refinerías gasóleo (\(G\)), gasolina sin plomo (\(P\)) y gasolina súper (\(S\)) a partir de dos tipos de crudos, \(C_1\) y \(C_2\). Las refinerías están dotadas de dos tipos de tecnologías. La tecnología nueva (\(T_n\)) utiliza en cada sesión de destilación \(7\) unidades de \(C_1\) y \(12\) de \(C_2\), para producir \(8\) unidades de \(G\), \(6\) de \(P\) y \(5\) de \(S\). Con la tecnología antigua (\(T_a\)), se obtienen en cada destilación \(10\) unidades de \(G\), \(7\) de \(P\) y \(4\) de \(S\), con un gasto de \(10\) unidades de \(C_1\) y \(8\) de \(C_2\).

Estudios de la demanda permiten estimar que para el próximo mes se deben producir al menos \(900\) unidades de \(G\), \(300\) de \(P\) y entre \(800\) y \(1700\) de \(S\). La disponibilidad de \(C_1\) es de \(1400\) unidades y de \(C_2\) de \(2000\) unidades. Los beneficios por unidad producida son:
\begin{table}[ht]
  \centering
  \begin{tabular}{|c|c|c|c|}
  \hline
  Gasolina & \textit{G} & \textit{P} & \textit{S} \\ \hline
  Beneficio/u & 4 & 6 & 7 \\ \hline
  \end{tabular}
\end{table}

La compañía desea conocer cómo utilizar ambos procesos de destilación, que se pueden utilizar total o parcialmente, y los crudos disponibles para que el beneficio sea el máximo.

\vspace{5mm}
\hrule
\vspace{5mm}

Este problema es un ejemplo típico de un PPL verbal. Hay una cantidad que se desea optimizar y un conjunto de restricciones que se deben cumplir. En este caso, se desea maximizar el beneficio y las restricciones son las cantidades de crudos disponibles y las cantidades de gasolina que se deben producir. Más adelante vamos a plantear el PPL asociado a este problema y vamos a resolver otros problemas. De momento con darnos a la idea de cómo es un problema verbal es suficiente.

\subsubsection{Métodos de planteamiento de PPLs}

Como dijimos anteriormente, existen múltiples formas de plantear un problema de optimización, por lo que los dos métodos que se proponen son el método recomendado por el libro Bronson (sacado de la cátedra) y el método que se propone en el libro de Alfaomega (\cite{PPL_Alfaomega}). 

Luego de describir los métodos de planteamiento de PPLs, el ejemplo \ref{ej:modelo_de_ppl_verbal} muestra el planteo del ejemplo \ref{ej:ppl_verbal} con el método de Alfaomega.

\begin{tcolorbox}[title=Método del libro Bronson]
  Luego de tener un enunciado verbal de un problema de optimización, se deben seguir los siguientes pasos:
  \begin{enumerate}
    \item Determínese la cantidad que se optimizará y exprésese como una función matemática. Hacer esto sirve para definir las variables de decisión.
    \item Identifíquese todos los requerimientos, restricciones y limitaciones estipulados, y exprésense matemáticamente. Estos requerimientos constituyen las restricciones.
    \item Exprésense todas aquellas condiciones ocultas. Tales condiciones no están estipuladas explícitamente, pero se hacen evidentes a partir de la situación física para la que se está planteando el modelo. Generalmente, involucran requerimientos de no negatividad o de ser enteras, para las variables de decisión.
  \end{enumerate}  
\end{tcolorbox}

\noindent Por otro lado, el método que se proponen en el libro PPL de Alfaomega es el siguiente:
\begin{tcolorbox}[title=Método del libro Alfaomega]
  \begin{enumerate}
    \item \textit{Reconocimiento de las variables de decisión}: las variables de decisión son las variables sobre las que el decisor tiene control y que se suponen continuas. Representan productos o bienes a producir, almacenar o vender, disponibilidad o adquisición de materias primas, etc.
    \item \textit{Identificación de las restricciones}: las restricciones representan las limitaciones o requisitos y definen la \textit{región factible} del problema. Representan el deseo de no exceder un valor específico (\(\leq\)), no descender por debajo de un valor particular (\(\geq\)) o ser igual a un valor particular (\(=\)).
    \item \textit{Obtención de la función objetivo}: la función objetivo es la que se quiere maximizar o minimizar. Representa el beneficio, renta, ganancias, costos, etc.
    \item \textit{Formulación del PPL}: cuando se tienen los tres pasos anteriores se puede formular el PPL. La solución de este PPL se denomina \textit{solución óptima} y se estudiará en la proxima sección.
  \end{enumerate}
\end{tcolorbox}

\begin{tcolorbox}[mydanger]
  \textbf{Cuidado:} El paso 2 de la metodología de Alfaomega incluye \textit{todas} las restricciones, incluso aquellas que no están en el enunciado verbal (como no negatividad por ejemplo).
\end{tcolorbox}

\vspace{5mm}
\hrule
\vspace{5mm}

\ejemplo\label{ej:modelo_de_ppl_verbal}: Tomemos el PPL verbal del ejemplo \ref{ej:ppl_verbal}. Propongo al lector que intente plantear el PPL asociado (no resolver) y luego comparar con la solución. En este caso vamos a modelar el PPL siguiendo el método recomendado por el libro de Alfaomega (segundo método).

\noindent\textbf{Paso 1: Reconocimiento de las variables de decisión}

Si bien la consigna puede resultar un poco enredada al principio, si prestamos atención, vemos que al final nos pregunta sobre \hl{cómo usar los procesos de destilación}. Esto nos da un indicio de que lo que se quiere es saber cuánto usar el proceso \(T_n\) y el proceso \(T_a\) para maximizar el beneficio.

Entonces, las variables sobre las que el vendedor tiene control son:
\begin{itemize}
  \item \(x_1 ~\rightarrow\) cantidad de sesiones de destilación usando el proceso \(T_n\)
  \item \(x_2 ~\rightarrow\) cantidad de sesiones de destilación usando el proceso \(T_a\)
\end{itemize}

\noindent\textbf{Paso 2: Identificación de las restricciones}

Tenemos restricciones debidas a las limitaciones en la disponibilidad de ambos tipos de crudos:
\begin{itemize}
  \item Para \(C_1\): \(7x_1 + 10x_2 \leq 1400\)
  \item Para \(C_2\): \(12x_1 + 8x_2 \leq 2000\)
\end{itemize}

\noindent También se tienen restricciones según las necesidades de refinado que se requieren:
\begin{itemize}
  \item Para Gasóleo (\(G\)): \(8x_1 + 10x_2 \geq 900\)
  \item Para Sin Plomo (\(P\)): \(6x_1 + 7x_2 \geq 300\)
  \item Para Super (\(S\)): \((5x_1 + 4x_2 \geq 800) \wedge (5x_1 + 4x_2 \leq 1700)\)
\end{itemize}

\noindent Además no debemos olvidar de tener en cuenta las restricciones implícitas, ya que es obvio que no se pueden realizar destilaciones negativas, se tiene:
\begin{itemize}
  \item \(x_1 \geq 0\)
  \item \(x_2 \geq 0\)
\end{itemize}

\noindent\textbf{Paso 3: Identificación de la función objetivo}

El beneficio será la suma de las cantidades de cada refinado que se vende, por lo que:
\begin{itemize}
  \item Cantidad de Gasóleo: \(8x_1 + 10x_2 ~ \rightarrow G \times \text{Precio: } 4(8x_1 + 10x_2) = 32x_1 + 40x_2\)
  \item Cantidad de Sin Plomo: \(6x_1 + 7x_2 ~ \rightarrow P \times \text{Precio: } 6(6x_1 + 7x_2) = 36x_1 + 42x_2\)
  \item Cantidad de Super: \(5x_1 + 4x_2 ~ \rightarrow S \times \text{Precio: } 7(5x_1 + 4x_2) = 35x_1 + 28x_2\) 
\end{itemize}

\noindent Por lo tanto, la función objetivo es:
\begin{align*}
  z &= 32x_1 + 40x_2 + 36x_1 + 42x_2 + 35x_1 + 28x_2 \\
    &= 103x_1 + 110x_2
\end{align*}

\noindent\textbf{Paso 4: Formulación del PPL}
\begin{align*}
  \text{maximizar:} \quad   &z = 103x_1 + 110x_2 \\[3pt]
  \text{sujeto a:} \quad    &7x_1 + 10x_2 \leq 1400 \\
                            &12x_1 + 8x_2 \leq 2000 \\
                            &8x_1 + 10x_2 \geq 900 \\
                            &6x_1 + 7x_2 \geq 300 \\
                            &5x_1 + 4x_2 \geq 800 \\
                            & 5x_1 + 4x_2 \leq 1700 \\
                            &x_1 \geq 0 \\
                            &x_2 \geq 0
\end{align*}

Es muy importante entender la metodología del planteo, ya que si el modelo realizado no es adecuado, el resultado obtenido cuando se resuelva el problema será, muy probablemente, incorrecto.

Al intentar plantear el problema del ejemplo \ref{ej:ppl_verbal} tal vez te hayan surgido algunas dudas, así que vamos a realizar un pequeño repaso sobre algunos detalles en el procedimiento del ejemplo \ref{ej:modelo_de_ppl_verbal}.

\noindent\textbf{¿Por qué se tomaron \(T_n\) y \(T_a\) como variables de decisión y no \(C_1\) y \(C_2\)?}

Tal vez podrías pensar que \(C_1\) y \(C_2\) pueden ser buenos candidatos a variables de decisión, ya que el vendedor puede controlar cuánta cantidad de cada tipo de curdo usa en cada tecnología. Entonces intentemos generar un modelo con \(C_1\) y \(C_2\) como variables de decisión.

Como ya sabemos que las variables de decisión son \(C_1\) y \(C_2\) entonces el paso 1 resulta:
\begin{itemize}
  \item \(x_1 ~\rightarrow\) cantidad de crudo \(C_1\) utilizado
  \item \(x_2 ~\rightarrow\) cantidad de crudo \(C_2\) utilizado
\end{itemize}

\noindent En el paso 2 tenemos las restricciones:
\begin{enumerate}
  \item Según la tecnología usada
  \item Según la cantidad de crudo disponible
  \item Según la cantidad de refinado que se debe producir dependiendo de la tecnología usada
  \item Restricción de no negatividad (no se pueden refinar cantidades negativas)
\end{enumerate}
Si bien de alguna forma rebuscada puede llegarse a un PPL equivalente, vemos que es mucho más complicado, ya que la tercer restricción está relacionando los crudos con el destilado y la tecnología, por lo que serán restricciones largas y complejas de modelar correctamente.

Es de mucha importancia tomarse el tiempo de analizar bien cuáles son las variables de decisión, ya que de ahí partirá todo el modelo. En caso de que usted elija utilizar el procedimiento de formulación de modelos del libro de Bronson, el paso 1 tiene este riesgo implícito, ya que si se genera una función objetivo cuyas variables de decisión no sean las adecuadas, el modelo puede terminar por ser incorrecto.

En este documento se ha usado el enfoque de Alfaomega ya que la idea de tener un paso específico para analizar el enunciado y tomar las variables de decisión es muy importante y no debe pasarse por alto.

\subsection{Resolución de Programas Lineales}

En esta materia se ven dos métodos para resolver PPLs:
\begin{itemize}
  \item \textbf{Método Gráfico}: se basa en la representación gráfica de las restricciones y de la solución. Tiene la ventaja de ser muy intuitivo y sencillo de entender, pero tiene la desventaja de ser poco eficiente y muy limitado en cuanto al número de variables.
  \item \textbf{Método Analítico}: Es más general que el método gráfico ya que no tiene la desventaja de límite de variables. Este método se conoce como el método Simplex y es muy eficiente y algorítmico. La desventaja es que requiere la transformación previa al \textit{formato estándar} añadiendo variables de holgura y/o variables artificiales.
\end{itemize}
Más adelante veremos en profundidad los dos métodos, así que no se preocupe por el vocabulario desconocido.

\subsubsection{Típos de solución}

Todo PPL puede tener alguna de las siguientes soluciones:
\begin{itemize}
  \item Solución optima única
  \item Solución optima múltiple
  \item Problema no acotado
  \item Problema infactible
  \item Rayo óptimo
\end{itemize}

A continuación vamos a profundizar un poco sobre qué significa cada tipo de solución, para que al momento de resolver PPLs sepa identificar qué tipo de solución tiene.

\paragraph{Solución óptima única}

\noindent Es el caso más común y deseable. Existe exactamente un punto en la región factible donde la función objetivo alcanza su valor óptimo (máximo o mínimo) y presenta las siguientes características:
\begin{itemize}
  \item La función objetivo tiene una pendiente única que no es paralela a ninguna cara de la región factible
  \item Gráficamente, la línea de la función objetivo ``toca'' la región factible en un solo vértice
  \item Matemáticamente, existe un único vector \(x^*\) que optimiza la función
\end{itemize}

\paragraph{Solución óptima múltiple (infinitas soluciones óptimas)}

Existen infinitos puntos que proporcionan el mismo valor óptimo de la función objetivo. Esto ocurre cuando:
\begin{itemize}
  \item La función objetivo es paralela a una de las caras (aristas) de la región factible
  \item Todos los puntos en esa arista proporcionan el mismo valor óptimo
\end{itemize}
Este tipo de soluciones tienen las siguientes características:
\begin{itemize}
  \item Cualquier combinación convexa de dos vértices óptimos adyacentes también es óptima
  \item La solución es un segmento de línea completo en la frontera de la región factible
\end{itemize}

Si el problema tiene una solución óptima múltiple, entonces no existe una preferencia por una solución sobre otra, ya que todas son óptimas. La respuesta que usted elija darle al problema será cualquiera de las soluciones óptimas, y será considerada correcta.

\paragraph{Problema no acotado}

La función objetivo puede crecer (o decrecer) indefinidamente sin violar ninguna restricción. Estos problemas presentan las siguientes características:
\begin{itemize}
  \item La región factible se extiende infinitamente en la dirección que mejora la función objetivo
  \item No existe un valor máximo (o mínimo) finito para la función objetivo
  \item Indica generalmente un error en la formulación del problema
\end{itemize}

\ejemplo : Maximizar \(z = x_1 + x_2\) sujeto solo a \(x_1 \geq 0, x_2 \geq 0\) (sin restricciones superiores).

\paragraph{Problema infactible}

No existe ningún punto que satisfaga simultáneamente todas las restricciones del problema. Este tipo de problemas pueden ocurrir en dos casos:
\begin{itemize}
  \item Las restricciones son contradictorias entre sí
  \item El conjunto de restricciones no tiene intersección común
\end{itemize}
Y presentan las siguientes características:
\begin{itemize}
  \item La región factible está vacía
  \item No hay solución posible al problema
  \item Matemáticamente: el conjunto \(\{x : Ax \leq b, x \geq 0\} = \emptyset\)
\end{itemize}

\ejemplo : Sea el siguiente PPL:
\begin{align*}
  \text{maximizar:} \quad   &z = x_1 + x_2 \\[3pt]
  \text{sujeto a:} \quad    &x_1 + x_2 \leq 5 \\
                            &x_1 + x_2 \geq 10 \\
                            &x_1, x_2 \geq 0
\end{align*}
Estas restricciones son imposibles de satisfacer simultáneamente.

\paragraph{Rayo óptimo}

Este es un caso especial del problema no acotado donde existe una dirección específica (rayo) a lo largo de la cual la función objetivo mejora indefinidamente. Este tipo de problemas presentan las siguientes características:
\begin{itemize}
  \item Existe un punto factible \(x_0\) y una dirección \(d\) tal que \(x_0 + \lambda d\) es factible para todo \(\lambda \geq 0\)
  \item La función objetivo mejora a lo largo de esta dirección: \(c^T d > 0\) (para maximización)
  \item El rayo representa la dirección de crecimiento ilimitado
\end{itemize}
La diferencia con la solución de tipo ``no acotado'' es que el rayo óptimo especifica exactamente la dirección del crecimiento infinito, mientras que ``no acotado'' es la conclusión general.

\paragraph{Identificación en el método simplex}

En el método simplex, la identificación de los tipos de soluciones se realiza de la siguiente manera:
\begin{itemize}
  \item \textbf{Única/Múltiple:} Se identifica en la tabla final del simplex
  \item \textbf{No acotado:} Aparece cuando una variable no básica tiene coeficientes no positivos en su columna
  \item \textbf{Infactible:} Se detecta cuando aparecen variables artificiales con valor positivo en la solución final
  \item \textbf{Rayo óptimo:} Se determina por la dirección correspondiente a la variable que causa el comportamiento no acotado
\end{itemize}

Cada tipo de solución tiene implicaciones importantes para la interpretación y aplicación práctica del modelo de optimización. De momento no se preocupe por entender el método simplex, ya que se verá en detalle en la siguiente sección.

\subsubsection{Método Gráfico}

Como se mencionó anteriormente, el método gráfico es muy intuitivo, y por ende, ideal para aprender los conceptos de PPLs. Para entender el método gráfico lo más sencillo es analizarlo con un ejemplo.

\ejemplo\label{ej:mtd_grfico}: Un expendio de carnes de la cuidad acostumbra a preparar carne para albondigón con una combinación de carne molida de res y carne molida de cerdo. La carne de res contiene \(80\%\) de carne y \(20\%\) de grasa, y le cuesta a la tienda \textcent \textit{80} por libra; la carne de cerdo contiene \(68\%\) de carne y \(32\%\) de grasa y cuesta \textcent \textit{60} por libra 

¿Qué cantidad de cada tipo de carne debe emplear la tienda en cada libra de albondigón, si se desea minimizar el costo y mantener el contenido de grasa no mayor al \(25\%\)?

\noindent\textbf{Resolución del ejemplo \ref{ej:mtd_grfico}:}
\begin{quote}
  \textbf{1. Identificación de variables de decisión:}
  Vemos que el resultado es un albondigón, y este se prepara con \(x_1\) libras de carne de res y \(x_2\) libras de carne de cerdo. Por lo que las variables de decisión son:
  \begin{itemize}
    \item \(x_1 ~\rightarrow\) cantidad de carne de res por libra de albondigón
    \item \(x_2 ~\rightarrow\) cantidad de carne de cerdo por libra de albondigón
  \end{itemize}

  \textbf{2. Identificación de restricciones:}
  Vemos que la cantidad total de grasa del albondigón debe ser menor o igual al 25\%, y que la cantidad de carne de res más la cantidad de carne de cerdo debe ser igual a 1 libra.

  Sumado a esto, también se tiene que las cantidades de carne de res y carne de cerdo no pueden ser negativas. Entonces, resulta:

  \begin{align*}
    0.2x_1 + 0.32x_2 &\leq 0.25 \\
    x_1 + x_2 &= 1 \\
    x_1, x_2 &\geq 0
  \end{align*}

  \textbf{3. Identificación de la función objetivo:}

  Vemos que la función objetivo es el costo total, que es la suma del costo de la carne de res y la carne de cerdo. Por lo que la función objetivo es:
  \[
    z = 80x_1 + 60x_2
  \]

  \textbf{4. Formulación del PPL:}

  Juntando la función objetivo con las restricciones, resulta en el siguiente PPL:
  \begin{align*}
    \text{minimizar:} \quad   &z = 80x_1 + 60x_2 \\[3pt]
    \text{sujeto a:} \quad    &0.2x_1 + 0.32x_2 \leq 0.25 \\
                              &x_1 + x_2 = 1 \\
                              &x_1, x_2 \geq 0
  \end{align*}

  \textbf{5. Resolución del PPL:}

  Para resolver el PPL usando el método gráfico, debemos graficar las restricciones, la región factible y la función objetivo. 

  \noindent ¿Cómo se realiza la gráfica de la función objetivo? 

  Para realizar la gráfica de la función objetivo debemos establecer un valor para el costo, y verificar si se encuentra en la región factible. Por ejemplo, nosotros vamos a elegir un valor inicial de \(z = 50\), y resolver la ecuación \(50 = 80x_1 + 60x_2\) para obtener la ecuación de la recta que representa la función objetivo.
  \begin{align*}
    50 &= 80x_1 + 60x_2 \\
    x_2 &= -\frac{4}{3}x_1 + \frac{5}{6}
  \end{align*}
  Y listo, ahora solo debemos graficar las restricciones e identificar la región factible, que la marcaremos de color \textcolor{cyan}{cyan}. 
    
  \begin{figure}[ht]
  \centering
  \begin{tikzpicture}
  \begin{axis}[
      xlabel={$x_1$},
      ylabel={$x_2$},
      xmin=0, xmax=1.5,
      ymin=0, ymax=1.2,
      grid=major,
      axis lines=center,
      legend pos=north east,
      width=10cm,
      height=8cm
  ]

  % Restricción 1: 0.2x1 + 0.32x2 <= 0.25
  \addplot[orange, thick, domain=0:1.25, name path=A] {-(4/7)*x + (5/7)};
  \addlegendentry{\(0.2x_1 + 0.32x_2 \leq 0.25\)}

  % Restricción 2: x1 + x2 <= 1
  \addplot[blue, thick, domain=0:1, name path=B] {1-x};
  \addlegendentry{\(x_1 + x_2 = 1\)}

  % Función objetivo (para un valor de z=50): 50 = 80x1 + 60x2
  \addplot[teal, thick, domain=0:1, name path=C] {-(4/3)*x + (5/6)};
  \addlegendentry{\(50 = 80x_1 + 60x_2\)}

  % Restricciones de no negatividad
  \addplot[black, thick, domain=0:10, name path=D] {0};
  \addplot[black, thick] coordinates {(0,0) (0,10)};

  % % Región factible para la restricción 1 sombreada
  % \addplot[orange!30, opacity=0.3] fill between[
  %     of=A and D,
  %     soft clip={domain=0:1.25}
  % ];

  % Región factible
  \addplot[cyan, ultra thick , domain=2/3:1, name path=E] {1-x};

  % Puntos vértices de la región factible
  \addplot[only marks, mark=*, mark size=3pt, color=black] 
  coordinates {(1,0) (2/3,1/3)};

  % Etiquetas de los vértices
  \node at (axis cs:1,0) [above right] {B (1,0)};
  \node at (axis cs:2/3,1/3) [above right] {A \(\left(\frac{2}{3},\frac{1}{3}\right)\)};

  \end{axis}
  \end{tikzpicture}
  \caption{Gráfico del PPL}
  \label{fig:ppl}
  \end{figure}

  \noindent En el gráfico se puede observar que la región factible es el segmento color cyan. En este caso, el valor de costo \(z=50\) no se encuentra dentro de la región factible. Buscar este valor a prueba y error no es lo más conveniente, entonces vamos a usar un teorema que nos asegura que la solución óptima se encuentra en un vértice del \textbf{polígono} de la región factible. Más adelante se verá la demostración de este teorema.

  Entonces, basado en el teorema, la solución óptima se encuentra en el vértice \textit{A} o en el vértice \textit{B}. A simple vista en el gráfico podemos ver que la función objetivo está más cerca del origen de coordenadas en el punto \textit{A} y por lo tanto el costo será menor, sin embargo veremos el costo en ambos puntos para mostrar que el costo aumenta a medida que la función objetivo se aleja del origen de coordenadas.

  \begin{align*}
    z_A &= 80\left(\frac{2}{3}\right) + 60\left(\frac{1}{3}\right) \\
    z_A &= \frac{220}{3} \approx \boxed{73.33} \\[5pt]
    z_B &= 80(1) + 60(0) = 80
  \end{align*}

  \textbf{6. Respuesta:} 

  En base a la resolución del PPL, se debe emplear \({2}/{3}\) de carne de res y \({1}/{3}\) de carne de cerdo por libra de albondigón para mantener el contenido de grasa no mayor al 25\%, y que el costo sea el menor posible, que es \$\(73.33\).

\end{quote}

\hrule
\vspace{5mm}

\paragraph{Resolución Bónus}

\ejemplo\label{ej:resolucion_ppl_petroleos_grafico}: Resolución del PPL del problema de la compañía de petróleos. 
\begin{quote}
  \textbf{1. Formulación del PPL:}

  A continuación se mostrará la resolución del ejercicio visto en el ejemplo \ref{ej:modelo_de_ppl_verbal}. Recordando el PPL ya modelado, se tiene:
  \begin{align*}
    \text{maximizar:} \quad   &z = 103x_1 + 110x_2 \\[3pt]
    \text{sujeto a:} \quad    &7x_1 + 10x_2 \leq 1400 \\
                              &12x_1 + 8x_2 \leq 2000 \\
                              &8x_1 + 10x_2 \geq 900 \\
                              &6x_1 + 7x_2 \geq 300 \\
                              &5x_1 + 4x_2 \geq 800 \\
                              & 5x_1 + 4x_2 \leq 1700 \\
                              &x_1 \geq 0 \\
                              &x_2 \geq 0
  \end{align*}

  \textbf{2. Resolución gráfica:}

  En este caso se tienen muchas más restricciones que en el ejemplo \ref{ej:mtd_grfico}, sin embargo siempre se terminará formando un polígono, donde los vertices indicarán las posibles soluciones óptimas. Para facilitar la resolución gráfica, escribiremos las restricciones en la forma de ecuaciones y reordenaremos los términos de cada ecuación para que queden en la forma de \(y = mx + b\).
  \begin{align*}
    7x_1 + 10x_2 = 1400 \quad &\rightarrow \quad x_2 = -\frac{7}{10}x_1 + 140 \\
    12x_1 + 8x_2 = 2000 \quad &\rightarrow \quad x_2 = -\frac{3}{2}x_1 + 250 \\
    8x_1 + 10x_2 = 900  \quad &\rightarrow \quad x_2 = -\frac{4}{5}x_1 + 90
  \end{align*}

  \begin{align*}
    6x_1 + 7x_2 = 300  \quad &\rightarrow \quad x_2 = -\frac{6}{7}x_1 + \frac{300}{7} \\
    5x_1 + 4x_2 = 800  \quad &\rightarrow \quad x_2 = -\frac{5}{4}x_1 + 200 \\
    5x_1 + 4x_2 = 1700 \quad &\rightarrow \quad x_2 = -\frac{5}{4}x_1 + 425 \\
  \end{align*}
  Y para las restricciones de no negatividad, sabemos que la región factible se tiene que encontrar el el primer cuadrante.

  Por último, antes de graficar, vamos a suponer algún valor de ganancia de la función objetivo, por ejemplo \(z = 17000\), y resolver la ecuación \(17000 = 103x_1 + 110x_2\) para obtener la ecuación de la recta que representa la función objetivo.
  \[
    17000 = 103x_1 + 110x_2 \quad \rightarrow \quad x_2 = -\frac{103}{110}x_1 + \frac{1700}{11}
  \]

  Entonces, el gráfico de la región factible y la función objetivo con la ganancia \(z = 17000\) es el siguiente:
  \begin{figure}[ht]
    \centering
    \begin{tikzpicture}
    \begin{axis}[
        xlabel={$x_1$},
        ylabel={$x_2$},
        xmin=0, xmax=300,
        ymin=0, ymax=200,
        grid=major,
        legend pos=north east,
        width=13cm,
        height=10cm,
        legend style={font=\footnotesize}
    ]
    
    % Restricción 1: 7x₁ + 10x₂ ≤ 1400 → x₂ ≤ 140 - 0.7x₁
    \addplot[blue, thick, dashed, domain=0:200] {140-0.7*x};
    \addlegendentry{$7x_1 + 10x_2 \leq 1400$}
    
    % Restricción 2: 12x₁ + 8x₂ ≤ 2000 → x₂ ≤ 250 - 1.5x₁
    \addplot[red, thick, dashed, domain=0:166.67] {250-1.5*x};
    \addlegendentry{$12x_1 + 8x_2 \leq 2000$}
    
    % Restricción 3: 8x₁ + 10x₂ ≥ 900 → x₂ ≥ 90 - 0.8x₁
    \addplot[green!70!black, thick, dashed, domain=0:112.5] {90-0.8*x};
    \addlegendentry{$8x_1 + 10x_2 \geq 900$}
    
    % Restricción 4: 6x₁ + 7x₂ ≥ 300 → x₂ ≥ 42.86 - 0.857x₁
    \addplot[orange, thick, dashed, domain=0:50] {(300/7)-(6/7)*x};
    \addlegendentry{$6x_1 + 7x_2 \geq 300$}
    
    % Restricción 5: 5x₁ + 4x₂ ≥ 800 → x₂ ≥ 200 - 1.25x₁
    \addplot[purple, thick, dashed, domain=0:160] {200-1.25*x};
    \addlegendentry{$5x_1 + 4x_2 \geq 800$}
    
    % Restricción 6: 5x₁ + 4x₂ ≤ 1700 → x₂ ≤ 425 - 1.25x₁
    \addplot[brown, thick, dashed, domain=0:300] {425-1.25*x};
    \addlegendentry{$5x_1 + 4x_2 \leq 1700$}
    
    % Región factible aproximada (necesitarías calcular los vértices exactos)
    % Esta es una aproximación visual de la región factible
    \fill[cyan!25, opacity=0.5] 
        (1200/11,700/11) -- (275/2,43.75) -- (500/3,0) -- (160,0) -- cycle;

    % Función objetivo
    \addplot[magenta, very thick, domain=0:165] {-(103/110)*x + (1700/11)};
    \addlegendentry{$z = 17000$}
    
    % Algunos puntos de intersección importantes (aproximados)
    \addplot[only marks, mark=*, mark size=3pt, color=black] 
    coordinates {(1200/11,700/11) (275/2,43.75) (500/3,0) (160,0)};
    
    % Etiquetas para algunos puntos clave
    \node at (axis cs:1200/11,700/11) [below left] {\footnotesize A \(\left(\frac{1200}{11},\frac{700}{11}\right)\)};
    \node at (axis cs:275/2,43.75) [above right] {\footnotesize B \(\left(\frac{275}{2},\frac{175}{4}\right)\)};
    \node at (axis cs:160,0) [above right] {\footnotesize C \(\left(160,0\right)\)};
    \node at (axis cs:500/3,0) [above left] {\footnotesize D \(\left(\frac{500}{3},0\right)\)};
    \end{axis}
    \end{tikzpicture}
    \caption{Región factible del PPL de ejemplo \ref{ej:modelo_de_ppl_verbal} en color \textcolor{cyan}{cyan}}
    \label{fig:ppl-maximizacion}
  \end{figure}

  Vemos que la la función objetivo \(z\) se encuentra dentro de la región factible, sin embargo, no es la solución óptima. Se puede ver que el punto más alejado del origen de coordenadas que puede tomar la función objetivo es el vértice \textit{B}. Por lo tanto, la solución óptima es el vértice \textit{B}, que es el punto \(\left(275/2,175/4\right)\). Entonces:
  \[
    z_B = 103\left(\frac{275}{2}\right) + 110\left(\frac{175}{4}\right) = \boxed{18975}
  \]

  \textbf{3. Respuesta:}

  En base a la resolución del PPL, se deben realizar \(137.5\) sesiones de destilación usando la tecnología nueva (\(T_n\)) y \(43.75\) sesiones de destilación usando la tecnología antigua (\(T_a\)), para obtener la mayor ganancia posible, que es \$\(18975\).

  \begin{tcolorbox}[remember]
    Aunque la respuesta pueda resultar confusa, recordemos que el enunciado decía que los procesos de destilación \textit{se pueden utilizar total o parcialmente}, por lo que la respuesta toma valores fraccionarios. Si se quisiera que la respuesta fuera un número entero, entonces se agregan las respectivas restricciones y se resuelve el PPL de la misma manera, tomando únicamente los puntos enteros de \(x_1\) y \(x_2\).
  \end{tcolorbox}

\end{quote}

\subsubsection{Introducción al método Simplex}

Como anteriormente mencionamos, el método simplex es un algoritmo que nos permite resolver problemas de optimización lineal. Aunque disponemos del método gráfico, cuando el número de \textit{variables de decisión} es grande (\(> 2\)), el método gráfico es impracticable, por lo que utilizar el método analítico se vuelve una necesidad.

Además, en general, la mayoría de los problemas de optimización suelen tener más de dos variables de decisión, por lo que la necesidad de generalizar la búsqueda de la solución optima es un tema de interés.

\paragraph{¿Qué hace el método Simplex?}

El método Simplex es un algoritmo iterativo que parte de una solución básica factible (es decir, un vértice del conjunto factible del problema) y se \hl{desplaza de un vértice a otro adyacente}, mejorando en cada paso el valor de la función objetivo, hasta que no se puede mejorar más, lo que significa que se ha llegado a la \textbf{solución óptima}.

Visualmente, imagina un poliedro (la región factible), donde cada vértice es una posible solución básica. El Simplex camina por los bordes del poliedro en dirección ascendente (en problemas de maximización) hasta llegar al punto más alto.

\paragraph{Estructura general del método}

El método se basa en representar el problema en \textbf{forma tabular}. Cada paso del algoritmo consiste en:
\begin{enumerate}
  \item \textbf{Iniciar en una solución básica factible inicial}: esta primera solución se obtiene en la \textit{Fase I} del Simplex. 
  \item \textbf{Seleccionar una variable entrante}: mediante la \textit{regla del criterio} se selecciona aquella que más mejora la función objetivo
  \item \textbf{Seleccionar una variable saliente}: para mantener la factibilidad (cumplir restricciones)
  \item \textbf{Actualizar la tabla (pivoteo)}: para generar una nueva solución básica
  \item \textbf{Repetir el proceso}: hasta que no haya más mejoras posibles
\end{enumerate} 
No se preocupe por entender el concepto de \textit{Fase I} y la \textit{regla del criterio}, ya que se explicarán en detalle más adelante.

\begin{tcolorbox}[interesting_data, title=Forma tabular]
  Que el método Simplex se base en representar el problema en ``forma tabular'' significa que todas las ecuaciones (la función objetivo y las restricciones) del problema de Programación Lineal se organizan en una matriz estructurada, una tabla, llamada ``tableau Simplex''
\end{tcolorbox}

Antes de ver el método Simplex en profundidad, vamos a ver algunos conceptos necesarios para comprender la formulación del método.

\vspace{3mm}

\paragraph{Forma matricial de un PPL}

Un PPL puede ser representado en forma matricial de la siguiente manera:
\begin{align*}
  \text{optimizar:} \quad   &z = C^TX \\[3pt]
  \text{sujeto a:} \quad    &AX \thicksim  B \\
\end{align*}
donde:
\begin{itemize}
  \item \(C\) es la matriz (o el vector) de coeficientes de la función objetivo,
  \item \(X\) es la matriz (o el vector) de variables de decisión,
  \item \(A\) es la matriz de coeficientes de las restricciones,
  \item \(B\) es la matriz (o el vector) de términos independientes de las restricciones.
\end{itemize}
\vspace{5pt}
\noindent De forma explícita se puede ver como:
\begin{align*}
  \text{optimizar:} \quad   &z = c_{1}x_{1} + c_{2}x_{2} + \cdots + c_{n}x_{n} \\[3pt]
  \text{sujeto a:} \quad    &a_{11}x_{1} + a_{12}x_{2} + \cdots + a_{1n}x_{n} \thicksim b_{1} \\[3pt]
                            &a_{21}x_{1} + a_{22}x_{2} + \cdots + a_{2n}x_{n} \thicksim b_{2} \\[3pt]
                            &\quad \vdots \\[3pt]
                            &a_{m1}x_{1} + a_{m2}x_{2} + \cdots + a_{mn}x_{n} \thicksim b_{m}
\end{align*}
donde:
\[
  C = \begin{pmatrix} c_1 \\ c_2 \\ \vdots \\ c_n \end{pmatrix},\ X = \begin{pmatrix} x_1 \\ x_2 \\ \vdots \\ x_n \end{pmatrix},\ A = \begin{pmatrix} a_{11} & a_{12} & \cdots & a_{1n} \\ a_{21} & a_{22} & \cdots & a_{2n} \\ \vdots & \vdots & \ddots & \vdots \\ a_{m1} & a_{m2} & \cdots & a_{mn} \end{pmatrix},\ B = \begin{pmatrix} b_1 \\ b_2 \\ \vdots \\ b_m \end{pmatrix}
\]

Esta forma es conveniente ya que permite representar el PPL de una manera más compacta y permite referirnos a una parte específica como la matriz de los coeficientes \(C\) para referirnos a los coeficientes de la función objetivo.

Con estos conceptos estamos listos para ver la forma de resolución de PPLs mediante el método Simplex.

\subsection{Requisitos para la resolución de PPL mediante el método Simplex}

El método simplex se mueve de un vértice del poliedro factible a un vértice adyacente que mejora el valor de la función objetivo. Repasando lo que vimos anteriormente cada iteración corresponde a:
\begin{itemize}
  \item \textbf{Evaluación del vértice actual}: Verificar si es óptimo mediante los costos reducidos
  \item \textbf{Selección de arista}: Elegir la arista (dirección) que más mejora la función objetivo
  \item \textbf{Movimiento}: Desplazarse a lo largo de la arista hasta el siguiente vértice
  \item \textbf{Actualización de la tabla}: El nuevo vértice se convierte en la nueva solución básica
\end{itemize}
Más adelante veremos un ejemplo simple para comprender mejor el método Simplex, pero antes necesitamos ver qué necesitamos para usar el método Simplex.

\paragraph{Condiciones para usar el método Simplex}

Entender con claridad \textit{cuáles son las condiciones necesarias para aplicar el método Simplex} es fundamental, ya que no todo problema lineal puede resolverse directamente con él sin alguna transformación previa. De forma resumida, para aplicar Simplex se necesita:
\begin{enumerate}
  \item Convertir el problema a forma estándar,
  \item Tener una solución básica factible inicial,
  \item Asegurar que el problema tiene solución factible y acotada.
\end{enumerate}
En general, para poder aplicar el método simplex, se expresa el PPL en su forma matricial, ya que permite realizar las transformaciones y operaciones de forma ordenada.

A continuación veremos cada una de las condiciones en detalle.

\subsubsection{Forma estándar del problema}
\label{sec:forma_estandar}

Un PPL está en su forma estándar si cumple las siguientes condiciones:
\begin{itemize}
  \item La función objetivo debe ser de maximización,
  \item Todas las variables deben ser mayores o iguales a cero (restricción de no negatividad).
  \item Todas las restricciones deben estar escritas como igualdades (no desigualdades).
\end{itemize}
Ya puede intuir, por los ejemplos vistos, que no todos los PPLs cumplen con estas condiciones, por lo que se requiere transformar el PPL a su forma estándar.

\paragraph{Transformación de un PPL en su forma estándar}

Como vimos anteriormente, un PPL está en su forma estándar si cumple las condiciones vistas. Para trabajar de forma ordenada vamos a llamar a cada condición de la siguiente manera:
\begin{enumerate}
  \item \textbf{Condición de maximización}: la función objetivo debe ser de maximización,
  \item \textbf{Condición de no negatividad}: todas las variables de decisión deben ser positivas: \(\forall x_i,\ x_i \geq 0;\ i = 1,2,\ldots,n\)
  \item \textbf{Condición no desigualdad}: todas las restricciones deben ser de la forma \(A_iX = b_i\). Para lograr esto se introducen \hl{variables de holgura} y \hl{variables superfluas}.
\end{enumerate}
En un momento se explicará el concepto de variables de holgura y variables superfluas.

Para tener un contexto previo, ver el siguiente video: \href{https://www.youtube.com/watch?v=6f5K3O7yUzU}{\texttt{forma estándar en programación lineal - YouTube}}, donde se muestran algunos ejemplos de cómo transformar un PPL a forma estándar. De igual manera, en este documento se mostrará el proceso de transformación a formato estándar de un PPL. 

\ejemplo\label{ej:transformacion_ppl_forma_estandar}: Transformación de un PPL a forma estándar.
\begin{quote}
  Consideremos el siguiente PPL (ejemplo del video recomendado):
  \begin{align*}
    \text{maximizar:} \quad   &z = -8x_1 + 16x_2 - 4x_3 \\[3pt]
    \text{sujeto a:} \quad    &5x_1 + 8x_3 \geq 100 \\
                              &-x_1 + 6x_2 \geq 100 \\
                              &-6x_2 + 4x_3 \leq 100 \\
                              &x_1 \geq 0 \\
                              &x_2 \text{ Libre} \\
                              &x_3 \leq0
  \end{align*}

  \noindent\textbf{1. Condición de maximización:}

  El PPL ya cumple con la condición de maximización, por lo que no se requiere ninguna transformación. Para aquellos PPL que buscan \textit{minimizar}, que no cumplen esta condición, al multiplicar la función objetivo por \(-1\) se puede transformar en una maximización. 

  \noindent\textbf{2. Condición de no negatividad:}

  En este caso, tenemos las tres posibilidades en las variables de decisión:
  \begin{itemize}
    \item Libre: \(x_2\)
    \item Positiva: \(x_1\)
    \item Negativa: \(x_3\)
  \end{itemize}

  La única variable que cumple con la condición de no negatividad es \(x_1\), por lo tanto esta variable se mantiene como está, es decir, \(x_1 \geq 0\). 

  Las variables que no cumplen con esta condición son \(x_2\) y \(x_3\), por lo que debemos operar sobre ellas para hacer cumplir la condición. Veamos el caso de \(x_3\) primero: 

  La variable \(x_3\) es negativa, para transformar la restricción a positiva debemos realizar un cambio de variable, tal que:
  \[
    x_3 = -y_3 \quad \rightarrow \quad y_3 \geq 0
  \]

  \noindent donde \(y_3\) es una variable que reemplaza a \(x_3\) y que es positiva. Esta variable debe cambiarse en todo el PPL, es decir, en la función objetivo y en las restricciones. 

  Para el caso de \(x_2\), la variable puede tomar cualquier valor, por lo que para que estrictamente cumpla la condición de no negatividad podemos realizar lo siguiente:
  \[
    x_2 = y_2 - y_2' \quad \rightarrow \quad y_2,\ y_2' \geq 0
  \]

  \noindent donde \(y_2\) y \(y_2'\) son variables que reemplazan a \(x_2\) y que cumplen con la condición de no negatividad. Estas variables deben reemplazar a \(x_2\) en todo el PPL, al igual que \(x_3\) se reemplaza por \(y_3\). Reemplazando las variables en el PPL, obtenemos:
  \begin{align*}
    \text{maximizar:} \quad   &z = -8x_1 + 16y_2 - 16y_2' + 4y_3 \\[3pt]
    \text{sujeto a:} \quad    &5x_1 - 8y_3 \geq 100 \\
                              &-x_1 + 6y_2 - 6y_2' \geq 100 \\
                              &-6y_2 + 6y_2' - 4y_3 \leq 100 \\
                              &x_1,\ y_2,\ y_2',\ y_3 \geq 0
  \end{align*}
  Listo, la condición de no negatividad se cumple, ahora veamos la tercera condición.

  \noindent\textbf{3. Condición de no desigualdad:}

  La tercera condición es que todas las restricciones deben ser de la forma \(Ax = b\). En este caso, ninguna de las restricciones cumple con esta condición, por lo que debemos operar sobre todas ellas. 

  Cuando analizamos la tercera condición, podemos encontrarnos con tres casos posibles:
  \begin{enumerate}
    \item Restricción de la forma \(A_i x \leq b_i\): agregar una \hl{\textit{variable de holgura}}
    \item Restricción de la forma \(A_i x \geq b_i\): agregar una \hl{\textit{variable superflua}}
    \item Restricción de la forma \(A_i x = b_i\): cumple con la condición de forma estándar, por lo que no se necesita ninguna variable adicional
  \end{enumerate}
  \noindent donde \(A_i\) es la i-ésima restricción y \(b_i\) es su término independiente.
  \begin{tcolorbox}[interesting_data, title=¿Qué significa variable de holgura?]
    La \textbf{variable de holgura} es aquella que se agrega a una restricción de la forma \(A_i x \leq b_i\) para transformarla en una igualdad, y representa la cantidad de recursos de los que se disponían pero no se usaron. 

    \textit{Ejemplo}: si se disponen de 10 horas totales para fabricar un producto, pero solo se usan 8 horas, entonces la variable de holgura representa las \(2\) horas que se disponían pero no se usaron. 
  \end{tcolorbox}

  \begin{tcolorbox}[interesting_data, title=¿Qué significa variable superflua?]
    Por otro lado, la \textbf{variable superflua} es aquella que se agrega a una restricción de la forma \(A_i x \geq b_i\) para transformarla en una igualdad, y representa la cantidad en que el lado izquierdo de la restricción \textbf{excede} el requisito mínimo establecido por el lado derecho.

    \textit{Ejemplo}: si se requieren como mínimo 54 toneladas de mineral, pero se extraen 60 toneladas, entonces la variable superflua vale \(60 - 54 = 6\) toneladas, y representa el exceso de toneladas de mineral que se extrae. 
  \end{tcolorbox}

  \noindent Volviendo al problema original, las restricciones que no cumplen la tercer condición son:
  \begin{align*}
    5x_1 - 8y_3 &\geq 100 \\
    -x_1 + 6y_2 - 6y_2' &\geq 100 \\
    -6y_2 + 6y_2' - 4y_3 &\leq 100
  \end{align*}
  Para la tercera restricción, tenemos el caso 1. Para este caso debemos agregar una \textit{variable de holgura}:
  \[
    -6y_2 + 6y_2' - 4y_3 \leq 100 \quad \rightarrow \quad -6y_2 + 6y_2' - 4y_3 + h_1 = 100
  \]
  Luego, para la primera y la segunda restricción tenemos el caso 2. Para este caso debemos agregar una \textit{variable superflua}:
  \begin{align*}
    5x_1 - 8y_3 \geq 100 \quad &\rightarrow \quad 5x_1 - 8y_3 - s_1 = 100 \\
    -x_1 + 6y_2 - 6y_2' \geq 100 \quad &\rightarrow \quad -x_1 + 6y_2 - 6y_2' - s_2 = 100
  \end{align*}

  Estas variables que se han agregado a las restricciones deben aparecer en la función objetivo, sin embargo, como no son variables de decisión, su coeficiente en la función objetivo debe ser cero. Por lo que el PPL transformado a forma estándar es:
  \begin{align*}
    \text{maximizar:} \quad   &z = -8x_1 + 16y_2 - 16y_2' + 4y_3 - 0s_1 - 0s_2 + 0h_1 \\[3pt]
    \text{sujeto a:} \quad    &5x_1 + 0y_2 - 0y_2' - 8y_3 - s_1 - 0s_2 + 0h_1 = 100 \\
                              &-x_1 + 6y_2 - 6y_2' + 0y_3 - 0s_1 - s_2 + 0h_1 = 100 \\
                              &0x_1 -6y_2 + 6y_2' - 4y_3 - 0s_1 - 0s_2 + h_1 = 100 \\
                              &x_1,\ y_2,\ y_2',\ y_3,\ h_1,\ s_1,\ s_2 \geq 0
  \end{align*}
  Si quitamos todos los términos con coeficiente nulo de las restricciones queda:
  \begin{align*}
    \text{maximizar:} \quad   &z = -8x_1 + 16y_2 - 16y_2' + 4y_3 - 0s_1 - 0s_2 + 0h_1 \\[3pt]
    \text{sujeto a:} \quad    &5x_1 - 8y_3 - s_1 = 100 \\
                              &-x_1 + 6y_2 - 6y_2' - s_2 = 100 \\
                              &-6y_2 + 6y_2' - 4y_3 + h_1 = 100 \\
                              &x_1,\ y_2,\ y_2',\ y_3,\ h_1,\ s_1,\ s_2 \geq 0
  \end{align*}
\end{quote}

\subsubsection{Existencia de una solución básica factible inicial (SBF)}

Este es el ``\textit{punto más delicado}''. Para comenzar el método Simplex, se necesita un punto inicial que cumpla todas las restricciones (factible) y sea una solución básica (es decir, con tantas variables básicas como ecuaciones, y el resto en cero).

Como ya vimos, si las restricciones son de tipo \(\leq\) y se agregan variables de \textit{holgura positiva}, entonces la \textbf{SBF} inicial está dada simplemente por poner en cero las variables originales y tomar las variables de holgura como solución. Por otro lado, si las restricciones son de tipo \(\geq\) o \(=\), o si el sistema \textbf{no} tiene una SBF evidente, entonces se debe recurrir a un método auxiliar como el método de la fase I del Simplex o el método de las dos fases.

\paragraph{¿Qué es una solución básica?}

En términos del álgebra lineal, una \textit{solución básica} es una solución del sistema lineal:
\[
AX = B
\]
Este sistema, representado de forma matricial, no es más que las restricciones del PPL en forma estándar. Si desea repasar el tema de sistemas de ecuaciones lineales puede consultar el capítulo \ref{sec:sel}.

\paragraph{¿Cómo se define una \textit{solución básica}?}

Para construir una \textit{solución básica}, se hace lo siguiente:
\begin{enumerate}
  \item Se eligen arbitrariamente \(m\) variables del vector \(X\), a las que se llama \textbf{variables básicas}.
  \item El resto de las \(n - m\) variables se fijan en cero; estas se llaman \textbf{variables no básicas}.
  \item Luego se resuelve el sistema lineal con esas \(m\) incógnitas (las básicas), usando las \(m\) ecuaciones.
\end{enumerate}
Esto es posible si las columnas de \(A\) correspondientes a las variables básicas son \textbf{linealmente independientes}, lo que permite resolver el sistema.

\paragraph{¿Y qué es una \textit{solución básica factible} (SBF)?}

Una solución básica factible es una solución básica que además satisface:
\[
x \geq 0
\]
Es decir, todas las variables (básicas y no básicas) deben ser \textbf{mayores o iguales a cero}. Esta condición es necesaria porque el método Simplex trabaja solamente en la región factible del espacio, que está limitada por las restricciones del problema.

\paragraph{Intuición geométrica}

En geometría, la región factible de un problema lineal es un \textit{poliedro} (en 2D, un polígono; en 3D, un poliedro tridimensional). Cada \textit{vértice} (o esquina) de esa región corresponde a una \textbf{solución básica factible}.

El método Simplex camina de un vértice al siguiente, buscando mejorar el valor de la función objetivo, hasta que ya no puede mejorar más.

Por eso se necesita comenzar desde un vértice: es decir, desde una \textbf{SBF}.

\ejemplo\label{ej:busqueda_sbf}{: Búsqueda de SBF}

Supón que tienes este sistema (ya convertido a forma estándar):
\begin{align*}
  x_1 + x_2 + x_3 &= 4 \\
  2x_1 + 3x_2 + x_4 &= 7 \\
  x_1, x_2, x_3, x_4 &\geq 0
 \end{align*}

Aquí hay 4 variables y 2 ecuaciones. Una solución básica se obtiene eligiendo, por ejemplo, las variables \(x_1\) y \(x_3\) como básicas, y fijando \(x_2 = x_4 = 0\).

Al resolver el sistema para \(x_1\) y \(x_3\), obtenemos \(x_1 = 3.5;\, x_3 = 0.5\). Como todas las variables son no negativas, entonces \(x_1 = 3.5;\, x_3 = 0.5\) es una \hl{solución básica factible}.

Ahora veamos qué pasa si elegimos las variables \(x_1\) y \(x_2\) como básicas, y fijamos \(x_3 = x_4 = 0\). Al resolver el sistema para \(x_1\) y \(x_2\) el resultado da \(x_1 = 5;\, x_2 = -1\). Como \(x_2\) es negativo, entonces \textbf{no es} una solución básica factible, simplemente es una solución básica al sistema.
\vspace{5mm}
\hrule

\subsubsection{Condición de factibilidad y acotación}

Esta es la última condición que debe cumplir el problema para que el Simplex sea aplicable.

El método Simplex solo es aplicable si el problema tiene una solución factible. Si no la tiene, el Simplex lo detectará (normalmente en la fase I).

El problema también debe ser acotado, es decir, si la función objetivo puede crecer indefinidamente sin violar las restricciones, el Simplex lo reportará como no acotado.

En términos simples, con el método simplex buscamos una solución factible \textbf{única}.

\subsection{Inicialización del método Simplex}

\subsubsection{Generación de solución factible inicial}

Como vimos, Simplex también requiere de una solución básica factible (SBF) inicial ya que el algoritmo se mueve de una solución factible básica a otra, mejorando el valor de la función objetivo en cada iteración. Anteriormente, en el ejemplo \ref{ej:busqueda_sbf} vimos, tras elegir arbitrariamente las variables básicas, que dimos con una SBF y luego, tras elegir nuevamente otras variables básicas, dimos con una solución básica pero no factible. Ese problema tenía pocas variables de decisión, por lo que si queremos encontrar una SBF podemos ir probando hasta dar con alguna. Sin embargo, si el problema tiene muchas variables de decisión, es difícil encontrar una SBF inicial, y probar todas las posibilidades no es lo más eficiente. Por lo que existen dos métodos para generar una SBF inicial:
\begin{itemize}
  \item \hl{Inspección directa}: si es fácil identificar un punto que satisface todas las restricciones, se puede usar directamente.
  \item \hl{Método de la Fase I del Simplex}: para aquellos casos donde no es fácil aplicar la inspección directa, se utiliza este método. Este método consiste en cuatro pasos:
  \begin{enumerate}
    \item Se construye un problema auxiliar donde se agregan \textit{variables artificiales} para convertir el problema original en uno con una solución factible evidente
    \item Se minimiza la suma de esas variables artificiales
    \item Si en el óptimo esta suma es cero, se ha encontrado una solución factible al problema original
    \item Si no es cero, el problema original no tiene solución factible (es incompatible)
  \end{enumerate}
\end{itemize}

\noindent El método de la inspección directa se vió en el ejemplo \ref{ej:busqueda_sbf}. Básicamente consiste en analizar las restricciones y verificar si se puede generar una solución básica que sea factible. Ahora veamos el método de la Fase I.

\subsubsection{Método de la Fase I: construcción del problema auxiliar}

Esta fase tiene como objetivo encontrar una solución básica factible inicial para un problema de programación lineal (PPL) cuando no está disponible de forma directa, aunque también puede usarse para aquellos problemas que tienen una SBF evidente.

Recordemos que el método Simplex parte de una SBF y se mueve de una a otra mejorando el valor de la función objetivo. Por eso, si no contamos con una SBF al comienzo, necesitamos construir un problema auxiliar que nos la proporcione. Esto es lo que hace la Fase I.

El objetivo de la Fase I es formular y resolver un problema auxiliar que:
\begin{itemize}
  \item Sea fácil de resolver
  \item Tenga una SBF evidente
  \item Su solución, si es factible, nos permita obtener una SBF \textbf{del problema original}
\end{itemize}

Este método consiste en cuatro pasos generales:
\begin{enumerate}

  \item \textbf{Convertir el PPL original a forma estándar}: Esto incluye que todas las restricciones sean igualdades (introduciendo variables de holgura, exceso, etc.) y todas las variables estén acotadas inferiormente por cero.

  \item \textbf{Identificar las restricciones problemáticas}: Por ejemplo, si una ecuación tiene una constante en el lado derecho negativa, o si las variables artificiales son necesarias para armar una base inicial.

  \item \textbf{Agregar variables artificiales}: Se introducen \hl{variables auxiliares (artificiales)} para poder construir una base inicial. Esto se hace cuando:
    \begin{itemize}
      \item Tenemos ecuaciones con signo \(\geq\) o \(=\), donde las variables de holgura o exceso no permiten construir directamente una base.
      \item La matriz de restricciones no contiene directamente columnas de la forma \(e_i\) necesarias para formar una base.
    \end{itemize}

  \item \textbf{Formular un problema auxiliar (problema de la Fase I)}: Se define una nueva función objetivo auxiliar: Minimizar la suma de todas las variables artificiales. Esta función objetivo representa el “costo” de alejarse del espacio factible del problema original.

  \item \textbf{Resolver el problema auxiliar con el método Simplex}: 
  \begin{itemize}
    \item Si la solución óptima del problema auxiliar tiene valor cero, significa que se ha encontrado una SBF para el problema original.
    \item Si la solución óptima tiene valor distinto de cero, el problema original no es factible (no existe ninguna solución que satisfaga todas las restricciones).
  \end{itemize}

  \item \textbf{Eliminar las variables artificiales}: Si la Fase I fue exitosa, se eliminan las variables artificiales (si aún están presentes en la base se deben reemplazar mediante pivoteos), y se continúa con la \textit{Fase II}, ahora con una SBF válida y la función objetivo original.
\end{enumerate}

\paragraph{¿Por qué funciona?}

El problema auxiliar busca un punto factible \hl{minimizando} el ``uso'' de las variables artificiales. Si puede lograr que todas ellas sean cero, significa que existe una combinación de las variables reales (originales y de holgura) que satisfacen todas las restricciones. Esa es justamente una solución básica factible. Veamos un ejemplo de la Fase I.

\ejemplo\label{ej:fase_1} Sea el siguiente problema de programación lineal:
\begin{quote}
  \begin{align*}
    \text{maximizar} \quad  &z = 3x_1 + 2x_2\\[3pt]
    \text{sujeto a:} \quad  &x_1 + x_2 = 4\\
                            &x_1 - x_2 \geq 2\\
                            &x_1, x_2 \geq 0
  \end{align*}
  Observamos que:
  \begin{itemize}
    \item La primera restricción ya está en forma de igualdad, pero no tiene una variable de holgura asociada que permita incluir una columna identidad.
    \item La segunda es una desigualdad \(\geq\), por lo tanto debemos restar una variable superflua y, además, agregar una variable artificial para poder construir una base inicial.
  \end{itemize}
  
  \subparagraph{Paso 1: Convertimos a forma estándar}
  \begin{enumerate}
    \item A la primera ecuación (que ya es igualdad), agregamos una variable artificial \(a_1\), porque no hay variable de holgura ni exceso que permita formar la base inicial.
    \item A la segunda desigualdad, restamos una variable de exceso \(s_2\), y agregamos una variable artificial \(a_2\).
  \end{enumerate}
  El sistema queda:
  \begin{align*}
    x_1 + x_2 + a_1 &= 4\\
    x_1 - x_2 - s_2 + a_2 &= 2
  \end{align*}  
  Con condiciones:
  \[
    x_1, x_2, s_2, a_1, a_2 \geq 0
  \]

  \subparagraph{Paso 2: Definimos el problema auxiliar}
  
  La función objetivo auxiliar es:
  \[
    \omega = a_1 + a_2
  \]
  Nuestro nuevo problema (auxiliar) es:
  \begin{align*}
    \text{minimizar} \quad  &\omega = a_1 + a_2\\[3pt]
    \text{sujeto a:} \quad  &x_1 + x_2 + a_1 = 4\\
                            &x_1 - x_2 - s_2 + a_2 = 2\\
                            &x_1, x_2, s_2, a_1, a_2 \geq 0
  \end{align*}

  Para las restricciones, la base inicial está formada por las variables artificiales \(a_1\) y \(a_2\), porque aparecen con coeficiente 1 y sólo en una ecuación (forman una matriz identidad).
  \begin{align*}
    \begin{pmatrix}
      1 & 1 & 0 & 1 & 0\\
      1 & -1 & -1 & 0 & 1
    \end{pmatrix}
    \begin{pmatrix}
      x_1\\
      x_2\\
      s_2\\
      a_1\\
      a_2
    \end{pmatrix}
    =
    \begin{pmatrix}
      4\\
      2
    \end{pmatrix}
  \end{align*}
  Esto significa que, si establecemos como variables básicas a \(a_1\) y \(a_2\), y las demás en cero, obtenemos una solución básica factible para el problema auxiliar, ya que \(a_1 = 4\) y \(a_2 = 2\).

  De este modo, ya tenemos una \hl{SBF inicial} para el problema auxiliar.
  
  \subparagraph{Paso 3: Aplicamos el método Simplex}
  
  A partir de esta situación, podríamos construir la tabla Simplex con \(a_1\) y \(a_2\) en la base, y comenzar a iterar para minimizar \(\omega\). Si al finalizar el valor óptimo de \(\omega\) es cero, habremos encontrado una combinación de \(x_1\), \(x_2\) y \(s_2\) que satisface las restricciones sin necesidad de variables artificiales, es decir, una solución básica factible del problema original.
  
  Si en cambio \(\omega > 0\), entonces el problema original no tiene solución factible.  
\end{quote}

Antes de continuar con la construcción de la tabla Simplex, es absolutamente conveniente (e incluso necesario) comprender los elementos que definen el problema auxiliar de la Fase I en profundidad, en particular:
\begin{itemize}
  \item El \textbf{vector de costos} (función objetivo auxiliar),
  \item La \textbf{base inicial} y su interpretación,
  \item El \textbf{costo reducido} o \textbf{función Z},
  \item Y cómo se arma correctamente la primera \textit{tabla Simplex}.
\end{itemize}

Entonces, siguiendo con el ejemplo \ref{ej:fase_1}, vamos a desarrollar estos tres puntos.

\paragraph{Contrucción del vector de costos \(c\) para la Fase I}

Recordando que la función objetivo del problema auxiliar es \(\omega = a_1 + a_2\), por lo tanto el \textbf{vector de costos} asociado a todas las variables (en el orden \(x_1, x_2, s_2, a_1, a_2\)) es: 
\[c = (0, 0, 0, 1, 1)\]
Ya que \(x_1\), \(x_2\) y \(s_2\) no participan en la función objetivo del problema auxiliar, su coeficiente es cero.

\paragraph{Elección de la base inicial}

Para poder aplicar el método Simplex, se necesita una \textit{base inicial} formada por un conjunto de variables básicas tales que el sistema:
\[
A_{\beta}X_{\beta} = B
\]
tenga una solución factible (es decir, \(X_\beta \geq 0\)).
\begin{tcolorbox}[remember, title=Aclaración]
  Cuando se usa \(A_\beta\) o \(X_\beta\) se refiere al vector de \textbf{variables básicas} y sus respectivos coeficientes. Eso no tiene nada que ver con la matriz \(B\).
\end{tcolorbox}

En el problema auxiliar hemos introducido \(a_1\) y \(a_2\) de modo que:
\begin{itemize}
  \item Cada una aparece \textbf{una sola vez} en una ecuación,
  \item Con coeficiente 1,
  \item Y no aparece en las demás ecuaciones.
\end{itemize}
Este tipo de estructura es ideal para que las variables artificiales sirvan como base inicial. Por lo tanto, tomamos como base inicial:
\[
  \beta = \{a_1, a_2\}
\]
Esto garantiza que la matriz base \(A_\beta\) es la \textit{matriz identidad} \(I_2\), y por lo tanto, \(X_\beta = B\) tiene solución inmediata. En forma matricial es:
\begin{align*}
  A_\beta =
  \begin{pmatrix}
    1 & 0\\
    0 & 1
  \end{pmatrix}
  \quad
  X_\beta =
  \begin{pmatrix}
    a_1\\
    a_2
  \end{pmatrix}
  \quad
  B =
  \begin{pmatrix}
    4\\
    2
  \end{pmatrix}
\end{align*}
Entonces:
\begin{align*}
  A_\beta X_\beta = B \quad \rightarrow \quad x_{a_1} = 4, ~~ x_{a_2} = 2
\end{align*}
Entonces ya tenemos una \textbf{SBF} inicial (aunque no es deseable por contener variables artificiales).

\paragraph{Cálculo de \(Z_j\) y \(Z_{j}-c_j\) (costos reducidos)}

Ahora preparamos la tabla Simplex. Para cada variable \(x_j\) en la fila correspondiente, se calcula:
\[
  Z_j = \sum_{i \in \beta} c_i \cdot a_{ij}
\]
donde:
\begin{itemize}
  \item \(\beta=\{a_1, a_2\}\),
  \item \(c_i\) son los costos de la función objetivo auxiliar (\(\omega = a_1 + a_2\)),
  \item \(a_{ij}\) son los coeficientes de la variable \(x_j\) en la fila correspondiente.
\end{itemize}
Luego se calcula el \textbf{costo reducido}:
\[
  Z_j - c_j
\]
La interpretación es:
\begin{itemize}
  \item Si \(Z_j - c_j > 0\), introducir la variable reducirá el valor de \(omega\) (recordar que estamos \textit{minimizando}).
  \item Si todos los \(Z_j - c_j \geq 0\), la solución es óptima para la Fase I.
\end{itemize}
Con esto completo, ahora \textbf{si} estamos listos para armar la primera tabla Simplex de la Fase I con total seguridad.

% Ampliación de los costos reducidos

% Excelente que lo señales, porque entender el **significado y cálculo de los costos reducidos** es clave para aplicar correctamente el método Simplex. Vamos a repasarlo de forma clara, con base en el ejemplo de Fase I que estamos utilizando. Primero daré una explicación general, luego la aplicaremos paso a paso.

% ---

% ### ¿Qué son los costos reducidos?

% En el método Simplex, el valor $Z_j - c_j$ (llamado **costo reducido** de la variable $x_j$) indica **cuánto cambia la función objetivo** si introducimos esa variable en la base (es decir, si la hacemos básica).

% En la **Fase I**, queremos **minimizar** la función auxiliar $w = a_1 + a_2$. Por eso:

% * Si $Z_j - c_j > 0$, introducir esa variable **aumenta** el valor de $w$ → no conviene.
% * Si $Z_j - c_j < 0$, introducirla **disminuye** el valor de $w$ → sí conviene.
% * Si todos los $Z_j - c_j \geq 0$, entonces ya estamos en el **mínimo** de $w$.

% ---

% ### ¿Cómo se calcula?

% Primero, definimos:

% * $B$: conjunto de variables **básicas** actuales. En la primera tabla son $a_1$ y $a_2$.
% * $c_B$: vector de **costos** asociados a las variables básicas (en Fase I: ambos son 1).
% * $A_B$: matriz de coeficientes de las variables básicas (es una matriz identidad).
% * $a_j$: columna de la variable $x_j$ en el sistema de restricciones.
% * $c_j$: costo asociado a la variable $x_j$ en la función objetivo auxiliar (es 0 para $x_1$ y $x_2$, 1 para $a_1$ y $a_2$).

% La fórmula del costo reducido es:

% $$
% Z_j - c_j = c_B^\top A_B^{-1} a_j - c_j
% $$

% En palabras: **el costo reducido se calcula multiplicando los costos de las básicas por la solución del sistema con la columna $a_j$, y restando el costo directo de $x_j$**.

% ---

% ### Aplicación al ejemplo

% Retomamos el problema auxiliar:

% $$
% \begin{aligned}
% x_1 + x_2 + a_1 &= 4 \quad \text{(F1)}\\
% 2x_1 + 3x_2 + a_2 &= 9 \quad \text{(F2)}
% \end{aligned}
% $$

% Orden de las variables: $x_1,\ x_2,\ a_1,\ a_2$

% Coeficientes:

% $$
% \begin{array}{c|cccc|c}
% \text{Base} & x_1 & x_2 & a_1 & a_2 & b \\
% \hline
% a_1 & 1 & 1 & 1 & 0 & 4 \\
% a_2 & 2 & 3 & 0 & 1 & 9 \\
% \end{array}
% $$

% Ahora, los costos $c_B$ son:

% $$
% c_B = \begin{bmatrix} 1 \\ 1 \end{bmatrix}, \quad \text{(porque la base es } a_1,\ a_2)
% $$

% Y como $A_B = I_2$, tenemos $A_B^{-1} = I_2$ también. Entonces para cada variable no básica (en este caso, $x_1$ y $x_2$):

% **Costo reducido para $x_1$:**

% La columna de $x_1$ es $a_{x_1} = \begin{bmatrix} 1 \\ 2 \end{bmatrix}$

% Entonces:

% $$
% Z_{x_1} = c_B^\top a_{x_1} = [1\ 1] \cdot \begin{bmatrix} 1 \\ 2 \end{bmatrix} = 1 \cdot 1 + 1 \cdot 2 = 3
% $$

% $$
% Z_{x_1} - c_{x_1} = 3 - 0 = 3
% $$

% **Costo reducido para $x_2$:**

% La columna de $x_2$ es $a_{x_2} = \begin{bmatrix} 1 \\ 3 \end{bmatrix}$

% Entonces:

% $$
% Z_{x_2} = c_B^\top a_{x_2} = [1\ 1] \cdot \begin{bmatrix} 1 \\ 3 \end{bmatrix} = 1 \cdot 1 + 1 \cdot 3 = 4
% $$

% $$
% Z_{x_2} - c_{x_2} = 4 - 0 = 4
% $$

% **Conclusión**:

% Ambos costos reducidos son positivos:

% * $Z_1 - c_1 = 3$,
% * $Z_2 - c_2 = 4$.

% Entonces introducir $x_1$ o $x_2$ aumentaría el valor de $w$. Como **estamos minimizando**, esto **no conviene**, por lo tanto la solución actual es **óptima para la Fase I**.

% ---

% ¿Deseas que revisemos ahora cómo interpretar esta información para decidir si pasamos a la Fase II? ¿O te gustaría que volvamos a repasar este concepto con una representación gráfica o una analogía más intuitiva?


\paragraph{Costos de Penalización}

La introducción de variables de holgura y superfluas no altera ni la naturaleza de las restricciones ni al objetivo. Por consiguiente, estas variables se incorporan a la función objetivo con coeficientes cero. Las variables artificiales, sin embargo, cambian la naturaleza de als restricciones. Ya que se agregan a un solo lado de una desigualdad, el nuevo sistema es equivalente al sistema anterior de restricciones sólo si las variables artificiales son cero. Para garantizar estas condiciones en la solución óptima las variables artificiales se introducen a la función objetivo con coeficientes muy grandes (si hay que maximizar la ganancia, una ganancia muy pequeña o negativa implica una gran penalización, por el contrario si hay que minimizar costos, un costo grande implica una gran penalización para esa variable). Estos coeficientes se denotan generalmente como \(+M\) (para minimización) o \(-M\) (para maximización).

Los coeficientes \(\pm M\) donde \(M\) se considera un número positivo muy grande, representan el severo costo de penalización que se impone a las variables artificiales cuando se activan en la solución óptima. 
  \newpage
  
  \section{Espacios vectoriales}

\subsection{Cuerpo matemático}

Un cuerpo (o campo) \(K\) es, a grandes rasgos, un conjunto no vacío provisto de dos operaciones binarias, la suma \((+)\) y el producto \((\cdot)\), que satisfacen los siguientes requisitos:

\noindent \textbf{1. Estructura aditiva}
\begin{itemize}
  \item La suma es asociativa y conmutativa.
  \item Existe un elemento neutro aditivo, denotado \(0\), tal que \(a+0=0+a=a\) para todo \(a\in K\).
  \item Cada elemento \(a\in K\) tiene un inverso aditivo \(-a\) tal que \(a+(-a)=0\).
\end{itemize}

\noindent \textbf{2. Estructura multiplicativa}
\begin{itemize}
  \item El producto es asociativo y conmutativo.
  \item Existe un elemento neutro multiplicativo, denotado \(1\neq 0\), tal que \(a\cdot 1 = 1\cdot a = a\) para todo \(a\in K\).
  \item Cada elemento distinto de cero, \(a\neq 0\), tiene un inverso multiplicativo \(a^{-1}\) tal que \(a\cdot a^{-1}=1\).
\end{itemize}

\noindent \textbf{3. Distributividad}

El producto distribuye sobre la suma:
\[
  a\cdot(b+c) = a\cdot b + a\cdot c,\quad
  (a+b)\cdot c = a\cdot c + b\cdot c
  \quad\forall\,a,b,c\in K.
\]

En conjunto, estas propiedades garantizan que en \(K\) podemos manejar sumas, restas, productos y divisiones (salvo por cero) con la misma libertad aritmética que conocemos en \(\mathbb{R}\).

\ejemplo{ Cuerpos matemáticos habituales}
\begin{itemize}
  \item El cuerpo de los números reales, \(\mathbb{R}\).
  \item El cuerpo de los números complejos, \(\mathbb{C}\).
  \item El cuerpo de los números racionales, \(\mathbb{Q}\).
\end{itemize}

Cuando definimos un espacio vectorial, escogemos uno de esos cuerpos como conjunto de escalares; de este modo, todas las multiplicaciones por escalares vienen ``heredadas'' de la estructura de \(K\). Por ejemplo: En un espacio vectorial sobre \(\mathbb{R}\), cualquier \(k\in\mathbb{R}\) puede multiplicar a los vectores. En un espacio vectorial sobre \(\mathbb{C}\), permitimos escalares complejos, lo cual es fundamental, por ejemplo, en espacios de funciones de variable compleja.

Así, el cuerpo \(K\) proporciona el marco algebraico (con sus operaciones y axiomas) que garantiza que la multiplicación escalar en el espacio vectorial se comporte correctamente.

\subsection{Espacio vectorial}

Sea \textit{V} un conjunto no vacío, y \(\left(K, +, \cdot\right)\) un cuerpo (se trabajará con el cuerpo de los números reales y el cuerpo de los números complejos), se definen una operación binaria interna y una operación binaria externa en \textit{V} como sigue:
\begin{align*}
  +:& V \times V \rightarrow V \quad \text{tal que}~~ (u,v) \rightarrow u+v \\
  \cdot ~ :& K \times V \rightarrow V \quad \text{tal que}~~ (k,u) \rightarrow k\cdot u
\end{align*}

La notación:
\[
+\;:\;V\times V\longrightarrow V
\]
se interpreta de la manera siguiente:
\begin{itemize}
  \item \textbf{Dominio}: \(
  V \times V
  \) es el \textbf{producto cartesiano} de \(V\) consigo mismo. Un elemento de \(V \times V\) es un \textbf{par ordenado} donde \(u \in V\) y \(v \in V\). Decimos ``relación de \(V\) consigo mismo'' porque la suma toma dos vectores de \(V\).
  \item \textbf{Imagen o codominio}: \(V\) indica que el resultado de aplicar la operación ``\(+\)'' a cualquier par \((u,v)\) debe ser otro elemento de \(V\). Es la condición de \textit{clausura}: sumar dos vectores nos vuelve a dar un vector \hl{del mismo espacio}.
\end{itemize}

En términos coloquiales: ``\(+\) es una regla que, a cada par de vectores \((u,v)\), le asigna un único vector \(u+v\), y este resultado siempre pertenece a \(V\).''

De modo análogo, la multiplicación por escalar se escribe
\[
  \cdot\;:\;K\times V\;\longrightarrow\;V,
\]
donde:
\begin{itemize}
  \item El dominio \(K\times V\) agrupa un escalar \(k\in K\) y un vector \(u\in V\).
  \item El codominio \(V\) exige que el producto \(k\cdot u\) tenga como resultado siempre un vector en \(V\).
\end{itemize}
Así quedan definidas de manera precisa las dos operaciones básicas de cualquier espacio vectorial:
\begin{itemize}
  \item La suma toma dos vectores y los combina en uno solo.
  \item El producto escalar toma un escalar y un vector, y produce un vector.
\end{itemize}

Definir explícitamente estos mapeos te permite luego verificar los axiomas (asociatividad, conmutatividad, distributividad, existencia de neutros e inversos, etc.), pues todos ellos se expresan como igualdades entre resultados de estas dos funciones.

Se dice que \textit{V} con dichas operaciones es un \textit{K} espacio vectorial si se verifican los siguientes axiomas:
\begin{itemize}
  \item \textbf{A1} - Asociativa: Cualesquiera sean \(u,v\) y \(w\) de \(V\): \[
    (u+v)+w = u + (v+w)
  \]
  \item \textbf{A2} - Existencia de elemento neutro: Existe \(\vec{0}\) en \(V\) para todo \(u\) de \(V\):\[
    u+\vec{0}=\vec{0}+u = u
  \]
  \item \textbf{A3} - Existencia de elementos opuestos: Para todo \(u\) de \(V\) existe \(-u\) también en \(V\):\[
    u + (-u) = (-u) + u = \vec{0}
  \]
  \item \textbf{A4} - Conmutativa: Cualesquiera sean \(u\) y \(v\) de \(V\): \[
    u + v = v + u
  \]
  \item \textbf{A5} - Para todo \(k\) de \(K\), cualesquiera sean \(u\) y \(v\) de \(V\): \[
    k\cdot (u+v)=k\cdot u+k\cdot v
  \]
  \item \textbf{A6} - Cualesquiera sean \(k\) y \(k'\) de \(K\), para todo \(u\) de \(V\):\[
    (k + k')\cdot u = k \cdot u + k' \cdot u
  \]
  \item \textbf{A7} - Cualesquiera sean \(k\) y \(k'\) de \(K\), para todo \(u\) de \(V\):\[
    (k\cdot k')\cdot u = k \cdot (k' \cdot u)
  \]
  \item \textbf{A8} - Para todo \(u\) de \(V\): \[
    1 \cdot u = u \quad \text{(siendo 1 la unidad del cuerpo)}
  \]
\end{itemize}
A los elementos del espacio vectorial se los llama \hl{vectores}.

\begin{quote}
  \ejemplo{ Defina un espacio vectorial de los polinomios de grado menor o igual a dos}

  \textbf{Solución}: Para definir el espacio vectorial de polinomios de grado a lo sumo dos, procedemos del siguiente modo:

  Sea \(K\) el cuerpo de los números reales (o complejos) y consideremos el conjunto
  \[
  V = \bigl\{\,p(x) = a_0 + a_1x + a_2x^2 \;\big|\; a_0,a_1,a_2 \in K\bigr\}.
  \]
  En otras palabras, \(V\) es el conjunto de todas las funciones polinómicas cuyo grado es menor o igual a dos.

  A continuación, definimos en \(V\) las operaciones de suma y multiplicación por escalar exactamente como las de los polinomios:
  \begin{itemize}
    \item Para cualesquiera \(p(x),q(x)\in V\), la suma se define punto a punto:
    \[
      (p+q)(x) \;=\; p(x)+q(x) \;=\; \bigl(a_0+b_0\bigr) + \bigl(a_1+b_1\bigr)x + \bigl(a_2+b_2\bigr)x^2,
    \]
    donde \(p(x)=a_0+a_1x+a_2x^2\) y \(q(x)=b_0+b_1x+b_2x^2\).

  \item Para cualquier escalar \(k\in K\) y \(p(x)\in V\):
    \[
      (k\cdot p)(x) \;=\; k\,p(x) \;=\; (k\,a_0) + (k\,a_1)x + (k\,a_2)x^2.
    \]
  \end{itemize}

  Ahora bien, para ver que efectivamente \((V,+,\cdot)\) es un espacio vectorial sobre \(K\), basta observar que:
  \begin{enumerate}
    \item La suma de dos polinomios de grado \(\le2\) sigue siendo de grado \(\le2\) y lo mismo vale para la multiplicación por escalar.

    \item Asociatividad y conmutatividad de la suma, existencia del elemento neutro (el polinomio \(0(x)=0\)) y de inversos aditivos (para \(p(x)\), su inverso es \(-p(x)\)) se heredan directamente de las propiedades algebraicas de los coeficientes en \(K\).

    \item Distributividad de la multiplicación escalar respecto de la suma de polinomios y de escalares, compatibilidad de la multiplicación de escalares ( \((k\,k')p = k(k'p)\) ) y existencia del escalar unidad como operador neutro (\(1\cdot p = p\)) se verifican ecuación a ecuación, ya que se reducen a propiedades en el cuerpo \(K\).
  \end{enumerate}

  Por tanto, todos los axiomas A1 a A8 se cumplen en este caso concreto.
  \begin{tcolorbox}[myconclusion]
    Si no se está seguro de que todos los axiomas se cumplen (como en este caso), se puede demostrar cada uno de ellos para confirmarlo. Si los ocho axiomas se cumplen, entonces el conjunto es un espacio vectorial
  \end{tcolorbox}
\end{quote}

\begin{tcolorbox}[interesting_data, title=¿Puedo crear un conjunto cualquiera y verificar si es un espacio vectorial?]
  Exacto. El procedimiento general es precisamente ese: uno elige un conjunto \(V\) y un cuerpo de escalares \(K\) (por ejemplo \(\mathbb{R}\) o \(\mathbb{C}\)), define en \(V\) dos operaciones: la suma de dos elementos de \(V\) y la multiplicación de un escalar de \(K\) por un elemento de \(V\) y, si logra demostrar que estas operaciones satisfacen los ocho axiomas (A1-A8), entonces \((V,+,\cdot)\) es un espacio vectorial sobre \(K\). A partir de ese momento, cualquier objeto que pertenezca a \(V\) se denomina vector.

  No hay más requisitos ocultos: lo importante es que la suma y el producto por escalar estén bien definidos (es decir, que siempre produzcan un elemento de \(V\)) y que cumplan las propiedades de asociatividad, conmutatividad, existencia de neutros e inversos aditivos, distributividad y compatibilidad con la estructura del cuerpo.
\end{tcolorbox}

\paragraph{Consecuencias de la definición de espacio vectorial}

La definición de espacio vectorial implica que se cumplan las siguientes propiedades:
\begin{itemize}
  \item \textbf{P1}: \(0\cdot u = \vec{0} \qquad \forall u \in V\)
  \item \textbf{P2}: \(k \cdot \vec{0} = \vec{0} \qquad \forall k \in K\)
  \item \textbf{P3}: \((-1) \cdot u = -u \qquad \forall u \in V\)
\end{itemize}

\begin{tcolorbox}[mydanger]
  Cuidado: Es muy importante tener en cuenta que \[
  \text{Axiomas A1-A8} \quad \implies \quad \text{Propiedades P1-P3}
  \]
  Si solo se te dan P1, P2 y P3 sin garantizar que las operaciones están bien definidas, que la suma sea asociativa, que haya un neutro, o que se cumplan las reglas de distribución y compatibilidad, entonces no puedes concluir que estás en un espacio vectorial.
\end{tcolorbox}

\subsection{Conjunto de vectores: familia libre y familia ligada}

Un conjunto ordenado de vectores: \(F = \left\{u_1,u_2,\cdots,u_m\right\}\) de un espacio vectorial \(V(K)\) se denomina familia de vectores.

\subsubsection{Combinación lineal}

Una combinación lineal de los vectores de una familia con escalares \(a_1, a_2, \cdots, a_m\) es \textit{siempre} un vector del espacio vectorial \(V(K)\).
\[
  u = a_1 u_1 + a_2 u_2 + \cdots + a_m u_m \qquad \forall u \in V
\]
Si el vector \(u\) de \(V\) es nulo, entonces la combinación lineal es:
\[
  \vec{0} = a_1 u_1 + a_2 u_2 + \cdots + a_m u_m
\]
Esta combinación lineal es un sistema de ecuaciones lineales homogéneo, si al resolverlo resulta que los escalares son todos nulos \((a_1,a_2,\cdots,a_m = 0)\) entonces la familia de vectores \(F=\left\{u_1, u_2, \cdots, u_m \right\}\) recibe el nombre de \textbf{familia libre}.

En el caso contrario, si al menos uno de los escalares es no nulo, la familia \(F\) se denomina \textbf{ligada}.

Los vectores de una familia libre son linealmente independientes, esto significa que ninguno de ellos se puede expresar como combinación lineal de los demás de la familia. Los vectores de una familia ligada son linealmente dependientes.

\subsubsection{Propiedades de una familia libre}

Los vectores de una familia libre (linealmente independientes) presentan las siguientes propiedades:
\begin{enumerate}
  \item Si \(F\) es una familia libre, el vector nulo no pertenece a ella. \\ Demostración: Sea \(F=\left\{\vec{0},u_1, u_2, \cdots, u_m\right\}\) una familia de vectores \(V(K)\), en la combinación lineal: \[
    \vec{0} = a_0 \vec{0} + a_1 u_1 + a_2 u_2 + \cdots + a_m u_m
  \]
  Si suponemos que \(a_1, a_2, \cdots, a_m = 0\) resulta entonces \(\vec{0}=a_0 \vec{0}\). Pero por la propiedad \textbf{P2} estamos admitiendo que \(a_0\) no es necesariamente nulo, lo que implica que la familia no es libre.
  \item Si una familia de vectores de \(V\) consta de un solo vector no nulo, es una familia libre. \\ Demostración: Sea \(u_1 \neq \vec{0}\) y \(F=\left\{u_1\right\}\) una familia de vectores de \(V(K)\), la combinación lineal \(\vec{0} = a_1 u_1\) resulta que es verdadera si \(a_1 = 0\) o bien si \(u_1 = \vec{0}\). Sin embargo partimos de que \(u_1 \neq \vec{0}\), por lo que \(a_1 = 0\), lo que implica que \(F\) es una familia libre.
\end{enumerate}

\subsubsection{Propiedades de una familia ligada}

Los vectores de una familia ligada (linealmente dependientes) presentan las siguientes propiedades:
\begin{enumerate}
  \item Si el vector nulo pertenece a una familia \(F\) de vectores de \(V\), \(F\) es una familia ligada. (La demostración es análoga a la propiedad 1 del encabezado anterior).
  \item Si \(F\) es una familia ligada de vectores de \(V\), al menos uno de sus vectores se puede expresar como combinación lineal de los demás. \\ Demostración: Sea \(F=\left\{u_1, u_2, \cdots, u_m\right\}\) una familia ligada en \(V(K)\), y sea la combinación lineal del vector nulo:\[
    \vec{0} = a_1 u_1 + a_2 u_2 + \cdots + a_m u_m
  \]
  Por ser \(F\) ligada, al menos uno de los escalares \(a_i \neq 0\). Supongamos que \(a_1 \neq 0\), entonces: \[
    \vec{0} = a_1 u_1 + a_2 u_2 + \cdots + a_m u_m, ~~ a_1 \neq 0 \implies u_1 = -\frac{a_2}{a_1}u_2 - \cdots - \frac{a_m}{a_1}u_m
  \]
  Lo que nos está indicando que el vector \(u_1\) es combinación lineal de los demás vectores de la familia \(F\).
\end{enumerate}

\subsubsection{Familia generatriz}

Una familia de vectores de \(V(K): F=\left\{u_1,u_2,\cdots,u_m\right\}\) recibe el nombre de familia generatriz o generadora si \textit{todo} vector del espacio vectorial \(V\) se puede expresar como combinación lineal de los vectores de \(F\). Los vectores de \(F\) reciben el nombre de \textit{vectores generadores}.

Es decir, cualquiera sea \(u\) de \(V\), es posible expresarlo de la siguiente manera:\[
  u = a_1 u_1 + a_2 u_2 + \cdots + a_m u_m
\]

\teorema{Dos vectores de un espacio vectorial \(V\) son linealmente dependientes (L.D.) si y sólo si uno de ellos es un múltiplo escalar del otro.}

\teorema{Sea \(K\) un cuerpo, y \(K^n\) el espacio vectorial de dimensión \(n\) sobre \(K\). Entonces, todo conjunto de \(n\) vectores linealmente independientes en \(K^n\) forma una base y, en consecuencia, genera todo \(K^n\).}
\label{teo:generador_li}

Por ejemplo, si \(K=\mathbb{R}\) (cuerpo de los números reales), el teorema se aplica como sigue: Todo conjunto de \(n\) vectores linealmente independientes (L.I.) en \(\mathbb{R}^n\) genera \(\mathbb{R}^n\)

\subsection{Base y dimensión}

Una familia de vectores \(F = \left\{u_1, u_2, \cdots, u_m\right\}\) de \(V(K)\) se denomina \hl{base} de \(V\) si es a la vez familia \textbf{libre y generatriz}.

Esto equivale a decir que todo \(u \in V\) se puede expresar como combinación lineal de los vectores de \(F\), o bien:
\[
  u = a_1 u_1 + a_2 u _2 + \cdots + a_m u_m
\] 
y además si \(u\) es el vector nulo entonces:
\[
  \vec{0} = a_1 u_1 + a_2 u_2 + \cdots + a_m u_m \qquad a_i = 0; ~~ i=1,2,\cdots,m
\]
es decir, los vectores de \(F\) son L.I.

Hemos visto en el teorema \ref{teo:generador_li} que todo conjunto de \(n\) vectores linealmente independientes (L.I.) en \(\mathbb{R}^n\) genera a \(\mathbb{R}^n\). Entonces, de acuerdo a la definición de base, se tiene que:
\begin{tcolorbox}
  \centering
  Todo conjunto de \(n\) vectores L.I. en \(K^n\), constituye una base de \(K^n\)
\end{tcolorbox}

Si el espacio vectorial \(V\) tiene una base finita, es decir, con finitos elementos, entonces la \textbf{dimensión} de \(V\) que se denota \(\text{dim}V\), es el número de vectores que tiene una base cualquiera de \(V\). Este último recibe el nombre de \textit{espacio vectorial de dimensión finita}. En cualquier otro caso, se dice que \(V\) es un espacio vectorial de dimensión infinita.

\begin{itemize}
  \item Todo vector de \(V\) se expresa de manera única en cada base.
  \item El número de vectores de una base de \(V\) se denomina \hl{dimensión} del espacio vectorial.
  \item Todas las bases de un espacio vectorial de dimensión finita \textit{n} tienen exactamente \textit{n} vectores.
  \item Existen espacios vectoriales de dimensión infinita.
  \item El espacio vectorial \(\left\{\vec{0}\right\}\) tiene dimensión cero por definición.
  \item Si \textit{V} es un espacio vectorial de dimensión finita \textit{n}:
  \begin{enumerate}
    \item \(n+1\) vectores de \(V\) son linealmente dependientes, lo que equivale a decir que una familia libre tiene a lo más \(n\) elementos (ver teorema \ref{teo:base_ld}).
    \item una familia generatriz tiene como mínimo \textit{n} elementos (ver teorema \ref{teo:elementos_de_una_base}).
    \item toda familia libre de \(V\) con \(n\) elementos es una base de \(V\).
    \item toda familia generatriz de \(V\) con \(n\) elementos es una base de \(V\).
  \end{enumerate}
\end{itemize}

\teorema{Si \(\left\{v_1, v_2, \cdots, v_n\right\}\) es una base de un espacio vectorial \(V\), entonces todo conjunto con más de \(n\) vectores es linealmente dependiente (L.D.).}
\label{teo:base_ld}

\teorema{Si \(\left\{u_1, u_2, \cdots, u_m\right\}\) y \(\left\{v_1, v_2, \cdots, v_n\right\}\) son bases del espacio vectorial \(V\), entonces \(m=n\)}
\label{teo:elementos_de_una_base}

\subsubsection{Componentes de un vector relativas a una base}

Sea \(V(K)\) un espacio vectorial de dimensión \(m\) y \(F = \left\{u_1,u_2,\cdots,u_m\right\}\) una base de \(V\), un vector \(u\) cualquiera de \(V\) se expresa de manera única:
\[
  u = a_1 u_1 + a_2 u_2 + \cdots + a_m u_m
\]
Los escalares que permiten esta combinación lineal reciben el nombre de componentes del vector \(u\) y se puede anotar asi:
\[
  u = a_1 u_1 + a_2 u_2 + \cdots + a_m u_m = \begin{pmatrix}
    a_1 \\ a_2 \\ \vdots \\ a_m
  \end{pmatrix}
\]
El vector de coordenadas de \(u\) relativo a la base \(B\) se denota \((u)_B\):
\[
  (u)_B = (a_1, a_2, \cdots, a_m)
\]
Si \(v\) es otro vector de \(V\) se expresa en \(B\) de la siguiente manera:
\[
  v = b_1 u_1 + b_2 u_2 + \cdots + b_m u_m = \begin{pmatrix}
    b_1 \\ b_2 \\ \vdots \\ b_m
  \end{pmatrix}
\]
Las componentes del vector \(u+v\) son las suma de las componentes de \(u\) más las componentes de \(v\), y las componentes del vector \(k\cdot u\) son las que se obtienen de multiplicar \(k\) por cada una de las componentes de \(u\). De forma explícita:
\begin{align*}
  u + v &= (a_1 u_1 + a_2 u_2 + \cdots + a_m u_m) + (b_1 u_1 + b_2 u_2 + \cdots + b_m u_m) \\[3pt]
  u + v &= \begin{pmatrix}
    a_1 \\ a_2 \\ \vdots \\ a_m
  \end{pmatrix} + \begin{pmatrix}
    b_1 \\ b_2 \\ \vdots \\ b_m
  \end{pmatrix} = \begin{pmatrix}
    a_1 + b_1 \\ a_2 + b_2 \\ \vdots \\ a_m + b_m
  \end{pmatrix} \\[10pt]
  k\cdot u &= k \cdot \begin{pmatrix}
    a_1 \\ a_2 \\ \vdots \\ a_m
  \end{pmatrix} = \begin{pmatrix}
    k \cdot a_1 \\ k \cdot a_2 \\ \vdots \\ k \cdot a_m
  \end{pmatrix}
\end{align*}
Es decir, si tienes dos vectores expresados en coordenadas relativas a una base:
\[
u = a_1 u_1 + \dots + a_m u_m,\quad
v = b_1 u_1 + \dots + b_m u_m,
\]
Entonces las coordenadas relativas del vector \(u + v\) son:
\[
(u+v)_B = (a_1 + b_1,\; a_2 + b_2,\; \dots,\; a_m + b_m)
\]
Y si multiplicas \(u\) por un escalar \(k\), las coordenadas se escalan:
\[
(k\cdot u)_B = (k a_1,\; k a_2,\; \dots,\; k a_m)
\]
Esto es una consecuencia directa de la linealidad del espacio vectorial: suma y producto por escalar se reflejan componente a componente.

\begin{quote}
  \ejemplo{ Supongamos que tienes un espacio vectorial \(V(K)\), por ejemplo \(V = \mathbb{R}^2\), y eliges una base \(B = \{u_1, u_2\}\), donde los vectores \(u_1\) y \(u_2\) son linealmente independientes.}

  Entonces, cualquier vector \(v \in \mathbb{R}^2\) puede escribirse de forma única como combinación lineal:
  \[
  v = a_1 u_1 + a_2 u_2
  \]
  Los escalares \(a_1\) y \(a_2\) se llaman coordenadas del vector \(v\) relativas a la base \(B\). Veamos un ejemplo concreto.

  Tomemos una base no canónica de \(\mathbb{R}^2\):
  \[
  B = \left\{ u_1 = \begin{pmatrix} 1 \\ 1 \end{pmatrix}, \quad
  u_2 = \begin{pmatrix} -1 \\ 2 \end{pmatrix} \right\}
  \]
  Y el vector:
  \[
  v = \begin{pmatrix} 2 \\ 5 \end{pmatrix}
  \]
  Queremos expresar \(v\) como combinación lineal de \(u_1\) y \(u_2\):
  \[
  v = a_1 u_1 + a_2 u_2
  \]
  Sustituyendo:
  \[
  \begin{pmatrix} 2 \\ 5 \end{pmatrix}
  =
  a_1 \begin{pmatrix} 1 \\ 1 \end{pmatrix}
  + a_2 \begin{pmatrix} -1 \\ 2 \end{pmatrix}
  =
  \begin{pmatrix} a_1 - a_2 \\ a_1 + 2a_2 \end{pmatrix}
  \]
  Entonces, debemos resolver el sistema:
  \[
  \begin{cases}
  a_1 - a_2 = 2 \\
  a_1 + 2a_2 = 5
  \end{cases}
  \]
  Resolviendo:
  \begin{itemize}
    \item De la primera: \(a_1 = 2 + a_2\)
    \item Sustituyendo en la segunda:
    \[
    (2 + a_2) + 2a_2 = 5 \quad\Rightarrow\quad 2 + 3a_2 = 5 \quad\Rightarrow\quad a_2 = 1,\; a_1 = 3
    \]
  \end{itemize}
  Entonces,
  \[
  v = 3u_1 + 1u_2
  \quad\Longrightarrow\quad
  (v)_B = \begin{pmatrix} 3 \\ 1 \end{pmatrix}
  \]
  Esto significa que respecto de la base \(B\), el vector \(v\) se representa mediante las coordenadas \((3, 1)\), aunque geométricamente sigue siendo el mismo vector \(\begin{pmatrix} 2 \\ 5 \end{pmatrix}\) en el plano.
\end{quote}

\begin{tcolorbox}
  Cuando dices que un vector tiene coordenadas relativas a una base \(B\), estás indicando cómo se escribe ese vector como combinación lineal de los vectores de la base \(B\). Esta representación depende totalmente de la base elegida, y por eso el mismo vector puede tener coordenadas distintas en diferentes bases.
\end{tcolorbox}

\subsubsection{Cambio de base}

\paragraph{¿Qué significa cambiar de base?}

En un espacio vectorial como \(\mathbb{R}^2\), cada vector \(w\) puede escribirse de muchas maneras geométricas, pero algebraicamente su representación depende de la base que uses.

Lo desarrollaremos para un espacio vectorial de dimensión 2 y luego lo generalizaremos a un espacio de dimensión finita \textit{n}.

\paragraph{Cambio de base en \(\mathbb{R}^2\)}

Sea \(V(K)\) un espacio vectorial y \(F_1 = \left\{u_1, u_2\right\}\) y \(F_2 = \left\{v_1, v_2\right\}\) dos bases de \(V\), un vector \(w\) de \(V\) se expresa de manera única en \(F_1\) y de manera única en \(F_2\).

Entonces, las dos bases de \(\mathbb{R}^2\) son:
\begin{itemize}
  \item \(F_1 = \{u_1, u_2\}\)
  \item \(F_2 = \{v_1, v_2\}\)
\end{itemize}
Cada vector \(w\in \mathbb{R}^2\) tiene una única representación lineal respecto de cada base:
\[
w = \alpha_1 u_1 + \alpha_2 u_2 = \beta_1 v_1 + \beta_2 v_2
\]
Entonces:
\begin{itemize}
  \item El vector columna \(\begin{pmatrix} \alpha_1 \\ \alpha_2 \end{pmatrix}\) representa a \(w\) en la base \(F_1\).
  \item El vector columna \(\begin{pmatrix} \beta_1 \\ \beta_2 \end{pmatrix}\) representa a \(w\) en la base \(F_2\).
\end{itemize}
El objetivo es pasar de una representación a otra, es decir, dado el vector \(w\) expresado en la base \(F_2\), ¿cómo se obtiene su representación en la base \(F_1\)? Y viceversa.

Para esto, se usa la relación entre las bases. Si conoces cómo se expresan los vectores de \(F_2\) (es decir, \(v_1\) y \(v_2\)) en la base \(F_1\), puedes construir la llamada matriz de cambio de base o matriz de pasaje.

Como sabemos cómo expresar un vector \(w\) en \(F_1\) y en \(F_2\), vemos que
\begin{gather*}
  (w)_{F1} = \alpha_1 u_1 + \alpha_2 u_2 = \begin{pmatrix}
    \alpha_1 \\ \alpha_2
  \end{pmatrix} \qquad (w)_{F2} = \beta_1 v_1 + \beta_2 v_2 = \begin{pmatrix}
    \beta_1 \\ \beta_2
  \end{pmatrix} \\[5pt]
  w = \alpha_1 u_1 + \alpha_2 u_2 = \beta_1 v_1 + \beta_2 v_2
\end{gather*}
Si los vectores \(v_1\) y \(v_2\) de la base \(F_2\) se escriben como combinaciones lineales de la base \(F_1 = \{u_1, u_2\}\):
\[
  v_1 = \begin{pmatrix}
    a_{11} \\ a_{21}
  \end{pmatrix} = a_{11} u_1 + a_{21} u_2 \qquad \text{y} \qquad v_2 = \begin{pmatrix}
    a_{12} \\ a_{22}
  \end{pmatrix} = a_{12} u_1 + a_{22} u_2
\]
entonces podemos escribir el vector \(w\) como:
\[
w = \beta_1 v_1 + \beta_2 v_2 = \beta_1(a_{11}u_1 + a_{21}u_2) + \beta_2(a_{12}u_1 + a_{22}u_2),
\]
agrupando términos:
\[
w = (\beta_1 a_{11} + \beta_2 a_{12}) u_1 + (\beta_1 a_{21} + \beta_2 a_{22}) u_2,
\]
y comparando con la expresión \(w = \alpha_1 u_1 + \alpha_2 u_2\), se deduce que:
\[
\begin{cases}
\alpha_1 = \beta_1 a_{11} + \beta_2 a_{12} \\
\alpha_2 = \beta_1 a_{21} + \beta_2 a_{22}
\end{cases}
\quad\Longrightarrow\quad
\begin{pmatrix}
\alpha_1 \\
\alpha_2
\end{pmatrix}
=
\begin{pmatrix}
a_{11} & a_{12} \\
a_{21} & a_{22}
\end{pmatrix}
\begin{pmatrix}
\beta_1 \\
\beta_2
\end{pmatrix}
\]
Que es un sistema de dos ecuaciones con dos incógnitas, con solución única, ya que dijimos que cada vector se expresa de manera única en cada base. Resolver el sistema equivale a hallar las componentes de \(w\) en \(F_2\) a partir de \(F_1\) o viceversa.

En forma matricial lo podemos expresar así:
\begin{align*}
  X = P \cdot X' ~ : ~ \begin{pmatrix}
    \alpha_1 \\ \alpha_2
  \end{pmatrix} = \begin{pmatrix}
    a_{11} & a_{12} \\ 
    a_{21} & a_{22}
  \end{pmatrix} \cdot \begin{pmatrix}
    \beta_1 \\ \beta_2
  \end{pmatrix}
\end{align*}
Entonces se define la \hl{matriz de pasaje} de \(F_2\) a \(F_1\) como:
\[
P = 
\begin{pmatrix}
a_{11} & a_{12} \\
a_{21} & a_{22}
\end{pmatrix}
\]
donde cada columna de \(P\) contiene las coordenadas de un vector de \(F_2\) expresado en la base \(F_1\), en este caso \((v_1)_{F1} = (a_{11}, a_{21})\) y \((v_2)_{F1} = (a_{12}, a_{22})\).

\newpage

\paragraph{¿Y para ir de \(F_1\) a \(F_2\)?}

Si quisieras obtener las coordenadas de \(w\) en \(F_2\) a partir de las coordenadas en \(F_1\), simplemente debes invertir la matriz:
\[
X' = P^{-1} \cdot X
\]

\paragraph{Generalización}

Todo esto se extiende sin dificultad a espacios \(K^n\) de dimensión mayor a 2: las matrices de pasaje serán de tamaño \(n\times n\), y cumplirán la misma lógica: cada columna de la matriz contiene un vector de una base expresado en la otra base.

\subsection{Subespacio vectorial}

Sea \(V(K)\) un espacio vectorial y \(S\) un subconjunto no vacío de \(V\), si \(S\) es espacio vectorial sobre \(K\) respecto a las mismas operaciones definidas en \(V\), es un subespacio de \(V\). En otras palabras: \(S(K)\) subespacio de \(V(K)\)

\paragraph{Parte estable (propiedad)}

Un subconjunto \(S\) no vacío de \(V(K)\) es estable para las combinaciones lineales si se verifica que para cualesquier par de elementos \(u,v\) de \(S\) y para cualquier escalar \(k\) de \(K\), los vectores \(u+v\) y \(k\cdot u\) pertenecen también a \(S\). \(S\) recibe el nombre de parte estable.
\[
  S ~ \text{es parte estable} ~ \Longleftrightarrow ~ S \subset V \land S \neq \emptyset \land \begin{cases}
    u + v \in S \quad \forall u, v \in S\\
    k \cdot u \in S \quad \forall k \in K \land \forall u \in S
  \end{cases}
\]
Se puede demostrar que si ``\(S\) es parte estable de \(V(K)\) es un subespacio vectorial de \(V\)''.

Por tanto, para comprobar que \(S\) es parte estable basta con verificar:
\begin{enumerate}
  \item \(S\neq\emptyset\) (o, mejor, que \(\mathbf{0}\in S\)).
  \item Si \(u,v\in S\) entonces \(u+v\in S\).
  \item Si \(u\in S\) y \(k\in K\) entonces \(k\cdot u\in S\).
\end{enumerate}

Con eso ya queda demostrado que \(S\) hereda todos los axiomas de espacio vectorial y es, en consecuencia, un subespacio de \(V\).


\subsubsection{Subespacios triviales}

Sea \(V(K)\) un espacio vectorial, los siguientes reciben el nombre de \textit{subespacios triviales} o impropios: \(\left\{\vec{0}\right\} (K) ~~ \text{y} ~~ V(K)\), es decir el conjunto formado exclusivamente por el vector nulo y el mismo \(V\), cualquier otro subespacio de \(V\) recibe el nombre de subespacio propio.

Todo espacio vectorial admite al menos los subespacios triviales.

\subsubsection{Dimensión de los subespacios}

Sea el espacio vectorial \(V(K)\), de dimensión finita \(n\), todo subespacio de \(V\) tiene dimensión finita \(m\) tal que \(m \leq n\).

\begin{itemize}
  \item Por convención el subespacio trivial \(U=\left\{\vec{0}\right\}\) como carece de base, entonces se dice que su dimensión es cero, \(\text{dim}\left\{\vec{0}\right\}=0\), en este caso el único subespacio es él mismo, por lo tanto mantiene la dimensión nula.
  \item Si \(\text{dim }V=1\), admite subespacios de dimensión 0 y dimensión 1. Esto significa que sólo admite a los subespacios triviales, ya que no existe otro subespacio de la misma dimensión incluido él.
  \item Si \(\text{dim }V = 2\), admite los subespacios triviales, de dimensión 0 y de dimensión 2, pero también admite subespacios de dimensión 1.
  \item En general si \(\text{dim }V=n\) admite a los subespacios triviales de dimensión 0 y \textit{n} y además todos los de dimensión \(m\) tal que \(0 \leq m \leq n\)
\end{itemize}
  \newpage
  
  \section{Matrices}

Si desea repasar la unidad de matrices, de forma resumida, puede ver la sección \ref{sec:repaso_matrices}.
  \newpage
  
  \section{Espacios Métricos}

\subsection{Función distancia}

Sea \(E\) un conjunto no vacío de ``puntos'' se denomina función distancia a aquella función que a cada par de ``puntos'' asigna un número real, verificando determinadas condiciones:
\[
  d:E\times E \rightarrow \mathbb{R} \quad \text{tal que} \quad d(a,b) = h
\]
Condiciones:
\begin{itemize}
  \item \(d(a,b) = d(b,a)\)
  \item \(d(a,b) = 0 \quad \Longleftrightarrow \quad a=b\)
  \item \(d(a,c) \leq d(a,b) + d(b,c)\)
\end{itemize}
Recuerde que todo par ordenado de puntos de \(E\) determina un vector del espacio vectorial \(V(K):(a,b) \rightarrow \vec{ab}\) por esto al conjunto de puntos siempre se asocia un espacio vectorial.

Definida una función distancia, se designa al conjunto \(E\) como ``espacio métrico'', asociado al espacio vectorial \(V\).

\subsection{Función norma}

\subsubsection{En un espacio vectorial real \(V(\mathbb{R})\)}

En todo espacio vectorial \(V(\mathbb{R})\) es posible definir una función que designe a cada vector un escalar, verificando ciertas condiciones, esa función recibe el nombre de función norma:
\begin{align*}
  \left\lVert ~~ \right\rVert : V &\rightarrow \mathbb{R} \\
  u &\rightarrow \left\lVert u\right\rVert  
\end{align*}
Condiciones: para todo vector de \(V\):
\begin{itemize}
  \item \(\left\lVert u\right\rVert \geq 0 ~~ \land ~~ (\left\lVert u\right\rVert = 0 \Longleftrightarrow) u = \vec{0}\)
  \item \(\left\lVert t \cdot u\right\rVert = \left|t\right| \cdot \left\lVert u\right\rVert \quad\) para cualquier \(t \in \mathbb{R}\)
  \item \(\left\lVert u + v\right\rVert \leq \left\lVert u\right\rVert + \left\lVert v\right\rVert\) 
\end{itemize}
Definida esta función norma sobre \(V\), decimos que \(V, \left\lVert ~~\right\rVert\) es un \textbf{espacio normado}.

\paragraph{Consecuencias de la definición}

\[
  \left\lVert \vec{0}\right\rVert = 0 \qquad \left\lVert -u\right\rVert = \left\lVert u\right\rVert \qquad \left\lVert u - v\right\rVert = \left\lVert v - u\right\rVert
\]
\[
\left\lVert u\right\rVert = 1 \qquad \text{Es un vector normado}
\]
\[
\left\lVert u\right\rVert \neq 1 \qquad \text{Se puede normalizar:} \quad u' = \frac{1}{\left\lVert u\right\rVert} \rightarrow \left\lVert u'\right\rVert = 1
\]

\begin{quote}
  \ejemplo{ Algunas normas son:}
  \begin{enumerate}
    \item Sea el espacio vectorial \(\mathbb{R}(\mathbb{R})\), se define la función norma: \(\left\lVert ~~ \right\rVert: \mathbb{R} \rightarrow \mathbb{R}\), llamada \textbf{valor absoluto}. \[
      \left\lVert x\right\rVert = \left|x\right|
    \]
    \item En el espacio vectorial \(\mathbb{R}^n(\mathbb{R})\), se definen:
    \begin{itemize}
      \item La función \hl{norma uno}: \(\left\lVert ~~\right\rVert _1 : \mathbb{R}^n \rightarrow \mathbb{R}\)\[
        \left\lVert x\right\rVert _1 = \sum_{i=1}^{n} \left|x_i\right|
      \]
      \item La función \hl{norma dos}: \(\left\lVert ~~\right\rVert _2 : \mathbb{R}^n \rightarrow \mathbb{R}\) \[
        \left\lVert x\right\rVert _2 = \sqrt{\sum_{i=1}^{n}x_i^2}
      \]
      \item La función \hl{norma p}: \(\left\lVert ~~\right\rVert _p : \mathbb{R}^n \rightarrow \mathbb{R}\) \[
        \left\lVert x\right\rVert _p = \sqrt[p]{\sum_{i=1}^{n}x_i^p}
      \]
      \item La función \hl{norma infinito}: \(\left\lVert ~~\right\rVert _\infty : \mathbb{R}^n \rightarrow \mathbb{R}\) \[
        \left\lVert x\right\rVert _\infty = \max \left\{\left|x_i\right|; ~~ i = 1,2,\cdots,n\right\}
      \]
    \end{itemize}
    \item Sea el espacio vectorial \(C_{\left[a,b\right]} (\mathbb{R})\), el espacio vectorial de las funciones continuas definidas en un intervalo real \([a,b]\), se definen:
    \begin{itemize}
      \item La función \hl{norma dos}: \(\left\lVert ~~\right\rVert _2 : C_{[a,b]} \rightarrow \mathbb{R}\) \[
        \left\lVert f\right\rVert _2 = \sqrt{\int_{a}^{b}f^2 ~~ dx}
      \]
      \item La función \hl{norma infinito}: \(\left\lVert ~~\right\rVert _\infty : C_{[a,b]} \rightarrow \mathbb{R}\) \[
        \left\lVert f\right\rVert _\infty = \max \left\{\left|f(x)\right|; ~~ x \in [a,b]\right\}
      \]
    \end{itemize}
    \item Normas matriciales:
    \begin{itemize}
      \item La \hl{norma matricial a uno}: \(\left\lVert A\right\rVert _1 = \max _j \sum_{i_1}^{n} \left|a_{ij}\right|\), es el máximo de la suma de los coeficientes de las columnas en valor absoluto.
      \item La \hl{norma matricial infinito}: \(\left\lVert A\right\rVert _\infty = \max _i \sum_{j=1}^{n}\left|a_{ij}\right|\), es el máximo de la suma de los coeficientes del renglón o filas e valor absoluto.
      \item La \hl{norma matricial a dos}: \(\left\lVert A\right\rVert _2 = \sqrt{\max_{1\leq k \leq n}{(\lambda_k)}}\), siendo \(\lambda_k\) los valores propios de la matriz \(A \cdot A^T\). 
      \item La \hl{norma de Frobenius}: \(\left\lVert A\right\rVert _F = \sqrt{\sum_{i,k=1}^{n} a_{i,k}^2}\)
    \end{itemize}
  \end{enumerate}
\end{quote}

\subsubsection{Norma y distancia}

Siempre es posible definir una distancia inducida por una función norma tal que:
\[
  d(u,v) = \left\lVert u-v\right\rVert \qquad \text{o también} \qquad d(a,b) = \left\lVert a-b \right\rVert
\]
Por lo tanto, todo \textbf{espacio normado}, es también \textbf{espacio métrico}.

\subsection{Producto interior (Producto escalar)}

\subsubsection{En un espacio vectorial real \(V(\mathbb{R})\)}

En todo espacio vectorial \(V(\mathbb{R})\) es posible definir una función tal que a cada par de vectores le asigne un número real, cumpliendo ciertos requisitos, esta función recibe el nombre de \textbf{producto escalar} o producto interior.
\begin{align*}
  \left\langle ~\right\rangle  : V \times V &\rightarrow \mathbb{R} \\
  (u,v) & \rightarrow \left\langle u, v\right\rangle 
\end{align*}
Condiciones: Para todo \(u\), para todo \(v\) y para todo \(w\) del espacio \(V\):
\begin{itemize}
  \item \(\left\langle u,u \right\rangle \geq 0 ~ \land ~ \left(\left\langle u,u\right\rangle = 0 \Longleftrightarrow u = \vec{0}\right)\)
  \item \(\left\langle u,v\right\rangle = \left\langle v, u\right\rangle\)
  \item \(\left\langle u, v+w\right\rangle = \left\langle u,v\right\rangle + \left\langle u,w\right\rangle\)
  \item \(\left\langle u, t\cdot v\right\rangle = t \left\langle u,v\right\rangle \quad\) para cualquier \(t\in \mathbb{R}\) 
\end{itemize}
Definida esta función producto escalar sobre \(V\), decimos que \(\left(V, \left\langle ~ \right\rangle \right)\) es un espacio \textbf{pre-hilbertiano}.

\paragraph{Consecuencia de la definición}
\[
\left\langle u,v\right\rangle = 0 \qquad \text{entonces los vectores son ortogonales} 
\]
s
\begin{quote}
  \ejemplo{ Algunos ejemplos de producto interior:}
  \begin{enumerate}
    \item Sea el espacio vectorial \(\mathbb{R}^n (\mathbb{R})\), definimos el producto interior o producto escalar usual:
    \[
      \left\langle u,v\right\rangle = \sum_{i=1}^{n} u_i ~ v_i 
    \]
    \item Sea el espacio \(\mathbb{R}^2(\mathbb{R})\), para \(u = (x,y),~ v=(x',y')\), definimos el producto escalar:
    \[
      \left\langle u,v\right\rangle = x\cdot x' - y\cdot x' - x \cdot y' + 4 (y\cdot y') 
    \]
    \item En el espacio \(P_n (\mathbb{R})\) de los polinomios de grado menor o igual a \(n\):
    \[
      \left\langle u,v \right\rangle = \int_{0}^{1} u(x) ~ v(x) ~~ dx 
    \]
    \item En el espacio vectorial \(M_{n\times n} (\mathbb{R})\), de las matrices cuadradas de orden \(n\):
    \[
      \left\langle A, B\right\rangle = \text{tr} (B^T \cdot A) 
    \]
  \end{enumerate} 
\end{quote}

\subsubsection{Norma y distacia inducidas por el producto interior}

Siempre es posible definir una norma inducida por el producto escalar:
\[
\left\lVert u\right\rVert = + \sqrt{\left\langle u,u\right\rangle } 
\]
Por lo cual todo \textbf{espacio pre-hilbertiano} es también \textbf{espacio normado}.

Siempre es posible definir una distancia inducida por un producto escalar y su norma inducida:
\[
  d(u,v) = \left\lVert u-v\right\rVert = + \sqrt{\left\langle u-v, u-v\right\rangle } 
\]
Por lo cual todo \textbf{espacio pre-hilbertiano} es también un \textbf{espacio métrico}

\subsection{Espacios Complejos con producto interno}

Sea \(V\) un espacio vectorial sobre el cuerpo \(K=\mathbb{C}\) de números complejos, se define la función producto interior sobre \(\mathbb{C}\), que a cada par de vectores le asigna un número complejo:
\begin{align*}
  \left\langle ~ \right\rangle : V \times V &\rightarrow \mathbb{C} \\
  (u,v) &\rightarrow \left\langle u,v\right\rangle 
\end{align*}
Que cumple las siguientes propiedades:
\begin{itemize}
  \item \(\left\langle u, v+w\right\rangle = \left\langle u,v\right\rangle + \left\langle v,w\right\rangle\)
  \item \(\left\langle t \cdot u, v\right\rangle = \bar{t} \cdot \left\langle u,v\right\rangle \quad\) donde \(t\in \mathbb{C}\) y \(\bar{t}\) es su conjugado.
  \item \(\left\langle u,v\right\rangle = \overline{\left\langle v,u\right\rangle} \quad\) (conjugado de \(\left\langle u,v\right\rangle\))
  \item \(\left\langle u,u\right\rangle \geq 0 \land \left\langle u,u\right\rangle = 0 \Longleftrightarrow u=\vec{0}\)
\end{itemize}
Este producto escalar se denomina ``\textbf{producto hermítico}''

Definida esta función producto interior sobre el espacio complejo \(V(\mathbb{C})\), decimos que \(V, \left\langle ~ \right\rangle\) es un \textbf{espacio unitario}.

\begin{quote}
  \ejemplo{ Algunos ejemplos de producto interior sobre \(\mathbb{C}\)}:
  \begin{enumerate}
    \item Sea el espacio complejo \(\mathbb{C}^n (\mathbb{C})\), para \(u=(x_1, \cdots , x_n), ~ v=(x'_1, \cdots, x'_n)\), se define el \textbf{producto interno canónico}:
    \[
      \left\langle u,v\right\rangle = x_1 \cdot \bar{x}'_1 + \cdots + x_n \cdot \bar{x}'_n
    \]
    \item Sea \(C^*_{[a,b]}\) el espacio de las funciones definidas en el intervalo \([a,b]\) con valores complejos, es decir, cada función es de la forma: \(f(t)=f_1(t)+if_2(t)\), donde \(f_1(t)\) y \(f_2(t)\) pertenece a \(C_{[a,b]}\), se define el producto interior:
    \[
      \left\langle f, g\right\rangle = \int_{a}^{b} f(t) ~\overline{g(t)} ~~ dt 
    \]
    \item En el espacio \(M_n (C)\) de matrices cuadradas complejas de orden \(n\geq 1\), se define el producto interior canónico dado por:
    \[
      \left\langle A,B\right\rangle = \sum_{i,j=1}^{n} a_{ij} ~ \bar{b}_{ij} 
    \]
  \end{enumerate}
\end{quote}
La mayoría de las propiedades de los espacios pre-hilbertianos se mantienen para los espacios unitarios.

Siempre es posible definir una función norma inducida por un producto hermítico, del mismo modo que en el espacio real:
\[
  d(w_1, w_2) = \left\lVert (w_1 + w_2)\right\rVert = + \sqrt{\left\langle (w_1 - w_2), (w_1 - w_2)\right\rangle } 
\]
Siempre teniendo en cuenta que la distancia asigna siempre un numero real, por lo tanto, en los espacios unitarios también se verifica esa condición.

\subsection{Bases ortogonales y ortonormadas}

\subsubsection{Vectores y bases ortogonales}

Sea \(V(K)\) un espacio vectorial, en el que se ha definido un producto escalar, y resulta que para un par cualquiera de vectores, \(u,v\) de \(V\) el producto escalar es nulo, entonces esos vectores son ortogonales.
\[
(u \in V ~ \land ~ v \in V) \land \left\langle u,v\right\rangle = 0 \quad \rightarrow \quad \text{Los vectores son ortogonales}
\]
\textbf{Nota}: El vector nulo es ortogonal a cualquier vector \(u\) de \(V\).

Una familia de vectores de \(V:F=\left\{u_1,u_2,\cdots, u_n\right\}\) es un \textbf{conjunto ortogonal} si para todo par de vectores distintos de \(F\) se verifica que son ortogonales, es decir, que todos los vectores del conjunto \(F\) son ortogonales entre sí.
\[
F ~ \text{es conjunto ortogonal} ~ \Longleftrightarrow ~ \left(\forall (u_i,v_i)\in V^2\right) | i \neq j : \left\langle u_i, u_j\right\rangle = 0 
\]
Una familia \(F\) de vectores de \(V\) que sea base de dicho espacio vectorial y también conjunto ortogonal, se denomina \textbf{base ortogonal}.

\subsubsection{Vectores y bases normados}

Sea \(V(K)\) un espacio vectorial en el cual se ha definido una norma, si resulta que para algún vector \(u\) de \(V\) se verifica que su norma es la unidad, ese vector se denomina normado.
\[
  u \in V ~ \land ~ \left\lVert u\right\rVert = 1 \quad \rightarrow \quad u~\text{es un vector normado}
\]
Como vimos un vector que no es normado es posible normalizarlo, siempre y cuando no sea el vector nulo, de la siguiente manera:
\[
  \left\lVert u\right\rVert \neq 1 \land u \neq 0 \implies u' = \frac{1}{\left\lVert u\right\rVert} \cdot u \quad \rightarrow \quad \text{es un vector normado}
\]

Si \(F\) es una familia de vectores de \(V:F={u_1,u_2,\cdots,u_n}\) en la cual todos sus vectores son normados, \(F\) se denomina \textbf{conjunto normado}.
\[
  F ~ \text{es conjunto normado} \quad \Longleftrightarrow \quad \left(\forall u_i \in F\right): \left\lVert u_i\right\rVert = 1 
\]
Si la familia \(F\) es conjunto normado y además es una base de \(V\) se denomina \textbf{base normada}.

\textbf{Nota}: si una familia de vectores de \(V\) es base ortogonal y normada se denomina \textbf{base ortonormada}.
  \newpage
  
  \section{Funciones Lineales}

% Bloque introductorio: definiciones fundamentales
\subsection{Aplicación lineal}

Una función lineal (también llamada transformación lineal o aplicación lineal) es una función entre dos espacios vectoriales que preserva la estructura algebraica de los vectores, es decir, respeta la suma de vectores y la multiplicación por escalares.

\textbf{Definición}: Sea \(f: V \to U\) una función entre espacios vectoriales sobre el mismo cuerpo \(K\),
\begin{align*}
  f: V &\rightarrow U\\
  v &\rightarrow u
\end{align*}
que a cada vector \(v\) de \(V\) le hace corresponder un único vector \(f(v)\) de \(U\).

Diremos que \(f\) es una aplicación lineal si para todo \(v, w \in V\) y todo escalar \(t \in K\), se cumple:
\begin{enumerate}
  \item Aditividad (respeta la suma):
    \[
     f(v + w) = f(v) + f(w)
    \]
  \item Homogeneidad (respeta el producto por escalares):
    \[
      f(t \cdot v) = t \cdot f(v)
    \]
\end{enumerate}
La primera condición se la conoce como \textit{aditividad}, indica que la función respeta las operaciones internas definidas en los espacios vectoriales involucrados, la segunda se denomina \textit{homogeneidad} y está indicando que la función también respeta las operaciones externas. 

Estas dos propiedades garantizan que la función ``conserva'' la estructura lineal. En otras palabras, las combinaciones lineales en el espacio de partida se transforman en combinaciones lineales en el espacio de llegada, de la misma forma.

\begin{tcolorbox}[remember, title=Aclaración]
  \(V(K)\) y \(U(K)\) son dos espacios vectoriales definidos sobre un mismo cuerpo \(K\) (por ejemplo, \(\mathbb{R}\) o \(\mathbb{C}\)).
  
  La función \(f: V \rightarrow U\) toma un vector \(v \in V\) y lo lleva a un vector \(f(v) \in U\).
  
  Es importante entender que tanto el dominio como el codominio son espacios vectoriales, por lo tanto, en ambos hay operaciones de suma y producto por escalar.
\end{tcolorbox}

\subsubsection{Propiedades de una función lineal}

Si \(f\) es una función lineal de \(V(K)\) en \(U(K)\) entonces:
\begin{enumerate}
  \item \(f\left(\vec{0}\right) = \vec{0}, \qquad \left(\vec{0} \in V,\vec{0} \in U\right)\)

  Demostración:\[
    f\left(\vec{0}\right) = f\left(0 \cdot v\right) = 0 \cdot f(v) = \vec{0}
  \]
  \item \(f(-v)=-f(v), \quad \forall v \in V\)
  
  Demostración: \[
    f(-v) = f(-1 \cdot v) = -1 \cdot f(v) = -f(v)
  \]
  \item \(f(v-w)=f(v) - f(w) \quad \forall v,w \in V\)
  
  Demostración: \[
    f(v-w) = f(v) + f(-w) = f(v) + (-f(w)) = f(v) - f(w)
  \]
\end{enumerate}

\begin{tcolorbox}[interesting_data, title=Nota conceptual]
  Las propiedades desarrolladas son sumamente importantes. Por ejemplo la primer propiedad distingue claramente las funciones lineales de otras funciones. Por ejemplo, si una función \(f\) no cumple que \(f\left(\vec{0}\right) = \vec{0}\), entonces automáticamente no es lineal.
\end{tcolorbox}

\paragraph{Propiedad fundamental de las funciones lineales}

Como consecuencia directa de la aditividad y la homogeneidad, para todo par de escalares \(a, b \in K\) y todo par de vectores \(v, w \in V\), se cumple:
\begin{equation}
  f(av + bw) = a f(v) + b f(w)
  \label{eq:lineal_combinacion_binaria}
\end{equation}

Esta propiedad se generaliza de manera natural a cualquier combinación lineal finita:
\begin{equation}
  f\left( \sum_{i=1}^n a_i v_i \right) = \sum_{i=1}^n a_i f(v_i)
  \label{eq:lineal_combinacion_general}
\end{equation}
donde \(a_i \in K\) y \(v_i \in V\) para \(i = 1, \ldots, n\).

\begin{tcolorbox}[interesting_data, title=Uso de la propiedad]
  Las expresiones \eqref{eq:lineal_combinacion_binaria} y \eqref{eq:lineal_combinacion_general} se utilizan frecuentemente para demostrar que una función es lineal o para verificar propiedades relacionadas con combinaciones lineales en el contexto de espacios vectoriales.
\end{tcolorbox}

\ejemplo{ Veamos algunos ejemplos}
\begin{enumerate}[label=\alph*.]
  \item Sea \( A \in M_{m \times n}(K) \) una matriz con \( m \) filas y \( n \) columnas, cuyas entradas pertenecen a un cuerpo \( K \). Esta matriz define una aplicación lineal \( f: K^n \rightarrow K^m \) mediante la asignación:
  \[
    f(v) = Av
  \]
  donde \( v \in K^n \) se considera como un vector columna. Verifiquemos que \( f \) es lineal:
  \begin{align*}
    f(v + w) &= A(v+w) = Av + Aw = f(v) + f(w) \\
    f(kv) &= A(kv) = kAv = kf(v)
  \end{align*}
  Por lo tanto, \( f \) es lineal.
  \item Sea \(f:\mathbb{R}^3 \rightarrow \mathbb{R}^3\) la aplicación <<proyección>> en el plano \(xy: f(x,y,z)= (x,y,0)\). Probemos que \(f\) es lineal. Sean \(v=(a,b,c)\) y \(w=(a',b',c')\). Entonces:
  \begin{align*}
    f(v+w) &= f(a+a', b+b', c+c') = (a+a', b+b',0) = \\
          &= (a,b,0) + (a',b',0) = f(v) + f(w)
  \end{align*}
  y para todo \(k \in \mathbb{R}\):
  \[
    f(kv) = f(ka,kb,kc) = (ka,kb,0) = k(a,b,0) = kf(v)
  \]
  O sea, \(f\) es lineal.
  \item Sea \(f: \mathbb{R}^2 \rightarrow \mathbb{R}^2\) la aplicación de <<traslación>> definida según \(f(x,y) = (x+1,y+2)\). Obsérvese que \(f(0)=(0,0)=(1,2)\neq 0\). Es decir, el vector cero no se aplica sobre el vector cero. Por consiguiente \(f\) no es lineal.
  \item Sea \(f:V\rightarrow U\) la aplicación que asigna \(0 \in U\) a todo \(v \in V\). Para todo par de vectores \(v,w \in V\) y todo \(k \in K\) tenemos:
  \[
    f(v+w) = 0 = 0+0 = f(v) + f(w) \qquad \text{y} \qquad f(kv) = 0 = k0 = kf(v)
  \]
  Así \(f\) es lineal. Llamamos a \(f\) la \textit{aplicación cero} y la denotaremos normalmente por \(0\).
  \item Sea \( V \) el espacio vectorial de los polinomios en la variable \( t \) sobre \( \mathbb{R} \). Definimos dos aplicaciones:
  \begin{itemize}
    \item La derivada: \( \mathbf{D}: V \rightarrow V \), dada por \( \mathbf{D}(p) = p' \)
    \item La integral definida desde 0: \( \mathbf{J}: V \rightarrow V \), dada por
    \[
      \mathbf{J}(p)(t) = \int_0^t p(s) \, ds
    \]
  \end{itemize}
  Ambas son lineales, porque se cumple:
  \[
    \mathbf{D}(u + v) = \mathbf{D}(u) + \mathbf{D}(v), \qquad \mathbf{D}(ku) = k\mathbf{D}(u)
  \]
  y de forma análoga:
  \[
    \mathbf{J}(u + v) = \mathbf{J}(u) + \mathbf{J}(v), \qquad \mathbf{J}(ku) = k\mathbf{J}(u)
  \]
  para todo \( u, v \in V \) y todo \( k \in \mathbb{R} \). Estas propiedades se demuestran en cursos de Cálculo.
\end{enumerate}
\subsection{Clasificación de las funciones lineales}
\label{sec:clasificacion_de_funciones}

Sea \( f: V(K) \rightarrow W(K) \) una función lineal entre espacios vectoriales sobre un cuerpo \( K \). Podemos clasificar a \( f \) según las siguientes propiedades:

\begin{itemize}
  \item \textbf{Monomorfismo}: \( f \) es inyectiva si \( f(v_1) = f(v_2) \Rightarrow v_1 = v_2 \).  
  Por ejemplo, la función \( f(x) = x \) es inyectiva, mientras que \( f(x) = x^2 \) no lo es, ya que \( f(-1) = f(1) = 1 \).

  \item \textbf{Epimorfismo}: \( f \) es sobreyectiva si su imagen es todo el codominio: \( \text{Im } f = W \).  
  Por ejemplo, \( f: \mathbb{R} \to \mathbb{R} \), \( f(x) = x \), es sobreyectiva, ya que todo valor real se alcanza. En cambio, \( f(x) = e^x \) no lo es, ya que su imagen es \( (0, +\infty) \), no todo \( \mathbb{R} \).

  \item \textbf{Isomorfismo}: \( f \) es biyectiva (inyectiva y sobreyectiva).  
  Por ejemplo, \( f: \mathbb{R}^2 \to \mathbb{R}^2 \), \( f(x, y) = (x + y, x - y) \), es un isomorfismo. Su inversa también es lineal y se puede calcular explícitamente.

  \item \textbf{Endomorfismo}: \( f: V \to V \) (el dominio y el codominio coinciden).  
  Por ejemplo, la proyección \( f(x, y, z) = (x, y, 0) \) es un endomorfismo de \( \mathbb{R}^3 \), pero no es inyectiva ni sobreyectiva.

  \item \textbf{Automorfismo}: endomorfismo que además es isomorfismo (inyectivo y sobreyectivo).  
  Por ejemplo, \( f(x, y) = (x + y, x) \) es un automorfismo de \( \mathbb{R}^2 \); se puede verificar que es lineal y que tiene inversa.
\end{itemize}

% Núcleo teórico: propiedades estructurales
\subsection{Núcleo e imagen de una función lineal}

Sea \(f\) una función lineal de \(V(K)\) en \(U(K)\) \(\left(f:V\rightarrow U\right)\), asociados a esta función existen dos subconjuntos, uno del conjunto de partida y otro del conjunto de llegada, que reciben el nombre de \textit{núcleo} (o \textit{Ker}) de la función e imagen de la función, los definimos:
\[
\text{Ker } f = N(f) = \left\{ v \in V \mid f(v) = \vec{0}_U \right\}
\]
\begin{center}
  son todos los vectores de \(V\) que tienen como imagen al vector nulo de \(U\)
\end{center}
\[
\text{Im } f = \left\{u\in U \mid \exists v \in V ~\text{ tal que }~ f(v) = u \right\}
\]
o también en una forma más abreviada:
\[
\text{Im } f = \left\{f(v) \mid v \in V\right\}
\]
\begin{center}
  son todos los vectores de \(U\) que son imagen de algún vector \(v \in V\)
\end{center}

Los subconjuntos \(N(f)\) e \(\text{Im } f\) son subespacios de \(V\) y de \(U\) respectivamente.

\begin{quote}
  \ejemplo{ Consideremos la siguiente función lineal:}
  \[
    f(x) = 3x
  \]
  Esta es una función lineal \(f:\mathbb{R} \rightarrow \mathbb{R}\).
  \begin{itemize}
    \item \textbf{Núcleo:}
    \[
      \text{Ker } f = \left\{x \in \mathbb{R} \mid f(x) = 0\right\} = \{0\}
    \]
    \item \textbf{Imagen:}
    \[
      \text{Im } f = \left\{f(x) \mid x \in \mathbb{R}\right\} = \left\{3x \mid x \in \mathbb{R}\right\} = \mathbb{R}
    \]
  \end{itemize}
  Aquí el \textit{único} elemento que se anula es el \(0\), pero la imagen es \textit{todo} \(\mathbb{R}\).

  \vspace{5mm}

  \ejemplo{ Ahora consideremos la función trivial de \(\mathbb{R}\) a \(\mathbb{R}\):}
  \[
    f(x) = 0
  \]
  Esta función también es lineal, y \(f:\mathbb{R} \rightarrow \mathbb{R}\)
  \begin{itemize}
    \item \textbf{Núcleo:}
    \[
      \text{Ker } f = \left\{x \in \mathbb{R} \mid f(x) = 0\right\} = \mathbb{R}
    \]
    \item \textbf{Imagen:}
    \[
      \text{Im } f = \{0\}
    \]
  \end{itemize}
  En este caso, \textit{todo} el dominio se anula y la imagen es el conjunto reducido que contiene solo al cero.
\end{quote}

\begin{tcolorbox}[title=Resumen para fijar la idea]
  \begin{itemize}
    \item El \textbf{núcleo} es un subconjunto del \textbf{dominio} \(V\): son los vectores que van a parar al cero del codominio.
    \item La \textbf{imagen} es un subconjunto del codominio \(U\): son todos los valores posibles que puede tomar \(f(v)\)
  \end{itemize}

  \vspace{5pt}

  El núcleo responde a la pregunta: ``¿Qué vectores se anulan bajo \(f\)?''

  La imagen responde a: ``¿Qué vectores pueden obtenerse como salida de \(f\)''
\end{tcolorbox}

\subsubsection{Los subespacios Núcleo y Imagen de \(f\)}

\paragraph{El conjunto núcleo de \(f\) es un subespacio del dominio}

\textbf{Proposición}: Sea \(f: V \to U\) una aplicación lineal entre espacios vectoriales. Entonces, el conjunto núcleo de \(f\),
\[
\text{Ker } f = \{ v \in V \mid f(v) = 0 \}
\]
es un subespacio vectorial de \(V\).

\textit{Demostración}
\begin{quote}
  \begin{enumerate}
    \item \textbf{No vacío}:
    
    El vector nulo del dominio \(\vec{0}_V \in V\) cumple que:
      \[
      f(\vec{0}_V) = \vec{0}_U
      \]
    por lo tanto, \(\vec{0}_V \in \text{Ker } f\), y así \(\text{Ker } f \ne \emptyset\).
  
    \item \textbf{Cerrado bajo la suma}:
    
      Sean \(v_1, v_2 \in \text{Ker } f\). Entonces:
     \[
     f(v_1) = 0 \quad \text{y} \quad f(v_2) = 0
     \]
     Como \(f\) es lineal:
     \[
     f(v_1 + v_2) = f(v_1) + f(v_2) = 0 + 0 = 0
     \]
     Entonces \(v_1 + v_2 \in \text{Ker } f\).
  
    \item \textbf{Cerrado bajo el producto por escalares}:
     
      Sea \(t \in K\) y \(v \in \text{Ker } f\), es decir, \(f(v) = 0\). Entonces:
     \[
     f(t \cdot v) = t \cdot f(v) = t \cdot 0 = 0
     \]
     Por lo tanto, \(t \cdot v \in \text{Ker } f\).
  \end{enumerate}
  
  Conclusión: Se cumplen las tres condiciones necesarias para que \(\text{Ker } f\) sea un subespacio de \(V\).
\end{quote}

\paragraph{El conjunto imagen de \(f\) es un subespacio del codominio}


\textbf{Proposición}: Sea \(f: V \to U\) una aplicación lineal entre espacios vectoriales sobre un mismo cuerpo \(K\). Entonces el conjunto
\[
\text{Im } f = \{ u \in U \mid \exists v \in V \text{ tal que } f(v) = u \}
\]
es un subespacio vectorial de \(U\).

\textit{Demostración:}
\begin{quote}
  \begin{enumerate}
    \item \textbf{No vacío}:
      Como \(f\) es lineal, se cumple que:
      \[
      f(\vec{0}_V) = \vec{0}_U
      \]
      Entonces, \(\vec{0}_U \in \text{Im } f\), por lo tanto \(\text{Im } f \ne \emptyset\).

    \item \textbf{Cerrado bajo suma y producto por escalares} (en una sola propiedad):

      Sean \(u_1, u_2 \in \text{Im } f\).
      Por definición, existen \(v_1, v_2 \in V\) tales que:
      \[
      f(v_1) = u_1 \quad \text{y} \quad f(v_2) = u_2
      \]
      Sean \(a, b \in K\) escalares arbitrarios. Como \(f\) es lineal:
      \[
      f(a v_1 + b v_2) = a f(v_1) + b f(v_2) = a u_1 + b u_2
      \]
      Esto significa que \(a u_1 + b u_2 \in \text{Im } f\).
  \end{enumerate}

  Conclusión: La imagen de \(f\) cumple las condiciones necesarias para ser subespacio de \(U\): no es vacía, y es cerrada bajo combinaciones lineales.
\end{quote}

\begin{tcolorbox}[title=Observaciones]
  Note que no necesitamos comprobar ``cerrado bajo suma'' y ``cerrado bajo producto escalar'' por separado, ya que probar cerrado bajo combinaciones lineales es más general y suficiente.
\end{tcolorbox}

\subsubsection{Ejemplos de cálculo del núcleo e imagen}

\ejemplo{ Sea \(F:\mathbb{R}^3 \rightarrow \mathbb{R}^3\) la aplicación definida por}
\label{ej:aplicacion_proyeccion_xy}
\[
F(x,y,z) = (x,y,0)
\]
Esta función corresponde a la proyección ortogonal sobre el plano \(xy\).

\begin{figure}[ht]
  \centering
  % Configurar el punto de vista 3D
  \tdplotsetmaincoords{70}{110}

  \begin{tikzpicture}[tdplot_main_coords, scale=1.5]

  % Definir las coordenadas del vector (para el dibujo)
  \def\vx{3}
  \def\vy{2}
  \def\vz{2.5}

  % Dibujar los ejes coordenados
  \draw[thick, ->] (0,0,0) -- (4,0,0) node[anchor=north east]{$x$};
  \draw[thick, ->] (0,0,0) -- (0,3,0) node[anchor=north west]{$y$};
  \draw[thick, ->] (0,0,0) -- (0,0,2) node[anchor=south]{$z$};

  % Dibujar el plano xy con una cuadrícula sutil
  \draw[gray!30] (0,0,0) -- (4,0,0) -- (4,3,0) -- (0,3,0) -- cycle;
  \foreach \x in {1,2,3} {
      \draw[gray!20] (\x,0,0) -- (\x,3,0);
  }
  \foreach \y in {1,2,3} {
      \draw[gray!20] (0,\y,0) -- (4,\y,0);
  }

  % Dibujar el vector original v
  \draw[very thick, blue, ->] (0,0,0) -- (\vx,\vy,\vz) 
      node[midway, above left] {$\vec{v}$};

  % Dibujar la proyección del vector en el plano xy
  \draw[very thick, red, ->] (0,0,0) -- (\vx,\vy,0) 
      node[midway, above right] {$\text{proy}_{xy}(\vec{v})$};

  % Dibujar líneas auxiliares para mostrar la proyección
  \draw[dashed, gray] (\vx,\vy,\vz) -- (\vx,\vy,0);
  \draw[dashed, gray] (\vx,0,0) -- (\vx,\vy,0);
  \draw[dashed, gray] (0,\vy,0) -- (\vx,\vy,0);

  % Marcar puntos importantes
  \fill[blue] (\vx,\vy,\vz) circle (2pt) node[above right] {$(x,y,z)$};
  \fill[red] (\vx,\vy,0) circle (2pt) node[below right] {$(x,y,0)$};
  \fill[black] (0,0,0) circle (2pt) node[left] {$O$};

  % Agregar un arco para mostrar el ángulo (opcional)
  \tdplotdrawarc[dashed, gray]{(0,0,0)}{0.8}{0}{90}{}{$\theta$}
  % Gráfico generado con Claude.ai
  \end{tikzpicture}
\end{figure}

\begin{itemize}
  \item \textbf{Núcleo:} son todos los vectores que se proyectan en el origen. Es decir:
  \[
  \text{Ker } F = \{(0,0,z) \in \mathbb{R}^3 \mid z \in \mathbb{R}\}
  \]
  que corresponde al eje \(z\).
  \item \textbf{Imagen:} son todos los vectores de la forma \((x,y,0)\), es decir, el plano \(xy\):
  \[
  \text{Im } F = \{(x,y,0) \in \mathbb{R}^3 \mid x,y \in \mathbb{R}\}
  \]
\end{itemize}

\ejemplo{ Sea \(V\) el espacio de polinomios reales y sea \(\mathbf{T}:V \to V\) el operador derivada tercera:}
\[
\mathbf{T}[f] = \frac{d^3 f}{dt^3}
\]

\begin{itemize}
  \item \textbf{Núcleo:} está formado por todos los polinomios que se anulan al derivar tres veces, es decir:
  \[
  \text{Ker } \mathbf{T} = \{f(t) \in V \mid \deg f \leq 2\}
  \]
  \item \textbf{Imagen:} todo polinomio de \(V\) puede obtenerse como tercera derivada de otro polinomio, por lo tanto:
  \[
  \text{Im } \mathbf{T} = V
  \]
\end{itemize}

Este ejemplo ilustra un caso donde el núcleo no es trivial, pero la imagen es todo el espacio.

\subsubsection{Relación entre generadores del dominio y la imagen}

Supongamos que \(v_1, v_2, \dots, v_n\) generan el espacio vectorial \(V\), y que \(F:V \rightarrow U\) es una función lineal. Entonces los vectores \(F(v_1), F(v_2), \dots, F(v_n)\) generan \(\text{Im } F\).

\begin{tcolorbox}[title=Idea intuitiva]
Dado que todo vector del dominio puede escribirse como combinación lineal de los \(v_i\), y que \(F\) preserva combinaciones lineales, la imagen de cualquier vector será una combinación de las imágenes de los \(v_i\).
\end{tcolorbox}

\textit{Demostración:} Sea \(v \in V\). Como los \(v_i\) generan \(V\), existe una combinación lineal:
\[
v = a_1 v_1 + \cdots + a_n v_n
\]
Aplicando \(F\):
\[
F(v) = F(a_1 v_1 + \cdots + a_n v_n) = a_1 F(v_1) + \cdots + a_n F(v_n)
\]
Esto demuestra que todo vector en \(\text{Im } F\) es combinación lineal de \(F(v_1), \dots, F(v_n)\).

\subsection{Rango y nulidad de una aplicación lineal}

Hasta aquí hemos estudiado el núcleo y la imagen de una aplicación lineal, pero aún no los hemos relacionado con la dimensión del espacio de partida. Cuando el dominio es un espacio vectorial de dimensión finita, se cumple una igualdad fundamental.

\subsubsection{Teorema del núcleo y la imagen}
\label{sec:teorema_dimensión_nucleo_imagen}

\teorema{ Sea \(F: V \rightarrow U\) una aplicación lineal, con \(V\) un espacio vectorial de dimensión finita. Entonces:}
\label{teo:teorema_dimensión_nucleo_imagen}
\begin{equation}
  \dim V = \dim(\text{Ker } F) + \dim(\text{Im } F)
\end{equation}

Este teorema establece que la suma de las dimensiones del núcleo y de la imagen de \(F\) coincide con la dimensión del dominio \(V\). Es decir, la información contenida en el espacio vectorial de partida se reparte entre los vectores que se anulan bajo \(F\) (núcleo) y los que efectivamente se proyectan al espacio de llegada (imagen).

\vspace{5pt}

\ejemplo{ Aplicación del teorema}

Consideremos el ejemplo \ref{ej:aplicacion_proyeccion_xy} de la proyección sobre el plano \(xy\) en \(\mathbb{R}^3\), donde:
\[
  F(x,y,z) = (x,y,0)
\]
En este caso:
\begin{itemize}
  \item \(\dim(\text{Im } F) = 2\), ya que la imagen es el plano \(xy\)
  \item \(\dim(\text{Ker } F) = 1\), ya que el núcleo es el eje \(z\)
  \item \(\dim(\mathbb{R}^3) = 3\)
\end{itemize}
Se verifica así la identidad:
\[
  \dim(\text{Ker } F) + \dim(\text{Im } F) = 1 + 2 = 3 = \dim(\mathbb{R}^3)
\]

\vspace{1em}

\subsubsection{Definiciones de rango y nulidad}

Dados estos conceptos, se introducen dos definiciones asociadas a una aplicación lineal \(F: V \rightarrow U\), con \(V\) de dimensión finita:

\begin{itemize}
  \item El \textbf{rango} de \(F\) es la dimensión de su imagen:
    \[
      \text{rango } F = \dim(\text{Im } F)
    \]
  \item La \textbf{nulidad} de \(F\) es la dimensión de su núcleo:
    \[
      \text{nulidad } F = \dim(\text{Ker } F)
    \]
\end{itemize}

Por lo tanto, el teorema anterior puede expresarse como:
\[
  \text{rango } F + \text{nulidad } F = \dim V
\]

\vspace{5pt}

\begin{tcolorbox}[title=Relación con matrices]
  Recordemos que el rango de una matriz se definió originalmente como la dimensión de su espacio columna (o espacio fila). Si consideramos una matriz \(A\) como una aplicación lineal \(F_A: K^n \rightarrow K^m\), entonces:
  \begin{itemize}
    \item \(\text{Im } F_A\) es el espacio columna de \(A\)
    \item \(\text{rango } A = \dim(\text{Im } F_A)\)
  \end{itemize}
  Esto confirma que la noción de rango coincide tanto en la teoría matricial como en la teoría de aplicaciones lineales.
\end{tcolorbox}

\subsubsection{Ejemplos}

Veamos algunos ejemplos que ilustran la aplicación del teorema anterior y consolidan las nociones de núcleo, imagen, rango y nulidad.

\ejemplo{ Aplicación lineal desde \(\mathbb{R}^4\) a \(\mathbb{R}^3\)}

Sea \(F:\mathbb{R}^4 \rightarrow \mathbb{R}^3\) la aplicación definida por:
\[
  F(x,y,s,t) = (x-y+s+t,\quad x+2s-t,\quad x+y+3s-3t)
\]

\begin{enumerate}[label=\alph*.]

  \item \textbf{Cálculo de una base y dimensión de la imagen de \(F\)}.

  Consideramos la base canónica de \(\mathbb{R}^4\) y calculamos:
  \[
  \begin{aligned}
    F(1,0,0,0) &= (1,1,1) \\
    F(0,1,0,0) &= (-1,0,1) \\
    F(0,0,1,0) &= (1,2,3) \\
    F(0,0,0,1) &= (1,-1,-3)
  \end{aligned}
  \]

  Estos vectores generan la imagen de \(F\). Para hallar una base, colocamos estos vectores como filas en una matriz y reducimos por filas:
  \[
  \begin{pmatrix}
    1 & 1 & 1 \\
    -1 & 0 & 1 \\
    1 & 2 & 3 \\
    1 & -1 & -3
  \end{pmatrix}
  \sim
  \begin{bmatrix}
    1 & 1 & 1 \\
    0 & 1 & 2 \\
    0 & 0 & 0 \\
    0 & 0 & 0
  \end{bmatrix}
  \]

  Así, una base de \(\text{Im } F\) está dada por los vectores \((1,1,1)\) y \((0,1,2)\). Por lo tanto:
  \[
    \dim(\text{Im } F) = 2 \quad \Rightarrow \quad \text{rango}(F) = 2
  \]

  \item \textbf{Cálculo de una base y dimensión del núcleo de \(F\)}.

  Sea \(v = (x,y,s,t)\). Buscamos las soluciones del sistema:
  \[
    F(x,y,s,t) = (0,0,0)
  \]

  Lo que equivale al sistema lineal:
  \[
  \begin{cases}
    x - y + s + t = 0 \\
    x + 2s - t = 0 \\
    x + y + 3s - 3t = 0
  \end{cases}
  \Rightarrow
  \begin{cases}
    x - y + s + t = 0\\
    \phantom{x+} y + s - 2t = 0
  \end{cases}
  \]

  De las dos ecuaciones:
  \begin{enumerate}
    \item \(y + s - 2t = 0 \;\Rightarrow\; y = 2t - s\)
    \item \(x - y + s + t = 0 \;\Rightarrow\; x = y - s - t = (2t - s) - s - t = t - 2s\)
  \end{enumerate}
  Elegimos los parámetros libres \(s\) y \(t\). Entonces
  \begin{align*}
    v = (x,y,s,t) &= (\,t - 2s,\;2t - s,\;s,\;t\,) \\
    &= s\,(-2,-1,1,0)\;+\; t\,(1,2,0,1)
  \end{align*}

  Por lo tanto, una \hl{base} de \(\text{Ker } F\) está dada por:
  \[
    \{(-2,-1,1,0),\ (1,2,0,1)\}, \quad \text{y} \quad \dim(\text{Ker } F) = 2
  \]

  Finalmente, verificamos el teorema:
  \[
    \text{rango } F + \text{nulidad } F = 2 + 2 = 4 = \dim(\mathbb{R}^4)
  \]

\end{enumerate}

\ejemplo{ Aplicación inyectiva con núcleo trivial}

Sea \(F : \mathbb{R}^2 \rightarrow \mathbb{R}^2\) la aplicación definida por:
\[
F(x,y) = (x + y,\ 2x + 3y)
\]

\begin{enumerate}[label=\alph*.]
  \item \textbf{Determinamos el núcleo:}

  Buscamos \((x,y)\) tal que \(F(x,y) = (0,0)\), es decir:
  \[
  \begin{cases}
    x + y = 0 \\
    2x + 3y = 0
  \end{cases}
  \Rightarrow \text{Sustituyendo: } y = -x \Rightarrow 2x - 3x = -x = 0 \Rightarrow  ~ \begin{array}{c}
    x = 0 \\ y = 0
  \end{array}
  \]

  Por lo tanto:
  \[
  \text{Ker } F = \{(0,0)\}, \quad \text{nulidad } F = 0
  \]

  \item \textbf{Determinamos la imagen:}

  Como el sistema no tiene restricciones adicionales, la imagen es todo \(\mathbb{R}^2\). Los vectores columna de la matriz asociada
  \[
  A = \begin{pmatrix}
    1 & 1 \\
    2 & 3
  \end{pmatrix}
  \]
  son linealmente independientes, ya que el determinante \(\det A = 1 \cdot 3 - 1 \cdot 2 = 1 \ne 0\). Entonces:
  \[
  \text{Im } F = \mathbb{R}^2, \quad \text{rango } F = 2
  \]

  \item \textbf{Verificamos el teorema:}

  \[
  \text{rango } F + \text{nulidad } F = 2 + 0 = 2 = \dim(\mathbb{R}^2)
  \]

\end{enumerate}

\ejemplo{ Aplicación derivada en un espacio de polinomios}

\begin{quote}
  Sea \(V = \mathbb{P}_3(\mathbb{R})\), el espacio de los polinomios reales de grado a lo sumo 3, y sea \(\mathbf{D} : V \rightarrow V\) la aplicación derivada:
\end{quote}
\[
\mathbf{D}(p(t)) = p'(t)
\]

\begin{enumerate}[label=\alph*.]
  \item \textbf{Núcleo:}

  El núcleo está formado por los polinomios cuya derivada es nula. Esto es:
  \[
  \text{Ker } \mathbf{D} = \{p \in V \mid p'(t) = 0\} = \{\text{constantes}\}
  \]
  Luego:
  \[
  \dim(\text{Ker } \mathbf{D}) = 1
  \]

  \item \textbf{Imagen:}

  La imagen está compuesta por todos los polinomios de grado a lo sumo 2, ya que derivar un polinomio de grado 3 reduce el grado en 1:
  \[
  \text{Im } \mathbf{D} = \mathbb{P}_2(\mathbb{R}), \quad \dim(\text{Im } \mathbf{D}) = 3
  \]

  \begin{quote}
    \begin{tcolorbox}[remember, title=Aclaración de la Imagen]
      La imagen son los polinomios de grado a lo sumo dos ya que, dado cualquier polinomio de grado a lo sumo tres:
      \begin{align*}
        \frac{d}{dx}\left(a+bx+cx^2+dx^3\right) = b+cx+dx^2
      \end{align*}
    \end{tcolorbox}
  \end{quote}

  \item \textbf{Verificación del teorema:}
  \[
  \text{rango } \mathbf{D} + \text{nulidad } \mathbf{D} = 3 + 1 = 4 = \dim(\mathbb{P}_3)
  \]
\end{enumerate}

% Primera aplicación: interpretación de SELs
\subsection{Aplicación a los sistemas de ecuaciones lineales I}
\label{sec:appl_sels_1}

\begin{tcolorbox}[remember, title=Aclaración]
  En este apartado se verá una introducción con los conocimientos que se tienen, sobre cómo se aplican las transformaciones lineales a los SEL's. Más adelante veremos la segunda parte después de abordar otros conceptos fundamentales.
\end{tcolorbox}

Consideremos un sistema de \(m\) ecuaciones lineales con \(n\) incógnitas sobre un cuerpo \(K\):
\begin{align*}
  a_{11} x_1 + a_{12} x_2 + \cdots + a_{1n} x_n &= b_1 \\
  a_{21} x_1 + a_{22} x_2 + \cdots + a_{2n} x_n &= b_2\\
  \vdots  \qquad \qquad ~ & \\
  a_{m1} x_1 + a_{m2} x_2 + \cdots + a_{mn} x_n &= b_m
\end{align*}
Este sistema puede escribirse de forma matricial como:
\[
  A x = b
\]
donde \(A = (a_{ij})\) es la matriz de coeficientes del sistema, \(x = (x_i)\) es el vector columna de incógnitas, y \(b = (b_i)\) es el vector columna de constantes.

La matriz \(A\), vista como una aplicación lineal
\[
  A: K^n \rightarrow K^m
\]
permite interpretar el sistema \(Ax = b\) como la búsqueda de un vector \(x \in K^n\) cuya imagen bajo \(A\) sea \(b\). Es decir, estamos buscando preimágenes de \(b\) bajo la transformación lineal definida por \(A\).

En particular, el conjunto de soluciones del sistema homogéneo asociado \(Ax = 0\) coincide con el núcleo de la aplicación lineal \(A: K^n \to K^m\), es decir:
\[
  \text{Ker } A = \left\{ x \in K^n \mid Ax = 0 \right\}
\]
Aplicando el teorema del rango y la nulidad (teorema \ref{teo:teorema_dimensión_nucleo_imagen}), que establece que para toda aplicación lineal \(F: V \rightarrow U\) con \(V\) de dimensión finita:
\[
  \dim V = \dim(\text{Ker } F) + \dim(\text{Im } F)
\]
se obtiene, en este contexto particular, que:
\[
  \dim(\text{Ker } A) = \dim(K^n) - \dim(\text{Im } A) = n - \text{rango } A
\]
donde el rango de \(A\) coincide con la dimensión de la imagen de la transformación lineal asociada a la matriz \(A\).

\begin{tcolorbox}[title=Interpretación]
  La \textbf{nulidad} del sistema (número de parámetros libres en la solución del sistema homogéneo) está dada por la diferencia entre el número de incógnitas \(n\) y el rango de la matriz \(A\). Esta relación resulta especialmente útil para determinar la existencia y el número de soluciones, tanto del sistema homogéneo como del sistema no homogéneo.
\end{tcolorbox}


\subsection{Isomorfismos y operadores lineales}

\subsubsection{Aplicaciones lineales singulares y no singulares}

Sea \(F: V \rightarrow U\) una aplicación lineal entre espacios vectoriales sobre un cuerpo \(K\). Se dice que \(F\) es \textit{singular} si existe un vector no nulo \(v \in V\) tal que \(F(v) = 0\). En otras palabras, la aplicación anula al menos un vector distinto de cero. 

Por el contrario, se dice que \(F\) es \textit{no singular} si únicamente el vector nulo se mapea en el vector nulo del codominio, es decir, si \(F(v) = 0\) implica necesariamente que \(v = 0\). Esta condición es equivalente a afirmar que el núcleo de \(F\) está formado únicamente por el vector nulo:
\[
\text{Ker}(F) = \{0\}.
\]

\teorema{Sea \(F: V \rightarrow U\) una aplicación lineal no singular. Entonces, la imagen de cualquier conjunto linealmente independiente de vectores en \(V\) es también linealmente independiente en \(U\).}

\begin{proof}
Sean \(v_1, v_2, \dots, v_n\) vectores linealmente independientes en \(V\), y supongamos que existe una combinación lineal nula de sus imágenes:
\[
a_1 F(v_1) + a_2 F(v_2) + \cdots + a_n F(v_n) = 0,
\]
con \(a_i \in K\). Por linealidad de \(F\), esto equivale a
\[
F(a_1 v_1 + a_2 v_2 + \cdots + a_n v_n) = 0.
\]
De aquí se deduce que el vector
\[
a_1 v_1 + a_2 v_2 + \cdots + a_n v_n
\]
pertenece al núcleo de \(F\). Como \(F\) es no singular, su núcleo es trivial, por lo tanto:
\[
a_1 v_1 + a_2 v_2 + \cdots + a_n v_n = 0.
\]
Dado que los vectores \(v_1, \dots, v_n\) son linealmente independientes, se concluye que \(a_1 = a_2 = \cdots = a_n = 0\). En consecuencia, las imágenes \(F(v_1), \dots, F(v_n)\) son linealmente independientes.
\end{proof}

\subsubsection{Isomorfismos}

Sea \(F: V \rightarrow U\) una aplicación lineal entre espacios vectoriales. Recordemos que \(F\) se denomina \textit{isomorfismo} si es lineal, inyectiva y suprayectiva, es decir, si es una biyección que preserva la estructura lineal (ver sección \ref{sec:clasificacion_de_funciones}). En tal caso, decimos que los espacios vectoriales \(V\) y \(U\) son \textit{isomorfos}, y lo denotamos por
\[
V \simeq U.
\]

\textit{Proposición}: Una aplicación lineal \(F: V \rightarrow U\) es inyectiva si y sólo si es no singular.

\begin{proof}
Supongamos primero que \(F\) es inyectiva. Entonces, si \(F(v) = 0\), se tiene que \(v = 0\), pues de lo contrario existirían dos vectores distintos con la misma imagen, contradiciendo la inyectividad. Por lo tanto, \(\text{Ker}(F) = \{0\}\), y \(F\) es no singular.

Recíprocamente, si \(F\) es no singular, entonces \(\text{Ker}(F) = \{0\}\). Supongamos que \(F(v) = F(w)\). Entonces
\[
F(v - w) = F(v) - F(w) = 0,
\]
por lo que \(v - w \in \text{Ker}(F)\). Como el núcleo es trivial, se deduce que \(v - w = 0\), es decir, \(v = w\). Por lo tanto, \(F\) es inyectiva.
\end{proof}

En el caso de espacios de dimensión finita, puede establecerse un criterio adicional para caracterizar los isomorfismos:

\teorema{Sea \(V\) un espacio vectorial de dimensión finita y \(F: V \rightarrow U\) una aplicación lineal. Entonces \(F\) es un isomorfismo si y sólo si es no singular.}

\begin{proof}
Si \(F\) es un isomorfismo, en particular es inyectiva, por lo que \(\text{Ker}(F) = \{0\}\). Luego, \(F\) es no singular.

Recíprocamente, supongamos que \(F\) es no singular. Entonces \(\dim(\text{Ker}(F)) = 0\). Por el teorema del rango (teorema \ref{teo:teorema_dimensión_nucleo_imagen}), se cumple que:
\[
\dim(V) = \dim(\text{Ker}(F)) + \dim(\text{Im}(F)).
\]
Dado que \(\dim(\text{Ker}(F)) = 0\), se tiene que \(\dim(\text{Im}(F)) = \dim(V)\). Como \(U\) también es de dimensión finita y \(\dim(\text{Im}(F)) = \dim(U)\), se deduce que \(\text{Im}(F) = U\), es decir, \(F\) es suprayectiva.

Como ya hemos demostrado que \(F\) es inyectiva (por ser no singular), y ahora que es suprayectiva, entonces \(F\) es biyectiva. Por lo tanto, \(F\) es un isomorfismo.
\end{proof}

\subsubsection{Operadores lineales e invertibilidad}

\paragraph{Operadores lineales}
\label{sec:operadores_lineales}

Una aplicación lineal \(T: V \rightarrow V\), es decir, cuyo dominio y codominio coinciden, se denomina \textbf{operador lineal} sobre el espacio vectorial \(V\). En este caso, \(T\) transforma vectores de \(V\) en vectores del mismo espacio, conservando las propiedades de linealidad:
\[
  T(a u + b v) = a T(u) + b T(v), \qquad \text{para todo } u, v \in V,~ a, b \in K.
\]

Los operadores lineales permiten estudiar transformaciones internas del espacio vectorial, y son objeto central de muchas teorías algebraicas, como el estudio de autovalores, autovectores, diagonalización, formas canónicas, etc.

Cuando \(V = \mathbb{R}^n\), todo operador lineal puede representarse mediante una matriz cuadrada de orden \(n\). Más generalmente, en cualquier base de un espacio vectorial de dimensión finita, a cada operador lineal se le puede asociar una matriz cuadrada que describe su acción.

\ejemplo{ Sea \(T: \mathbb{R}^2 \rightarrow \mathbb{R}^2\) definido por \(T(x, y) = (x + y, y)\). Esta transformación es lineal y tiene el mismo dominio y codominio, por lo tanto es un operador lineal sobre \(\mathbb{R}^2\).}

\paragraph{Operadores invertibles}

Un operador lineal \(T: V \rightarrow V\) se dice \textbf{invertible} si existe otro operador \(T^{-1}: V \rightarrow V\) tal que:
\[
  T \circ T^{-1} = T^{-1} \circ T = \text{id}_V,
\]
donde \(\text{id}_V\) es el operador identidad en \(V\). En este caso, \(T^{-1}\) se denomina \textit{inverso} de \(T\).

\begin{tcolorbox}[title=Observación]
  Un operador lineal \(T\) es invertible si y sólo si es inyectivo y suprayectivo. En particular, si es invertible, entonces es no singular. Sin embargo, la recíproca no es válida en espacios de dimensión infinita.
\end{tcolorbox}

\ejemplo{ Sea \(V\) el espacio vectorial de los polinomios sobre \(K\), y sea \(T: V \rightarrow V\) el operador definido por}
\[
  T(p(t)) = t \cdot p(t),
\]
es decir, \(T(a_0 + a_1t + \cdots + a_n t^n) = a_0t + a_1 t^2 + \cdots + a_n t^{n+1}\). Esta aplicación es lineal y no singular, ya que \(T(p) = 0\) implica \(p = 0\). Sin embargo, \(T\) no es suprayectivo: no existe ningún polinomio \(p(t)\) tal que \(T(p) = 1\), ya que \(T(p)\) nunca tiene término constante. Por tanto, \(T\) no es invertible.

En el caso de espacios de dimensión finita, se tiene un resultado más fuerte:

\teorema{Sea \(T: V \rightarrow V\) un operador lineal sobre un espacio vectorial \(V\) de dimensión finita. Son equivalentes las siguientes afirmaciones:}
\begin{enumerate}
  \item \(T\) es no singular, es decir, \(\text{Ker}(T) = \{0\}\).
  \item \(T\) es inyectivo.
  \item \(T\) es suprayectivo.
  \item \(T\) es invertible.
\end{enumerate}

\begin{proof}
Ya se ha demostrado que \(T\) es no singular si y sólo si es inyectivo. Para completar el ciclo de equivalencias, probemos que 1) y 3) son equivalentes.

Por el teorema del rango (teorema \ref{teo:teorema_dimensión_nucleo_imagen}):
\[
  \dim(V) = \dim(\text{Ker}(T)) + \dim(\text{Im}(T)).
\]
Si \(T\) es no singular, entonces \(\dim(\text{Ker}(T)) = 0\), por lo que \(\dim(\text{Im}(T)) = \dim(V)\). Como \(\text{Im}(T) \subseteq V\), se concluye que \(\text{Im}(T) = V\), es decir, \(T\) es suprayectivo.

Recíprocamente, si \(T\) es suprayectivo, entonces \(\dim(\text{Im}(T)) = \dim(V)\), por lo que necesariamente \(\dim(\text{Ker}(T)) = 0\), es decir, \(T\) es no singular. El resto de las equivalencias siguen de forma inmediata.
\end{proof}

\ejemplo{ Sea \(T: \mathbb{R}^2 \rightarrow \mathbb{R}^2\) definido por}
\[
  T(x, y) = (y, 2x - y).
\]
Verifiquemos que \(T\) es invertible y hallemos su inverso. Para ello, planteamos:
\[
  T(x, y) = (s, t) \quad \Rightarrow \quad s = y,\quad t = 2x - y.
\]
Sustituyendo \(y = s\) en la segunda ecuación, se obtiene:
\[
  t = 2x - s \quad \Rightarrow \quad x = \frac{1}{2}(s + t).
\]
Así, \(T^{-1}(s, t) = \left(\frac{1}{2}(s + t),~s\right)\).

\subsection{Álgebra de operadores lineales}

\subsubsection{Operaciones con aplicaciones lineales}

Las aplicaciones lineales o funciones lineales pueden combinarse de diversas formas para generar nuevas aplicaciones también lineales. Estas operaciones son fundamentales, ya que permiten dotar al conjunto de las aplicaciones lineales de una estructura algebraica relevante que será empleada reiteradamente a lo largo del estudio.

Supongamos que \(F, G: V \to U\) son aplicaciones lineales entre espacios vectoriales sobre un cuerpo \(K\). Definimos las siguientes operaciones:

\begin{itemize}
  \item \textbf{Suma}: la aplicación \(F + G: V \to U\) se define por
  \[
    (F + G)(v) = F(v) + G(v), \quad \text{para todo } v \in V.
  \]

  \item \textbf{Producto por escalar}: para todo escalar \(k \in K\), la aplicación \(kF: V \to U\) se define por
  \[
    (kF)(v) = k \cdot F(v), \quad \text{para todo } v \in V.
  \]
\end{itemize}

Estas operaciones están bien definidas, en el sentido de que preservan la linealidad. Veámoslo a continuación.

Sean \(a, b \in K\) y \(v, w \in V\). Entonces:
\begin{align*}
  (F + G)(a v + b w) &= F(a v + b w) + G(a v + b w) \\
                     &= a F(v) + b F(w) + a G(v) + b G(w) \\
                     &= a(F(v) + G(v)) + b(F(w) + G(w)) \\
                     &= a (F + G)(v) + b (F + G)(w), \\[6pt]
  (kF)(a v + b w) &= k F(a v + b w) \\
                  &= k (a F(v) + b F(w)) \\
                  &= a (k F(v)) + b (k F(w)) \\
                  &= a (kF)(v) + b (kF)(w).
\end{align*}

Por lo tanto, \(F + G\) y \(kF\) son también aplicaciones lineales.

\teorema{Sean \(V\) y \(U\) espacios vectoriales sobre un cuerpo \(K\). El conjunto de todas las aplicaciones lineales de \(V\) en \(U\), provisto con la suma de aplicaciones y el producto por escalares definidos anteriormente, forma un espacio vectorial sobre \(K\).}

Este espacio vectorial se denota habitualmente por
\[
  \text{Hom}(V, U),
\]
donde el prefijo ``Hom''\footnote{La página siguiente proporciona la definición general de homomorfismo} hace referencia a los homomorfismos entre espacios vectoriales.

Cuando \(V\) y \(U\) son de dimensión finita, el espacio \(\text{Hom}(V,U)\) también tiene dimensión finita. En particular, si \(\dim V = m\) y \(\dim U = n\), se cumple:

\begin{equation}
  \dim(\text{Hom}(V, U)) = m \cdot n.
  \label{eq:dim_homomorfismo}
\end{equation}

En la sección \ref{sec:operadores_lineales} se estudiará el tema de \textit{Operadores Lineales}, que no es más que un caso particular de operaciones lineales. En particular es cuando \(V=U\).

\begin{tcolorbox}[remember, title=Homomorfismo]
  Si tienes dos estructuras algebraicas \((G,\ast)\) y \((H,\cdot)\) del mismo tipo, una función \(f:G\rightarrow H\) es un \textbf{homomorfismo} si para cualesquiera elementos \(a,b\) en \(G\) se cumple que:
  \[
    f(a\ast b) = f(a) \cdot f(b)
  \]
  Esta propiedad significa que ``no importa si primero operas y luego aplicas la función, o si primero aplicas la función y luego operas; el resultado es el mismo''.

  \textbf{Nota}: Los isomorfismos, endomorfismos y automorfismos son tipos particulares de homomorfismos.
\end{tcolorbox}

\subsubsection{Composición de aplicaciones lineales}

Sean \(V\), \(U\) y \(W\) espacios vectoriales sobre un cuerpo \(K\). Sean \(F : V \rightarrow U\) y \(G : U \rightarrow W\) aplicaciones lineales. La \textbf{composición} de \(F\) con \(G\), denotada por \(G \circ F\), es la aplicación de \(V\) en \(W\) definida por:
\[
  (G \circ F)(v) := G(F(v)) \qquad \text{para todo } v \in V.
\]

\begin{center}
  \begin{tikzpicture}[
    node distance=3cm,
    every node/.style={font=\large},
    set/.style={circle, draw, minimum size=1cm, align=center},
    arrow/.style={-Stealth, thick}
    ]

    \node[set] (V) {\(V\)};
    \node[set, right=of V] (U) {\(U\)};
    \node[set, right=of U] (W) {\(W\)};

    \draw[arrow] (V) -- node[above] {\(F\)} (U);
    \draw[arrow] (U) -- node[above] {\(G\)} (W);

  \end{tikzpicture}
\end{center}

\textit{Proposición}: La composición de aplicaciones lineales es también una aplicación lineal. Es decir, si \(F : V \to U\) y \(G : U \to W\) son lineales, entonces \(G \circ F : V \to W\) es lineal.

\begin{proof}
Sean \(v, w \in V\) y \(a, b \in K\). Se tiene:
\begin{align*}
  (G \circ F)(av + bw) &= G(F(av + bw)) \\
                       &= G(aF(v) + bF(w)) \quad \text{(por linealidad de \(F\))} \\
                       &= aG(F(v)) + bG(F(w)) \quad \text{(por linealidad de \(G\))} \\
                       &= a(G \circ F)(v) + b(G \circ F)(w),
\end{align*}
lo que prueba que \(G \circ F\) es lineal.
\end{proof}

\textit{Observación}: La operación de composición de aplicaciones lineales es \textbf{asociativa}, pero en general no es \textbf{conmutativa}. Es decir, si las composiciones están definidas,
\[
  H \circ (G \circ F) = (H \circ G) \circ F, \quad \text{pero usualmente } G \circ F \ne F \circ G.
\]

\paragraph{Compatibilidad con suma y producto por escalar}

Sean \(F, F' : V \to U\), \(G, G' : U \to W\), y \(k \in K\). Las siguientes identidades muestran que la composición de aplicaciones lineales es compatible con la estructura de espacio vectorial:
\begin{align*}
  G \circ (F + F') &= G \circ F + G \circ F', \\
  (G + G') \circ F &= G \circ F + G' \circ F, \\
  k(G \circ F) &= (kG) \circ F = G \circ (kF).
\end{align*}

\begin{tcolorbox}[remember, title=Observación]
  Estas propiedades permiten considerar, en particular, que el conjunto de operadores lineales \(A(V) = \text{Hom}(V, V)\), con la suma, el producto por escalar y la composición, constituye una \textbf{álgebra asociativa} sobre \(K\). Este hecho será aprovechado en secciones posteriores al estudiar operadores invertibles, polinomios de operadores, autovalores y formas canónicas.
\end{tcolorbox}

\subsubsection{Estructura algebraica de \(A(V)\)}

Una estructura algebraica es un conjunto equipado con una o más operaciones que satisfacen ciertas propiedades. Por ejemplo:
\begin{itemize}
  \item Los números enteros \(\mathbb{Z}\) con la suma forman una estructura algebraica (un grupo)
  \item Los números reales \(\mathbb{R}\) con suma y multiplicación forman otra estructura (un cuerpo)
  \item Las matrices \(n\times n\) con suma y multiplicación forman un anillo
\end{itemize}
Ahora, veamos en el contexto de las aplicaciones lineales.

Sea \(V\) un espacio vectorial sobre un cuerpo \(K\). Consideramos ahora el conjunto de todas las aplicaciones lineales de \(V\) en sí mismo, es decir, aquellas \(T : V \rightarrow V\). Estas aplicaciones se denominan \textbf{operadores lineales} o \textbf{transformaciones lineales en} \(V\), y se agrupan en el conjunto usualmente denotado por:
\[
  A(V) := \text{Hom}(V, V).
\]

\textit{Proposición}: El conjunto \(A(V)\), con las operaciones de suma y producto por escalar, forma un espacio vectorial sobre \(K\). Además, si \(\dim(V) = n\), entonces \(\dim(A(V)) = n^2\) (ver ecuación \ref{eq:dim_homomorfismo}).

\textit{Observación}: A diferencia de \(\text{Hom}(V, U)\), donde \(V \ne U\), en \(A(V)\) se puede definir además la \textbf{composición} de operadores, dado que el codominio de cada operador coincide con su dominio. Esta operación permite considerar un producto interno en \(A(V)\): si \(F, G \in A(V)\), entonces \(GF := G \circ F \in A(V)\).

En términos simples, \(A(V)\) es el conjunto de todas las transformaciones lineales de un espacio vectorial \(V\) hacia sí mismo. Este conjunto tiene múltiples estructuras algebraicas superpuestas:

\noindent\textbf{Primera estructura: Espacio vectorial} - \(A(V)\) con las operaciones:
\begin{itemize}
  \item Suma de transformaciones: \((T + S)(v) = T(v) + S(v)\)
  \item Multiplicación por escalar: \((kT)(v) = k\cdot T(v)\)
\end{itemize}
Esto convierte a \(A(V)\) en un espacio vectorial sobre \(K\).

\textbf{Segunda estructura: Anillo (lo que enuncia la observación)}
Pero aquí viene lo especial: como las transformaciones van de \(V\) hacia \(V\) (mismo dominio y codominio), también puedes componer transformaciones: \((G \circ F)(v)=G(F(v))\)

Esta composición actúa como una ``multiplicación'' entre elementos de \(A(V)\).

\begin{tcolorbox}[title=¿Por qué es importante esta estructura múltiple?]
  Porque \(A(V)\) se convierte en lo que se llama un \textbf{álgebra} que es el próximo tema.
\end{tcolorbox}

\paragraph{Definición de álgebra}

Un \textbf{álgebra} es una estructura algebraica \(A(V)\) que es simultáneamente un espacio vectorial y tiene una operación de ``multiplicación'' (la composición) que es compatible con la estructura de espacio vectorial.

\ejemplo{ Si \(\mathbb{R}^2\), entonces \(A(V)\) son todas las matrices \(2\times 2\). Puedes:}
\begin{itemize}
  \item Sumar matrices
  \item Multiplicar por escalares  
  \item Multiplicar matrices (que corresponde a componer transformaciones)
\end{itemize}

\textit{Definición}: Un \textbf{álgebra} \(A\) sobre un cuerpo \(K\) es un espacio vectorial sobre \(K\) provisto de una operación de producto bilineal (no necesariamente conmutativo), que cumple, para todo \(F, G, H \in A\) y \(k \in K\):
\begin{enumerate}
  \item \(F(G + H) = FG + FH\),
  \item \((G + H)F = GF + HF\),
  \item \(k(FG) = (kF)G = F(kG)\).
\end{enumerate}
Si además el producto es asociativo, es decir, si \((FG)H = F(GH)\) para todos los elementos \(F, G, H \in A\), se dice que el álgebra es \textbf{asociativa}.

\teorema{Sea \(V\) un espacio vectorial sobre un cuerpo \(K\). Entonces \(A(V)\), provisto de la suma, el producto por escalar y la composición de aplicaciones, es un \textbf{álgebra asociativa} sobre \(K\). Además, si \(\dim(V) = n\), entonces \(\dim(A(V)) = n^2\).}

\begin{tcolorbox}[remember, title=Observaciones]
El álgebra \(A(V)\) juega un rol fundamental en Álgebra Lineal. No solo es el entorno natural para estudiar operadores lineales y sus propiedades estructurales, sino que proporciona la base teórica para introducir polinomios de operadores, diagonalización, autovalores, formas triangulares y más.
\end{tcolorbox}

La ``estructura algebraica'' de \(A(V)\) es precisamente esta riqueza: no es solo un conjunto, sino un conjunto con múltiples operaciones interrelacionadas que le dan propiedades muy útiles para el álgebra lineal.

\paragraph{Ejemplo de una transformación lineal}

\ejemplo{ Consideremos el espacio vectorial \(V = \mathbb{R}^2\). Definimos los siguientes operadores lineales \(T_1, T_2 \in A(\mathbb{R}^2)\), expresados matricialmente respecto de la base canónica:}

\[
T_1 =
\begin{bmatrix}
1 & 0 \\
0 & -1
\end{bmatrix}
\qquad \text{y} \qquad
T_2 =
\begin{bmatrix}
0 & 1 \\
1 & 0
\end{bmatrix}
\]

Entonces:

\begin{itemize}
  \item La suma \(T_1 + T_2\) es:

  \[
  T_1 + T_2 =
  \begin{bmatrix}
  1 & 0 \\
  0 & -1
  \end{bmatrix}
  +
  \begin{bmatrix}
  0 & 1 \\
  1 & 0
  \end{bmatrix}
  =
  \begin{bmatrix}
  1 & 1 \\
  1 & -1
  \end{bmatrix}
  \]

  \item El producto por un escalar, por ejemplo \(2T_1\), es:

  \[
  2T_1 =
  2 \cdot
  \begin{bmatrix}
  1 & 0 \\
  0 & -1
  \end{bmatrix}
  =
  \begin{bmatrix}
  2 & 0 \\
  0 & -2
  \end{bmatrix}
  \]

  \item La composición \(T_2 \circ T_1\), que en el álgebra se escribe \(T_2 T_1\), corresponde al producto matricial:

  \[
  T_2 T_1 =
  \begin{bmatrix}
  0 & 1 \\
  1 & 0
  \end{bmatrix}
  \cdot
  \begin{bmatrix}
  1 & 0 \\
  0 & -1
  \end{bmatrix}
  =
  \begin{bmatrix}
  0 & -1 \\
  1 & 0
  \end{bmatrix}
  \]

  Obsérvese que \(T_2 T_1 \ne T_1 T_2\), ya que:

  \[
  T_1 T_2 =
  \begin{bmatrix}
  1 & 0 \\
  0 & -1
  \end{bmatrix}
  \cdot
  \begin{bmatrix}
  0 & 1 \\
  1 & 0
  \end{bmatrix}
  =
  \begin{bmatrix}
  0 & 1 \\
  -1 & 0
  \end{bmatrix}
  \]

  De este modo, la composición en \(A(V)\) no es conmutativa en general.
\end{itemize}

Este ejemplo ilustra que \(A(\mathbb{R}^2)\) no solo es un espacio vectorial, sino también un \textbf{álgebra no conmutativa} con multiplicación dada por composición de operadores. Existen casos donde si es conmutativa, pero lo importante es entender que no siempre es así.

Si gustas de ver ``gráficamente'' cómo una transformación lineal modifica el ``espacio'', te recomiendo ver el siguiente video: \href{https://www.youtube.com/watch?v=YJfS4_m_0Z8}{Transformaciones lineales y matrices | Esencia del álgebra lineal, capítulo 3 (3Blue1Brown Español)} 

(si no tienes acceso al enlace, copia el texto en YouTube, que es el título del video.)

\subsubsection{Polinomios aplicados a operadores}

Dado un espacio vectorial \(V\) sobre un cuerpo \(K\), el conjunto de todos los operadores lineales \(T: V \rightarrow V\) se denota por \(A(V)\). Entre estos, la aplicación identidad \(I: V \rightarrow V\), definida por \(I(v) = v\) para todo \(v \in V\), cumple que, para todo \(T \in A(V)\),
\[
  T \circ I = I \circ T = T.
\]

Con base en la composición, es posible definir potencias de un operador lineal \(T \in A(V)\) como sigue:
\[
  T^1 = T,\quad T^2 = T \circ T,\quad T^3 = T \circ T \circ T,\quad \ldots,\quad T^0 = I.
\]

Dado un polinomio con coeficientes en \(K\),
\[
  p(x) = a_0 + a_1 x + a_2 x^2 + \cdots + a_n x^n,
\]
se puede construir el operador lineal \(p(T)\), definido como:
\[
  p(T) = a_0 I + a_1 T + a_2 T^2 + \cdots + a_n T^n.
\]
(En esta expresión, cada escalar \(a_i \in K\) actúa multiplicando al operador correspondiente). En particular, si se cumple que \(p(T) = 0\) (la aplicación nula en \(A(V)\)), se dice que \(T\) es una \textbf{raíz} del polinomio \(p(x)\).

\ejemplo{Definamos el operador \(T: \mathbb{R}^3 \rightarrow \mathbb{R}^3\) por:}
\[
  T(x, y, z) = (0, x, y).
\]
Evaluamos las iteraciones de \(T\) sobre un vector arbitrario \((a, b, c) \in \mathbb{R}^3\):
\[
  T(a, b, c) = (0, a, b), \qquad T^2(a, b, c) = T(0, a, b) = (0, 0, a), \qquad T^3(a, b, c) = T(0, 0, a) = (0, 0, 0).
\]
Por tanto, se cumple que \(T^3 = 0\), es decir, la aplicación nula. En consecuencia, \(T\) es una raíz del polinomio \(p(x) = x^3\).

\subsection{Aplicaciones a los sistemas de ecuaciones lineales II}

En la sección \ref{sec:appl_sels_1} vimos una breve introducción acerca de cómo se plantea una aplicación lineal a un SEL. En esta sección profundizaremos más acerca de la resolución de sistemas de ecuaciones lineales a través de transformaciones lineales (o la aplicación lineal).

\subsubsection{Solución de sistemas cuadrados mediante operadores}

Consideremos un sistema de \(n\) ecuaciones lineales con \(n\) incógnitas sobre un cuerpo \(K\). Tal sistema puede representarse matricialmente mediante la ecuación:
\begin{equation}
  Ax = b
  \label{eq:sistema_lineal_5}  
\end{equation}
donde \(A\) es una matriz cuadrada de orden \(n\) con entradas en \(K\), \(x \in K^n\) es el vector incógnita, y \(b \in K^n\) es el vector de términos independientes.

Interpretamos a \(A\) como un operador lineal \(T_A : K^n \rightarrow K^n\) definido por \(T_A(x) = Ax\). Esta interpretación nos permite analizar el sistema desde una perspectiva funcional: buscamos el elemento \(x\) que el operador \(T_A\) transforma en \(b\).

\paragraph{Caso 1: \(A\) no singular}

Supongamos que la matriz \(A\) es \hl{no singular}, es decir, que el sistema homogéneo asociado \(Ax = 0\) tiene únicamente la solución trivial \(x = 0\). En ese caso, el operador \(T_A\) es inyectivo. Como trabajamos en dimensión finita, se sigue del Teorema de la dimensión (teorema \ref{teo:teorema_dimensión_nucleo_imagen}) que:
\[
\dim(K^n) = \dim(\operatorname{Ker} T_A) + \dim(\operatorname{Im} T_A)
\]
y como \(\dim(\operatorname{Ker} T_A) = 0\), se deduce que \(\operatorname{Im} T_A = K^n\), es decir, \(T_A\) es también suprayectivo. En consecuencia, el sistema \(Ax = b\) admite una solución única para todo \(b \in K^n\).

\begin{tcolorbox}[remember,title=Recordatorio]
  Es importante recordar que, por definición \(\dim\left({\vec{0}}\right) = 0\) (es decir, la dimensión del espacio nulo es cero).
\end{tcolorbox}

\paragraph{Caso 2: \(A\) singular}

Por el contrario, si \(A\) es singular, es decir, si \(Ax = 0\) posee soluciones no triviales, entonces el operador \(T_A\) no es inyectivo. En ese caso, la imagen del operador no puede abarcar todo \(K^n\), por lo tanto, no es suprayectivo. Esto significa que existen vectores \(b \in K^n\) para los cuales el sistema no tiene solución. Además, si una solución existe, entonces no puede ser única, ya que el núcleo de \(T_A\) contiene infinitas soluciones homogéneas que pueden sumarse a cualquier solución particular.

\teorema Sea \(Ax = b\) un sistema lineal con igual número de ecuaciones e incógnitas. Entonces:
\begin{enumerate}[label=\alph*)]
  \item Si el sistema homogéneo asociado \(Ax = 0\) posee únicamente la solución trivial, entonces el sistema \(Ax = b\) admite una solución única para todo \(b \in K^n\).
  \item Si el sistema homogéneo asociado admite soluciones no triviales, entonces:
  \begin{itemize}
    \item[i)] Existen valores de \(b \in K^n\) para los cuales el sistema \(Ax = b\) no admite solución.
    \item[ii)] Siempre que exista una solución, esta no será única.
  \end{itemize}
\end{enumerate}

\paragraph{Ejemplo ilustrativo}

Veamos estos conceptos en un caso concreto.

\ejemplo{ Sea el sistema en \(\mathbb{R}^2\) dado por:}
\[
  A = \begin{bmatrix}
    2 & 1 \\
    4 & 2
  \end{bmatrix}, \qquad b = \begin{bmatrix}
    3 \\
    6
  \end{bmatrix}
\]

Estudiamos el sistema \(Ax = b\). Observamos que:
\[
  \det A = 2 \cdot 2 - 4 \cdot 1 = 4 - 4 = 0
\]
Por lo tanto, \(A\) es singular. El sistema homogéneo \(Ax = 0\) tiene infinitas soluciones no triviales. Verifiquemos si \(b\) pertenece a la imagen de \(A\). Como la segunda fila es un múltiplo de la primera, el rango de \(A\) es 1. Notamos que \(b_2 = 2 \cdot b_1\), luego \(b\) pertenece a la imagen de \(A\), y el sistema tiene soluciones, pero no son únicas.

Si en cambio tomamos \(b = \begin{bmatrix} 3 \\ 7 \end{bmatrix}\), ya no cumple esa proporción, y el sistema no tiene solución.

\paragraph{Interpretación}

Este ejemplo muestra cómo el operador lineal \(T_A(x) = Ax\) puede o no ser inversible dependiendo de la estructura de \(A\), y cómo esta estructura influye directamente en la existencia y unicidad de soluciones. Esta conexión entre álgebra lineal y análisis funcional es uno de los fundamentos del estudio de los sistemas de ecuaciones en espacios vectoriales.

\subsubsection{Matriz asociada a una aplicación lineal}

Sea \(f:V \rightarrow W\) una aplicación lineal entre espacios vectoriales sobre un cuerpo \(K\). Sean \(B = \{u_1, u_2, \ldots, u_n\}\) una base de \(V\) y \(B' = \{u'_1, u'_2, \ldots, u'_m\}\) una base de \(W\). Dado que \(f\) es lineal, cada vector \(f(u_j)\) puede escribirse como combinación lineal de los vectores de \(B'\):
\[
f(u_j) = a_{1j} u'_1 + a_{2j} u'_2 + \cdots + a_{mj} u'_m, \quad j = 1, \dots, n
\]
Los coeficientes \(a_{ij}\) se organizan como columnas para formar una matriz \(A = (a_{ij}) \in M_{m \times n}(K)\). Esta matriz se llama \hl{matriz asociada} a la aplicación lineal \(f\) en las bases \(B\) y \(B'\).

Entonces, si un vector \(v \in V\) tiene coordenadas \((\alpha_1, \ldots, \alpha_n)\) respecto de la base \(B\), su imagen \(w = f(v)\) tiene coordenadas en \(B'\) dadas por:
\[
[w]_{B'} = A \cdot [v]_B
\]

\begin{ejemplo}
Supongamos que \(f: \mathbb{R}^3 \to \mathbb{R}^2\) está definida por:
\[
f\begin{pmatrix}x \\ y \\ z\end{pmatrix} = \begin{pmatrix}x+y \\ y - z\end{pmatrix}
\]
Queremos hallar la matriz asociada a \(f\) respecto de las bases canónicas de \(\mathbb{R}^3\) y \(\mathbb{R}^2\).

Calculamos la imagen de cada vector de la base canónica de \(\mathbb{R}^3\):
\[
f\begin{pmatrix}1 \\ 0 \\ 0\end{pmatrix} = \begin{pmatrix}1 \\ 0\end{pmatrix}, \quad
f\begin{pmatrix}0 \\ 1 \\ 0\end{pmatrix} = \begin{pmatrix}1 \\ 1\end{pmatrix}, \quad
f\begin{pmatrix}0 \\ 0 \\ 1\end{pmatrix} = \begin{pmatrix}0 \\ -1\end{pmatrix}
\]

Luego, la matriz asociada \(A\) es:
\[
A = \begin{pmatrix}
1 & 1 & 0 \\
0 & 1 & -1
\end{pmatrix}
\]

Podemos usar esta matriz para hallar la imagen de cualquier vector \(v \in \mathbb{R}^3\). Por ejemplo, si
\[
v = \begin{pmatrix}2 \\ 1 \\ -1\end{pmatrix}, \quad \text{entonces} \quad f(v) = A \cdot v = \begin{pmatrix}3 \\ 2\end{pmatrix}
\]
\end{ejemplo}

\begin{tcolorbox}
La matriz asociada a una aplicación lineal \(f: V \to W\) en bases dadas tiene tantas filas como la dimensión de \(W\) y tantas columnas como la dimensión de \(V\). Esta matriz permite traducir la acción de \(f\) en términos de productos matriciales.
\end{tcolorbox}

\paragraph{Ejemplo con bases no canónicas}

\begin{ejemplo}
Sea \(f: \mathbb{R}^2 \to \mathbb{R}^2\) la aplicación lineal definida por
\[
f\begin{pmatrix}x \\ y\end{pmatrix} = \begin{pmatrix}x + 2y \\ 3x + y\end{pmatrix}
\]
Consideremos las siguientes bases no canónicas:

Base del dominio:
\[
B = \left\{ \begin{pmatrix}1 \\ 1\end{pmatrix},\; \begin{pmatrix}1 \\ -1\end{pmatrix} \right\}
\]

Base del codominio:
\[
B' = \left\{ \begin{pmatrix}1 \\ 0\end{pmatrix},\; \begin{pmatrix}1 \\ 1\end{pmatrix} \right\}
\]

Queremos hallar la **matriz asociada** a \(f\) en las bases \(B\) y \(B'\), y luego usarla para calcular \(f(v)\), donde
\[
v = \begin{pmatrix}2 \\ 3\end{pmatrix}
\]

\vspace{0.5em}
\textbf{Paso 1:} Hallamos la imagen por \(f\) de los vectores de la base \(B\):
\[
f\begin{pmatrix}1 \\ 1\end{pmatrix} = \begin{pmatrix}1 + 2 \\ 3 + 1\end{pmatrix} = \begin{pmatrix}3 \\ 4\end{pmatrix}
\quad \text{y} \quad
f\begin{pmatrix}1 \\ -1\end{pmatrix} = \begin{pmatrix}1 - 2 \\ 3 - 1\end{pmatrix} = \begin{pmatrix}-1 \\ 2\end{pmatrix}
\]

\vspace{0.5em}
\textbf{Paso 2:} Escribimos estos vectores en coordenadas respecto de la base \(B'\). Es decir, buscamos escalares \(\alpha_1, \alpha_2\) tales que:
\[
\begin{pmatrix}3 \\ 4\end{pmatrix} = \alpha_1 \begin{pmatrix}1 \\ 0\end{pmatrix} + \alpha_2 \begin{pmatrix}1 \\ 1\end{pmatrix}
\Rightarrow
\begin{cases}
\alpha_1 + \alpha_2 = 3 \\
\alpha_2 = 4
\end{cases}
\Rightarrow \alpha_2 = 4,\; \alpha_1 = -1
\]

\[
\begin{pmatrix}-1 \\ 2\end{pmatrix} = \beta_1 \begin{pmatrix}1 \\ 0\end{pmatrix} + \beta_2 \begin{pmatrix}1 \\ 1\end{pmatrix}
\Rightarrow
\begin{cases}
\beta_1 + \beta_2 = -1 \\
\beta_2 = 2
\end{cases}
\Rightarrow \beta_2 = 2,\; \beta_1 = -3
\]

\vspace{0.5em}
\textbf{Paso 3:} La matriz asociada a \(f\) en las bases \(B\) y \(B'\) es:
\[
[f]_{B,B'} = \begin{pmatrix}
-1 & -3 \\
4 & 2
\end{pmatrix}
\]

\vspace{0.5em}
\textbf{Paso 4:} Expresamos el vector \(v = \begin{pmatrix}2 \\ 3\end{pmatrix}\) en coordenadas respecto de la base \(B\). Buscamos escalares \(\lambda_1, \lambda_2\) tales que:
\[
\begin{pmatrix}2 \\ 3\end{pmatrix} = \lambda_1 \begin{pmatrix}1 \\ 1\end{pmatrix} + \lambda_2 \begin{pmatrix}1 \\ -1\end{pmatrix}
\Rightarrow
\begin{cases}
\lambda_1 + \lambda_2 = 2 \\
\lambda_1 - \lambda_2 = 3
\end{cases}
\Rightarrow \lambda_1 = \dfrac{5}{2},\; \lambda_2 = -\dfrac{1}{2}
\]

\vspace{0.5em}
\textbf{Paso 5:} Multiplicamos la matriz asociada por las coordenadas del vector en la base \(B\):
\[
[f]_{B,B'} \cdot [v]_B = \begin{pmatrix}
-1 & -3 \\
4 & 2
\end{pmatrix} \cdot \begin{pmatrix}
\dfrac{5}{2} \\[4pt] -\dfrac{1}{2}
\end{pmatrix}
=
\begin{pmatrix}
-1 \cdot \dfrac{5}{2} + (-3) \cdot \left(-\dfrac{1}{2} \right) \\
4 \cdot \dfrac{5}{2} + 2 \cdot \left(-\dfrac{1}{2} \right)
\end{pmatrix}
=
\begin{pmatrix}
-1 \\ 9
\end{pmatrix}
\]

Estas son las coordenadas de \(f(v)\) respecto de la base \(B'\).

\vspace{0.5em}
\textbf{Paso 6:} Finalmente, para obtener \(f(v)\) como vector de \(\mathbb{R}^2\), desarrollamos en la base \(B'\):
\[
f(v) = -1 \cdot \begin{pmatrix}1 \\ 0\end{pmatrix} + 9 \cdot \begin{pmatrix}1 \\ 1\end{pmatrix}
= \begin{pmatrix}-1 \\ 0\end{pmatrix} + \begin{pmatrix}9 \\ 9\end{pmatrix} = \begin{pmatrix}8 \\ 9\end{pmatrix}
\]

\textbf{Verificación directa:}
\[
f\begin{pmatrix}2 \\ 3\end{pmatrix} = \begin{pmatrix}2 + 6 \\ 6 + 3\end{pmatrix} = \begin{pmatrix}8 \\ 9\end{pmatrix}
\quad \checkmark
\]
\end{ejemplo}

  \newpage
  
  \section{Diagonalización}

Sea \(f\) un endomorfismo en el espacio vectorial \(V(K)\), y \(A\) su matriz asociada en cierta base \(B\)
\begin{align*}
  f:V&\rightarrow V \\
  u &\rightarrow f(u) \quad \text{tal que} \quad A\cdot [u] = [f(u)]
\end{align*}
Un \textbf{vector} \(u\) de \(V\), no nulo, es un \textbf{vector propio} o característico del endomorfismo \(f\), si y sólo si existe un escalar \(\lambda\) tal que:
\[
  f(u) = \lambda \cdot u \quad \text{lo que equivale a que} \quad A \cdot [u] = \lambda[u]
\]
\(u\) es un vector propio relativo al \textbf{escalar \(\lambda\)} que recibe el nombre de \textbf{valor propio}.

\textbf{Proposición}: Si \(u\) es un vector propio del endomorfismo \(f\) relativo al valor propio \(\lambda\), resulta que todo vector linealmente dependiente de \(u\) también es un vector propio correspondiente al mismo valor propio \(\lambda\).

\textit{Demostración}: \(f\) es un endomorfismo y \(u\) es un vector propio de él relativo al valor propio \(\lambda\), por lo tanto:
\[
  f(u) = \lambda \cdot u
\]
Busquemos ahora la imagen de \(v\) por \(f\):
\[
  f(v) = f(t \cdot u) = t \cdot f(u) = t \cdot (\lambda \cdot u) = \lambda \cdot (t\cdot u) = \lambda \cdot v
\]
Lo que indica que \(v\) también es un vector propio relativo al valor propio \(\lambda\): \(f(v) = \lambda \cdot v\)

Como \(u\) es no nulo, el conjunto de los vectores linealmente dependientes a él, \(v = t \cdot u\), determinan un subespacio de \(V\):
\[
  U = \left\{v \in V \mid \exists t \in K ~~ \text{tal que} ~~ v = t\cdot u\right\}
\]
Este \textbf{conjunto que contiene a los vectores propios} generados por el vector propio \(u\), es un subespacio vectorial de \(V\) que recibe el nombre de \textbf{espacio propio o característico} del endomorfismo \(f\), respecto al valor propio \(\lambda\).

Veamos ahora de qué forma se pueden determinar los valores, vectores y espacios propios:

\(f\) es un endomorfismo en \(V\) del cual conocemos su matriz asociada \(A\) y queremos hallar los vectores que hacen posible \(A \cdot [u] = \lambda [u]\) con \(A\in M_{n\times n}\)

Esta expresión la podemos escribir:
\[
  A \cdot [u] = \lambda \cdot I \cdot [u] \qquad I : \text{matriz identidad en } M_{n\times n}
\]
Por lo tanto:
\[
  A \cdot [u] - \lambda \cdot I \cdot [u] = 0 \qquad 0:\text{matriz nula}
\]
O bien:
\[
  (A-\lambda \cdot I) \cdot [u] = 0
\]
Para que el vector \(u\), \textbf{no nulo}, verifique esta última expresión para algún \(\lambda\), que es un sistema de ecuaciones homogéneo, debe verificarse que:
\[
\det(A-\lambda \cdot I) = 0
\]
Lo que equivale a decir que el sistema homogéneo admite solución no nula.

La expresión: \(\det(A-\lambda \cdot I) = 0\) se conoce como ecuación característica, los escalares \(\lambda\) que la satisfacen son los valores propios.

Sustituyendo los valores propios \(\lambda\) en el sistema de ecuaciones lineales homogéneo:
\[
  (A-\lambda \cdot I) \cdot [u] = 0
\]
se obtiene los vectores propios y espacios característicos respectivos.

\ejemplo{ Sea \(f\) un endomorfismo en \(\mathbb{R}^2\) cuya matriz asociada en la base canónica es:}
\[
  A = \begin{pmatrix}
    3 & 0 \\
    8 & -1
  \end{pmatrix}
\]
\begin{enumerate}
  \item Buscar valores propios a partir de la ecuación característica: \(\det(A-\lambda \cdot I) = 0\)
  \[
    A = \begin{pmatrix}
      3 & 0 \\ 8 & -1
    \end{pmatrix}, \quad \lambda \cdot I = \lambda \cdot \begin{pmatrix}
      1 & 0 \\ 0 & 1
    \end{pmatrix} = \begin{pmatrix}
      \lambda & 0 \\
      0 & \lambda
    \end{pmatrix}
  \]
  \[ 
    \quad A - \lambda \cdot I = \begin{pmatrix}
      3 - \lambda & 0 \\ 8 & -1-\lambda
    \end{pmatrix}
  \]
  \begin{align*}
    \det(A-\lambda \cdot I) &= \begin{vmatrix}
      3-\lambda & 0 \\ 8 & -1-\lambda
    \end{vmatrix} = (3-\lambda) \cdot (-1-\lambda) - 0 \\
    &= \lambda ^2 -2 \lambda  - 3 = 0 \implies \lambda_1 = 3 ~,~~ \lambda_2 = -1
  \end{align*}
  Resolviendo la ecuación característica obtuvimos los valores propios del endomorfismo dado:
  \[
    \lambda_1 = 3 \quad \text{y} \quad \lambda_2 = -1
  \]
  \item Buscar los vectores propios relativos a los valores propios hallados anteriormente, para ello planteamos el sistema homogéneo: \((A-\lambda \cdot I)\cdot [u] = 0\)
  \[
    (A-\lambda \cdot I)\cdot [u] = \begin{pmatrix}
      3-\lambda & 0 \\ 8 & -1-\lambda
    \end{pmatrix} \cdot \begin{pmatrix}
      x \\ y
    \end{pmatrix} = \begin{pmatrix}
      0 \\ 0
    \end{pmatrix}, ~~ \text{multiplicando matricialmente:}
  \]
  \[
    \begin{cases}
      (3-\lambda)x + 0y = 0 \\
      8x + (-1-\lambda)y = 0
    \end{cases} \quad \text{sustituimos } \lambda \text{ por los valores encontrados}
  \]
  Si \(\lambda = \lambda_1 = 3\):
  \[
    \begin{cases}
      (3-3)x + 0y = 0 \\
      8x + (-1-3)y = 0
    \end{cases} \qquad \implies y = 2x
  \]
  Por lo tanto los vectores \(u = (x ~~~ 2x)^T\) son solución del sistema para \(\lambda = \lambda_1 = 3\), de aquí que:
  \[
    U_1 = \left\{u \in \mathbb{R}^2 \mid u = \begin{pmatrix}
      x \\ 2x
    \end{pmatrix} ~~ \text{para algún} ~ x \in \mathbb{R}\right\}
  \]
  es el espacio propio relativo al valor propio \(\lambda_1 = 3\), un vector propio a este mismo escalar sería por ejemplo: \(u_1 = (1 ~~~ 2)^T\)

  Si \(\lambda = \lambda_2 = -1\):
  \[
    \begin{cases}
      (3-(-1))x + 0y = 0 \\
      8x + (-1-(-1))y = 0
    \end{cases} \quad \implies\quad \begin{array}{c}
      4x = 0 \\
      8x = 0
    \end{array} \quad \implies x=0
  \]
  Por lo tanto los vectores \(u=(0 ~~~ y)^T\) son solución del sistema para \(\lambda = \lambda_2 = -1\), de aquí que:
  \[
    U_2 = \left\{u \in \mathbb{R}^2 \mid u = \begin{pmatrix}
      0 \\ y
    \end{pmatrix} ~~ \text{para algún} ~ y \in \mathbb{R}\right\}
  \]
  es el espacio propio relativo al valor propio \(\lambda_2 = -1\), un vector propio relativo a este mismo escalar sería por ejemplo: \(u_2 = (0 ~~~ 1)^T\)
\end{enumerate}

\subsection{Diagonalización por vectores propios}

\subsubsection{Diagonalización}

Dada una matriz cuadrada \(A\) asociada a un endomorfismo \(f\) en \(V\):

\textbf{La matriz \textit{A} es diagonalizable} si existe una matriz \(P\) \textbf{invertible}, tal que se verifique que:
\[
  P^{-1} \cdot A \cdot P = D~ , ~~~ \text{siendo } D \text{ una matriz diagonal}
\]
\paragraph{Procedimiento para diagonalizar una matriz a partir de los vectores propios}

\begin{enumerate}[label=\arabic{*}^\circ]
  \item se buscan los vectores propios del endomorfismo \(f\)
  \item se forma con ellos una matriz \(P\)
  \item se analiza si \(P\) es invertible, en caso afirmativo se determina \(P^{-1}\)
  \item se realiza la multiplicación matricial \(P^{-1}\cdot A \cdot P = D\). Esta matriz \(D\) tiene en su diagonal principal los valores propios del endomorfismo \(f\). 
\end{enumerate}
Si \(P\) no es invertible, entonces es imposible diagonalizar a la matriz \(A\).

\ejemplo{ Retomemos el ejemplo anterior, teníamos un endomorfismo en \(\mathbb{R}^2\) cuya matriz asociada es}
\[
  A = \begin{pmatrix}
    3 & 0 \\ 8 & -1
  \end{pmatrix}
\]
dijimos que los vectores \(u_1=(1 ~~~ 2)^T\) y \(u_2 =(0 ~~~ 1)\) eran vectores propios relativos a los vectores propios \(\lambda_1 = 3\) y \(\lambda_2 = -1\)

Formamos con ellos una matriz \(P\):
\[
  P = \begin{pmatrix}
    1 & 0 \\
    2 & 1 
  \end{pmatrix}
\]
Analizamos si \(P\) es invertible: \(\det P = 1\), por lo tanto existe \(P^{-1}\):
\[
P^{-1} = \begin{pmatrix}
  1 & 0 \\
  -2 & 1
\end{pmatrix}
\]
Esto implica que \(A\) es diagonalizable, relizamos la multiplicación matricial:
\[
P^{-1} \cdot A \cdot P = D = \begin{pmatrix}
  3 & 0 \\
  0 & -1
\end{pmatrix}
\]
Observemos que en la diagonal están los valores propios del endomorfismo.

\teorema{Si la matriz \(A\), asociada al endomorfismo \(f\) de \(V_n\), tiene \(n\) valores propios diferentes, entonces \(A\) es diagonalizable.}

\teorema{Si el endomorfismo \(f\) de \(v_n\) tiene \(n\) vectores propios linealmente independientes, \(A\) es diagonalizable.}

\subsection{Matrices simétricas congruentes}

Se dice que una matriz \(M\) es congruente a otra \(A\) si existe una matriz no singular \(P\) tal que:
\[
  M = P^T \cdot A \cdot P
\]
La congruencia es una relación de equivalencia.

Si la matriz \(A\) es simétrica significa que \(A = A^T\), entonces podemos hacer:
\[
  M^T = (P^T \cdot A \cdot P)^T = P^T \cdot A \cdot (P^T)^T = P^T \cdot A \cdot P = M
\]
Por lo tanto, si \(A\) es simétrica, entonces \(M\) también lo es.

Las matrices diagonales son simétricas, se puede demostrar que únicamente matrices simétricas son congruentes a matrices diagonales.

\subsubsection{Diagonalización ortogonal}

Una matriz cuadrada y simétrica \(A\), asociada a un endomorfismo \(f\) en \(V\), es \textbf{ortogonalmente diagonalizable} si existe una matriz \(P\) tal que:
\[
  P^T \cdot A \cdot P = D \qquad \text{sea diagonal}
\]
\(A\) es diagonalizable ortogonalmente solamente si es \textbf{simétrica}, es decir, que \(A^T=A\)

\paragraph{Procedimiento para diagonalizar ortogonalmente una matriz simétrica a partir de los vectores propios}

\begin{enumerate}[label=\arabic{*}^\circ]
  \item Se hallan los valores propios relativos al endomorfismo \(f\)
  \item Se determinan vectores propios correspondientes a los valores propios
  \item Se ortonormalizan dichos vectores propios
  \item Se forma con los vectores ya ortonormalizados una matriz \(P\) y se realiza la multiplicación matricial \(P^T \cdot A \cdot P\)
\end{enumerate}
La matriz que se obtiene es diagonal.

\ejemplo

Sea \(A\) una matriz asociada a un endomorfismo en \(\mathbb{R}^2\), tal que \(A^T = A\), definida a continuación
\[
  A = \begin{pmatrix}
    3 & 1 \\
    1 & 3
  \end{pmatrix}
\]
Buscamos los valores propios a partir de la ecuación característica: \(\det (A - \lambda \cdot I) = 0\)
\begin{align*}
A - \lambda \cdot I = \begin{pmatrix}
  3-\lambda & 1 \\
  1 & 3-\lambda
\end{pmatrix} \quad \implies \det (A-\lambda \cdot I) &= (3-\lambda)(3-\lambda)-1\\
&= \lambda^2 -6\lambda + 8 = 0
\end{align*}
Los valores propios son: \(\lambda_1 = 4\) y \(\lambda_2=2\)

Buscamos vectores propios a relativos a esos valores propios:
\begin{align*}
  (A-\lambda \cdot I) \cdot [u] &= [0] \\
  \begin{pmatrix}
    3-\lambda & 1 \\
    1 & 3-\lambda
  \end{pmatrix} \cdot \begin{pmatrix}
    x\\ y
  \end{pmatrix} &= \begin{pmatrix}
    0 \\ 0
  \end{pmatrix} \quad \implies \quad \begin{cases}
    (3-\lambda)x + y =0\\
    x +(3-\lambda) \cdot y = 0
  \end{cases} 
\end{align*}
Resolviendo el sistema para \(\lambda_1 = 4\) resulta que un vector propio puede ser \(u_1 = (1 ~~~ 1)^T\) y para el valor \(\lambda_2 = 2\) un vector propio sería por ejemplo \(u_2 = (-1 ~~~ 1)^T\)

Tenemos que normalizar a estos vectores, 
  \newpage

  \section{Recordatorios}

\subsection{Sistemas de ecuaciones lineales}
\label{sec:sel}

Un sistema de ecuaciones lineales (SEL) es un conjunto de ecuaciones de la forma
\begin{align*}
  a_{1,1}x_1 + a_{1,2}x_2 + &\cdots + a_{1,n}x_n = b_1,\\
  a_{2,1}x_1 + a_{2,2}x_2 + &\cdots + a_{2,n}x_n = b_2,\\
  &\vdots\\
  a_{m,1}x_1 + a_{m,2}x_2 + &\cdots + a_{m,n}x_n = b_m,
\end{align*}
donde los coeficientes \(a_{i,j}\) y los términos independientes \(b_i\) son números dados, y las incógnitas \(x_1,\dots,x_n\) son las que buscamos. El SEL puede representarse de forma compacta usando la notación matricial
\[
  A\,\mathbf{x} = \mathbf{b},
\]
con \(A\in\mathbb{R}^{m\times n}\), \(\mathbf{x}\in\mathbb{R}^n\) y \(\mathbf{b}\in\mathbb{R}^m\).

\vspace{5pt}

\paragraph{Clasificación mediante el rango}

Para decidir si un SEL tiene solución, y en qué forma, se emplea el concepto de rango de una matriz. El rango de \(A\), denotado \(\mathrm{rang}(A)\), es el número máximo de filas (o columnas) linealmente independientes. Si se forma la matriz aumentada \([A\mid \mathbf{b}]\in\mathbb{R}^{m\times(n+1)}\), entonces el teorema de Rouché-Frobenius afirma que:
\begin{itemize}
  \item El sistema es \hl{compatible} (tiene al menos una solución) si y solo si

  \[
  \mathrm{rang}(A) = \mathrm{rang}([A\mid \mathbf{b}]).
  \]
  \item Dentro de los compatibles, es \hl{determinado} (solución única) cuando
  \(\mathrm{rang}(A) = n\) (el número de incógnitas);
  en caso contrario, se dice \hl{indeterminado} y posee infinitas soluciones.
  \item El sistema es \hl{incompatible} (no tiene solución) si
  \(\mathrm{rang}(A) < \mathrm{rang}([A\mid \mathbf{b}])\).
\end{itemize}

En la práctica el rango se calcula por eliminación de Gauss: al triangular la matriz, el número de filas no nulas en la forma escalonada es precisamente el rango.

\vspace{5pt}

\paragraph{Métodos de resolución}

Desde el punto de vista práctico, existen varias técnicas para obtener las soluciones:

En primer lugar, el \textit{método de sustitución} consiste en despejar una incógnita de una ecuación y sustituirla en las restantes, reduciendo progresivamente el número de incógnitas hasta resolver una ecuación en una sola variable. A continuación se retrocede sustituyendo cada valor hallado en los despejes previos. Este procedimiento es sencillo pero engorroso cuando hay más de tres incógnitas.

Una variante, el \textit{método de igualación}, despeja la misma incógnita en dos ecuaciones diferentes, iguala esas expresiones y elimina así una variable de todo el sistema. Se repite hasta obtener una ecuación con una sola incógnita y luego se retroalimenta.

Para casos con muchas ecuaciones y variables, el \textit{método de eliminación de Gauss} resulta más sistemático: se forman operaciones elementales (intercambio de filas, multiplicar una fila por escalar distinto de cero, sumar un múltiplo de una fila a otra) para llevar la matriz aumentada a una forma triangular superior. En esa forma, cada ecuación sencilla permite el despeje directo de una incógnita (proceso llamado ``descenso'' o ``back-substitution''). Si se continúa hasta reducir a la matriz identidad en el bloque de coeficientes, se obtiene el \textit{método de Gauss-Jordan}, que produce la solución inmediatamente sin necesidad de retroceder.

Cuando el sistema es cuadrado (\(m=n\)) y el determinante de \(A\) es distinto de cero, se puede invocar la \textit{regla de Cramer}: cada incógnita \(x_j\) se calcula como
\[
x_j = \frac{\det(A_j)}{\det(A)},
\]
donde \(A_j\) se obtiene de \(A\) reemplazando su columna \(j\) por el vector \(\mathbf{b}\). El coste de calcular determinantes hace que este método sea eficiente sólo para pequeñas dimensiones. Alternativamente, si \(\det(A)\neq 0\), es posible computar directamente la \textit{matriz inversa} \(A^{-1}\) y resolver
\[
\mathbf{x} = A^{-1}\,\mathbf{b}.
\]
Para sistemas de gran tamaño o con estructuras especiales (por ejemplo, muy dispersos), se emplean métodos iterativos como Jacobi, Gauss-Seidel o mínimos cuadrados, que aproximan la solución por sucesivas iteraciones y suelen converger cuando la matriz de coeficientes cumple propiedades de diagonal dominante o positividad definida.

\subparagraph{Ejemplo ilustrativo}

Supongamos el sistema
\[
\begin{cases}
x + 2y - z = 3,\\
2x - y + 3z = -1,\\
-\,x + y + 2z = 4.
\end{cases}
\]
La matriz de coeficientes y vector independiente son
\[
A = \begin{pmatrix}
1 & 2 & -1\\
2 & -1 & 3\\
-1 & 1 & 2
\end{pmatrix},\quad
\mathbf{b} = \begin{pmatrix}3\\-1\\4\end{pmatrix}.
\]
Al aplicar eliminación de Gauss sobre \([A\mid \mathbf{b}]\) obtenemos tras operaciones elementales una forma escalonada de rango 3 igual al número de incógnitas, luego el sistema es compatible y determinado. El back-substitution conduce a la única solución \((x,y,z) = (1,\,0,\,2)\).

Si en cambio reemplazáramos \(\mathbf{b}\) por otro vector que rompiera la consistencia de última fila, podríamos obtener un pivote cero en el bloque de coeficientes pero un término independiente distinto de cero, lo que revelaría rango aumentado y, por tanto, un sistema incompatible.

En síntesis, para abordar un SEL conviene en primer lugar formularlo en notación matricial, calcular rangos de \(A\) y de \([A\mid\mathbf{b}]\) para determinar su compatibilidad y grado de determinación, y luego aplicar el método que resulte más adecuado según el tamaño, la estructura y la dimensión: sustitución o igualación para casos muy pequeños, Gauss o Gauss-Jordan para sistemas de mediano tamaño, Cramer o inversa cuando \(A\) es cuadrada y no singular, e iterativos cuando el sistema es muy grande o disperso.

\vspace{5pt}

\paragraph{Método de Gauss-Jordan para resolución de SEL}

El método de Gauss-Jordan es una extensión del procedimiento de eliminación de Gauss que lleva la matriz aumentada \([A\mid \mathbf b]\) no sólo a forma escalonada, sino a la forma escalonada reducida (RREF). En esta forma cada columna de pivote contiene un único 1 y ceros en todas las demás posiciones, de modo que la solución del sistema queda inmediatamente visible en la columna de términos independientes.

El procedimiento general es el siguiente. Partimos de la matriz aumentada
\[
\bigl[A\mid \mathbf b\bigr]=
\begin{pmatrix}
a_{11} & a_{12} & \cdots & a_{1n} & \bigm| & b_1\\
a_{21} & a_{22} & \cdots & a_{2n} & \bigm| & b_2\\
\vdots & \vdots & \ddots & \vdots & \bigm| & \vdots\\
a_{m1} & a_{m2} & \cdots & a_{mn} & \bigm| & b_m
\end{pmatrix}.
\]

Empleamos operaciones elementales de fila (intercambiar filas, multiplicar una fila por escalar distinto de cero, sumar un múltiplo de una fila a otra) para alcanzar RREF:
\begin{itemize}
  \item Primero se localiza el primer pivote, es decir, el primer coeficiente no nulo de la fila superior izquierda. Si no está en la fila adecuada, se intercambian filas para llevarlo a la posición \((1,1)\).
  \item A continuación se escala la fila 1 para que el pivote valga exactamente 1.
  \item Luego se eliminan todos los demás elementos de esa columna: para cada fila \(i\neq1\), se resta de la fila \(i\) el múltiplo adecuado de la fila 1 de forma que el resto quede 0.
  \item Se prosigue al siguiente pivote en la submatriz que queda descartando la fila 1 y la columna 1, y se repite el proceso (incluir intercambio si es necesario, escala a 1 y eliminación de arriba y abajo).
\end{itemize}

Al finalizar estos pasos para todas las columnas de pivote, la parte izquierda será una matriz identidad (o bien una matriz escalonada reducida si el sistema tiene infinitas soluciones), y la columna de la derecha contendrá los valores de las incógnitas.

Para ver con claridad cómo funciona, consideremos de nuevo el sistema
\[
\begin{cases}
x + 2y - z = 3,\\
2x - y + 3z = -1,\\
-\,x + y + 2z = 4.
\end{cases}
\]
La matriz aumentada inicial es
\[
\left[\begin{array}{ccc|c}
1 & 2 & -1 & 3\\
2 & -1 & 3 & -1\\
-1 & 1 & 2 & 4
\end{array}\right].
\]

\noindent \textbf{1.} Ya tenemos pivote \(a_{11}=1\). Para anular en columna 1:
\begin{itemize}
  \item \(\text{Fila} ~ 2 \leftarrow ~ \text{Fila} ~ 2 - 2\cdot\text{Fila} ~ 1\)
  \item \(\text{Fila} ~ 3 \leftarrow ~ \text{Fila} ~ 3 + \text{Fila} ~ 1\)
\end{itemize}
Esto arroja
\[
\left[\begin{array}{ccc|c}
1 & 2 & -1 & 3\\
0 & -5 & 5 & -7\\
0 & 3 & 1 & 7
\end{array}\right].
\]

\noindent \textbf{2.} El siguiente pivote deseamos en posición \((2,2)\). Dividimos la fila 2 por \(-5\) para obtener un 1:
\[
\left[\begin{array}{ccc|c}
1 & 2 & -1 & 3\\
0 & 1 & -1 & 7/5\\
0 & 3 & 1 & 7
\end{array}\right].
\]

Luego anular en esa columna sobre y bajo:
\begin{itemize}
  \item \(\text{Fila} ~ 1 \leftarrow ~ \text{Fila} ~ 1 - 2\cdot\text{Fila} ~ 2\)
  \item \(\text{Fila} ~ 3 \leftarrow ~ \text{Fila} ~ 3 - 3\cdot\text{Fila} ~ 2\)
\end{itemize}

\[
\left[\begin{array}{ccc|c}
1 & 0 & 1 & 1/5\\
0 & 1 & -1 & 7/5\\
0 & 0 & 4 & 14/5
\end{array}\right].
\]

\noindent \textbf{3.} En \((3,3)\) el pivote es 4; dividimos fila 3 por 4:
\[
\left[\begin{array}{ccc|c}
1 & 0 & 1 & 1/5\\
0 & 1 & -1 & 7/5\\
0 & 0 & 1 & 7/10
\end{array}\right].
\]

\noindent \textbf{4.} Finalmente, eliminamos arriba:
\begin{itemize}
  \item \(\text{Fila} ~ 1 \leftarrow ~ \text{Fila} ~ 1 - 1\cdot\text{Fila} ~ 3\)
  \item \(\text{Fila} ~ 2 \leftarrow ~ \text{Fila} ~ 2 + 1\cdot\text{Fila} ~ 3\)
\end{itemize}

\[
\left[\begin{array}{ccc|c}
1 & 0 & 0 & -1/2\\
0 & 1 & 0 & 21/10\\
0 & 0 & 1 & 7/10
\end{array}\right].
\]

La solución es \(x=-1/2\), \(y=21/10\), \(z=7/10\).
  \newpage

    \section*{Recursos Adicionales}
  \noindent El código fuente de este resumen está disponible en el siguiente repositorio de GitHub: \url{https://github.com/EVAnci/algebra_lineal}
  \vspace{5pt}

  \begin{quote}
  Si ha encontrado errores en el documento, o no está seguro si tiene la última versión del documento puede descargarla desde la pestaña ``Releases'' de GitHub. A continuación se deja el enlace:

  \url{https://github.com/EVAnci/algebra_lineal/releases}

  \vspace{5pt}

  Si tiene la última versión y esta contiene errores, puede crear un issue de GitHub para que pueda ser revisado. Los issues requieren que usted tenga una cuenta en GitHub.

  Para crear un issue puede ir al siguiente enlace:

  \url{https://github.com/EVAnci/algebra_lineal/issues}

  El procedimiento es sencillo:
  \begin{enumerate}
    \item Una vez en el enlace, con su cuenta de GitHub presiona el botón ``New issue''.
    \item Describe el problema indicando la sección o página. También puede indicarlo describiendo el contenido de forma detallada.
    \item Publica el issue y listo.
  \end{enumerate}
  Una vez haya publicado el ``issue'' se analizará y, si se puede arreglar, se publicará en la pestaña de releases la nueva versión.

  \vspace{5pt}

  Si desea realizar una contribución al documento, tiene las instrucciones de como hacerlo en el \texttt{README.md} del repositorio. 

  Básicamente el procedimiento es:
  \begin{enumerate}
    \item Crea un fork del repositorio,
    \item agrega los cambios y contribuciones que desea al documento,
    \item y realiza un ``Pull Request''
  \end{enumerate}
  \end{quote}
  \newpage
  
  \printbibliography
\end{document}
