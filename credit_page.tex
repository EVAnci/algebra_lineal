  \section*{Recursos Adicionales}
  \noindent El código fuente de este resumen está disponible en el siguiente repositorio de GitHub: \url{https://github.com/EVAnci/algebra_lineal}
  \vspace{5pt}

  \begin{quote}
  Si ha encontrado errores en el documento, o no está seguro si tiene la última versión del documento puede descargarla desde la pestaña ``Releases'' de GitHub. A continuación se deja el enlace:

  \url{https://github.com/EVAnci/algebra_lineal/releases}

  \vspace{5pt}

  Si tiene la última versión y esta contiene errores, puede crear un issue de GitHub para que pueda ser revisado. Los issues requieren que usted tenga una cuenta en GitHub.

  Para crear un issue puede ir al siguiente enlace:

  \url{https://github.com/EVAnci/algebra_lineal/issues}

  El procedimiento es sencillo:
  \begin{enumerate}
    \item Una vez en el enlace, con su cuenta de GitHub presiona el botón ``New issue''.
    \item Describe el problema indicando la sección o página. También puede indicarlo describiendo el contenido de forma detallada.
    \item Publica el issue y listo.
  \end{enumerate}
  Una vez haya publicado el ``issue'' se analizará y, si se puede arreglar, se publicará en la pestaña de releases la nueva versión.

  \vspace{5pt}

  Si desea realizar una contribución al documento, tiene las instrucciones de como hacerlo en el \texttt{README.md} del repositorio. 

  Básicamente el procedimiento es:
  \begin{enumerate}
    \item Crea un fork del repositorio,
    \item agrega los cambios y contribuciones que desea al documento,
    \item y realiza un ``Pull Request''
  \end{enumerate}
  \end{quote}