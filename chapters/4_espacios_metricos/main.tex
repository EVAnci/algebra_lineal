\section{Espacios Métricos}

\subsection{Función distancia}

Sea \(E\) un conjunto no vacío de ``puntos'' se denomina función distancia a aquella función que a cada par de ``puntos'' asigna un número real, verificando determinadas condiciones:
\[
  d:E\times E \rightarrow \mathbb{R} \quad \text{tal que} \quad d(a,b) = h
\]
Condiciones:
\begin{itemize}
  \item \(d(a,b) = d(b,a)\)
  \item \(d(a,b) = 0 \quad \Longleftrightarrow \quad a=b\)
  \item \(d(a,c) \leq d(a,b) + d(b,c)\)
\end{itemize}
Recuerde que todo par ordenado de puntos de \(E\) determina un vector del espacio vectorial \(V(K):(a,b) \rightarrow \vec{ab}\) por esto al conjunto de puntos siempre se asocia un espacio vectorial.

Definida una función distancia, se designa al conjunto \(E\) como ``espacio métrico'', asociado al espacio vectorial \(V\).

\subsection{Función norma}

\subsubsection{En un espacio vectorial real \(V(\mathbb{R})\)}

En todo espacio vectorial \(V(\mathbb{R})\) es posible definir una función que designe a cada vector un escalar, verificando ciertas condiciones, esa función recibe el nombre de función norma:
\begin{align*}
  \left\lVert ~~ \right\rVert : V &\rightarrow \mathbb{R} \\
  u &\rightarrow \left\lVert u\right\rVert  
\end{align*}
Condiciones: para todo vector de \(V\):
\begin{itemize}
  \item \(\left\lVert u\right\rVert \geq 0 ~~ \land ~~ (\left\lVert u\right\rVert = 0 \Longleftrightarrow) u = \vec{0}\)
  \item \(\left\lVert t \cdot u\right\rVert = \left|t\right| \cdot \left\lVert u\right\rVert \quad\) para cualquier \(t \in \mathbb{R}\)
  \item \(\left\lVert u + v\right\rVert \leq \left\lVert u\right\rVert + \left\lVert v\right\rVert\) 
\end{itemize}
Definida esta función norma sobre \(V\), decimos que \(V, \left\lVert ~~\right\rVert\) es un \textbf{espacio normado}.

\paragraph{Consecuencias de la definición}

\[
  \left\lVert \vec{0}\right\rVert = 0 \qquad \left\lVert -u\right\rVert = \left\lVert u\right\rVert \qquad \left\lVert u - v\right\rVert = \left\lVert v - u\right\rVert
\]
\[
\left\lVert u\right\rVert = 1 \qquad \text{Es un vector normado}
\]
\[
\left\lVert u\right\rVert \neq 1 \qquad \text{Se puede normalizar:} \quad u' = \frac{1}{\left\lVert u\right\rVert} \rightarrow \left\lVert u'\right\rVert = 1
\]

\begin{quote}
  \ejemplo{ Algunas normas son:}
  \begin{enumerate}
    \item Sea el espacio vectorial \(\mathbb{R}(\mathbb{R})\), se define la función norma: \(\left\lVert ~~ \right\rVert: \mathbb{R} \rightarrow \mathbb{R}\), llamada \textbf{valor absoluto}. \[
      \left\lVert x\right\rVert = \left|x\right|
    \]
    \item En el espacio vectorial \(\mathbb{R}^n(\mathbb{R})\), se definen:
    \begin{itemize}
      \item La función \hl{norma uno}: \(\left\lVert ~~\right\rVert _1 : \mathbb{R}^n \rightarrow \mathbb{R}\)\[
        \left\lVert x\right\rVert _1 = \sum_{i=1}^{n} \left|x_i\right|
      \]
      \item La función \hl{norma dos}: \(\left\lVert ~~\right\rVert _2 : \mathbb{R}^n \rightarrow \mathbb{R}\) \[
        \left\lVert x\right\rVert _2 = \sqrt{\sum_{i=1}^{n}x_i^2}
      \]
      \item La función \hl{norma p}: \(\left\lVert ~~\right\rVert _p : \mathbb{R}^n \rightarrow \mathbb{R}\) \[
        \left\lVert x\right\rVert _p = \sqrt[p]{\sum_{i=1}^{n}x_i^p}
      \]
      \item La función \hl{norma infinito}: \(\left\lVert ~~\right\rVert _\infty : \mathbb{R}^n \rightarrow \mathbb{R}\) \[
        \left\lVert x\right\rVert _\infty = \max \left\{\left|x_i\right|; ~~ i = 1,2,\cdots,n\right\}
      \]
    \end{itemize}
    \item Sea el espacio vectorial \(C_{\left[a,b\right]} (\mathbb{R})\), el espacio vectorial de las funciones continuas definidas en un intervalo real \([a,b]\), se definen:
    \begin{itemize}
      \item La función \hl{norma dos}: \(\left\lVert ~~\right\rVert _2 : C_{[a,b]} \rightarrow \mathbb{R}\) \[
        \left\lVert f\right\rVert _2 = \sqrt{\int_{a}^{b}f^2 ~~ dx}
      \]
      \item La función \hl{norma infinito}: \(\left\lVert ~~\right\rVert _\infty : C_{[a,b]} \rightarrow \mathbb{R}\) \[
        \left\lVert f\right\rVert _\infty = \max \left\{\left|f(x)\right|; ~~ x \in [a,b]\right\}
      \]
    \end{itemize}
    \item Normas matriciales:
    \begin{itemize}
      \item La \hl{norma matricial a uno}: \(\left\lVert A\right\rVert _1 = \max _j \sum_{i_1}^{n} \left|a_{ij}\right|\), es el máximo de la suma de los coeficientes de las columnas en valor absoluto.
      \item La \hl{norma matricial infinito}: \(\left\lVert A\right\rVert _\infty = \max _i \sum_{j=1}^{n}\left|a_{ij}\right|\), es el máximo de la suma de los coeficientes del renglón o filas e valor absoluto.
      \item La \hl{norma matricial a dos}: \(\left\lVert A\right\rVert _2 = \sqrt{\max_{1\leq k \leq n}{(\lambda_k)}}\), siendo \(\lambda_k\) los valores propios de la matriz \(A \cdot A^T\). 
      \item La \hl{norma de Frobenius}: \(\left\lVert A\right\rVert _F = \sqrt{\sum_{i,k=1}^{n} a_{i,k}^2}\)
    \end{itemize}
  \end{enumerate}
\end{quote}

\subsubsection{Norma y distancia}

Siempre es posible definir una distancia inducida por una función norma tal que:
\[
  d(u,v) = \left\lVert u-v\right\rVert \qquad \text{o también} \qquad d(a,b) = \left\lVert a-b \right\rVert
\]
Por lo tanto, todo \textbf{espacio normado}, es también \textbf{espacio métrico}.

\subsection{Producto interior (Producto escalar)}

\subsubsection{En un espacio vectorial real \(V(\mathbb{R})\)}

En todo espacio vectorial \(V(\mathbb{R})\) es posible definir una función tal que a cada par de vectores le asigne un número real, cumpliendo ciertos requisitos, esta función recibe el nombre de \textbf{producto escalar} o producto interior.
\begin{align*}
  \left\langle ~\right\rangle  : V \times V &\rightarrow \mathbb{R} \\
  (u,v) & \rightarrow \left\langle u, v\right\rangle 
\end{align*}
Condiciones: Para todo \(u\), para todo \(v\) y para todo \(w\) del espacio \(V\):
\begin{itemize}
  \item \(\left\langle u,u \right\rangle \geq 0 ~ \land ~ \left(\left\langle u,u\right\rangle = 0 \Longleftrightarrow u = \vec{0}\right)\)
  \item \(\left\langle u,v\right\rangle = \left\langle v, u\right\rangle\)
  \item \(\left\langle u, v+w\right\rangle = \left\langle u,v\right\rangle + \left\langle u,w\right\rangle\)
  \item \(\left\langle u, t\cdot v\right\rangle = t \left\langle u,v\right\rangle \quad\) para cualquier \(t\in \mathbb{R}\) 
\end{itemize}
Definida esta función producto escalar sobre \(V\), decimos que \(\left(V, \left\langle ~ \right\rangle \right)\) es un espacio \textbf{pre-hilbertiano}.

\paragraph{Consecuencia de la definición}
\[
\left\langle u,v\right\rangle = 0 \qquad \text{entonces los vectores son ortogonales} 
\]
s
\begin{quote}
  \ejemplo{ Algunos ejemplos de producto interior:}
  \begin{enumerate}
    \item Sea el espacio vectorial \(\mathbb{R}^n (\mathbb{R})\), definimos el producto interior o producto escalar usual:
    \[
      \left\langle u,v\right\rangle = \sum_{i=1}^{n} u_i ~ v_i 
    \]
    \item Sea el espacio \(\mathbb{R}^2(\mathbb{R})\), para \(u = (x,y),~ v=(x',y')\), definimos el producto escalar:
    \[
      \left\langle u,v\right\rangle = x\cdot x' - y\cdot x' - x \cdot y' + 4 (y\cdot y') 
    \]
    \item En el espacio \(P_n (\mathbb{R})\) de los polinomios de grado menor o igual a \(n\):
    \[
      \left\langle u,v \right\rangle = \int_{0}^{1} u(x) ~ v(x) ~~ dx 
    \]
    \item En el espacio vectorial \(M_{n\times n} (\mathbb{R})\), de las matrices cuadradas de orden \(n\):
    \[
      \left\langle A, B\right\rangle = \text{tr} (B^T \cdot A) 
    \]
  \end{enumerate} 
\end{quote}

\subsubsection{Norma y distacia inducidas por el producto interior}

Siempre es posible definir una norma inducida por el producto escalar:
\[
\left\lVert u\right\rVert = + \sqrt{\left\langle u,u\right\rangle } 
\]
Por lo cual todo \textbf{espacio pre-hilbertiano} es también \textbf{espacio normado}.

Siempre es posible definir una distancia inducida por un producto escalar y su norma inducida:
\[
  d(u,v) = \left\lVert u-v\right\rVert = + \sqrt{\left\langle u-v, u-v\right\rangle } 
\]
Por lo cual todo \textbf{espacio pre-hilbertiano} es también un \textbf{espacio métrico}

\subsection{Espacios Complejos con producto interno}

Sea \(V\) un espacio vectorial sobre el cuerpo \(K=\mathbb{C}\) de números complejos, se define la función producto interior sobre \(\mathbb{C}\), que a cada par de vectores le asigna un número complejo:
\begin{align*}
  \left\langle ~ \right\rangle : V \times V &\rightarrow \mathbb{C} \\
  (u,v) &\rightarrow \left\langle u,v\right\rangle 
\end{align*}
Que cumple las siguientes propiedades:
\begin{itemize}
  \item \(\left\langle u, v+w\right\rangle = \left\langle u,v\right\rangle + \left\langle v,w\right\rangle\)
  \item \(\left\langle t \cdot u, v\right\rangle = \bar{t} \cdot \left\langle u,v\right\rangle \quad\) donde \(t\in \mathbb{C}\) y \(\bar{t}\) es su conjugado.
  \item \(\left\langle u,v\right\rangle = \overline{\left\langle v,u\right\rangle} \quad\) (conjugado de \(\left\langle u,v\right\rangle\))
  \item \(\left\langle u,u\right\rangle \geq 0 \land \left\langle u,u\right\rangle = 0 \Longleftrightarrow u=\vec{0}\)
\end{itemize}
Este producto escalar se denomina ``\textbf{producto hermítico}''

Definida esta función producto interior sobre el espacio complejo \(V(\mathbb{C})\), decimos que \(V, \left\langle ~ \right\rangle\) es un \textbf{espacio unitario}.

\begin{quote}
  \ejemplo{ Algunos ejemplos de producto interior sobre \(\mathbb{C}\)}:
  \begin{enumerate}
    \item Sea el espacio complejo \(\mathbb{C}^n (\mathbb{C})\), para \(u=(x_1, \cdots , x_n), ~ v=(x'_1, \cdots, x'_n)\), se define el \textbf{producto interno canónico}:
    \[
      \left\langle u,v\right\rangle = x_1 \cdot \bar{x}'_1 + \cdots + x_n \cdot \bar{x}'_n
    \]
    \item Sea \(C^*_{[a,b]}\) el espacio de las funciones definidas en el intervalo \([a,b]\) con valores complejos, es decir, cada función es de la forma: \(f(t)=f_1(t)+if_2(t)\), donde \(f_1(t)\) y \(f_2(t)\) pertenece a \(C_{[a,b]}\), se define el producto interior:
    \[
      \left\langle f, g\right\rangle = \int_{a}^{b} f(t) ~\overline{g(t)} ~~ dt 
    \]
    \item En el espacio \(M_n (C)\) de matrices cuadradas complejas de orden \(n\geq 1\), se define el producto interior canónico dado por:
    \[
      \left\langle A,B\right\rangle = \sum_{i,j=1}^{n} a_{ij} ~ \bar{b}_{ij} 
    \]
  \end{enumerate}
\end{quote}
La mayoría de las propiedades de los espacios pre-hilbertianos se mantienen para los espacios unitarios.

Siempre es posible definir una función norma inducida por un producto hermítico, del mismo modo que en el espacio real:
\[
  d(w_1, w_2) = \left\lVert (w_1 + w_2)\right\rVert = + \sqrt{\left\langle (w_1 - w_2), (w_1 - w_2)\right\rangle } 
\]
Siempre teniendo en cuenta que la distancia asigna siempre un numero real, por lo tanto, en los espacios unitarios también se verifica esa condición.

\subsection{Bases ortogonales y ortonormadas}

\subsubsection{Vectores y bases ortogonales}

Sea \(V(K)\) un espacio vectorial, en el que se ha definido un producto escalar, y resulta que para un par cualquiera de vectores, \(u,v\) de \(V\) el producto escalar es nulo, entonces esos vectores son ortogonales.
\[
(u \in V ~ \land ~ v \in V) \land \left\langle u,v\right\rangle = 0 \quad \rightarrow \quad \text{Los vectores son ortogonales}
\]
\textbf{Nota}: El vector nulo es ortogonal a cualquier vector \(u\) de \(V\).

Una familia de vectores de \(V:F=\left\{u_1,u_2,\cdots, u_n\right\}\) es un \textbf{conjunto ortogonal} si para todo par de vectores distintos de \(F\) se verifica que son ortogonales, es decir, que todos los vectores del conjunto \(F\) son ortogonales entre sí.
\[
F ~ \text{es conjunto ortogonal} ~ \Longleftrightarrow ~ \left(\forall (u_i,v_i)\in V^2\right) | i \neq j : \left\langle u_i, u_j\right\rangle = 0 
\]
Una familia \(F\) de vectores de \(V\) que sea base de dicho espacio vectorial y también conjunto ortogonal, se denomina \textbf{base ortogonal}.

\subsubsection{Vectores y bases normados}

Sea \(V(K)\) un espacio vectorial en el cual se ha definido una norma, si resulta que para algún vector \(u\) de \(V\) se verifica que su norma es la unidad, ese vector se denomina normado.
\[
  u \in V ~ \land ~ \left\lVert u\right\rVert = 1 \quad \rightarrow \quad u~\text{es un vector normado}
\]
Como vimos un vector que no es normado es posible normalizarlo, siempre y cuando no sea el vector nulo, de la siguiente manera:
\[
  \left\lVert u\right\rVert \neq 1 \land u \neq 0 \implies u' = \frac{1}{\left\lVert u\right\rVert} \cdot u \quad \rightarrow \quad \text{es un vector normado}
\]

Si \(F\) es una familia de vectores de \(V:F={u_1,u_2,\cdots,u_n}\) en la cual todos sus vectores son normados, \(F\) se denomina \textbf{conjunto normado}.
\[
  F ~ \text{es conjunto normado} \quad \Longleftrightarrow \quad \left(\forall u_i \in F\right): \left\lVert u_i\right\rVert = 1 
\]
Si la familia \(F\) es conjunto normado y además es una base de \(V\) se denomina \textbf{base normada}.

\textbf{Nota}: si una familia de vectores de \(V\) es base ortogonal y normada se denomina \textbf{base ortonormada}.