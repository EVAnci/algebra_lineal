\section{Espacios Métricos}

\subsection{Función distancia}

Sea \(E\) un conjunto no vacío cuyos elementos se denominan ``puntos''. Se llama \textit{función distancia} a toda aplicación que a cada par de puntos de \(E\) les asigna un número real, verificando las siguientes propiedades:

\[
  d: E \times E \rightarrow \mathbb{R}, \quad \text{tal que } d(a,b) = h
\]

\begin{itemize}
  \item Simetría: \(d(a,b) = d(b,a)\)
  \item Identidad: \(d(a,b) = 0 \iff a=b\)
  \item Desigualdad triangular: \(d(a,c) \leq d(a,b) + d(b,c)\)
\end{itemize}

Un par \((E, d)\), donde \(E\) es un conjunto y \(d\) una función distancia definida sobre él, se denomina \hl{espacio métrico}.

\begin{tcolorbox}[remember, title=Observación]
  En el caso particular en que \(E\) sea un subconjunto de un espacio vectorial \(V(K)\), como por ejemplo \(\mathbb{R}^n\), es posible asociar a cada par ordenado de vectores \((a,b)\) el vector de desplazamiento \(\vec{ab} = b - a = d(a,b)\). En tales contextos, la distancia entre puntos puede definirse a partir de la norma de dicho vector, por ejemplo, utilizando la norma euclídea (o pre-hilbertiana):
  \[
    d(a,b) = \| b - a \| = \sqrt{(b_1 - a_1)^2 + \cdots + (b_n - a_n)^2}
  \]
  En este caso, el espacio métrico sería \(\left(\mathbb{R}^n,\left\lVert a-b \right\rVert \right)\), aunque profundizaremos más sobre esto en la próxima sección.

  Sin embargo, es importante destacar que la definición general de espacio métrico no requiere que \(E\) tenga estructura vectorial; basta con que exista una función distancia que cumpla las propiedades mencionadas.
\end{tcolorbox}

\subsection{Función norma}

\subsubsection{En un espacio vectorial real \(V(\mathbb{R})\)}

Sea \(V(\mathbb{R})\) un espacio vectorial sobre \(\mathbb{R}\). Se denomina \textit{norma} a una función que asigna a cada vector \(u \in V\) un número real no negativo, cumpliendo las siguientes condiciones:
\begin{align*}
  \left\lVert \cdot \right\rVert : V &\longrightarrow \mathbb{R} \\
  u &\longmapsto \left\lVert u \right\rVert
\end{align*}
Para todo \(u, v \in V\) y \(t \in \mathbb{R}\), la función \(\left\lVert \cdot \right\rVert\) debe verificar:

\begin{itemize}
  \item \textbf{Positividad:} \(\left\lVert u \right\rVert \geq 0\), y \(\left\lVert u \right\rVert = 0 \iff u = \vec{0}\)
  \item \textbf{Homogeneidad absoluta:} \(\left\lVert t \cdot u \right\rVert = \left|t\right| \cdot \left\lVert u \right\rVert\)
  \item \textbf{Desigualdad triangular:} \(\left\lVert u + v \right\rVert \leq \left\lVert u \right\rVert + \left\lVert v \right\rVert\)
\end{itemize}

Cuando se ha definido una función norma sobre el espacio vectorial \(V\), al par \((V, \left\lVert \cdot \right\rVert)\) se lo denomina \textbf{espacio normado}.

\vspace{5mm}

\paragraph{Consecuencias de la definición}

A partir de las propiedades que cumple la norma, se deducen inmediatamente los siguientes resultados:

\[
  \left\lVert \vec{0} \right\rVert = 0, \qquad
  \left\lVert -u \right\rVert = \left\lVert u \right\rVert, \qquad
  \left\lVert u - v \right\rVert = \left\lVert v - u \right\rVert
\]

\paragraph{Vector unitario (o normalizado):}

Un vector \(u \in V\) tal que \(\left\lVert u \right\rVert = 1\) se denomina \textbf{vector unitario}. En general, si \(u \neq \vec{0}\), siempre es posible normalizarlo mediante:
\[
u' = \frac{u}{\left\lVert u \right\rVert} \quad \Rightarrow \quad \left\lVert u' \right\rVert = 1
\]

\noindent \ejemplo{ Algunas funciones norma son las siguientes}
\begin{enumerate}

  \item \textbf{En \(\mathbb{R}\): valor absoluto}

  Sea \(E = \mathbb{R}\). La norma natural sobre este espacio es el valor absoluto:
  \[
    \left\lVert x \right\rVert = |x|
  \]

  \item \textbf{En \(\mathbb{R}^n\): normas \(p\)}

  Sea \(x = (x_1, x_2, \dots, x_n) \in \mathbb{R}^n\). Se definen distintas normas en función de su forma:

  \begin{itemize}
    \item \textbf{Norma uno (o norma \(L^1\)):}
    \[
      \left\lVert x \right\rVert_1 = \sum_{i=1}^{n} |x_i|
    \]

    \item \textbf{Norma dos (o euclídea, o \(L^2\)):}
    \[
      \left\lVert x \right\rVert_2 = \sqrt{\sum_{i=1}^{n} x_i^2}
    \]

    \item \textbf{Norma \(p\) (para \(p \geq 1\)):}
    \[
      \left\lVert x \right\rVert_p = \left( \sum_{i=1}^{n} |x_i|^p \right)^{\frac{1}{p}}
    \]

    \item \textbf{Norma infinito (o \(L^\infty\)):}
    \[
      \left\lVert x \right\rVert_\infty = \max \{ |x_i| : 1 \leq i \leq n \}
    \]
  \end{itemize}

  \item \textbf{En \(C([a,b])\): funciones continuas en un intervalo}

  Sea \(f \in C([a,b])\), es decir, una función continua definida sobre \([a,b]\). Se definen:

  \begin{itemize}
    \item \textbf{Norma \(L^2\):}
    \[
      \left\lVert f \right\rVert_2 = \sqrt{ \int_a^b f(x)^2 \, dx }
    \]

    \item \textbf{Norma supremo (o norma \(L^\infty\)):}
    \[
      \left\lVert f \right\rVert_\infty = \max_{x \in [a,b]} |f(x)|
    \]
  \end{itemize}

  \item \textbf{Normas matriciales}

  Sea \(A = (a_{ij}) \in \mathbb{R}^{n \times n}\). Se definen:

  \begin{itemize}
    \item \textbf{Norma matricial uno:}
    \[
      \left\lVert A \right\rVert_1 = \max_{1 \leq j \leq n} \sum_{i=1}^{n} |a_{ij}|
    \]
    Es el máximo valor absoluto de las sumas por columnas.

    \item \textbf{Norma matricial infinito:}
    \[
      \left\lVert A \right\rVert_\infty = \max_{1 \leq i \leq n} \sum_{j=1}^{n} |a_{ij}|
    \]
    Es el máximo valor absoluto de las sumas por filas.

    \item \textbf{Norma matricial dos:}
    \[
      \left\lVert A \right\rVert_2 = \sqrt{ \lambda_{\max} }
    \]
    donde \(\lambda_{\max}\) es el mayor valor propio de \(A^T A\). Esta norma es inducida por la norma euclídea.

    \item \textbf{Norma de Frobenius:}
    \[
      \left\lVert A \right\rVert_F = \sqrt{ \sum_{i=1}^{n} \sum_{j=1}^{n} a_{ij}^2 }
    \]
    Coincide con la norma euclídea sobre el espacio \(\mathbb{R}^{n^2}\).
  \end{itemize}

\end{enumerate}

\subsubsection{Norma y distancia}

Siempre es posible definir una distancia inducida por una función norma tal que:
\[
  d(u,v) = \left\lVert u-v\right\rVert \qquad \text{o también} \qquad d(a,b) = \left\lVert a-b \right\rVert
\]
Por lo tanto, todo \textbf{espacio normado}, es también \textbf{espacio métrico}.

\subsection{Producto interior (producto escalar)}

\subsubsection{En un espacio vectorial real \(V(\mathbb{R})\)}

Sea \(V(\mathbb{R})\) un espacio vectorial sobre el cuerpo \(\mathbb{R}\). Se denomina \textbf{producto escalar} (o \textbf{producto interior}) a una aplicación que asigna a cada par de vectores un número real, cumpliendo ciertas propiedades:
\begin{align*}
  \left\langle \cdot, \cdot \right\rangle : V \times V &\longrightarrow \mathbb{R} \\
  (u,v) &\longmapsto \left\langle u, v \right\rangle
\end{align*}
Para todos \(u, v, w \in V\) y para todo escalar \(t \in \mathbb{R}\), el producto escalar debe verificar:

\begin{itemize}
  \item \textbf{Positividad:} \(\left\langle u, u \right\rangle \geq 0\), y \(\left\langle u, u \right\rangle = 0 \iff u = \vec{0}\)
  \item \textbf{Simetría:} \(\left\langle u, v \right\rangle = \left\langle v, u \right\rangle\)
  \item \textbf{Linealidad en la segunda variable:}
  \[
    \left\langle u, v + w \right\rangle = \left\langle u, v \right\rangle + \left\langle u, w \right\rangle
    \quad \text{y} \quad
    \left\langle u, t v \right\rangle = t \left\langle u, v \right\rangle
  \]
\end{itemize}

Cuando un producto escalar está definido sobre el espacio vectorial \(V\), al par \(\left(V, \left\langle \cdot, \cdot \right\rangle\right)\) se lo denomina un \textbf{espacio prehilbertiano}.

\paragraph{Consecuencias de la definición}
\[
\left\langle u, v \right\rangle = 0 \quad \Rightarrow \quad \text{los vectores \(u\) y \(v\) son ortogonales}
\]

\ejemplo{ Algunos productos interiores}

\begin{enumerate}
  \item \textbf{Producto escalar usual en \(\mathbb{R}^n\):}

  Sea \(u = (u_1, \dots, u_n)\) y \(v = (v_1, \dots, v_n)\), se define:
  \[
    \left\langle u, v \right\rangle = \sum_{i=1}^{n} u_i v_i
  \]

  \item \textbf{Producto escalar no usual en \(\mathbb{R}^2\):}

  Para \(u = (x, y)\), \(v = (x', y')\), definimos:
  \[
    \left\langle u, v \right\rangle = x x' - y x' - x y' + 4 y y'
  \]
  Este producto escalar es válido si cumple las propiedades definitorias.

  \item \textbf{Producto escalar en el espacio \(P_n(\mathbb{R})\):}

  Para polinomios reales de grado menor o igual a \(n\):
  \[
    \left\langle u, v \right\rangle = \int_0^1 u(x) v(x) \, dx
  \]

  \item \textbf{Producto escalar en el espacio \(M_{n \times n}(\mathbb{R})\):}

  Para matrices reales cuadradas de orden \(n\):
  \[
    \left\langle A, B \right\rangle = \mathrm{tr}(B^T A)
  \]
  donde \(\mathrm{tr}(\cdot)\) denota la traza de una matriz.
\end{enumerate}

\subsubsection{Norma y distancia inducidas por el producto interior}

Todo producto interior induce una norma sobre el espacio vectorial \(V\), definida por:
\[
\left\lVert u \right\rVert = \sqrt{ \left\langle u, u \right\rangle }
\]
Por lo tanto, todo \textbf{espacio prehilbertiano} es también un \textbf{espacio normado}.

Además, esta norma induce una distancia (o métrica) sobre \(V\), dada por:
\[
d(u, v) = \left\lVert u - v \right\rVert = \sqrt{ \left\langle u - v, u - v \right\rangle }
\]
Por lo tanto, todo \textbf{espacio prehilbertiano} es también un \textbf{espacio métrico}.

\subsection{Espacios complejos con producto interno}

Sea \(V\) un espacio vectorial sobre el cuerpo \(\mathbb{C}\) de los números complejos. Se denomina \textbf{producto interior} (o \textbf{producto hermítico}) a una función que asigna a cada par de vectores un número complejo, cumpliendo ciertas propiedades fundamentales:

\begin{align*}
  \left\langle \cdot , \cdot \right\rangle : V \times V &\longrightarrow \mathbb{C} \\
  (u, v) &\longmapsto \left\langle u, v \right\rangle
\end{align*}

Para todo \(u, v, w \in V\) y \(t \in \mathbb{C}\), se deben cumplir las siguientes propiedades:

\begin{itemize}
  \item \textbf{Linealidad en la segunda variable:}
  \[
    \left\langle u, v + w \right\rangle = \left\langle u, v \right\rangle + \left\langle u, w \right\rangle
  \]
  \item \textbf{Conjugación en la primera variable:}
  \[
    \left\langle t \cdot u, v \right\rangle = \overline{t} \cdot \left\langle u, v \right\rangle
  \]
  \item \textbf{Hermiticidad:}
  \[
    \left\langle u, v \right\rangle = \overline{ \left\langle v, u \right\rangle }
  \]
  \item \textbf{Positividad definida:}
  \[
    \left\langle u, u \right\rangle \in \mathbb{R}, \quad \left\langle u, u \right\rangle \geq 0, \quad \text{y} \quad \left\langle u, u \right\rangle = 0 \iff u = \vec{0}
  \]
\end{itemize}

Definido un producto interior que cumple estas propiedades, al par \(\left(V, \left\langle \cdot , \cdot \right\rangle\right)\) se lo denomina un \textbf{espacio unitario}.

\ejemplo{Ejemplos de producto interior sobre \(\mathbb{C}\):}

\begin{enumerate}
  \item \textbf{Producto interior canónico en \(\mathbb{C}^n\):}

  Para \(u = (x_1, \dots, x_n)\) y \(v = (x'_1, \dots, x'_n)\), se define:
  \[
    \left\langle u, v \right\rangle = x_1 \cdot \overline{x'_1} + \cdots + x_n \cdot \overline{x'_n}
    = \sum_{k=1}^{n} x_k \cdot \overline{x'_k}
  \]

  \item \textbf{Producto interior en el espacio \(C^*_{[a,b]}\):}

  Sea \(C^*_{[a,b]}\) el conjunto de funciones continuas con valores complejos definidas en \([a,b]\), es decir:
  \[
    f(t) = f_1(t) + i f_2(t), \quad \text{con } f_1, f_2 \in C_{[a,b]}(\mathbb{R})
  \]
  El producto interior se define como:
  \[
    \left\langle f, g \right\rangle = \int_{a}^{b} f(t) \cdot \overline{g(t)} \, dt
  \]

  \item \textbf{Producto interior en \(M_n(\mathbb{C})\):}

  Sea \(M_n(\mathbb{C})\) el espacio de matrices complejas cuadradas de orden \(n\). Se define:
  \[
    \left\langle A, B \right\rangle = \sum_{i,j=1}^{n} a_{ij} \cdot \overline{b_{ij}} 
  \]
  Este producto también puede expresarse como:
  \[
    \left\langle A, B \right\rangle = \mathrm{tr}(B^* A)
  \]
  donde \(B^*\) denota la matriz adjunta de \(B\) (conjugada transpuesta).
\end{enumerate}

Una gran parte de las propiedades de los espacios prehilbertianos se extiende naturalmente a los espacios unitarios.

\paragraph{Norma y distancia inducidas por el producto hermítico}

Como en el caso real, todo producto hermítico induce una norma sobre el espacio \(V\):
\[
  \left\lVert u \right\rVert = \sqrt{ \left\langle u, u \right\rangle } \quad \in \mathbb{R}_{\geq 0}
\]

A partir de esta norma, se puede definir una función distancia (o métrica) mediante:
\[
  d(u, v) = \left\lVert u - v \right\rVert = \sqrt{ \left\langle u - v, u - v \right\rangle }
\]

Esta distancia es siempre un número real no negativo, por lo que todo \textbf{espacio unitario} es también un \textbf{espacio métrico}.

\subsection{Bases ortogonales y ortonormadas}

\subsubsection{Vectores y bases ortogonales}

Sea \(V(K)\) un espacio vectorial en el que se ha definido un producto interior. Dados \(u, v \in V\), si
\[
\left\langle u, v \right\rangle = 0,
\]
entonces los vectores \(u\) y \(v\) se dicen \textbf{ortogonales}.

\textbf{Nota:} El vector nulo \(\vec{0}\) es ortogonal a todo vector de \(V\).

Una familia de vectores \(F = \{u_1, u_2, \dots, u_n\}\) se denomina un \textbf{conjunto ortogonal} si:
\[
\left\langle u_i, u_j \right\rangle = 0 \quad \text{para todo } i \neq j
\]

Si además el conjunto \(F\) es una base del espacio \(V\), entonces se dice que \(F\) es una \textbf{base ortogonal}.

\subsubsection{Vectores y bases normalizados}

Sea \(V(K)\) un espacio vectorial provisto de una norma. Un vector \(u \in V\) se dice \textbf{normalizado} o \textbf{unitario} si su norma es igual a uno:
\[
\left\lVert u \right\rVert = 1
\]

Todo vector no nulo puede ser normalizado mediante:
\[
u' = \frac{u}{\left\lVert u \right\rVert}, \quad \text{de modo que } \left\lVert u' \right\rVert = 1
\]

Una familia de vectores \(F = \{u_1, u_2, \dots, u_n\}\) se dice un \textbf{conjunto normalizado} si:
\[
\left\lVert u_i \right\rVert = 1 \quad \text{para todo } i
\]

Si además el conjunto \(F\) es una base del espacio \(V\), se dice que \(F\) es una \textbf{base normalizada}.

\subsubsection{Bases ortonormadas}

Una familia de vectores de \(V\) que sea simultáneamente ortogonal y normalizada se denomina \textbf{ortonormada}. Es decir:
\[
\left\langle u_i, u_j \right\rangle = 
\begin{cases}
1 & \text{si } i = j \\
0 & \text{si } i \neq j
\end{cases}
\]

Si esta familia constituye además una base de \(V\), entonces se dice que es una \textbf{base ortonormada}.

\begin{table}[H]
  \centering
  \begin{tabular}{c|c|c}
    \textbf{Tipo de base} & \textbf{Vectores ortogonales} & \textbf{Vectores normalizados} \\
    \hline
    Arbitraria & No necesariamente & No necesariamente \\
    Ortogonal & Sí: \(\left\langle u_i, u_j \right\rangle = 0\) si \(i \neq j\) & No necesariamente \\
    Ortonormal & Sí: \(\left\langle u_i, u_j \right\rangle = 0\) si \(i \neq j\) & Sí: \(\left\lVert u_i \right\rVert = 1\) \\
  \end{tabular}
  \caption{Comparación entre bases en espacios con producto interior}
\end{table}

Una base ortonormal es el caso más estructurado: los vectores son mutuamente ortogonales y además tienen norma unitaria. Toda base ortonormal es ortogonal, pero no toda base ortogonal está normalizada.