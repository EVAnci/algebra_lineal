\subsection{Aplicación a los sistemas de ecuaciones lineales I}
\label{sec:appl_sels_1}

\begin{tcolorbox}[remember, title=Aclaración]
  En este apartado se verá una introducción con los conocimientos que se tienen, sobre cómo se aplican las transformaciones lineales a los SEL's. Más adelante veremos la segunda parte después de abordar otros conceptos fundamentales.
\end{tcolorbox}

Consideremos un sistema de \(m\) ecuaciones lineales con \(n\) incógnitas sobre un cuerpo \(K\):
\begin{align*}
  a_{11} x_1 + a_{12} x_2 + \cdots + a_{1n} x_n &= b_1 \\
  a_{21} x_1 + a_{22} x_2 + \cdots + a_{2n} x_n &= b_2\\
  \vdots  \qquad \qquad ~ & \\
  a_{m1} x_1 + a_{m2} x_2 + \cdots + a_{mn} x_n &= b_m
\end{align*}
Este sistema puede escribirse de forma matricial como:
\[
  A x = b
\]
donde \(A = (a_{ij})\) es la matriz de coeficientes del sistema, \(x = (x_i)\) es el vector columna de incógnitas, y \(b = (b_i)\) es el vector columna de constantes.

La matriz \(A\), vista como una aplicación lineal
\[
  A: K^n \rightarrow K^m
\]
permite interpretar el sistema \(Ax = b\) como la búsqueda de un vector \(x \in K^n\) cuya imagen bajo \(A\) sea \(b\). Es decir, estamos buscando preimágenes de \(b\) bajo la transformación lineal definida por \(A\).

En particular, el conjunto de soluciones del sistema homogéneo asociado \(Ax = 0\) coincide con el núcleo de la aplicación lineal \(A: K^n \to K^m\), es decir:
\[
  \text{Ker } A = \left\{ x \in K^n \mid Ax = 0 \right\}
\]
Aplicando el teorema del rango y la nulidad (teorema \ref{teo:teorema_dimensión_nucleo_imagen}), que establece que para toda aplicación lineal \(F: V \rightarrow U\) con \(V\) de dimensión finita:
\[
  \dim V = \dim(\text{Ker } F) + \dim(\text{Im } F)
\]
se obtiene, en este contexto particular, que:
\[
  \dim(\text{Ker } A) = \dim(K^n) - \dim(\text{Im } A) = n - \text{rango } A
\]
donde el rango de \(A\) coincide con la dimensión de la imagen de la transformación lineal asociada a la matriz \(A\).

\begin{tcolorbox}[title=Interpretación]
  La \textbf{nulidad} del sistema (número de parámetros libres en la solución del sistema homogéneo) está dada por la diferencia entre el número de incógnitas \(n\) y el rango de la matriz \(A\). Esta relación resulta especialmente útil para determinar la existencia y el número de soluciones, tanto del sistema homogéneo como del sistema no homogéneo.
\end{tcolorbox}
