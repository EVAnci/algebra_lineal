\section{Funciones Lineales}

Una función lineal (también llamada transformación lineal o aplicación lineal) es una función entre dos espacios vectoriales que preserva la estructura algebraica de los vectores, es decir, respeta la suma de vectores y la multiplicación por escalares.

\textbf{Definición}: Sea \(f: V \to U\) una función entre espacios vectoriales sobre el mismo cuerpo \(K\). 
\begin{align*}
  f: V &\rightarrow U\\
  v &\rightarrow u
\end{align*}
que a cada vector \(v\) de \(V\) le hace corresponder un único vector \(f(v)\) de \(U\).

Diremos que \(f\) es una aplicación lineal si para todo \(v, w \in V\) y todo escalar \(t \in K\), se cumple:
\begin{itemize}
  \item \(f(v + w) = f(v) + f(w)\)
  \item \(f(t \cdot v) = t \cdot f(v)\)
\end{itemize}

La primera condición se la conoce como \textit{aditividad}, indica que la función respeta las operaciones internas definidas en los espacios vectoriales involucrados, la segunda se denomina \textit{homogeneidad} y está indicando que la función también respeta las operaciones externas. 

\begin{tcolorbox}[remember, title=Aclaración]
  \(V(K)\) y \(U(K)\) son dos espacios vectoriales definidos sobre un mismo cuerpo \(K\) (por ejemplo, \(\mathbb{R}\) o \(\mathbb{C}\)).
  
  La función \(f: V \rightarrow U\) toma un vector \(v \in V\) y lo lleva a un vector \(f(v) \in U\).
  
  Es importante entender que tanto el dominio como el codominio son espacios vectoriales, por lo tanto, en ambos hay operaciones de suma y producto por escalar.
\end{tcolorbox}

En otras palabras, una función \(f: V \rightarrow U\) es lineal si cumple dos condiciones fundamentales para todo \(v, w \in V\) y \(t \in K\):
\begin{enumerate}
  \item Aditividad (respeta la suma):
    \[
     f(v + w) = f(v) + f(w)
    \]
  \item Homogeneidad (respeta el producto por escalares):
    \[
      f(t \cdot v) = t \cdot f(v)
    \]
\end{enumerate}

Estas dos propiedades garantizan que la función ``conserva'' la estructura lineal. En otras palabras, las combinaciones lineales en el espacio de partida se transforman en combinaciones lineales en el espacio de llegada, de la misma forma.

\paragraph{Propiedades de una función lineal}

Si \(f\) es una función lineal de \(V(K)\) en \(U(K)\) entonces:
\begin{enumerate}
  \item \(f\left(\vec{0}\right) = \vec{0}, \qquad \left(\vec{0} \in V,\vec{0} \in U\right)\)

  Demostración:\[
    f\left(\vec{0}\right) = f\left(0 \cdot v\right) = 0 \cdot f(v) = \vec{0}
  \]
  \item \(f(-v)=-f(v), \quad \forall v \in V\)
  
  Demostración: \[
    f(-v) = f(-1 \cdot v) = -1 \cdot f(v) = -f(v)
  \]
  \item \(f(v-w)=f(v) - f(w) \quad \forall v,w \in V\)
  
  Demostración: \[
    f(v-w) = f(v) + f(-w) = f(v) + (-f(w)) = f(v) - f(w)
  \]
\end{enumerate}

\begin{tcolorbox}[interesting_data, title=Nota conceptual]
  Las propiedades desarrolladas son sumamente importantes. Por ejemplo la primer propiedad distingue claramente las funciones lineales de otras funciones. Por ejemplo, si una función \(f\) no cumple que \(f\left(\vec{0}\right) = \vec{0}\), entonces automáticamente no es lineal.
\end{tcolorbox}

Ahora para todo par de escalares \(a,b \in K\) y para todo par de vectores \(v,w \in V\), obtenemos imponiendo ambas condiciones de linealidad:
\[
  F(av + bw) = a\cdot f(v) + b\cdot f(w) \qquad \text{se utiliza para definirlas}
\]
Con mayor generalidad, para escalares \(a_i \in K\) y vectores \(v_i \in V\), llegamos:
\[
  F(a_1 v_1 + a_2 v_2 + \cdots + a_n v_n) = a_1 f(v_1) + a_2 f(v_2) + \cdots + a_n f(v_n)
\]
\begin{tcolorbox}[myconclusion]
  Expresión utilizada para demostrar los teoremas.
\end{tcolorbox}

\begin{quote}
  \ejemplo{ Veamos algunos ejemplos}
  \begin{enumerate}[label=\alph*.]
    \item Sea \(A\) cualquier matriz \(m \times n\) sobre un cuerpo \(K\). Como se señaló previamente, \(A\) determina una aplicación \(F:K^n \rightarrow K^m\) mediante la asignación \(v \rightarrow Av\) (aquí los vectores de \(K^n\) y \(K^m\) se escriben como columnas). Afirmamos que \(F\) es lineal. Esto es debido a que, por propiedades de las matrices:
    \begin{align*}
      F(v + w) &= A(v+w) = Av + Aw = F(v) + F(w) \\
      F(kv) &= A(kv) = kAv = kF(v)
    \end{align*}
    donde \(v, w \in K^n\) y \(k \in K\).
    \item Sea \(F:\mathbb{R}^3 \rightarrow \mathbb{R}^3\) la aplicación <<proyección>> en el plano \(xy: F(x,y,z)= (x,y,0)\). Probemos que \(F\) es lineal. Sean \(v=(a,b,c)\) y \(w=(a',b',c')\). Entonces:
    \begin{align*}
      F(v+w) &= F(a+a', b+b', c+c') = (a+a', b+b',0) = \\
            &= (a,b,0) + (a',b',0) = F(v) + F(w)
    \end{align*}
    y para todo \(k \in \mathbb{R}\):
    \[
      F(kv) = F(ka,kb,kc) = (ka,kb,0) = k(a,b,0) = kF(v)
    \]
    O sea, \(F\) es lineal.
    \item Sea \(F: \mathbb{R}^2 \rightarrow \mathbb{R}^2\) la aplicación de <<traslación>> definida según \(F(x,y) = (x+1,y+2)\). Obsérvese que \(F(0)=(0,0)=(1,2)\neq 0\). Es decir, el vector cero no se aplica sobre el vector cero. Por consiguiente \(F\) no es lineal.
    \item Sea \(F:V\rightarrow U\) la aplicación que asigna \(0 \in U\) a todo \(v \in V\). Para todo par ded vectores \(v,w \in V\) y todo \(k \in K\) tenemos:
    \[
      F(v+w) = 0 = 0+0 = F(v) + F(w) \qquad \text{y} \qquad F(kv) = 0 = k0 = kF(v)
    \]
    Así \(F\) es lineal. Llamamos a \(F\) la \textit{aplicación cero} y la denotaremos normalmente por \(0\).
    \item Consideremos la aplicación identidad \(I:V\rightarrow V\) que aplica cada \(v\in V\) en si mismo. Para todo par de vectores \(v, w \in V\) y todo par de escalares \(a,b \in K\):
    \[
      I(av+bw)= av + bw = aI(v) + bI(w)
    \]
    Así \(I\) es lineal.
    \item Sea \(V\) el espacio vectorial de los polinomios en la variable \(t\) sobre el cuerpo real \(\mathbb{R}\). La aplicación derivada \(\mathbf{D}:V\rightarrow V\) y la aplicación integral \(\mathbf{J}:V\rightarrow \mathbb{R}\), son lineales. La razón es que, según se demuestra en el cálculo, para todo par de vectores \(u,v \in V\) y todo \(k\in \mathbb{R}\):
    \[
      \frac{d(u+v)}{dt} = \frac{du}{dt} + \frac{dv}{dt} \qquad \text{y} \qquad \frac{ku}{dt} = k\frac{du}{dt}
    \]
  \end{enumerate}
 \end{quote}

\subsubsection{Clasificación de las funciones lineales}

Se tiene una función lineal \(f\) de \(V(K)\) en \(W(K)\):
\begin{itemize}
  \item Si \(f\) es inyectiva recibe el nombre de \textit{monomorfismo}
  \item Si \(f\) es suryectiva recibe el nombre de \textit{epimorfismo}
  \item Si \(f\) es biyectiva recibe el nombre de \textit{isomorfismo}
  \item Si en \(f \quad V=W\) recibe el nombre de \textit{endomorfismo}
  \item Si \(f\) es un endomorfismo y es biyectiva recibe el nombre de \textit{automorfismo} 
\end{itemize}

\subsubsection{Núcleo e imagen de una función lineal}

Sea \(f\) una función lineal de \(V(K)\) en \(U(K)\) \(\left(f:V\rightarrow U\right)\), asociados a esta función existen dos subconjuntos, uno del conjunto de partida y otro del conjunto de llegada, que reciben el nombre de \textit{núcleo} (o \textit{Ker}) de la función e imagen de la función, los definimos:
\[
\text{Ker } f = N(f) = \left\{ v \in V \mid f(v) = \vec{0}_U \right\}
\]
\begin{center}
  son todos los vectores de \(V\) que tienen como imagen al vector nulo de \(U\)
\end{center}
\[
\text{Im } f = \left\{u\in U \mid \exists v \in V ~\text{ tal que }~ f(v) = u \right\}
\]
o también en una forma más abreviada:
\[
\text{Im } f = \left\{f(v) \mid v \in V\right\}
\]
\begin{center}
  son todos los vectores de \(U\) que son imagen de algún vector \(v \in V\)
\end{center}

Los subconjuntos \(N(f)\) e \(\text{Im } f\) son subespacios de \(V\) y de \(U\) respectivamente.

\begin{quote}
  \ejemplo{ Consideremos la siguiente función lineal:}
  \[
    f(x) = 3x
  \]
  Esta es una función lineal \(f:\mathbb{R} \rightarrow \mathbb{R}\).
  \begin{itemize}
    \item \textbf{Núcleo:}
    \[
      \text{Ker } f = \left\{x \in \mathbb{R} \mid f(x) = 0\right\} = \{0\}
    \]
    \item \textbf{Imagen:}
    \[
      \text{Im } f = \left\{f(x) \mid x \in \mathbb{R}\right\} = \left\{3x \mid x \in \mathbb{R}\right\} = \mathbb{R}
    \]
  \end{itemize}
  Aquí el \textit{único} elemento que se anula es el \(0\), pero la imagen es \textit{todo} \(\mathbb{R}\).

  \vspace{5mm}

  \ejemplo{ Ahora consideremos la función trivial de \(\mathbb{R}\) a \(\mathbb{R}\):}
  \[
    f(x) = 0
  \]
  Esta función también es lineal, y \(f:\mathbb{R} \rightarrow \mathbb{R}\)
  \begin{itemize}
    \item \textbf{Núcleo:}
    \[
      \text{Ker } f = \left\{x \in \mathbb{R} \mid f(x) = 0\right\} = \mathbb{R}
    \]
    \item \textbf{Imagen:}
    \[
      \text{Im } f = \{0\}
    \]
  \end{itemize}
  En este caso, \textit{todo} el dominio se anula y la imagen es el conjunto reducido que contiene solo al cero.
\end{quote}

\begin{tcolorbox}[title=Resumen para fijar la idea]
  \begin{itemize}
    \item El \textbf{núcleo} es un subconjunto del \textbf{dominio} \(V\): son los vectores que van a parar al cero del codominio.
    \item La \textbf{imagen} es un subconjunto del codominio \(U\): son todos los valores posibles que puede tomar \(f(v)\)
  \end{itemize}

  \vspace{5mm}

  El núcleo responde a la pregunta: ``¿Qué vectores se anulan bajo \(f\)?''

  La imagen responde a: ``¿Qué vectores pueden obtenerse como salida de \(f\)''
\end{tcolorbox}

\subsubsection{Los subespacios Núcleo y Imagen de \(f\)}

\paragraph{El conjunto núcleo de \(f\) es un subespacio del dominio}

\textbf{Proposición}: Sea \(f: V \to U\) una aplicación lineal entre espacios vectoriales. Entonces, el conjunto núcleo de \(f\),
\[
\text{Ker } f = \{ v \in V \mid f(v) = 0 \}
\]
es un subespacio vectorial de \(V\).

\textit{Demostración}
\begin{quote}
  \begin{enumerate}
    \item \textbf{No vacío}:
    
    El vector nulo del dominio \(\vec{0}_V \in V\) cumple que:
      \[
      f(\vec{0}_V) = \vec{0}_U
      \]
    por lo tanto, \(\vec{0}_V \in \text{Ker } f\), y así \(\text{Ker } f \ne \emptyset\).
  
    \item \textbf{Cerrado bajo la suma}:
    
      Sean \(v_1, v_2 \in \text{Ker } f\). Entonces:
     \[
     f(v_1) = 0 \quad \text{y} \quad f(v_2) = 0
     \]
     Como \(f\) es lineal:
     \[
     f(v_1 + v_2) = f(v_1) + f(v_2) = 0 + 0 = 0
     \]
     Entonces \(v_1 + v_2 \in \text{Ker } f\).
  
    \item \textbf{Cerrado bajo el producto por escalares}:
     
      Sea \(t \in K\) y \(v \in \text{Ker } f\), es decir, \(f(v) = 0\). Entonces:
     \[
     f(t \cdot v) = t \cdot f(v) = t \cdot 0 = 0
     \]
     Por lo tanto, \(t \cdot v \in \text{Ker } f\).
  \end{enumerate}
  
  Conclusión: Se cumplen las tres condiciones necesarias para que \(\text{Ker } f\) sea un subespacio de \(V\).
  \(\blacksquare\)
\end{quote}

\paragraph{El conjunto imagen de \(f\) es un subespacio del codominio}


\textbf{Proposición}: Sea \(f: V \to U\) una aplicación lineal entre espacios vectoriales sobre un mismo cuerpo \(K\). Entonces el conjunto
\[
\text{Im } f = \{ u \in U \mid \exists v \in V \text{ tal que } f(v) = u \}
\]
es un subespacio vectorial de \(U\).

\textit{Demostración:}
\begin{quote}
  \begin{enumerate}
    \item \textbf{No vacío}:
      Como \(f\) es lineal, se cumple que:
      \[
      f(\vec{0}_V) = \vec{0}_U
      \]
      Entonces, \(\vec{0}_U \in \text{Im } f\), por lo tanto \(\text{Im } f \ne \emptyset\).

    \item \textbf{Cerrado bajo suma y producto por escalares} (en una sola propiedad):

      Sean \(u_1, u_2 \in \text{Im } f\).
      Por definición, existen \(v_1, v_2 \in V\) tales que:
      \[
      f(v_1) = u_1 \quad \text{y} \quad f(v_2) = u_2
      \]
      Sean \(a, b \in K\) escalares arbitrarios. Como \(f\) es lineal:
      \[
      f(a v_1 + b v_2) = a f(v_1) + b f(v_2) = a u_1 + b u_2
      \]
      Esto significa que \(a u_1 + b u_2 \in \text{Im } f\).
  \end{enumerate}

  Conclusión: La imagen de \(f\) cumple las condiciones necesarias para ser subespacio de \(U\): no es vacía, y es cerrada bajo combinaciones lineales.
  \(\blacksquare\)
\end{quote}

\begin{tcolorbox}[title=Observaciones]
  Note que no necesitamos comprobar ``cerrado bajo suma'' y ``cerrado bajo producto escalar'' por separado, ya que probar cerrado bajo combinaciones lineales es más general y suficiente.
\end{tcolorbox}
