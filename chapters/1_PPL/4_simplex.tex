\subsection{El método Simplex}
\label{sec:simplex}

El método simplex se mueve de un vértice del poliedro factible a un vértice adyacente que mejora el valor de la función objetivo. Repasando lo que vimos anteriormente cada iteración corresponde a:
\begin{itemize}
  \item \textbf{Evaluación del vértice actual}: Verificar si es óptimo mediante los costos reducidos
  \item \textbf{Selección de arista}: Elegir la arista (dirección) que más mejora la función objetivo
  \item \textbf{Movimiento}: Desplazarse a lo largo de la arista hasta el siguiente vértice
  \item \textbf{Actualización de la tabla}: El nuevo vértice se convierte en la nueva solución básica
\end{itemize}
Más adelante veremos un ejemplo simple para comprender mejor el método Simplex, pero antes necesitamos ver qué necesitamos para usar el método Simplex.

\paragraph{Condiciones para usar el método Simplex}

Entender con claridad \textit{cuáles son las condiciones necesarias para aplicar el método Simplex} es fundamental, ya que no todo problema lineal puede resolverse directamente con él sin alguna transformación previa. De forma resumida, para aplicar Simplex se necesita:
\begin{enumerate}
  \item Convertir el problema a forma estándar,
  \item Tener una solución básica factible inicial,
  \item Asegurar que el problema tiene solución factible y acotada.
\end{enumerate}
En general, para poder aplicar el método simplex, se expresa el PPL en su forma matricial, ya que permite realizar las transformaciones y operaciones de forma ordenada.

A continuación veremos cada una de las condiciones en detalle.

\subsubsection{Forma estándar del problema}
\label{sec:forma_estandar}

Un PPL está en su forma estándar si cumple las siguientes condiciones:
\begin{itemize}
  \item La función objetivo debe ser de maximización,
  \item Todas las variables deben ser mayores o iguales a cero (restricción de no negatividad).
  \item Todas las restricciones deben estar escritas como igualdades (no desigualdades).
\end{itemize}
Ya puede intuir, por los ejemplos vistos, que no todos los PPLs cumplen con estas condiciones, por lo que se requiere transformar el PPL a su forma estándar.

\paragraph{Transformación de un PPL en su forma estándar}

Como vimos anteriormente, un PPL está en su forma estándar si cumple las condiciones vistas. Para trabajar de forma ordenada vamos a llamar a cada condición de la siguiente manera:
\begin{enumerate}
  \item \textbf{Condición de maximización}: la función objetivo debe ser de maximización,
  \item \textbf{Condición de no negatividad}: todas las variables de decisión deben ser positivas: \(\forall x_i,\ x_i \geq 0;\ i = 1,2,\ldots,n\)
  \item \textbf{Condición no desigualdad}: todas las restricciones deben ser de la forma \(A_iX = b_i\). Para lograr esto se introducen \hl{variables de holgura} y \hl{variables superfluas}.
\end{enumerate}
En un momento se explicará el concepto de variables de holgura y variables superfluas.

Para tener un contexto previo, ver el siguiente video: \href{https://www.youtube.com/watch?v=6f5K3O7yUzU}{\texttt{forma estándar en programación lineal - YouTube}}, donde se muestran algunos ejemplos de cómo transformar un PPL a forma estándar. De igual manera, en este documento se mostrará el proceso de transformación a formato estándar de un PPL. 

\ejemplo\label{ej:transformacion_ppl_forma_estandar}: Transformación de un PPL a forma estándar.
\begin{quote}
  Consideremos el siguiente PPL (ejemplo del video recomendado):
  \begin{align*}
    \text{maximizar:} \quad   &z = -8x_1 + 16x_2 - 4x_3 \\[3pt]
    \text{sujeto a:} \quad    &5x_1 + 8x_3 \geq 100 \\
                              &-x_1 + 6x_2 \geq 100 \\
                              &-6x_2 + 4x_3 \leq 100 \\
                              &x_1 \geq 0 \\
                              &x_2 \text{ Libre} \\
                              &x_3 \leq0
  \end{align*}

  \noindent\textbf{1. Condición de maximización:}

  El PPL ya cumple con la condición de maximización, por lo que no se requiere ninguna transformación. Para aquellos PPL que buscan \textit{minimizar}, que no cumplen esta condición, al multiplicar la función objetivo por \(-1\) se puede transformar en una maximización. 

  \noindent\textbf{2. Condición de no negatividad:}

  En este caso, tenemos las tres posibilidades en las variables de decisión:
  \begin{itemize}
    \item Libre: \(x_2\)
    \item Positiva: \(x_1\)
    \item Negativa: \(x_3\)
  \end{itemize}

  La única variable que cumple con la condición de no negatividad es \(x_1\), por lo tanto esta variable se mantiene como está, es decir, \(x_1 \geq 0\). 

  Las variables que no cumplen con esta condición son \(x_2\) y \(x_3\), por lo que debemos operar sobre ellas para hacer cumplir la condición. Veamos el caso de \(x_3\) primero: 

  La variable \(x_3\) es negativa, para transformar la restricción a positiva debemos realizar un cambio de variable, tal que:
  \[
    x_3 = -y_3 \quad \rightarrow \quad y_3 \geq 0
  \]

  \noindent donde \(y_3\) es una variable que reemplaza a \(x_3\) y que es positiva. Esta variable debe cambiarse en todo el PPL, es decir, en la función objetivo y en las restricciones. 

  Para el caso de \(x_2\), la variable puede tomar cualquier valor, por lo que para que estrictamente cumpla la condición de no negatividad podemos realizar lo siguiente:
  \[
    x_2 = y_2 - y_2' \quad \rightarrow \quad y_2,\ y_2' \geq 0
  \]

  \noindent donde \(y_2\) y \(y_2'\) son variables que reemplazan a \(x_2\) y que cumplen con la condición de no negatividad. Estas variables deben reemplazar a \(x_2\) en todo el PPL, al igual que \(x_3\) se reemplaza por \(y_3\). Reemplazando las variables en el PPL, obtenemos:
  \begin{align*}
    \text{maximizar:} \quad   &z = -8x_1 + 16y_2 - 16y_2' + 4y_3 \\[3pt]
    \text{sujeto a:} \quad    &5x_1 - 8y_3 \geq 100 \\
                              &-x_1 + 6y_2 - 6y_2' \geq 100 \\
                              &-6y_2 + 6y_2' - 4y_3 \leq 100 \\
                              &x_1,\ y_2,\ y_2',\ y_3 \geq 0
  \end{align*}
  Listo, la condición de no negatividad se cumple, ahora veamos la tercera condición.

  \noindent\textbf{3. Condición de no desigualdad:}

  La tercera condición es que todas las restricciones deben ser de la forma \(Ax = b\). En este caso, ninguna de las restricciones cumple con esta condición, por lo que debemos operar sobre todas ellas. 

  Cuando analizamos la tercera condición, podemos encontrarnos con tres casos posibles:
  \begin{enumerate}
    \item Restricción de la forma \(A_i x \leq b_i\): agregar una \hl{\textit{variable de holgura}}
    \item Restricción de la forma \(A_i x \geq b_i\): agregar una \hl{\textit{variable superflua}}
    \item Restricción de la forma \(A_i x = b_i\): cumple con la condición de forma estándar, por lo que no se necesita ninguna variable adicional
  \end{enumerate}
  \noindent donde \(A_i\) es la i-ésima restricción y \(b_i\) es su término independiente.
  \begin{tcolorbox}[interesting_data, title=¿Qué significa variable de holgura?]
    La \textbf{variable de holgura} es aquella que se agrega a una restricción de la forma \(A_i x \leq b_i\) para transformarla en una igualdad, y representa la cantidad de recursos de los que se disponían pero no se usaron. 

    \textit{Ejemplo}: si se disponen de 10 horas totales para fabricar un producto, pero solo se usan 8 horas, entonces la variable de holgura representa las \(2\) horas que se disponían pero no se usaron. 
  \end{tcolorbox}

  \begin{tcolorbox}[interesting_data, title=¿Qué significa variable superflua?]
    Por otro lado, la \textbf{variable superflua} es aquella que se agrega a una restricción de la forma \(A_i x \geq b_i\) para transformarla en una igualdad, y representa la cantidad en que el lado izquierdo de la restricción \textbf{excede} el requisito mínimo establecido por el lado derecho.

    \textit{Ejemplo}: si se requieren como mínimo 54 toneladas de mineral, pero se extraen 60 toneladas, entonces la variable superflua vale \(60 - 54 = 6\) toneladas, y representa el exceso de toneladas de mineral que se extrae. 
  \end{tcolorbox}

  \noindent Volviendo al problema original, las restricciones que no cumplen la tercer condición son:
  \begin{align*}
    5x_1 - 8y_3 &\geq 100 \\
    -x_1 + 6y_2 - 6y_2' &\geq 100 \\
    -6y_2 + 6y_2' - 4y_3 &\leq 100
  \end{align*}
  Para la tercera restricción, tenemos el caso 1. Para este caso debemos agregar una \textit{variable de holgura}:
  \[
    -6y_2 + 6y_2' - 4y_3 \leq 100 \quad \rightarrow \quad -6y_2 + 6y_2' - 4y_3 + h_1 = 100
  \]
  Luego, para la primera y la segunda restricción tenemos el caso 2. Para este caso debemos agregar una \textit{variable superflua}:
  \begin{align*}
    5x_1 - 8y_3 \geq 100 \quad &\rightarrow \quad 5x_1 - 8y_3 - s_1 = 100 \\
    -x_1 + 6y_2 - 6y_2' \geq 100 \quad &\rightarrow \quad -x_1 + 6y_2 - 6y_2' - s_2 = 100
  \end{align*}

  Estas variables que se han agregado a las restricciones deben aparecer en la función objetivo, sin embargo, como no son variables de decisión, su coeficiente en la función objetivo debe ser cero. Por lo que el PPL transformado a forma estándar es:
  \begin{align*}
    \text{maximizar:} \quad   &z = -8x_1 + 16y_2 - 16y_2' + 4y_3 - 0s_1 - 0s_2 + 0h_1 \\[3pt]
    \text{sujeto a:} \quad    &5x_1 + 0y_2 - 0y_2' - 8y_3 - s_1 - 0s_2 + 0h_1 = 100 \\
                              &-x_1 + 6y_2 - 6y_2' + 0y_3 - 0s_1 - s_2 + 0h_1 = 100 \\
                              &0x_1 -6y_2 + 6y_2' - 4y_3 - 0s_1 - 0s_2 + h_1 = 100 \\
                              &x_1,\ y_2,\ y_2',\ y_3,\ h_1,\ s_1,\ s_2 \geq 0
  \end{align*}
  Si quitamos todos los términos con coeficiente nulo de las restricciones queda:
  \begin{align*}
    \text{maximizar:} \quad   &z = -8x_1 + 16y_2 - 16y_2' + 4y_3 - 0s_1 - 0s_2 + 0h_1 \\[3pt]
    \text{sujeto a:} \quad    &5x_1 - 8y_3 - s_1 = 100 \\
                              &-x_1 + 6y_2 - 6y_2' - s_2 = 100 \\
                              &-6y_2 + 6y_2' - 4y_3 + h_1 = 100 \\
                              &x_1,\ y_2,\ y_2',\ y_3,\ h_1,\ s_1,\ s_2 \geq 0
  \end{align*}
\end{quote}

\subsubsection{Existencia de una solución básica factible inicial (SBF)}

Este es el ``\textit{punto más delicado}''. Para comenzar el método Simplex, se necesita un punto inicial que cumpla todas las restricciones (factible) y sea una solución básica (es decir, con tantas variables básicas como ecuaciones, y el resto en cero).

Como ya vimos, si las restricciones son de tipo \(\leq\) y se agregan variables de \textit{holgura positiva}, entonces la \textbf{SBF} inicial está dada simplemente por poner en cero las variables originales y tomar las variables de holgura como solución. Por otro lado, si las restricciones son de tipo \(\geq\) o \(=\), o si el sistema \textbf{no} tiene una SBF evidente, entonces se debe recurrir a un método auxiliar como el método de la fase I del Simplex o el método de las dos fases.

\paragraph{¿Qué es una solución básica?}

En términos del álgebra lineal, una \textit{solución básica} es una solución del sistema lineal:
\[
AX = B
\]
Este sistema, representado de forma matricial, no es más que las restricciones del PPL en forma estándar. Si desea repasar el tema de sistemas de ecuaciones lineales puede consultar el capítulo \ref{sec:sel}.

\paragraph{¿Cómo se define una \textit{solución básica}?}

Para construir una \textit{solución básica}, se hace lo siguiente:
\begin{enumerate}
  \item Se eligen arbitrariamente \(m\) variables del vector \(X\), a las que se llama \textbf{variables básicas}.
  \item El resto de las \(n - m\) variables se fijan en cero; estas se llaman \textbf{variables no básicas}.
  \item Luego se resuelve el sistema lineal con esas \(m\) incógnitas (las básicas), usando las \(m\) ecuaciones.
\end{enumerate}
Esto es posible si las columnas de \(A\) correspondientes a las variables básicas son \textbf{linealmente independientes}, lo que permite resolver el sistema.

\paragraph{¿Y qué es una \textit{solución básica factible} (SBF)?}

Una solución básica factible es una solución básica que además satisface:
\[
x \geq 0
\]
Es decir, todas las variables (básicas y no básicas) deben ser \textbf{mayores o iguales a cero}. Esta condición es necesaria porque el método Simplex trabaja solamente en la región factible del espacio, que está limitada por las restricciones del problema.

\paragraph{Intuición geométrica}

En geometría, la región factible de un problema lineal es un \textit{poliedro} (en 2D, un polígono; en 3D, un poliedro tridimensional). Cada \textit{vértice} (o esquina) de esa región corresponde a una \textbf{solución básica factible}.

El método Simplex camina de un vértice al siguiente, buscando mejorar el valor de la función objetivo, hasta que ya no puede mejorar más.

Por eso se necesita comenzar desde un vértice: es decir, desde una \textbf{SBF}.

\ejemplo\label{ej:busqueda_sbf}{: Búsqueda de SBF}

Supón que tienes este sistema (ya convertido a forma estándar):
\begin{align*}
  x_1 + x_2 + x_3 &= 4 \\
  2x_1 + 3x_2 + x_4 &= 7 \\
  x_1, x_2, x_3, x_4 &\geq 0
 \end{align*}

Aquí hay 4 variables y 2 ecuaciones. Una solución básica se obtiene eligiendo, por ejemplo, las variables \(x_1\) y \(x_3\) como básicas, y fijando \(x_2 = x_4 = 0\).

Al resolver el sistema para \(x_1\) y \(x_3\), obtenemos \(x_1 = 3.5;\, x_3 = 0.5\). Como todas las variables son no negativas, entonces \(x_1 = 3.5;\, x_3 = 0.5\) es una \hl{solución básica factible}.

Ahora veamos qué pasa si elegimos las variables \(x_1\) y \(x_2\) como básicas, y fijamos \(x_3 = x_4 = 0\). Al resolver el sistema para \(x_1\) y \(x_2\) el resultado da \(x_1 = 5;\, x_2 = -1\). Como \(x_2\) es negativo, entonces \textbf{no es} una solución básica factible, simplemente es una solución básica al sistema.
\vspace{5mm}
\hrule

\subsubsection{Condición de factibilidad y acotación}

Esta es la última condición que debe cumplir el problema para que el Simplex sea aplicable.

El método Simplex solo es aplicable si el problema tiene una solución factible. Si no la tiene, el Simplex lo detectará (normalmente en la fase I).

El problema también debe ser acotado, es decir, si la función objetivo puede crecer indefinidamente sin violar las restricciones, el Simplex lo reportará como no acotado.

En términos simples, con el método simplex buscamos una solución factible \textbf{única}.