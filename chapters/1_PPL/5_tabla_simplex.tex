\subsection{Inicialización del método Simplex}

\subsubsection{Generación de solución factible inicial}

Como vimos, Simplex también requiere de una solución básica factible (SBF) inicial ya que el algoritmo se mueve de una solución factible básica a otra, mejorando el valor de la función objetivo en cada iteración. Anteriormente, en el ejemplo \ref{ej:busqueda_sbf} vimos, tras elegir arbitrariamente las variables básicas, que dimos con una SBF y luego, tras elegir nuevamente otras variables básicas, dimos con una solución básica pero no factible. Ese problema tenía pocas variables de decisión, por lo que si queremos encontrar una SBF podemos ir probando hasta dar con alguna. Sin embargo, si el problema tiene muchas variables de decisión, es difícil encontrar una SBF inicial, y probar todas las posibilidades no es lo más eficiente. Por lo que existen dos métodos para generar una SBF inicial:
\begin{itemize}
  \item \hl{Inspección directa}: si es fácil identificar un punto que satisface todas las restricciones, se puede usar directamente.
  \item \hl{Método de la Fase I del Simplex}: para aquellos casos donde no es fácil aplicar la inspección directa, se utiliza este método. Este método consiste en cuatro pasos:
  \begin{enumerate}
    \item Se construye un problema auxiliar donde se agregan \textit{variables artificiales} para convertir el problema original en uno con una solución factible evidente
    \item Se minimiza la suma de esas variables artificiales
    \item Si en el óptimo esta suma es cero, se ha encontrado una solución factible al problema original
    \item Si no es cero, el problema original no tiene solución factible (es incompatible)
  \end{enumerate}
\end{itemize}

\noindent El método de la inspección directa se vió en el ejemplo \ref{ej:busqueda_sbf}. Básicamente consiste en analizar las restricciones y verificar si se puede generar una solución básica que sea factible. Ahora veamos el método de la Fase I.

\subsubsection{Método de la Fase I: construcción del problema auxiliar}

Esta fase tiene como objetivo encontrar una solución básica factible inicial para un problema de programación lineal (PPL) cuando no está disponible de forma directa, aunque también puede usarse para aquellos problemas que tienen una SBF evidente.

Recordemos que el método Simplex parte de una SBF y se mueve de una a otra mejorando el valor de la función objetivo. Por eso, si no contamos con una SBF al comienzo, necesitamos construir un problema auxiliar que nos la proporcione. Esto es lo que hace la Fase I.

El objetivo de la Fase I es formular y resolver un problema auxiliar que:
\begin{itemize}
  \item Sea fácil de resolver
  \item Tenga una SBF evidente
  \item Su solución, si es factible, nos permita obtener una SBF \textbf{del problema original}
\end{itemize}

Este método consiste en cuatro pasos generales:
\begin{enumerate}

  \item \textbf{Convertir el PPL original a forma estándar}: Esto incluye que todas las restricciones sean igualdades (introduciendo variables de holgura, exceso, etc.) y todas las variables estén acotadas inferiormente por cero.

  \item \textbf{Identificar las restricciones problemáticas}: Por ejemplo, si una ecuación tiene una constante en el lado derecho negativa, o si las variables artificiales son necesarias para armar una base inicial.

  \item \textbf{Agregar variables artificiales}: Se utilizan cuando las restricciones del problema no permiten obtener una solución básica factible inicial (SBF inicial) de forma directa, es decir, cuando no se puede formar una matriz identidad a partir de las variables de holgura. Esto se hace cuando:
    \begin{itemize}
      \item En el caso \(\geq\), se agregan \textit{variables superfluas}, que restan una variable: \(-x_e\), lo cual da lugar a columnas con \(-1\) en vez de \(1\), y por tanto no son vectores \(e_i\).
      \item En el caso \(=\), al no haber ni suma ni resta de variables de holgura, no se genera ninguna columna tipo \(e_i\) automáticamente.
    \end{itemize}
    donde las columnas \(e_i\) son columnas que tienen un elemento 1 y el resto cero y permiten formar una matriz identidad.
  \begin{tcolorbox}[myconclusion]
    Las variables artificiales son variables auxiliares o ficticias que se introducen en el modelo de Programación Lineal para poder iniciar el algoritmo Simplex. No tienen un significado físico o económico en el problema original; son puramente un ``truco matemático''.
  \end{tcolorbox}

  \item \textbf{Formular un problema auxiliar (problema de la Fase I)}: Se define una nueva función objetivo auxiliar: Minimizar la suma de todas las variables artificiales. Esta función objetivo representa el “costo” de alejarse del espacio factible del problema original.

  \item \textbf{Resolver el problema auxiliar con el método Simplex}: 
  \begin{itemize}
    \item Si la solución óptima del problema auxiliar tiene valor cero, significa que se ha encontrado una SBF para el problema original.
    \item Si la solución óptima tiene valor distinto de cero, el problema original no es factible (no existe ninguna solución que satisfaga todas las restricciones).
  \end{itemize}

  \item \textbf{Eliminar las variables artificiales}: Si la Fase I fue exitosa, se eliminan las variables artificiales (si aún están presentes en la base se deben reemplazar mediante pivoteos), y se continúa con la \textit{Fase II}, ahora con una SBF válida y la función objetivo original.
\end{enumerate}

El problema auxiliar busca un punto factible \hl{minimizando} el ``uso'' de las variables artificiales. Si puede lograr que todas ellas sean cero, significa que existe una combinación de las variables reales (originales y de holgura) que satisfacen todas las restricciones. Esa es justamente una solución básica factible. Veamos un ejemplo de la Fase I.

\ejemplo\label{ej:fase_1} Sea el siguiente problema de programación lineal:
\begin{quote}
  \begin{align*}
    \text{maximizar} \quad  &z = 3x_1 + 2x_2\\[3pt]
    \text{sujeto a:} \quad  &x_1 + x_2 = 4\\
                            &x_1 - x_2 \geq 2\\
                            &x_1, x_2 \geq 0
  \end{align*}
  Observamos que:
  \begin{itemize}
    \item La primera restricción ya está en forma de igualdad, pero no tiene una variable de holgura asociada que permita incluir una columna identidad. Por lo tanto debemos agregar una variable artificial.
    \item La segunda es una desigualdad \(\geq\), por lo tanto debemos restar una variable superflua, por lo tanto, también debemos agregar una variable artificial.
  \end{itemize}
  
  \subparagraph{Paso 1: Convertimos a forma estándar y agregamos variables artificiales}
  \begin{enumerate}
    \item A la primera ecuación (que ya es igualdad), agregamos una variable artificial \(a_1\), porque no hay variable de holgura ni exceso que permita formar la base inicial.
    \item A la segunda desigualdad, restamos una variable de exceso \(s_2\), y agregamos una variable artificial \(a_2\).
  \end{enumerate}
  El sistema queda:
  \begin{align*}
    x_1 + x_2 + a_1 &= 4\\
    x_1 - x_2 - s_2 + a_2 &= 2
  \end{align*}  
  Con condiciones:
  \[
    x_1, x_2, s_2, a_1, a_2 \geq 0
  \]

  \subparagraph{Paso 2: Definimos el problema auxiliar}
  
  La función objetivo auxiliar es:
  \[
    \omega = a_1 + a_2
  \]
  Nuestro nuevo problema (auxiliar) es:
  \begin{align*}
    \text{minimizar} \quad  &\omega = a_1 + a_2\\[3pt]
    \text{sujeto a:} \quad  &x_1 + x_2 + a_1 = 4\\
                            &x_1 - x_2 - s_2 + a_2 = 2\\
                            &x_1, x_2, s_2, a_1, a_2 \geq 0
  \end{align*}

  Para las restricciones, la base inicial está formada por las variables artificiales \(a_1\) y \(a_2\), porque aparecen con coeficiente 1 y sólo en una ecuación (forman una matriz identidad).
  \begin{align*}
    \begin{pmatrix}
      1 & 1 & 0 & 1 & 0\\
      1 & -1 & -1 & 0 & 1
    \end{pmatrix}
    \begin{pmatrix}
      x_1\\
      x_2\\
      s_2\\
      a_1\\
      a_2
    \end{pmatrix}
    =
    \begin{pmatrix}
      4\\
      2
    \end{pmatrix}
  \end{align*}
  Esto significa que, si establecemos como variables básicas a \(a_1\) y \(a_2\), y las demás en cero, obtenemos una solución básica factible para el problema auxiliar, ya que \(a_1 = 4\) y \(a_2 = 2\).

  De este modo, ya tenemos una \hl{SBF inicial} para el problema auxiliar.
  
  \subparagraph{Paso 3: Aplicamos el método Simplex}
  
  A partir de esta situación, podríamos construir la tabla Simplex con \(a_1\) y \(a_2\) en la base, y comenzar a iterar para minimizar \(\omega\). Si al finalizar el valor óptimo de \(\omega\) es cero, habremos encontrado una combinación de \(x_1\), \(x_2\) y \(s_2\) que satisface las restricciones sin necesidad de variables artificiales, es decir, una solución básica factible del problema original.
  
  Si en cambio \(\omega > 0\), entonces el problema original no tiene solución factible.  
\end{quote}

Entonces, si resolvemos el problema auxiliar con Simplex, ya tenemos una SBF inicial para el problema original, y así poder aplicar el Simplex en nuestro problema original.

\begin{tcolorbox}[interesting_data, title=Siempre recordar lo que se busca en la Fase I]
  \begin{itemize}
    \item \textbf{Objetivo}: Minimizar \(\omega = \sum(\text{artificiales})\)
    \item \textbf{Costos}: Coeficientes de artificiales = 1, resto = 0
    \item \textbf{Costos reducidos}: Miden impacto en suma de artificiales
    \item \textbf{Criterio}: Costo reducido más negativo entra
    \item \textbf{Meta}: Llegar a \(\omega = 0\)
  \end{itemize}
\end{tcolorbox}
Ahora profundizaremos más en lo que dice este cuadro.

\subsubsection{Construcción de la primera tabla del método Simplex (problema auxiliar)}

Una vez que tienes el problema auxiliar planteado, el siguiente paso es construir la primera tabla del método Simplex para el problema auxiliar, resolverlo con iteraciones hasta obtener una SBFI, y a partir de ahí: si \(\omega = 0\), comenzar el Simplex para el problema original con esa base.

Entonces, siguiendo con el ejemplo \ref{ej:fase_1}, vamos a armar la primer tabla Simplex. Para armar la primer tabla Simplex debemos tener en cuenta algunos detalles, como la construcción del vector de costos y la selección de la base. Aunque en el ejemplo \ref{ej:fase_1} más o menos vimos estos conceptos, ahora los veremos más en detalle.

\paragraph{Contrucción del vector de costos \(c\) para la Fase I}

Recordando que la función objetivo del problema auxiliar es \(\omega = a_1 + a_2\), por lo tanto el \textbf{vector de costos} asociado a todas las variables \textbf{del PPL auxiliar} (en el orden \(x_1, x_2, s_2, a_1, a_2\)) es: 
\[
  \hat{c} = (0, 0, 0, 1, 1)
\]
Ya que \(x_1\), \(x_2\) y \(s_2\) no participan en la función objetivo del problema auxiliar, su coeficiente es cero.

Un detalle muy importante es que el problema auxiliar es de \textit{minimización}, por lo que no está en formato estándar. Por ello debemos multiplicar a la función objetivo por (\(-1\)), resultando:
\[
  \text{max} \quad -\omega = - a_1 - a_2 \quad \rightarrow \quad \hat{c}=(0,0,0,-1,-1)
\]



\paragraph{Elección de la base inicial}

Para poder aplicar el método Simplex, se necesita una \textit{base inicial} formada por un conjunto de variables básicas tales que el sistema:
\[
A_{\beta}X_{\beta} = B
\]
tenga una solución factible (es decir, \(X_\beta \geq 0\)).
\begin{tcolorbox}[remember, title=Aclaración]
  Cuando se usa \(A_\beta\) o \(X_\beta\) se refiere al vector de \textbf{variables básicas} y sus respectivos coeficientes. Eso no tiene nada que ver con la matriz \(B\).
\end{tcolorbox}

En el problema auxiliar hemos introducido \(a_1\) y \(a_2\) de modo que:
\begin{itemize}
  \item Cada una aparece \textbf{una sola vez} en una ecuación,
  \item Con coeficiente 1,
  \item Y no aparece en las demás ecuaciones.
\end{itemize}
Este tipo de estructura es ideal para que las variables artificiales sirvan como base inicial. Por lo tanto, tomamos como base inicial:
\[
  \beta = \{a_1, a_2\}
\]
Esto garantiza que la matriz base \(A_\beta\) es la \textit{matriz identidad} \(I_2\), y por lo tanto, \(X_\beta = B\) tiene solución inmediata. En forma matricial es:
\begin{align*}
  A_\beta =
  \begin{pmatrix}
    1 & 0\\
    0 & 1
  \end{pmatrix}
  \quad
  X_\beta =
  \begin{pmatrix}
    a_1\\
    a_2
  \end{pmatrix}
  \quad
  B =
  \begin{pmatrix}
    4\\
    2
  \end{pmatrix}
\end{align*}
Entonces:
\begin{align*}
  A_\beta X_\beta = B \quad \rightarrow \quad {a_1} = 4, ~~ {a_2} = 2
\end{align*}
Entonces ya tenemos una \textbf{SBF} inicial (como vimos anteriormente).

\paragraph{Introducción de costos reducidos}

Los costos reducidos (también llamados costos relativos) son un concepto fundamental para entender cómo el método Simplex decide qué variable entra en la base en cada iteración.

En un problema de programación lineal de la forma:
\begin{align*}
\text{Maximizar} \quad & z = c^T x \\
\text{s.a.} \quad & Ax = b \\
& x \geq 0
\end{align*}

El costo reducido de una variable \(x_j\) es:
\[
\bar{c}_j = c_j - c_B^T B^{-1} A_j
\]
Donde:
\begin{itemize}
  \item \(c_j\) es el coeficiente de \(x_j\) en la función objetivo original.
  \item \(c_B\) es el vector de los coeficientes de las variables básicas en la función objetivo.
  \item \(B\) es la matriz de las columnas de \(A\) asociadas a las variables básicas.
  \item \(A_j\) es la columna de la matriz \(A\) correspondiente a la variable \(x_j\).
\end{itemize}

Este valor mide cuánto aumentaría (o disminuiría) la función objetivo si se introdujera la variable \(x_j\) en la base.

El método Simplex se mueve de un vértice a otro del poliedro factible buscando mejorar la función objetivo. El costo reducido de una variable indica \textit{la pendiente} en la dirección asociada a esa variable:
\begin{itemize}
  \item Si \(\bar{c}_j < 0\) en un problema de maximización, entonces introducir \(x_j\) puede mejorar la solución (la variable entra a la base).
  \item Si \(\bar{c}_j > 0\), introducir \(x_j\) empeoraría la solución, por lo tanto no se la considera.
  \item Si \(\bar{c}_j = 0\), introducir esa variable no cambia el valor de la función objetivo (es decir, hay degeneración o soluciones múltiples).
\end{itemize}
Para nuestro ejemplo, teníamos la función objetivo auxiliar:
\[
\omega = a_1 + a_2
\Rightarrow z = -\omega = -a_1 - a_2
\]

En la primera tabla Simplex que armamos:

\[
\begin{array}{c|rrrrr|r}
\text{Base} & x_1 & x_2 & s_2 & a_1 & a_2 & \text{Término independiente} \\
\hline
a_1 & 1 & 1 & 0 & 1 & 0 & 4 \\
a_2 & 1 & -1 & -1 & 0 & 1 & 2 \\
\hline
z & -2 & 0 & 1 & 0 & 0 & -6 \\
\end{array}
\]

Fijate en la **última fila** de la tabla (la fila de $z$). Es allí donde se reflejan los **costos reducidos** de las variables **no básicas**. Cada entrada de esa fila es precisamente:

\[
\bar{c}_j = c_j - z_j
\]

* En este caso, $x_1$ tiene un costo reducido de $-2$
* $x_2$ tiene costo reducido 0
* $s_2$ tiene costo reducido $1$
* $a_1$ y $a_2$ tienen costos reducidos $0$ porque ya están en la base, y sus coeficientes están anulados por la forma canónica

¿Qué representa el valor $-2$ en $x_1$?

Significa que si 
reemplazás una de las variables básicas actuales por $x_1$, la función $z = -\omega$ aumentaría 2 unidades por cada unidad que entre $x_1$ en la base. Como estamos maximizando $-\omega$, esto equivale a **minimizar $\omega$**.

Es por eso que $x_1$ es la **mejor candidata para entrar a la base** en esta iteración.



Los costos reducidos en la fila de $z$ indican para cada variable **cuánto se modificaría la función objetivo si esa variable entra en la base**. En el método Simplex:

* Se elige como **entrante** a la variable con el **costo reducido más negativo** (en maximización)
* Se detiene el algoritmo cuando **todos los costos reducidos son mayores o iguales a cero**

¿Te gustaría que calcule explícitamente los costos reducidos en este ejemplo usando la fórmula $\bar{c}_j = c_j - c_B^T B^{-1} A_j$? Podemos hacerlo paso a paso.


Ahora si, el siguiente paso consiste en aplicar el método Simplex para hallar una \textbf{solución básica factible inicial (SBFI)} que sirva de punto de partida para el problema original. Para ello, se construye la \textbf{primera tabla del método Simplex correspondiente al problema auxiliar}.

\paragraph{Componentes de la tabla}

La tabla inicial debe contener:

\begin{itemize}
  \item Todas las \textbf{variables del sistema}: variables originales \(x_i\), variables de holgura \(h_i\) (si las hay), y variables artificiales \(a_i\).
  \item Una \textbf{columna de términos independientes} o la matriz \(B\) (lado derecho de las restricciones).
  \item Una \textbf{fila correspondiente a la función objetivo auxiliar} \(\omega = a_1 + a_2 + \cdots + a_k\), en forma canónica.
  \item Una \textbf{identificación de la base inicial}, que estará compuesta por las variables artificiales que se agregaron para formar una base identidad.
\end{itemize}

\subsubsection*{Forma general de la tabla}

La forma general de la tabla Simplex es la siguiente:

\[
\begin{array}{c|cccccc|c}
\text{Base} & x_1 & x_2 & \cdots & h_i & a_j & \cdots & \text{Término independiente} \\
\hline
a_1 & a_{11} & a_{12} & \cdots & a_{1i} & a_{1n} & \cdots & b_1 \\
a_2 & a_{21} & a_{22} & \cdots & a_{2i} & a_{2n} & \cdots & b_2 \\
\vdots & \vdots & \vdots & & \vdots & \vdots & & \vdots \\
a_k & a_{k1} & a_{k2} & \cdots & a_{ki} & a_{kn} & \cdots & b_k \\
\hline
\omega & c_1 & c_2 & \cdots & c_i & c_n & \cdots & -\sum b_i \\
\end{array}
\]

\textbf{Notas:}
\begin{itemize}
  \item No confundir la base \(\beta=\{a_1,a_2\}\) con los coeficientes de las restricciones \(A\) (\(a_{11},a_{12},\dots\)) 
  \item La fila de \(\omega\) está escrita en forma canónica. Es decir, los coeficientes se obtienen sustituyendo cada variable básica en la función objetivo auxiliar \(\omega\), lo cual produce la expresión: 
  \[
  \omega = -x_1 - x_2 - \cdots \quad \text{(solo si están presentes en la base)}
  \]
  y un término independiente igual a \( -\sum b_i \), si todos los coeficientes de las variables artificiales en \(\omega\) son 1.
  \item El signo negativo en la función objetivo responde al hecho de que el método Simplex estándar resuelve problemas de \textbf{maximización}, por lo que al resolver el problema auxiliar se transforma el objetivo \( \min \omega \) en \( \max (-\omega) \).
\end{itemize}

\begin{tcolorbox}[myconclusion]
  Ahora continuaremos con el ejemplo \ref{ej:fase_1} para visualizar el contenido de la tabla de forma numérica.
\end{tcolorbox}

\[
\begin{array}{c|rrrrr|r}
\text{Base} & x_1 & x_2 & s_2 & a_1 & a_2 & \text{Término independiente} \\
\hline
a_1 & 1 & 1 & 0 & 1 & 0 & 4 \\
a_2 & 1 & -1 & -1 & 0 & 1 & 2 \\
\hline
z = -\omega & -2 & 0 & 1 & 0 & 0 & -6 \\
\end{array}
\]


\paragraph{Procedimiento}

A partir de esta tabla inicial, se aplica el método Simplex como en cualquier problema estándar:

\begin{itemize}
  \item Se identifica la variable entrante mediante el criterio de optimalidad (el coeficiente más negativo en la fila de \(\omega\)).
  \item Se determina la variable saliente mediante el criterio del cociente mínimo.
  \item Se realizan operaciones fila para obtener la nueva tabla (pivoteo).
  \item Se repite el proceso hasta que todos los coeficientes en la fila de \(\omega\) sean mayores o iguales a cero (óptimo alcanzado).
\end{itemize}

Una vez obtenida la solución óptima del problema auxiliar, se evalúa el valor de \(\omega\). Si es cero, se ha obtenido una SBFI para el problema original. En ese caso, se eliminan las variables artificiales de la tabla (pivoteando si aún permanecen en la base) y se continúa con el método Simplex aplicado al problema original.