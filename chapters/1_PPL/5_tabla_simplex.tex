\subsection{Inicialización del método Simplex}

\subsubsection{Generación de solución factible inicial}

Como vimos, Simplex también requiere de una solución básica factible (SBF) inicial ya que el algoritmo se mueve de una solución factible básica a otra, mejorando el valor de la función objetivo en cada iteración. Anteriormente, en el ejemplo \ref{ej:busqueda_sbf} vimos, tras elegir arbitrariamente las variables básicas, que dimos con una SBF y luego, tras elegir nuevamente otras variables básicas, dimos con una solución básica pero no factible. Ese problema tenía pocas variables de decisión, por lo que si queremos encontrar una SBF podemos ir probando hasta dar con alguna. Sin embargo, si el problema tiene muchas variables de decisión, es difícil encontrar una SBF inicial, y probar todas las posibilidades no es lo más eficiente. Por lo que existen dos métodos para generar una SBF inicial:
\begin{itemize}
  \item \hl{Inspección directa}: si es fácil identificar un punto que satisface todas las restricciones, se puede usar directamente.
  \item \hl{Método de la Fase I del Simplex}: para aquellos casos donde no es fácil aplicar la inspección directa, se utiliza este método. Este método consiste en cuatro pasos:
  \begin{enumerate}
    \item Se construye un problema auxiliar donde se agregan \textit{variables artificiales} para convertir el problema original en uno con una solución factible evidente
    \item Se minimiza la suma de esas variables artificiales
    \item Si en el óptimo esta suma es cero, se ha encontrado una solución factible al problema original
    \item Si no es cero, el problema original no tiene solución factible (es incompatible)
  \end{enumerate}
\end{itemize}

\noindent El método de la inspección directa se vió en el ejemplo \ref{ej:busqueda_sbf}. Básicamente consiste en analizar las restricciones y verificar si se puede generar una solución básica que sea factible. Ahora veamos el método de la Fase I.

\subsubsection{Método de la Fase I: construcción del problema auxiliar}

Esta fase tiene como objetivo encontrar una solución básica factible inicial para un problema de programación lineal (PPL) cuando no está disponible de forma directa, aunque también puede usarse para aquellos problemas que tienen una SBF evidente.

Recordemos que el método Simplex parte de una SBF y se mueve de una a otra mejorando el valor de la función objetivo. Por eso, si no contamos con una SBF al comienzo, necesitamos construir un problema auxiliar que nos la proporcione. Esto es lo que hace la Fase I.

El objetivo de la Fase I es formular y resolver un problema auxiliar que:
\begin{itemize}
  \item Sea fácil de resolver
  \item Tenga una SBF evidente
  \item Su solución, si es factible, nos permita obtener una SBF \textbf{del problema original}
\end{itemize}

Este método consiste en cuatro pasos generales:
\begin{enumerate}

  \item \textbf{Convertir el PPL original a forma estándar}: Esto incluye que todas las restricciones sean igualdades (introduciendo variables de holgura, exceso, etc.) y todas las variables estén acotadas inferiormente por cero.

  \item \textbf{Identificar las restricciones problemáticas}: Por ejemplo, si una ecuación tiene una constante en el lado derecho negativa, o si las variables artificiales son necesarias para armar una base inicial.

  \item \textbf{Agregar variables artificiales}: Se utilizan cuando las restricciones del problema no permiten obtener una solución básica factible inicial (SBF inicial) de forma directa, es decir, cuando no se puede formar una matriz identidad a partir de las variables de holgura. Esto se hace cuando:
    \begin{itemize}
      \item En el caso \(\geq\), se agregan \textit{variables superfluas}, que restan una variable: \(-s_i\), lo cual da lugar a columnas con \(-1\) en vez de \(1\), y por tanto no son vectores \(e_i\).
      \item En el caso \(=\), al no haber ni suma ni resta de variables de holgura, no se genera ninguna columna tipo \(e_i\) automáticamente.
    \end{itemize}
    donde las columnas \(e_i\) son columnas que tienen un elemento 1 y el resto cero y permiten formar una matriz identidad.
  \begin{tcolorbox}[myconclusion]
    Las variables artificiales son variables auxiliares o ficticias que se introducen en el PPL para poder iniciar el algoritmo Simplex. No tienen un significado en el problema original; son puramente un ``truco matemático''.
  \end{tcolorbox}

  \item \textbf{Formular un problema auxiliar (problema de la Fase I)}: Se define una nueva función objetivo auxiliar: Minimizar la suma de todas las variables artificiales. Esta función objetivo representa el “costo” de alejarse del espacio factible del problema original.

  \item \textbf{Resolver el problema auxiliar con el método Simplex}: 
  \begin{itemize}
    \item Si la solución óptima del problema auxiliar tiene valor cero, significa que se ha encontrado una SBF para el problema original.
    \item Si la solución óptima tiene valor distinto de cero, el problema original no es factible (no existe ninguna solución que satisfaga todas las restricciones).
  \end{itemize}

  \item \textbf{Eliminar las variables artificiales}: Si la Fase I fue exitosa, se eliminan las variables artificiales (si aún están presentes en la base se deben reemplazar mediante pivoteos), y se continúa con la \textit{Fase II}, ahora con una SBF válida y la función objetivo original.
\end{enumerate}

El problema auxiliar busca un punto factible \hl{minimizando} el ``uso'' de las variables artificiales. Si puede lograr que todas ellas sean cero, significa que existe una combinación de las variables reales (originales y de holgura) que satisfacen todas las restricciones. Esa es justamente una solución básica factible. Veamos un ejemplo de la Fase I.

\begin{quote}
  \ejemplo\label{ej:fase_1} Sea el siguiente problema de programación lineal:
  \begin{align*}
    \text{maximizar} \quad  &z = 3x_1 + 2x_2\\[3pt]
    \text{sujeto a:} \quad  &x_1 + x_2 = 4\\
                            &x_1 - x_2 \geq 2\\
                            &x_1, x_2 \geq 0
  \end{align*}
  Observamos que:
  \begin{itemize}
    \item La primera restricción ya está en forma de igualdad, pero no tiene una variable de holgura asociada que permita incluir una columna identidad. Por lo tanto debemos agregar una variable artificial.
    \item La segunda es una desigualdad \(\geq\), por lo tanto debemos restar una variable superflua, y también debemos agregar una variable artificial.
  \end{itemize}
  
  \subparagraph{Paso 1: Convertimos a forma estándar y agregamos variables artificiales}
  \begin{enumerate}
    \item A la primera ecuación (que ya es igualdad), agregamos una variable artificial \(a_1\), porque no hay variable de holgura que permita formar la base inicial.
    \item A la segunda desigualdad, restamos una variable de exceso \(s_1\), y agregamos una variable artificial \(a_2\).
  \end{enumerate}
  El sistema queda:
  \begin{align*}
    x_1 + x_2 + a_1 &= 4\\
    x_1 - x_2 - s_1 + a_2 &= 2
  \end{align*}  
  Con condiciones:
  \[
    x_1, x_2, s_1, a_1, a_2 \geq 0
  \]

  \subparagraph{Paso 2: Definimos el problema auxiliar}
  
  La función objetivo auxiliar es:
  \[
    \omega = a_1 + a_2
  \]
  Nuestro nuevo problema (auxiliar) es:
  \begin{align*}
    \text{minimizar} \quad  &\omega = a_1 + a_2\\[3pt]
    \text{sujeto a:} \quad  &x_1 + x_2 + a_1 = 4\\
                            &x_1 - x_2 - s_1 + a_2 = 2\\
                            &x_1, x_2, s_1, a_1, a_2 \geq 0
  \end{align*}

  Para las restricciones, la base inicial\footnote{En la siguiente sección se verá qué es una base inicial. Por ahora puedes entenderlo como un conjunto de las variables básicas que dan solución factible al problema.} está formada por las variables artificiales \(a_1\) y \(a_2\), porque aparecen con coeficiente 1 y sólo en una ecuación (forman una matriz identidad).
  \begin{align*}
    \begin{pmatrix}
      1 & 1 & 0 & 1 & 0\\
      1 & -1 & -1 & 0 & 1
    \end{pmatrix}
    \begin{pmatrix}
      x_1\\
      x_2\\
      s_1\\
      a_1\\
      a_2
    \end{pmatrix}
    =
    \begin{pmatrix}
      4\\
      2
    \end{pmatrix}
  \end{align*}
  Esto significa que, si establecemos como variables básicas a \(a_1\) y \(a_2\), y las demás en cero, obtenemos una solución básica factible para el problema auxiliar, ya que \(a_1 = 4\) y \(a_2 = 2\).

  De este modo, ya tenemos una \hl{SBF inicial} para el problema auxiliar.
  
  \subparagraph{Paso 3: Aplicamos el método Simplex}
  
  A partir de esta situación, podríamos construir la tabla Simplex con \(a_1\) y \(a_2\) en la base, y comenzar a iterar para minimizar \(\omega\). Si al finalizar el valor óptimo de \(\omega\) es cero, habremos encontrado una combinación de \(x_1\), \(x_2\) y \(s_1\) que satisface las restricciones sin necesidad de variables artificiales, es decir, una solución básica factible del problema original.
  
  Si en cambio \(\omega > 0\), entonces el problema original no tiene solución factible.  
\end{quote}

\begin{tcolorbox}[interesting_data, title=Siempre recordar lo que se busca en la Fase I]
  \begin{itemize}
    \item \textbf{Objetivo}: Minimizar \(\omega = \sum(\text{artificiales})\)
    \item \textbf{Costos}: Coeficientes de artificiales = 1, resto = 0
    \item \textbf{Costos reducidos}: Miden impacto en suma de artificiales
    \item \textbf{Criterio}: Costo reducido más negativo entra
    \item \textbf{Meta}: Llegar a \(\omega = 0\)
  \end{itemize}
\end{tcolorbox}
Ahora profundizaremos más en lo que dice este cuadro. 

\subsubsection{Construcción de la primera tabla del método Simplex (problema auxiliar)}
\label{sec:tabla_simplex}

Una vez que tienes el problema auxiliar planteado, el siguiente paso es construir la primera tabla del método Simplex para el problema auxiliar, resolverlo con iteraciones hasta obtener una SBFI, y a partir de ahí: si \(\omega = 0\), comenzar el Simplex para el problema original con esa base.

Entonces, siguiendo con el ejemplo \ref{ej:fase_1}, vamos a armar la primer tabla Simplex. Para armar la primer tabla Simplex debemos tener en cuenta algunos detalles, como la construcción del vector de costos y la selección de la base. Aunque en el ejemplo \ref{ej:fase_1} más o menos vimos estos conceptos, ahora los veremos más en detalle.

\paragraph{Contrucción del vector de costos \(c\) para la Fase I}

Recordando que la función objetivo del problema auxiliar es \(\omega = a_1 + a_2\), por lo tanto el \textbf{vector de costos} asociado a todas las variables \textbf{del PPL auxiliar} (en el orden \(x_1, x_2, s_1, a_1, a_2\)) es: 
\[
  \hat{c} = (0, 0, 0, 1, 1)
\]
Ya que \(x_1\), \(x_2\) y \(s_1\) no participan en la función objetivo del problema auxiliar, su coeficiente es cero.

Un detalle muy importante es que el problema auxiliar es de \textit{minimización}, por lo que no está en formato estándar. Por ello debemos multiplicar a la función objetivo por (\(-1\)), resultando:
\[
  \text{max} \quad -\omega = - a_1 - a_2 \quad \rightarrow \quad \hat{c}=(0,0,0,-1,-1)
\]

\paragraph{Elección de la base inicial}

Para poder aplicar el método Simplex, se necesita una \textit{base inicial} formada por un conjunto de variables básicas tales que el sistema:
\[
A_{\beta}X_{\beta} = B
\]
tenga una solución factible (es decir, \(X_\beta \geq 0\)).
\begin{tcolorbox}[remember, title=Aclaración]
  Cuando se usa \(A_\beta\) o \(X_\beta\) se refiere al vector de \textbf{variables básicas} y sus respectivos coeficientes. Eso no tiene nada que ver con la matriz \(B\).
\end{tcolorbox}

En el problema auxiliar hemos introducido \(a_1\) y \(a_2\) de modo que:
\begin{itemize}
  \item Cada una aparece \textbf{una sola vez} en una ecuación,
  \item Con coeficiente 1,
  \item Y no aparece en las demás ecuaciones.
\end{itemize}
Este tipo de estructura es ideal para que las variables artificiales sirvan como base inicial. Por lo tanto, tomamos como base inicial:
\[
  \beta = \{a_1, a_2\}
\]
Esto garantiza que la matriz base \(A_\beta\) es la \textit{matriz identidad} \(I_2\), y por lo tanto, \(X_\beta = B\) tiene solución inmediata. En forma matricial es:
\begin{align*}
  A_\beta =
  \begin{pmatrix}
    1 & 0\\
    0 & 1
  \end{pmatrix}
  \quad
  X_\beta =
  \begin{pmatrix}
    a_1\\
    a_2
  \end{pmatrix}
  \quad
  B =
  \begin{pmatrix}
    4\\
    2
  \end{pmatrix}
\end{align*}
Entonces:
\begin{align*}
  A_\beta X_\beta = B \quad \rightarrow \quad {a_1} = 4, ~~ {a_2} = 2
\end{align*}
Entonces ya tenemos una \textbf{SBF} inicial (como vimos anteriormente).

\paragraph{Introducción de costos reducidos}

El costo reducido \(\bar{c}_j\) se define como:
\[
  \bar{c}_j = c_j - c_\beta^T A_\beta^{-1} A_j
\]
donde:
\begin{itemize}
  \item \(\bar{c}_j\) es el vector de costos reducidos correspondiente a la variable \(x_j\) que está fuera de la base (candidata a entrar),
  \item \(c_j\) es el valor \textit{j-ésimo} del vector de costos de la función objetivo,
  \item \(c_\beta^T\) es el vector de costos de la base (que es la base actual),
  \item \(A_\beta^{-1}\) es la matriz inversa de la matriz de los coeficientes de las restricciones que forman la base actual,
  \item \(A_j\) es la columna del sistema que corresponde a la variable \(x_j\) que está fuera de la base (candidata a entrar).
\end{itemize}

\noindent Esta fórmula nos dice: si agrego la variable \(x_j\) (actualmente fuera de la base), ¿cuánto cambia la función objetivo?
\begin{itemize}
  \item Si \(\bar{c}_j < 0\) en maximización (o \(\bar{c}_j > 0\) en minimización), es favorable introducir \(x_j\).
  \item Si \(\bar{c}_j = 0\), la variable es \textit{degenerada} (puede introducirse sin cambiar el valor de la función objetivo).
  \item Si \(\bar{c}_j > 0\) (en maximización), introducir esa variable empeora la función objetivo.
\end{itemize}

\begin{tcolorbox}[mydanger]
  Atención en esta parte, suele ser un poco complicada de entender al principio. Recomiendo ir tomando nota de los distintos costos para relacionarlos rápidamente con la fórmula de costos reducidos.
\end{tcolorbox}

\noindent A continuación vamos a ver detalladamente todos los costos que se utilizan en la fórmula de costos reducidos.

\subparagraph{1. Vector de costos \textit{c}}

Anteriormente vimos que el vector \textit{c} es el vector de coeficientes de la función objetivo del problema original. Si el problema original es:
\[
  \max z = c_1 x_1 + c_2 x_2 + \cdots + c_n x_n
\]
entonces el vector \(c\) es:
\[
  c = \begin{bmatrix} c_1 & c_2 & \cdots & c_n \end{bmatrix}
\]
Este vector se utiliza una vez que el problema auxiliar ha sido resuelto, y se quiere retomar el método Simplex para optimizar el problema original.

\subparagraph{2. Vector de costos \(c_\omega\) (función auxiliar)}

Este es el vector de coeficientes de la \hl{función auxiliar} \(\omega = a_1 + a_2 + \cdots + a_k\) que se introduce para hallar una solución básica factible inicial.

En este caso, el vector \(c_\omega\) es algo así como:
\[
  c_\omega = \begin{bmatrix} 0 & 0 & \cdots & 1 & 1 & \cdots \end{bmatrix}
\]
donde los ceros corresponden a las variables originales y de holgura, y los unos a las variables artificiales.

Este vector solo se usa mientras resolvemos el problema auxiliar (es decir, en la Fase I). Los costos reducidos que se calculan durante la Fase I (con \(c_\omega\)) determinan el pivoteo para minimizar \(\omega\).

\vspace{5mm}

\subparagraph{3. Vector de costos de la base \(c_\beta\)}

Este vector contiene los \textit{costos asociados a las variables que actualmente están en la base}.

Para ilustrar mejor este vector, supongamos el mismo ejemplo que vimos anteriormente (ejemplo \ref{ej:fase_1}).

\begin{table}[htbp]
  \centering
  \renewcommand\cellalign{tl}
  \renewcommand\cellgape{\Gape[4pt]}
  \begin{tabular}{c|c}
  \textit{PPL Original} & \textit{PPL Auxiliar} \\
  \hline
  \makecell[l]{
    \(\text{maximizar} \quad z = 3x_1 + 2x_2\)\\[3pt]
    \(\text{sujeto a:} \quad x_1 + x_2 = 4\)\\
    \hspace{4.5em}\(x_1 - x_2 - s_1 = 2\)\\
    \hspace{4.5em}\(x_1, x_2, s_1 \geq 0\)
  }
  &
  \makecell[l]{
    \(\text{maximizar} \quad -\omega = -a_1 - a_2\)\\[3pt]
    \(\text{sujeto a:} \quad x_1 + x_2 + a_1 = 4\)\\
    \hspace{4.5em}\(x_1 - x_2 - s_1 + a_2 = 2\)\\
    \hspace{4.5em}\(x_1, x_2, s_1, a_1, a_2 \geq 0\)
  }
  \\
  \end{tabular}
  \caption{Recordatorio de PPL Original y PPL Auxiliar de ejemplo \ref{ej:fase_1}}
  \label{tab:comparacion_ppl}
\end{table}

Ahora, recordando que la base \(\beta\) es el conjunto de las variables básicas que dan solución factible al problema, podríamos construir una base para cada uno de los PPLs del cuadro \ref{tab:comparacion_ppl}. Por ejemplo, para el PPL auxiliar, vamos a elegir la base \(\beta_\omega = \{a_1, a_2\}\)\footnote{la base del PPL auxiliar se ha elegido a propósito, para que se note como se forma una matriz identidad}, y para el PPL original, la base podría ser \(\beta = \{x_1, x_2\}\).

Entonces el vector \(c_\beta\) para cada uno de los PPLs sería:
\begin{itemize}
  \item Para el PPL original (\(\beta\)): \(c_\beta = \begin{bmatrix} 3 & 2 \end{bmatrix}\) 
  \item Para el PPL auxiliar (\(\beta_\omega\)): \(c_{\beta\omega} = \begin{bmatrix} -1 & -1 \end{bmatrix}\)
\end{itemize}
\begin{tcolorbox}[remember, title=Aclaración]
  Vea que el vector de costos \(c_\beta\) no es más que los costos asociados a la base \(\beta\) que se esté usando en ese momento.
\end{tcolorbox}

\subparagraph{4. La matriz \(A_\beta^{-1}\)}

Es la matriz inversa de la submatriz base (\(A_\beta\)), que está formada por las columnas del sistema \(A\) que corresponden a las variables en la base actual.

Continuemos con el ejemplo del cuadro \ref{tab:comparacion_ppl}. Entonces, las bases y las submatrices asociadas a cada base para cada PPL son:
\begin{align*}
  \beta = \{x_1,x_2\} \quad \rightarrow& \quad A_\beta = \begin{bmatrix} 1 & 1\\ 1 & -1 \end{bmatrix}\\[4pt]
  \beta_\omega = \{a_1, a_2\} \quad \rightarrow& \quad A_{\omega\beta} = \begin{bmatrix} 1 & 0\\ 0 & 1 \end{bmatrix}
\end{align*}
Entonces, debemos calcular la inversa para cada una de las matrices base:
\begin{align*}
  A_\beta^{-1} &= \begin{bmatrix} 1 & 1\\ 1 & -1 \end{bmatrix}^{-1} = \begin{bmatrix} 0.5 & 0.5\\ 0.5 & -0.5 \end{bmatrix}\\[4pt]
  A_{\omega\beta}^{-1} &= \begin{bmatrix} 1 & 0\\ 0 & 1 \end{bmatrix}^{-1} = \begin{bmatrix} 1 & 0\\ 0 & 1 \end{bmatrix}
\end{align*}

\subparagraph{5. Aplicación a los costos reducidos}

Con esto, podemos entonces, reemplazar en la fórmula de costos reducidos:
\[
  \bar{c}_j = c_j - c_\beta^T A_\beta^{-1} A_j
\]
Entonces, para el PPL \hl{auxiliar}, el costo reducido para la variable \(x_1\) sería:
\begin{enumerate}
  \item Costo de la variable en la función objetivo: \(c_j = 0\)
  \item Vector de costos de la base: \(c_\beta^T = \begin{bmatrix} -1 & -1 \end{bmatrix}\)
  \item Matriz inversa de la matriz de los coeficientes de las restricciones que forman la base actual: \(A_\beta^{-1} = \begin{bmatrix} 1 & 0\\ 0 & 1 \end{bmatrix}\)
  \item Columna del sistema que corresponde a la variable \(x_1\) que está fuera de la base (candidata a entrar): \(A_{x1} = \begin{bmatrix} 1\\ 1 \end{bmatrix}\)
\end{enumerate}
Entonces resulta:
\begin{align*}
  \bar{c}_{x1} = 0 - \begin{bmatrix} -1 & -1 \end{bmatrix} \begin{bmatrix} 1 & 0\\ 0 & 1 \end{bmatrix} \begin{bmatrix} 1\\ 1 \end{bmatrix} = 2
\end{align*}

\noindent Y para el PPL \hl{original}, el costo reducido para la variable \(s_1\):
\begin{enumerate}
  \item Costo de la variable en la función objetivo: \(c_j = 0\)
  \item Vector de costos de la base: \(c_\beta^T = \begin{bmatrix} 3 & 2 \end{bmatrix}\)
  \item Matriz inversa de la matriz de los coeficientes de las restricciones que forman la base actual: \(A_\beta^{-1} = \begin{bmatrix} 0.5 & 0.5\\ 0.5 & -0.5 \end{bmatrix}\)
  \item Columna del sistema que corresponde a la variable \(s_1\) que está fuera de la base (candidata a entrar): \(A_{s1} = \begin{bmatrix} 0\\ -1 \end{bmatrix}\)
\end{enumerate}
\begin{align*}
  \bar{c}_{s1} = 0 - \begin{bmatrix} 3 & 2 \end{bmatrix} \begin{bmatrix} 0.5 & 0.5\\ 0.5 & -0.5 \end{bmatrix} \begin{bmatrix} 0\\ -1 \end{bmatrix} = 0.5
\end{align*}

\vspace{5mm}

\begin{tcolorbox}[remember, title=¿Para qué usamos costos reducidos?]
  Los costos reducidos indican para cada variable cuánto se modificaría la función objetivo si esa variable entra en la base. Es decir, por ejemplo en el caso del PPL auxiliar, el costo reducido de la variable \(x_1\) es \(2\), lo que significa que si \(x_1\) entra en la base, la función objetivo \(z\) cambiaría en \(2\) unidades.
\end{tcolorbox}

\newpage

\paragraph{Armado de la tabla Simplex: Fase I}

Ahora si, ya estamos completamente listos para armar la tabla. Voy a comentar algunas cosas al lector:
\begin{quote}
  ``Hasta ahora hemos visto todo lo necesario para entender el armado de la tabla. Luego el algoritmo en sí es más sencillo. Este método es bastante largo y consiste en muchos pasos y conceptos previos antes de la formación de la tabla inicial. No se frustre si le toma tiempo entenderlo, a mi también me costó la primera vez que lo estudié.'' 
\end{quote}

Considerando la transformación de la función \(\omega\) del problema auxiliar a maximización (\(-\omega = z = -a_1 - a_2\)) y la base inicial \(\{a_1, a_2\}\) los pasos para armar la tabla consisten en juntar lo que hemos visto anteriormente.

\subparagraph{Paso 1: Identificar componentes}
\begin{itemize}
  \item Vector de costos (\(c_j\)) para \(z\):  
  \[
  c = \begin{bmatrix}
  c_{x_1} & c_{x_2} & c_{s_1} & c_{a_1} & c_{a_2}
  \end{bmatrix} = \begin{bmatrix}
  0 & 0 & 0 & -1 & -1
  \end{bmatrix}
  \]
  \item Matriz de restricciones (\(A\)):
  \[
  A = \begin{bmatrix}
  1 & 1 & 0 & 1 & 0 \\
  1 & -1 & -1 & 0 & 1
  \end{bmatrix}, \quad B = \begin{bmatrix} 4 \\ 2 \end{bmatrix}
  \]
  \item Base inicial: \(\beta = \{a_1, a_2\}\) → \(A_\beta = \begin{bmatrix} 1 & 0 \\ 0 & 1 \end{bmatrix}\) (matriz identidad).
\end{itemize}

\subparagraph{Paso 2: Calcular valores clave}
\begin{itemize}
  \item Solución básica:
   \[
   A_\beta^{-1} \cdot B = \begin{bmatrix} 1 & 0 \\ 0 & 1 \end{bmatrix} \begin{bmatrix} 4 \\ 2 \end{bmatrix} = \begin{bmatrix} 4 \\ 2 \end{bmatrix}
   \]
   Entonces \(\rightarrow\) variables básicas: \(a_1 = 4\), \(a_2 = 2\).

  \item Valor de \(z\) inicial:
   \(
   z = c_\beta^T \cdot B = \begin{bmatrix} -1 & -1 \end{bmatrix} \begin{bmatrix} 4 \\ 2 \end{bmatrix} = -6
   \)

  \item Costos reducidos (\(\bar{c}_j\)). Aplicamos la definición para cada variable:
   \[
   \bar{c}_j = c_j - c_\beta^T A_\beta^{-1} A_j
   \]
   \begin{itemize}
    \item Para \(x_1\):
     \(
     \bar{c}_{x_1} = 0 - \begin{bmatrix} -1 & -1 \end{bmatrix} \begin{bmatrix} 1 & 0\\ 0 & 1 \end{bmatrix} \begin{bmatrix} 1 \\ 1 \end{bmatrix} = 0 - (-2) = 2
     \)
    \item Para \(x_2\):
     \(
     \bar{c}_{x_2} = 0 - \begin{bmatrix} -1 & -1 \end{bmatrix} \begin{bmatrix} 1 & 0\\ 0 & 1 \end{bmatrix} \begin{bmatrix} 1 \\ -1 \end{bmatrix} = 0 - ( -1 + 1 ) = 0
     \)
    \item Para \(s_1\):
     \(
     \bar{c}_{s_1} = 0 - \begin{bmatrix} -1 & -1 \end{bmatrix} \begin{bmatrix} 1 & 0\\ 0 & 1 \end{bmatrix} \begin{bmatrix} 0 \\ -1 \end{bmatrix} = 0 - ( 0 + 1 ) = -1
     \)
    \item Para \(a_1\):
     \(
     \bar{c}_{a_1} = -1 - \begin{bmatrix} -1 & -1 \end{bmatrix} \begin{bmatrix} 1 & 0\\ 0 & 1 \end{bmatrix} \begin{bmatrix} 1 \\ 0 \end{bmatrix} = -1 - (-1) = 0
     \)
    \item Para \(a_2\):
     \(
     \bar{c}_{a_2} = -1 - \begin{bmatrix} -1 & -1 \end{bmatrix} \begin{bmatrix} 1 & 0\\ 0 & 1 \end{bmatrix} \begin{bmatrix} 0 \\ 1 \end{bmatrix} = -1 - (-1) = 0
     \)
   \end{itemize}
\end{itemize}

\begin{tcolorbox}[myconclusion]
  Vea que las matrices identidad pueden quitarse y no modifican el resultado final. Esto es porque la matriz identidad es el elemento neutro de la multiplicación en matrices.
\end{tcolorbox}

\subparagraph{Paso 3: Estructura general de la tabla Simplex}

Recordando que \(\beta\) es la base y \textit{B} es el vector de términos independientes la tabla es:

\[
\begin{array}{c|ccccc|c}
\beta & x_1 & x_2 & s_1 & a_1 & a_2 & B \\
\hline
a_1 & 1 & 1 & 0 & 1 & 0 & 4 \\
a_2 & 1 & -1 & -1 & 0 & 1 & 2 \\
\hline
z & -2 & 0 & 1 & 1 & 0 & -6 \\
\end{array}
\]

\begin{tcolorbox}[danger_box, title=Prevención de Riesgos]
  Es importante tener en cuenta que los costos reducidos se introducen con signo cambiado a la tabla, es decir \(-\bar{c}_j\). 
  
  La convención de mostrar \(-\bar{c}_j\) (en lugar de \(\bar{c}_j\)) en la fila objetivo (\(z\)) de la tabla Simplex es universal y aplica a todas las fases del método Simplex, no solo a la Fase I. Esta práctica es estándar y se mantiene tanto en problemas de maximización como de minimización, y en todas las etapas (Fase I, Fase II, o problemas sin variables artificiales).

  En la sección \ref{sec:iteraciones_simplex} veremos la regla del criterio, donde cobra importancia esta convención.
\end{tcolorbox}

\noindent las columnas de la tabla son:
\begin{itemize}
  \item \(\beta\): Variables básicas actuales (\(a_1, a_2\)). Estas irán cambiando en las iteraciones,
  \item \(x_1\): Coeficientes de cada variable en las restricciones,
  \item \(s_1\): Valores de las variables básicas (solución actual),
  \item \(a_1\): Coeficientes de cada variable en las restricciones,
  \item \(a_2\): Coeficientes de cada variable en las restricciones,
  \item \(B\): Valores de las variables básicas (solución actual),
  \item Fila \(z\): muestra los costos reducidos de la función objetivo y el valor actual de \(z\), que en este caso es \(-6\).
\end{itemize}
