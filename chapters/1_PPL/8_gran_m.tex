\subsection{El método de la Gran M: alternativa a Fase 1}

El método de la Fase I no es el único método para encontrar una SBFI. El método de la Gran M tiene el mismo propósito.

El motivo por el que se vió primero el método de la Fase I es porque el método de la Gran M presenta un pequeño riesgo cuando se elige el valor de M. Ahora que hemos visto cómo funciona el método de la Fase I, será más sencillo entender cómo hay que manipular este método. Además el método de la Fase I es preferido en ámbitos computacionales, ya que presenta mayor estabilidad. ¿Por qué? Bueno, estos detalles los veremos a continuación.

\subsubsection{Método de los Costos de Penalización (Método de la ``Gran M'')}

Es una técnica utilizada en programación lineal para manejar problemas que no tienen una solución básica factible inicial (SBFI) evidente. Se emplea cuando aparecen restricciones de igualdad (\(=\)) o desigualdad (\(\geq\)) que impiden formar una base inicial con variables de holgura. Su objetivo es forzar a las variables artificiales a cero en la solución óptima, penalizando su inclusión en la función objetivo con un coeficiente muy grande (``M''), donde \(M\) es un valor positivo arbitrariamente grande (o negativo para problemas de maximización).

Este método consiste en los siguientes pasos:

\noindent\textbf{1.} Agregar variables artificiales: En restricciones de igualdad o \(\geq\), se añaden variables artificiales \(a_i \geq 0\) para formar una base inicial. Ejemplo:  
\[
  x_1 - x_2 \geq 2 \quad \rightarrow \quad x_1 - x_2 - s_2 + a_2 = 2 \quad (s_2: \text{exceso}, \ a_2: \text{artificial})
\]

\noindent\textbf{2.} Modificar la función objetivo:
\begin{itemize}
  \item Minimización: Se suma \(M \cdot a_i\) a la función objetivo.  
  \[
  \text{Min } z = c^T x + M(a_1 + a_2 + \cdots)
  \]
  \item Maximización: Se resta \(M \cdot a_i\) a la función objetivo.  
  \[
    \text{Max } z = c^T x - M(a_1 + a_2 + \cdots)
  \]
  \item ``M'' representa una penalización arbitrariamente grande para que el algoritmo elimine \(a_i\) de la base (como si fuese una variable de decisión inconveniente).
\end{itemize}  

\noindent\textbf{3.} Resolver con el método Simplex: Se aplica el Simplex estándar a este nuevo problema. Si en la solución óptima todas \(a_i = 0\) la solución es factible para el problema original. Por el contrario si alguna \(a_i > 0\) el problema original es infactible.

\paragraph{¿Cuándo se usa?}

Se usa en los mismos casos en donde se usa la Fase I, es decir, se aplica en estos escenarios:
\begin{enumerate}
  \item Restricciones de igualdad:  
     \[
     x_1 + x_2 = 4 \quad \rightarrow \quad x_1 + x_2 + a_1 = 4
     \]
  
  \item Restricciones de desigualdad (\(\geq\)) sin holgura inicial:  
     \[
     x_1 - x_2 \geq 2 \quad \rightarrow \quad x_1 - x_2 - s_2 + a_2 = 2
     \]
\end{enumerate}
Además también se utiliza en problemas donde la Fase I del método de dos fases no es viable, o se busca resolver en una sola fase.

\ejemplo{ Veamos un ejemplo del método}:
\begin{quote}
  
Problema original:  
\[
\begin{aligned}
\text{Max } z = 3x_1 &+ 2x_2 \\
\text{s.t. } x_1 &+ x_2 = 4 \\
x_1 &- x_2 \geq 2 \\
x_1, x_2 &\geq 0
\end{aligned}
\]

Paso 1: Agregar variables artificiales:  
\[
\begin{aligned}
x_1 + x_2 + a_1 &= 4 \\
x_1 - x_2 - s_2 + a_2 &= 2
\end{aligned}
\]

Paso 2: Función objetivo penalizada:  
\[
\text{Max } z = 3x_1 + 2x_2 - M a_1 - M a_2
\]

Paso 3: Tabla inicial Simplex (base \(a_1, a_2\)):  
\[
  \begin{NiceMatrix}
    \beta & x_1 & x_2 & s_2 & a_1 & a_2 & B \\
    \hline
    a_1 & 1 & 1 & 0 & 1 & 0 & 4 \\
    a_2 & 1 & -1 & -1 & 0 & 1 & 2 \\
    \hline
    z & -3 & -2 & 0 & M & M & 0
  \end{NiceMatrix}
\]

Ahora se resuelve aplicando los 4 pasos del método simplex. 
\end{quote}

---

#### Interpretación de "M"
- "M" es un valor simbólico que representa un número muy grande (ej: \(10^6\)).  
- En los cálculos, se trata algebraicamente:  
  - Si \(M = 1000\), un coeficiente \(2M - 3\) se interpreta como \(2000 - 3 = 1997\).  
- Regla para pivoteo:  
  - Se priorizan las columnas con coeficientes más negativos en términos de \(M\) (ej: \(-3M + 2\) es "más negativo" que \(-2M + 5\)).

---

#### Ventajas y desventajas
| Ventajas                          | Desventajas                          |
|---------------------------------------|------------------------------------------|
| Resolución en una sola fase (sin Fase I). | Inestabilidad numérica si \(M\) es muy grande. |
| Útil para problemas pequeños.         | Dificultad para manejar \(M\) simbólicamente en software. |
| Fundamentos teóricos claros.          | Si \(M\) no es suficientemente grande, puede dar soluciones incorrectas. |

---

#### ¿Método de la Gran M vs. Dos Fases?
| Criterio           | Gran M                            | Dos Fases                          |
|------------------------|---------------------------------------|----------------------------------------|
| Función objetivo   | Original ± \(M \sum a_i\)             | Fase I: Min \(\sum a_i\); Fase II: Original |
| Eficiencia         | Menos iteraciones (en teoría)         | Más estable numéricamente              |
| Uso en software    | Poco común (por inestabilidad)       | Muy común (ej: CPLEX, Gurobi)         |
| Infactibilidad     | \(a_i > 0\) en óptimo                | \(\omega > 0\) en Fase I              |

---

### Conclusión
El método de los costos de penalización es una herramienta teórica importante para garantizar la factibilidad inicial en problemas complejos de programación lineal. Sin embargo, en la práctica, el método de dos fases es preferido por su estabilidad numérica. Su comprensión es fundamental para abordar problemas sin SBFI y profundizar en la teoría de optimización.