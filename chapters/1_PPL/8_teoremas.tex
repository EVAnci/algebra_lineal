\subsection{Teoría de soluciones}

En el método Simplex, la teoría de soluciones se refiere al conjunto de ideas matemáticas que justifican por qué este algoritmo puede encontrar una solución óptima a un problema de programación lineal (PPL), y por qué es suficiente evaluar los puntos extremos (vértices) de la región factible para encontrarla, en lugar de considerar todos los puntos del conjunto de soluciones.

\subsubsection{Dependencia e independencia lineal}

Un conjunto de vectores de \textit{m} dimensiones \(\left\{P_1,P_2,\cdots, P_n\right\}\), es \textbf{linealmente dependiente} si existen constantes, no todas cero, \(\alpha_1,\, \alpha_2,\, \cdots,\, \alpha_n\) tales que:
\begin{equation}
  \alpha_1 P_1 + \alpha_2 P_2 + \cdots + \alpha_n P_n = 0
  \label{eq:dependencia_lineal}
\end{equation}
donde \(0\) es el vector nulo de \textit{m} dimensiones.

\begin{quote}
  \ejemplo{ El conjunto de 5 dimensiones}
  \[
    \left\{
      \left[1,2,0,0,0\right];
      \left[1,0,0,0,0\right];
      \left[0,0,1,1,0\right];
      \left[0,1,0,0,0\right]
    \right\}
  \]
  es linealmente dependiente, ya que 
  \begin{align*}
    -1\begin{pmatrix}
      1 \\ 2 \\ 0 \\ 0 \\ 0
    \end{pmatrix} + 1 \begin{pmatrix}
      1 \\ 0 \\ 0 \\ 0 \\ 0
    \end{pmatrix} + 0 \begin{pmatrix}
      0 \\ 0 \\ 1 \\ 1 \\ 0
    \end{pmatrix} + 2 \begin{pmatrix}
      0 \\ 1 \\ 0 \\ 0 \\ 0
    \end{pmatrix} = \begin{pmatrix}
      0 \\ 0 \\ 0 \\ 0 \\ 0
    \end{pmatrix}
  \end{align*}
\end{quote}

\teorema{Cualquier conjunto de \(m+1\) o más vectores de \textit{m} dimensiones es linealmente dependiente.}

\noindent Es decir, si se tiene un conjunto de \textit{6} vectores de \textit{5} dimensiones, este conjunto es linealmente dependiente.

\vspace{5mm}

Un conjunto de vectores de \textit{m} dimensiones \(\left[P_1, P_2, \cdots, P_n\right]\), es \textit{linealmente independiente} si las únicas constantes para las que la definición de dependencia lineal (\ref{eq:dependencia_lineal}) se cumple, son \(\alpha_1, \alpha_2, \cdots, \alpha_n = 0\). Es decir, todas las constantes deben ser igual a cero.

A continuación veremos algunos ejemplos.
\vspace{5mm}

\begin{quote}
  \ejemplo{ Determínese si el conjunto \(\{\left[1,2\right]; \left[2,4\right]\}\)} es linealmente independiente.
  
  Si se denomina \(P_1\) y \(P_2\) a los dos vectores es obvio que \(P_2 = 2P_1\), o bien:
  \[
    2P_1 + (-1)P_2 = 0
  \]
  Entonces, el conjunto dado de vectores es \textbf{linealmente dependiente}.
  
  \vspace{5mm}

  \ejemplo{ ¿Es \(\left\{\left[1,1,3,1\right]; \left[1,2,1,1\right]; \left[1,0,0,0\right]\right\}\)} linealmente independiente?

  Aplicando la definición de dependencia lineal (referencia \ref{eq:dependencia_lineal}):
  \begin{align*}
    \alpha_1 \begin{pmatrix}
      1 \\ 1 \\ 3 \\ 1
    \end{pmatrix} + \alpha_2 \begin{pmatrix}
      1 \\ 2 \\ 1 \\ 1
    \end{pmatrix} + \alpha \begin{pmatrix}
      1 \\ 0 \\ 0 \\ 1
    \end{pmatrix} = \begin{pmatrix}
      0 \\ 0 \\ 0 \\ 0
    \end{pmatrix}
  \end{align*}
  Esto podemos escribirlo como un SEL de la siguiente manera:
  \[
    \begin{cases}
      \alpha_1 + \alpha_2 + \alpha_3 = 0 \\
      \alpha_1 + 2\alpha_2 \phantom{+ a_3} = 0 \\
      3\alpha_1 + \alpha_2 \phantom{+ a_3} = 0 \\
      \alpha_1 + \alpha_2 + \alpha_3 = 0
    \end{cases}
  \]
  Las primeras tres ecuaciones (la cuarta es redundante) tienen \(\alpha_1 = \alpha_2 = \alpha_3 = 0\) como única solución. Por lo tanto, el conjunto dado \textbf{es linealmente independiente}. 
\end{quote}

\subsubsection{Combinaciones convexas}

Un vector \textit{P} de \textit{m} dimensiones es una \hl{combinación convexa} de los vectores \(P_1, P_2, \cdots, P_n\) de \textit{m} dimensiones, si existen constantes no negativas \(\beta_1, \beta_2, \cdots, \beta_n\) cuya suma es 1, como:
\begin{equation}
  P = \beta_1 P_1 + \beta_2 P_2 + \cdots + \beta_n P_n
  \label{eq:def_convexo}
\end{equation}

\begin{quote}
  \ejemplo{ El vector de 2 dimensiones \((5/3,5/6)\) es una combinación convexa de los vectores \((1,1); (3,0); (1,2)\) porque}
  \[
    \begin{pmatrix}
      5/3 \\ 5/6
    \end{pmatrix} = \frac{1}{2}\begin{pmatrix}
      1 \\ 1
    \end{pmatrix} + \frac{1}{3} \begin{pmatrix}
      3 \\ 0
    \end{pmatrix} + \frac{1}{6}\begin{pmatrix}
      1 \\ 2
    \end{pmatrix} \quad \text{y} \quad \frac{1}{2}+\frac{1}{3}+\frac{1}{6} = 1
  \]
\end{quote}

Dados dos vectores de \textit{m} dimensiones, \(P_1, P_2\), se denomina \textit{segmento linea} entre los dos vectores al conjunto de todas las combinaciones convexas de \(P_1\) y \(P_2\). El significado geométrico de este término resulta claro en el caso \(m=3\).

\subsubsection{Conjuntos convexos}

Un conjunto de vectores de \textit{m} dimensiones es convexo, siempre que dos vectores pertenezcan al conjunto, si el segmento linea entre los vectores también pertenece al conjunto.

\begin{quote}
  \ejemplo{ El disco sombreado en la figura \ref{fig:conjuntos_convexos}a es un conjunto convexo, ya que el segmento linea entre cualquiera de sus puntos (vectores de \(\mathbb{R}^2\)) está completamente dentro del disco.} 
  \begin{figure}[ht]
    \centering
    \includegraphics[width=0.8\textwidth]{conjuntos_convexos.png}
    \caption{}
    \label{fig:conjuntos_convexos}
  \end{figure}
  La figura \ref{fig:conjuntos_convexos}b no es un conjunto convexo; aunque \textit{R} y \textit{S} pertenezcan al conjunto sombreado, existen puntos, como \textit{T}, que pertenecen al segmento de linea entre \textit{R} y \textit{S} y que no son parte de la estrella.
\end{quote}

Un vector \textit{P} es un \textit{punto extremo} de un conjunto, si no es posible expresarlo como una combinación convexa de otros dos vectores del conjunto; es decir, un punto extremo no descansa en el segmento linea entre cualquier otro par de vectores en el conjunto.

\begin{quote}
  \ejemplo{ Cualquier punto sobre la circunferencia del disco de la figura \ref{fig:conjuntos_convexos}a es un punto extremo del disco.}
\end{quote}

\teorema{Cualquier vector en un conjunto convexo cerrado y acotado que tenga un numero finito de puntos extremos, puede expresarse como una combinación convexa de los puntos extremos.}
\label{teo:combinacion_convexa_de_puntos_extremos}

\teorema{El espacio de solución de un conjunto de ecuaciones linales simultáneas es un conjunto convexo que tiene un número finito de puntos extremos.}
\label{teo:solucion_convexa}

\subsubsection{Soluciones de punto extremo}

Se designa \(\vartheta\) al conjunto de todas las soluciones factibles al programa lineal en forma estándar. Es decir \(\vartheta\) es el conjunto de todos los vectores \textbf{X} que satisfacen \(AX = B\) y \(X\geq 0\). Del teorema \ref{teo:solucion_convexa}, y por el hecho de que los conjuntos convexos se intersectan en conjuntos convexos\footnote{es decir, la intersección de dos conjuntos convexos \textit{A} y \textit{B} es otro conjunto \textit{C} que también es convexo} se tiene que \(\vartheta\) es un conjunto convexo con un número finito de puntos extremos.

\textbf{Nota:} La función objetivo logra su óptimo (ya sea máximo o mínimo) eun un punto extremo de \(\vartheta\), siempre y cuando dicho óptimo exista.