\subsection{Iteraciones del método Simplex}
\label{sec:iteraciones_simplex}

Una vez se ha construido la tabla Simplex inicial, se procede a iterar el método Simplex hasta alcanzar la solución óptima. El proceso de iteración no es complicado, de hecho, lo más largo y tedioso suele ser el proceso previo de construcción de la tabla Simplex inicial. De igual manera, depende mucho del problema que se esté resolviendo. Simplex es un método general que sirve para cualquier PPL que pueda ser transformado para cumplir con las condiciones del método Simplex.

A continuación veremos el proceso de iteración del método Simplex paso a paso.

\subsubsection{Algoritmo Simplex Paso a Paso (después de construir la tabla inicial)}
Para mostrar el algoritmo paso a paso, vamos a continuar con el ejemplo con el que trabajamos en el capítulo \ref{sec:tabla_simplex}. Para ello vamos a tener en cuenta las siguientes consideraciones:
\begin{itemize}
  \item Asumimos un problema de \textbf{maximización} y está en su forma estándar.
  \item La tabla Simplex muestra \(-\bar{c}_j\) en la fila \(z\).
  \item Variables artificiales están presentes (Fase I), pero el algoritmo es idéntico para Fase II.
\end{itemize}

Pero antes de continuar con el ejemplo, veamos el algoritmo paso a paso.

\paragraph{Paso 1: Verificar optimalidad}

El objetivo de este paso es evaluar si la solución básica actual (correspondiente a la tabla del método Simplex que ya has construido) es óptima. Esto se hace observando la fila de los coeficientes reducidos (comúnmente llamada fila \(z\)) en la tabla.

En términos prácticos, debes observar los coeficientes de la fila \(z\), excluyendo la columna del término independiente \(B\):
\begin{itemize}
  \item Si todos los coeficientes son mayores o iguales a cero, es decir, \(\geq 0\), entonces ya no es posible mejorar más el valor de la función objetivo. Esto indica que la solución actual es \hl{óptima}.
  \item Si algún coeficiente es negativo, significa que al aumentar el valor de la variable correspondiente, se podría mejorar (reducir, en caso de minimización) el valor de la función objetivo. Por tanto, aún no se ha alcanzado la solución óptima, y se debe continuar con la iteración (Paso 2).
\end{itemize}

El fundamento teórico tras este paso es el siguiente: En el método Simplex, el valor de cada coeficiente en la fila \(z\) representa el costo reducido de introducir la variable correspondiente en la base. Estos coeficientes suelen expresarse como \(-\bar{c}_j\), donde:
\begin{itemize}
  \item \(-\bar{c}_j = c_j - c_B^T B^{-1} A_j\), el costo reducido asociado a la variable \(x_j\).
  \item \(c_j\): coeficiente de \(x_j\) en la función objetivo.
  \item \(c_\beta\): vector de costos de las variables básicas actuales.
  \item \(A_\beta\): matriz de columnas básicas.
  \item \(A_j\): columna de la matriz de restricciones correspondiente a la variable \(x_j\).
\end{itemize}
Entonces:
\begin{itemize}
  \item Si \(-\bar{c}_j \geq 0\) para todos los \(j\), eso significa que \(\bar{c}_j \leq 0\). En este caso, introducir cualquier variable no básica en la base no mejora el valor de la función objetivo, por lo tanto, la solución es óptima.
  \item En cambio, si \(-\bar{c}_j < 0\) para algún \(j\), es decir, \(\bar{c}_j > 0\), entonces aumentar \(x_j\) mejoraría la función objetivo (en un problema de minimización), y se debe continuar iterando.
\end{itemize}

Este paso responde a la pregunta: ¿ya hemos optimizado la función objetivo con las variables básicas actuales?

Si todos los coeficientes en la fila \(z\) son positivos o cero, entonces sí: no hay variables no básicas que puedan mejorar el valor de \(z\).

Si hay coeficientes negativos, entonces no: existe al menos una variable que, al entrar a la base, podría mejorar el resultado. En ese caso, se procede al siguiente paso del algoritmo (selección de variable entrante y saliente).

\begin{tcolorbox}[title=Resumen del paso 1]
  \noindent \textbf{Objetivo}: Determinar si la solución actual es óptima.

  \noindent \textbf{Regla}:
  \begin{itemize}
    \item Si todos los coeficientes en la fila \(z\) (excluyendo la columna \textit{B}) son \(\geq 0\) (positivos) \(\rightarrow\) Solución óptima alcanzada.
    \item Si hay algún coeficiente \(< 0\) (negativo) \(\rightarrow\) Ir al Paso 2.
  \end{itemize}

  \noindent \textbf{Fundamento}:
  \begin{itemize}
    \item Coeficiente en fila \(z = -\bar{c}_j\).
    \item \(-\bar{c}_j \geq 0\) implica \(\bar{c}_j \leq 0\) (no hay variables que mejoren \(z\)).
  \end{itemize}
\end{tcolorbox}

\paragraph{Paso 2: Seleccionar variable entrante}

El paso 2 del método Simplex, denominado ``Seleccionar variable entrante'', es el segundo paso de cada iteración una vez que se ha verificado que la solución actual no es óptima. Aquí se toma una decisión clave: qué variable no básica se incorporará a la base, desplazando a alguna de las actuales.

El objetivo es identificar cuál variable \textbf{no básica} debe ingresar a la base. Esta elección busca mejorar el valor de la función objetivo \(z\) de la manera más eficiente posible, dado el estado actual del sistema.

Para hacer esta selección, se utiliza la fila \(z\) de la tabla simplex, donde están los costos reducidos \(-\bar{c}_j\) (es decir, la ganancia marginal de introducir cada variable no básica):
\begin{itemize}
  \item Se identifica el coeficiente más negativo en la fila \(z\) (excluyendo el término independiente). Esto se debe a que un coeficiente más negativo implica que aumentar esa variable no básica reduce el valor de \(z\) (recordando que en un problema de minimización, buscamos reducir \(z\) al máximo).
  \item Si hay empate entre varios coeficientes igualmente negativos, se puede:
  \begin{itemize}
    \item Elegir cualquiera (esto puede llevar a diferentes caminos, pero en teoría todos válidos), o
    \item Aplicar una regla de desempate, como la Regla de Bland, que elige la variable de menor índice para evitar ciclos.
  \end{itemize}
\end{itemize}

Desde el punto de vista teórico:
\begin{itemize}
  \item Los coeficientes de la fila \(z\) están dados por \(-\bar{c}_j = c_j - c_\beta^T A_\beta^{-1} A_j\). Como explicamos antes, este valor representa cuánto cambiará la función objetivo si se introduce una unidad de la variable \(x_j\) a la solución (es decir, si se la incorpora a la base).
  \item Cuanto más negativo sea \(-\bar{c}_j\), más positivo será \(\bar{c}_j\), y por tanto, mayor será la reducción de \(z\) si se introduce esa variable.
  \item En otras palabras, estamos buscando la variable con el mayor potencial para reducir el valor de la función objetivo, que es precisamente aquella con el \(-\bar{c}_j\) más negativo.
\end{itemize}

Este paso responde a la pregunta: ¿cuál de las variables no básicas, si la incorporamos a la base, más mejora la función objetivo?

La respuesta es: aquella cuya columna en la fila \(z\) tiene el valor más negativo. Aumentar esa variable generará la mayor mejora inmediata en \(z\), por lo tanto es la candidata ideal para entrar en la base.

\begin{tcolorbox}[title=Resumen del paso 2]
  \noindent \textbf{Objetivo}: Elegir la variable que ingresa a la base para mejorar \(z\).
  
  \noindent \textbf{Regla}:
  \begin{itemize}
    \item Seleccionar la columna con el coeficiente más negativo en la fila \(z\).
    \item Si hay empate, elegir cualquiera (o usar regla de Bland).
  \end{itemize}
  
  \noindent \textbf{Fundamento}: El coeficiente más negativo en fila \(z\) (\(-\bar{c}_j\)) corresponde al \(\bar{c}_j\) más positivo, que genera la mayor mejora en \(z\).
\end{tcolorbox}

\paragraph{Paso 3: Seleccionar variable saliente (Ratio Test)}

El \textit{paso 3 del método Simplex}, denominado ``\textit{Seleccionar variable saliente (Ratio Test)}'', completa la decisión de qué variable debe entrar y cuál debe salir de la base en la iteración actual. Este paso asegura que el movimiento hacia una nueva solución básica \textbf{sea factible}, es decir, no viole las restricciones del problema.

El objetivo es determinar \textit{cuál de las variables básicas actuales debe salir de la base} al incorporar la nueva variable entrante. Esto se hace asegurando que el nuevo punto al que se avanza permanezca dentro de la región factible (es decir, que todas las variables sigan siendo no negativas).

Una vez que se ha elegido la variable entrante (la columna con el coeficiente más negativo en la fila \(z\)), se aplica el llamado \textbf{``Test del cociente''} o \emph{Ratio Test} para decidir la variable saliente:
\begin{enumerate}
  \item Trabajar \textbf{solo con las filas correspondientes a las restricciones},
    es decir, las que representan variables básicas actuales.
  \item Observar la \textbf{columna de la variable entrante}.
  \item Considerar \textbf{únicamente los coeficientes positivos} de esa columna
    (coeficientes \(>0\)). Esto es fundamental porque un coeficiente negativo
    implicaría que al aumentar la variable entrante, la variable básica
    correspondiente aumentaría también, lo cual puede violar la factibilidad.
  \item Para cada fila con coeficiente positivo en la columna de la variable
    entrante, calcular el \textbf{cociente}:
    \[
      \mathrm{Ratio}_i
      = \frac{B_i}{A_{i,\mathrm{entrante}}}
    \]
    donde
    \begin{itemize}
      \item \(B_i\): el término independiente de la fila \(i\) (columna ``\textit{B}'').
      \item \(A_{i,\mathrm{entrante}}\): el valor de la columna
        de la variable entrante en la fila \(i\).
    \end{itemize}
  \item Elegir la fila con \textbf{el menor cociente positivo}: es la que
    \emph{llega primero a cero} al aumentar la variable entrante, es decir,
    la \emph{primera en volverse no factible} si seguimos avanzando en esa
    dirección.
  \item La \textbf{variable básica} correspondiente a esa fila es la que
    \textbf{debe salir} de la base.
\end{enumerate}

Desde el punto de vista teórico, este paso garantiza que el movimiento en la
dirección de la variable entrante:

\begin{itemize}
  \item Sea \textbf{factible}, es decir, que todas las variables básicas
    permanezcan no negativas.
  \item No viole ninguna restricción de desigualdad original.
\end{itemize}

Al elegir la \textbf{cantidad máxima que puede incrementarse la variable
entrante} sin hacer negativa alguna variable básica, se asegura que el nuevo
vértice factible es \emph{vecino del actual} y se mantiene dentro del
\emph{poliedro de soluciones factibles}.

Este paso refleja una transición a lo largo de una \emph{arista del poliedro
factible} en la dirección que más mejora \(z\), deteniéndose justo en el
siguiente vértice (nueva solución básica).

\textbf{Caso especial (problema no acotado)}: Si \textbf{todos los coeficientes en la columna de la variable entrante son
\(\leq 0\)}, entonces no se puede aplicar el Ratio Test, ya que aumentar la variable entrante haría que todas las variables básicas crezcan (o no se afecten), \emph{nunca decrezcan}, lo que implica que no hay ningún límite superior que detenga el crecimiento de la variable entrante.

Este escenario indica que el problema \textbf{no está acotado}, es decir, que la
función objetivo \emph{puede seguir mejorando indefinidamente}, y por lo tanto
\textbf{no existe una solución óptima finita}.

Este paso responde a la pregunta:
\begin{quote}
  \textbf{¿Qué variable debe salir de la base para mantener la factibilidad cuando
  una nueva variable entra?}
\end{quote}

La respuesta se obtiene aplicando el Ratio Test: se calcula el cociente entre
el término independiente y el coeficiente de la variable entrante en cada fila,
y se elige el menor cociente positivo. La variable básica correspondiente a
esa fila es la que sale.

\begin{tcolorbox}[title=Resumen del paso 3]
  \noindent \textbf{Objetivo}: Determinar qué variable abandona la base.
  
  \noindent \textbf{Regla}:
  \begin{enumerate}
    \item En la columna de la variable entrante, considerar solo coeficientes \(> 0\).
    \item Calcular ratio para cada fila \(i\):
     \[
     \text{Ratio}_i = \frac{B_i}{A_{i,\text{entrante}}}
     \]
    \item Seleccionar la fila con el menor ratio positivo.
    \item La variable básica de esa fila es la variable saliente.
  \end{enumerate}
  
  \noindent \textbf{Caso especial}: Si todos los coeficientes \(\leq 0\) \(\rightarrow\) Problema no acotado (no existe solución óptima finita).
\end{tcolorbox}

\paragraph{Paso 4: Realizar pivoteo}

El \textit{Paso 4 del método Simplex}, frecuentemente llamado
\textit{``Paso de pivoteo''}, es el proceso algebraico que actualiza toda la
tabla del método Simplex luego de decidir qué variable entra y cuál sale de la
base. Su objetivo es producir una \textbf{nueva solución básica factible}, en
la que se espera que el valor de la función objetivo haya mejorado.

El objetivo de este paso es \textbf{actualizar la tabla} del método Simplex de
acuerdo con el cambio de base (variable entrante y saliente) que se determinó
en el Paso 3. Esta actualización garantiza que la nueva tabla represente un
sistema equivalente, pero con una \textbf{nueva base} que contiene la variable
entrante.

\begin{enumerate}
  \item \textbf{Identificar el pivote}:\\
    El \textbf{pivote} es el elemento ubicado en la \textbf{intersección} de
    la columna de la \textbf{variable entrante} y la fila de la
    \textbf{variable saliente}. Este valor será utilizado para transformar la
    fila y la columna asociadas.

  \item \textbf{Normalizar la fila pivote}:\\
    Se transforma la fila pivote dividiendo cada uno de sus elementos
    (incluyendo el término independiente y el coeficiente en la fila \(z\), si
    corresponde) por el valor del pivote. El propósito es convertir el pivote
    en 1, lo que permitirá más adelante convertir los demás elementos de su
    columna en cero.\\
    \emph{Ejemplo}: si el pivote es 3 y la fila pivote es \((3,\;6,\;9)\), después
    de la normalización se convierte en \((1,\;2,\;3)\).

  \item \textbf{Hacer ceros en la columna pivote (excepto en fila pivote)}:\\
    Se aplica \emph{eliminación gaussiana} para anular todos los demás
    elementos de la \textbf{columna de la variable entrante}, excepto el 1 que
    ahora ocupa la posición del pivote. Para cada fila \(i\) (incluyendo la fila
    \(z\)), se realiza:
    \[
      \text{Nueva Fila}_i
      = \text{Fila}_i
      - \bigl(\text{Coeficiente}_{i,\mathrm{entrante}} \times \text{Fila Pivote}\bigr).
    \]
    Esto asegura que en todas las demás filas, la columna correspondiente a la
    variable entrante tenga valor 0, excepto en la fila pivote donde tiene 1,
    convirtiendo la columna en parte de una matriz identidad.

  \item \textbf{Actualizar la base}:\\
    En la columna de variables básicas (frecuentemente llamada ``Base''), se
    reemplaza la variable que salió con la variable que entró. Esto formaliza
    el cambio de base, reflejando la nueva estructura de la solución básica
    factible.
\end{enumerate}
Este paso es fundamental porque garantiza que:
\begin{itemize}
  \item La nueva tabla es \textbf{equivalente al sistema original}, pero
    reescrita en función de una nueva base.
  \item El sistema se mantiene en forma \textbf{canónica} (o forma estándar), es decir, la base
    sigue representada por una submatriz identidad.
  \item La nueva solución sigue siendo factible (todas las variables básicas
    siguen siendo \(\ge 0\)) y, si el procedimiento se realizó correctamente,
    \textbf{mejora o mantiene} el valor de la función objetivo.
\end{itemize}

Geométricamente, este paso corresponde a \textit{avanzar de un vértice del
poliedro (o politopo) factible al siguiente vértice adyacente}, a lo largo de una arista
que mejora el valor de la función objetivo.

Este paso responde a la pregunta:
\begin{quote}
  \textbf{¿Cómo actualizamos la tabla para reflejar el nuevo vértice factible
  al que nos hemos movido?}
\end{quote}

La respuesta es aplicar el \textbf{proceso de pivoteo}, que consiste en:
\begin{enumerate}
  \item Normalizar la fila pivote.
  \item Anular el resto de la columna pivote.
  \item Actualizar la base.
\end{enumerate}

Al final, se obtiene una nueva tabla lista para iniciar nuevamente el Paso 1.

\begin{tcolorbox}[title=Resumen del paso 4]
  \noindent \textbf{Objetivo}: Actualizar la tabla para la nueva base.
  
  \noindent \textbf{Pasos}:
  \begin{enumerate}
    \item \textbf{Identificar el pivote}: Intersección de columna entrante y fila saliente.
    \item \textbf{Normalizar la fila pivote}: Dividir toda la fila por el valor del pivote para convertirlo en 1. Por ejemplo si el pivote es 3, dividir toda la fila por 3.
    \item \textbf{Hacer ceros en la columna pivote (excepto en fila pivote)}: Para cada fila \(i\) (incluyendo fila \(z\)):
       \[
       \text{Nueva Fila}_i = \text{Fila}_i - (\text{Coeficiente}_{i,\text{entrante}} \times \text{Fila Pivote})
       \]
    \item \textbf{Actualizar la base}: Reemplazar variable saliente por variable entrante en la columna ``Base''.
  \end{enumerate}
\end{tcolorbox}

\paragraph{Paso 5: Actualizar la fila \(z\) y \textit{B} y repetir el proceso}

Realmente este no es un paso explícito como el resto de pasos, pero me parece útil mencionarlo para que puedas verificar que la tabla sigue siendo consistente. Además me parece muy importante destacar que si el problema ha llegado a su solución óptima, se detecta en el paso 1, no en este paso.

\begin{tcolorbox}[title=Resumen del paso 5]
  \noindent \textbf{Objetivo}: Actualizar la fila \(z\) y \textit{B} y repetir el proceso.
  
  \noindent \textbf{Nota}: La fila \(z\) se actualiza en el Paso 4 (operaciones fila). Si todo es consistente, repetir.
  
  \noindent \textbf{Verificar consistencia}:
  \begin{itemize}
    \item El nuevo \textit{B} en fila \(z\) debe ser el valor actual de \(z\).
    \item Coeficientes de variables básicas en fila \(z\) deben ser 0.
  \end{itemize}
\end{tcolorbox}

\subsubsection{Ejecución del método con un ejemplo}

Continuando con el ejemplo de la sección anterior:

\paragraph{Iteración 1}

\textbf{Paso 1}: Coeficientes en \(z\): \([-2, 0, 1, 0, 0]\). Hay negativo (\(-2\)) \(\rightarrow\) No óptimo.

Tabla inicial:
% Preguntar a chatgpt: 
% Cómo puedo hacer para que nicematrix incluya el famoso {c|ccccc|c} de {array}? 
\[
  \begin{NiceMatrix}[code-before = 
    \rectanglecolor{red!20}{4-2}{4-2}
  ]
  \beta & x_1 & x_2 & s_1 & a_1 & a_2 & B \\
  \hline
  a_1 & 1 & 1 & 0 & 1 & 0 & 4 \\
  a_2 & 1 & -1 & -1 & 0 & 1 & 2 \\
  \hline
  z & -2 & 0 & 1 & 0 & 0 & -6 \\
  \end{NiceMatrix}
\]

\textbf{Paso 2}: Variable entrante = \(x_1\) (coeficiente más negativo: \(-2\)).

\begin{enumerate}
  \item Paso 3 (Ratio Test):
  \begin{itemize}
    \item Columna \(x_1\): \([1, 1]^T\) (ambos \(> 0\)).
    \item Ratios: \(\frac{4}{1} = 4\), \(\frac{2}{1} = 2\).
    \item Menor ratio: \(2\) → Fila de \(a_2\) sale.
  \end{itemize}
  \item Paso 4 (Pivoteo):
  \begin{itemize}
    \item Pivote: \(1\) (intersección \(a_2\) y \(x_1\)).
    \item Normalizar fila \(a_2\): Ya es 1.
    \item Operaciones fila:
      \begin{itemize}
        \item Fila \(a_1\): \([1, 1, 0, 1, 0, 4] - (1 \times [1, -1, -1, 0, 1, 2]) = [0, 2, 1, 1, -1, 2]\)
        \item Fila \(z\): \([-2, 0, 1, 0, 0, -6] - (-2 \times [1, -1, -1, 0, 1, 2]) = [0, -2, -1, 0, 2, -2]\)
      \end{itemize}
    \item Nueva base: \(\{a_1, x_1\}\).
  \end{itemize}
\end{enumerate}

% Tabla actualizada:
% | Base | \(x_1\) | \(x_2\) | \(s_2\) | \(a_1\) | \(a_2\) | RHS  |
% |------|---------|---------|---------|---------|---------|------|
% | \(a_1\) | 0       | 2       | 1       | 1       | -1      | 2    |
% | \(x_1\) | 1       | -1      | -1      | 0       | 1       | 2    |
% | \(z\)  | 0       | -2      | -1      | 0       | 2       | -2   |

% ---

% #### Iteración 2:
% 1. Paso 1: Coeficientes en \(z\): \([0, -2, -1, 0, 2]\). Hay negativo (\(-2\)) → No óptimo.
% 2. Paso 2: Variable entrante = \(x_2\) (coeficiente más negativo: \(-2\)).
% 3. Paso 3 (Ratio Test):
%    - Columna \(x_2\): \([2, -1]^T\). Solo \(a_1\) tiene coeficiente \(> 0\) (2).
%    - Ratio: \(\frac{2}{2} = 1\) → Fila de \(a_1\) sale.
% 4. Paso 4 (Pivoteo):
%    - Pivote: \(2\) (intersección \(a_1\) y \(x_2\)).
%    - Normalizar fila \(a_1\): Dividir por 2 → \([0, 1, 0.5, 0.5, -0.5, 1]\).
%    - Operaciones fila:
%      - Fila \(x_1\): \([1, -1, -1, 0, 1, 2] - (-1 \times [0, 1, 0.5, 0.5, -0.5, 1]) = [1, 0, -0.5, 0.5, 0.5, 3]\)
%      - Fila \(z\): \([0, -2, -1, 0, 2, -2] - (-2 \times [0, 1, 0.5, 0.5, -0.5, 1]) = [0, 0, 0, 1, 1, 0]\)

% Tabla actualizada:
% | Base | \(x_1\) | \(x_2\) | \(s_2\)  | \(a_1\)  | \(a_2\)  | RHS  |
% |------|---------|---------|----------|----------|----------|------|
% | \(x_2\) | 0       | 1       | 0.5      | 0.5      | -0.5     | 1    |
% | \(x_1\) | 1       | 0       | -0.5     | 0.5      | 0.5      | 3    |
% | \(z\)  | 0       | 0       | 0        | 1        | 1        | 0    |

% ---

% #### Final de Fase I:
% - Paso 1: Coeficientes en \(z\): \([0, 0, 0, 1, 1] \geq 0\) → Solución óptima de Fase I.
% - Valor de \(z = 0\) → Factible. Pasamos a Fase II.

% ---

% ### Transición a Fase II
% 1. Eliminar columnas de variables artificiales (\(a_1, a_2\)).
% 2. Reemplazar fila \(z\) por la función objetivo original.
% 3. Recalcular costos reducidos para la base actual.

% Supongamos objetivo original: Maximizar \(x_1 + x_2\).  
% - Nueva fila \(z\): Coeficientes \([-1, -1, 0]\) (porque muestra \(-\bar{c}_j\)).
% - Cálculo:
%   - Base actual: \(\{x_1, x_2\}\), \(c_B = [1, 1]\).
%   - \(\bar{c}_{s_2} = 0 - [1, 1] \cdot [-0.5, 0.5]^T = 0 - (0) = 0\).
%   - Valor de \(z = [1, 1] \cdot [1, 3]^T = 4\).

% Tabla Fase II:
% | Base | \(x_1\) | \(x_2\) | \(s_2\)  | RHS  |
% |------|---------|---------|----------|------|
% | \(x_2\) | 0       | 1       | 0.5      | 1    |
% | \(x_1\) | 1       | 0       | -0.5     | 3    |
% | \(z\)  | 0       | 0       | 0        | 4    |

% - Solución óptima: \(x_1 = 3\), \(x_2 = 1\), \(z = 4\).

% ---

% ### Diagrama de Flujo del Algoritmo Simplex
% ```mermaid
% graph TD
%     A[Inicio con tabla inicial] --> B[Verificar optimalidad]
%     B -->|Todos coeficientes ≥ 0| C[¡Solución óptima!]
%     B -->|Hay coeficiente < 0| D[Seleccionar variable entrante]
%     D --> E[Ratio Test]
%     E -->|Todos coeficientes ≤ 0| F[¡Problema no acotado!]
%     E -->|Encontrar pivote| G[Pivoteo]
%     G --> H[Actualizar tabla]
%     H --> B
% ```

% ### Consejos clave
% 1. Fila \(z\) siempre muestra \(-\bar{c}_j\).
% 2. Ratio Test ignora coeficientes \(\leq 0\).
% 3. Variables artificiales:
%    - Fase I: Minimizar \(\omega = \sum a_i\) (transformada a \(z = -\omega\)).
%    - Si \(z_{\text{Fase I}} < 0\) → Problema infactible.
% 4. Fase II: Usar solución de Fase I y función objetivo original.